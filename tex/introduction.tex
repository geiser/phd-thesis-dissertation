\chapter{Introduction}
\label{chapter:introducao}

This chapter starts presenting the delimitations of the context and the research problem addressed in this PhD thesis dissertation  (\autoref{sec:problem-delimitation}).
After that, the chapter formulates the research questions and objectives (\autoref{sec:research-question-and-research-objectives}).
The research methodology is presented in \autoref{sec:research-methodology}.
The thesis statement and contributions are detailed in \autoref{sec:thesis-statement-and-claimed-contributions}. 
Finally, the chapter ends with the structure of this dissertation (\autoref{sec:structure-of-dissertation}).

\section{Context and Research Problem Delimitations}
\label{sec:problem-delimitation}

Over the last two decades or so, with the growing number of technologies that enable people to communicate and work in group activities using computers and Internet, researches and practitioners have developed technology and software applications that facilitate and foster the Collaborative Learning (CL) \cite{LehtinenHakkarainenLipponenRahikainenMuukkonen1999}.
The Computer-Supported Collaborative Learning (CSCL) is the research field that studies how this technology should link the advanced in computer science with the pedagogical approaches of collaborative learning, and it has been proved an important research field to support the instructional-learning process \cite{StahlKoschmannSuthers2006}.
Some of the most relevant benefits of the integration between computer sciences and collaborative learning are: to facilitate the sharing and distribution of knowledge among group members \cite{Lipponen2002,NordinKlobas2010},
to help the monitoring and evaluation of CL process \cite{Rodriguez-TrianaPrietoMartinez-MonesAsensio-PerezDimitriadis2018,CaballeDaradoumisXhafaJuan2011},
and to enhance the peer interactions and work in groups \cite{Wang2014,ZhaoGaoYang2018}.

The CSCL technology is more beneficial for the students when there is an adequate design of the CL process, when there is a mechanism to support and orchestrate the ways in which the students should collaborate to achieve pedagogical benefits \cite{Dillenbourg2013, Hewitt2005, IsotaniInabaIkedaMizoguchi2009}.
When there is not such support, students frequently fail to act or behave in a productive way.
In a CL session, the participants need to know with whom they must work,
they need to know what are their roles in the CL process, and
they need to know the steps to accomplish their learning goals.
Without this information, frequently, the participants will not adequately collaborate or leave to interact during the CL activities.
Hence, several researchers propose the use of scripts to guide and orchestrate the collaboration in these activities \cite{AlharbiAthaudaChiong2014}.

The scripted collaborative learning aims to engage the students in fruitful and significant interactions according to a design that has the purpose to attain a set of pedagogical objectives.
Thereby, the research and practitioners of the CSCL community have proposed the use of scripts to support the well-thought-out design of the CL scenarios through computer-based systems \cite{FischerKollarStegmannWeckerZottmann2013,KobbeWeinbergerDillenbourgHarrerHamalainenHakkinenFischer2007}.
The CSCL scripts are the technology that indicates how the interactions among students will be orchestrated and structured in a group activity to increase the possibility of achieving the pedagogical objectives \cite{WeinbergerErtlFischerMandl2005}. These scripts provide information that facilitates the group formation, the role distribution, and the sequencing of interaction for the participants in the CL process.
Despite of its benefits, there are situations in which the scripts may cause motivational problems.
For example, when the students prefer to work individually or when they do not want to play the role assigned by the scripts, they may neglect their personal behavior to get the task completed without effort, and the lack of choice over the interactions may produce in them a sense of obligation.
These issues cause troubles in the group dynamic - e.g. some students may dropout the CL activity, making superficial interactions - resulting in negative and widespread learning outcomes.

The motivational problems in scripted collabotive learning make more difficult the use of this technology over time.
Less motivated students prefer to spend more time in other activities instead of participated in the collaborative learning and, as consequence, the achievement of contemplated learning outcomes becomes more difficult \cite{Crook2000, FaloutElwoodHood2009, SchoorBannert2011}.
In this sense, motivating learners in the entire CL process is important.
However, the traditional instructional design practice often assumes that the motivation is a preliminary step that occurs outside to the learning process \cite{ChanAhern1999, Keller1987}.
This assumption is based in which the good quality of learning materials and content-domain can keep the students focused during the learning process, but if this process is long, there is a good chance that the students will lose their initial motivation.
To avoid this motivational problem known as demotivation, some researchers and practitioners currently propose the development and use of affective feedback systems \cite{FeidakisDaradoumisCaballeConesa2014,FeidakisCaballeDaradoumisJimenezConesa2014} based on emotion-aware systems and learning companions to motivate students along the entire learning process \cite{WoolfBurlesonArroyoDragonCooperPicard2009}.
These solutions assume that the students like the content-domain and that they have the desired to learn working in groups, so that the approach of using affective feedback systems does not motivate and engage students without the desire to learn or to work in groups.

In the last years, efforts of CSCL community have been directed to finding new innovative solutions that, beside to motivate and engage students during the entire CL process, are not completely tied to the domain-content and desired to learn working in groups.
In this direction, several researchers and practitioners have pointed Gamification as a promising technology to deal with motivational problems in educational contexts \cite{ChallcoMoreiraMizoguchiIsotani2014, SeabornFels2015, BorgesDurelliReisIsotani2014}.
Gamification \aspas{\emph{as the use of game design elements in non-game contexts}} \cite{DeterdingDixonKhaledNacke2011} aims to increase the students' motivation and engagement by making the learning process more game-like.
Through the introduction of game elements, such as points, rankings, competition, cooperation and so on, gamification intends to engage and motivate students who do not have the desire or interest in to learn the content-domain working in groups.
According to \cite{Kapp2012, KnutasIkonenNikulaPorras2014}, when a learning process is gamified, the benefits of introduced game elements will strongly depend on how well these game elements will be applied, and how well they are linked with the pedagogical approaches employed in the learning process.

When CL scenarios are gamified to deal with the motivational problems that can occur in a scripted collaborative learning, the thesis author hypothesizes that the chances to achieve engagement and educational benefits will be increased if there is a proper connection between the game elements and the CL process.
However, developing such well-though-out gamified CL scenario, hereinafter called gamified CL scenarios, is a non-trivial task.
The main difficulty to gamify CL scenarios as well as other non-game context is that the gamification is too context dependent \cite{HamariKoivistoSarsa2014, RichardsThompsonGraham2014}.
Its effects vary from individual to individual, from situation to situation, and occasionally.
Gamification depends of many factors such as the individual personality traits, preferences, and current student's emotions \cite{Nicholson2015, PedroLopesPratesVassilevaIsotani2015}
(e.g., a ranking of participation would motivate users who like competition but not users who want to customize their items and avatars).
Also, the expected effects of the game elements vary according to the non-game context and the target behavior that is being gamified \cite{DeterdingBjorkNackeDixonLawley2013, HeeterLeeMedlerMagerko2011}
(e.g., gamifying a learning scenario to promote the signing-up is not the same thing as gamifying an interactive environment to maintain the student's attention).
As consequence of this context-dependency, when a CL scenario is not well gamified, instead to have a positive effect, they may cause a detrimental on the students' motivation \cite{AndradeMizoguchiIsotani2016}, cheating \cite{NunesBittencourtIsotaniJaques2016}, embarrassment \cite{OhnoYamasakiTokiwa2013}, and lack of credibility on badges \cite{DavisSingh2015}.

Another difficulty to gamify CL scenarios, as well as other non-game contexts, it is the lack of approaches to systematically represent in an unambiguous way the gamification knowledge acquired in the last years by researchers and practitioners.
This knowledge, hereinafter referred to as the \emph{knowledge from theories and practices of gamification}, is constituted by gamification practices, games design models and the theoretical psychological employed by researchers and practitioners to gamify different non-game contexts; and it lacks of a formal and common vocabulary, definitions, and representation to be easily applied.
As can be appreciated in the current literature of gamification \cite{DichevaDichevAgreAngelova2015, HamariKoivistoSarsa2014, MoraRieraGonzalezArnedo-Moreno2015, SeabornFels2015}, each author proposes his/her own definitions, classifications and representations of concepts and characteristics about how to gamify a non-game context.
This fact hinders the creation of models/frameworks that formally represent the gamification and its application in a common understandable and shareable manner.
To the best of the thesis author's knowledge, there is no one approach to represent the knowledge about how to gamify CL scenarios, and how through the gamification is possible to deal with motivational problems in a scripted collaborative learning.

Owed to the variety of students who can participate in CL sessions, the diversity of subjects that can be under study in a CL activity, and the range of different CSCL scripts used to orchestrate the CL process, it is necessary to personalize the gamification for each student and situation, so providing a tailored gamified CL scenario is necessary to achieve better benefits of gamification.
Developing tailored gamified CL scenarios is a difficult and time-consuming task, so that a computational based-support to personalize the gamification is necessary and very helpful.
In this direction, in the context of CSCL, there is only one interesting approach that proposes to personalize gamification based on individual preference profiles estimated from an interaction analysis by machine learning techniques \cite{KnutasvanRoyHynninenGranatoKasurinenIkonen2017,KnutasIkonenMaggioriniRipamontiPorras2016,KnutasIkonenNikulaPorras2014}.
However, this solution is not oriented to deal with motivational problems in the scripted collaborative learning, its purpose is to increase the communication among the participants in any CL scenarios (not necessarily scripted).
Furthermore, this solution does not provide a model to share theoretical knowledge of gamification obtained by its computational mechanisms and procedures to personalize the gamification.
Solutions based on machine learning to personalize gamification require many data to support the personalization of gamification, and they may fall in an over-fitting or under-fitting problem with the data.
A computational mechanism based only in machine learning techniques to personalize gamification will always lack of theoretical-justification to explain why a game element has been introduced in the non-game context, and why a certain configuration of game elements engages and motivates the students as participants of a CL activity to continue and adequately interact in the CL scenario.

For the reason exposed above, to deal with motivational problems in a scripted collaborative learning through the gamification, a computational support is essential to overcome the challenges and difficulties of gamification, a computational system with common and shareable structures to represent knowledge from practices and theories of gamification.
In this direction, we have ontologies as the most advanced technology to support the representation of knowledge in a common understandable and shareable manner for computers and humans \cite{AsikriLaassiriKritChaib2016, Devedzic2006, MizoguchiBourdeau2016}.
Ontologies constitute an explicit mapping between the target world of interest and its representation with the purpose to delineate concepts without ambiguities providing a common way to represent the knowledge \cite{GuarinoOberleStaab2009}.
Taking advantages of this commonality and using the computer interconnection provided by Internet, computational mechanisms and procedures for intelligent tools are developed to use the ontologies as a language to share the understandings and interpretations of \emph{target world} - in this thesis dissertation, the target world is the gamification of CL scenario.

Employing ontologies to represent the knowledge from theories and practices of gamification, some interesting and practical results have been obtained by \citeonline{DermevalVilelaBittencourtCastroIsotaniBritoSilva2016, KarkarAlJaamFoufou2016, ZouaqNkambou2010}.
However, to the best of the thesis author knowledge, there is no one ontology that, from a philosophical perspective, gives support for the systematical representation of it knowledge and how to apply it in CL scenarios to deal with motivational problems.
Therefore, the general research goal in this thesis dissertation refers to the definition of this ontology from a philosophical perspective.



\section{Research Questions and Research Objectives}
\label{sec:research-question-and-research-objectives}

The research topic explored in this PhD thesis dissertation is addressed to answer the question:
\aspas{\emph{How to deal with motivational problems in scripted collaborative learning?}}

To answer this research question, the author of this thesis proposes the gamification of CL scenarios.
However, as the previous section explained the gamification is too context dependent, so that, to obtain tailored gamified CL scenarios, intelligent tools are necessary to provide support in the personalization of gamification. 
In this sense, it is necessary a common way to represent the knowledge extracted from practices and theories of gamification because an adequate gamification of CL scenarios should rely on this theoretical knowledge.
Thus, thesis author proposed the use of ontologies to represent the knowledge from theories and practices of gamification because ontologies have been consolidated as the technology in which computers and humans use a common language to build models/frameworks about the target world that is being represented. 
Thereby, the overarching research question (\textbf{RQ}) addressed in this thesis is: 

\aspas{\emph{How can gamification and ontologies be used to deal with motivational problems in scripted collaborative learning?}}

To answer this research question, the ontological engineering approach to gamify CL scenarios shown in \autoref{fig:ontological-engineering-approach-to-gamify-cl-scenarios} has been proposed by the author of this PhD thesis dissertation. 
This approach consists into three major stages described as follows:

\begin{figure}[htb]
 \caption{Ontological engineering approach to gamify CL scenarios}
 \label{fig:ontological-engineering-approach-to-gamify-cl-scenarios}
 \centering
 \includegraphics[width=0.95\textwidth]{images/chap-introduction/ontological-engineering-approach-to-gamify-cl-scenarios.png}
 \fautor
\end{figure}

\begin{enumerate}
\item
The first stage is: the formalization of the necessary knowledge about how to gamify CL scenarios for dealing with motivational problems in scripted collaborative learning into an ontology named \textbf{OntoGaCLeS} – \emph{\textbf{Onto}logy to \textbf{Ga}mify \textbf{C}ollaborative \textbf{Le}arning \textbf{S}cenarios}.
This ontology has been developed using ontology engineering in which, by extracting concepts from the theories and practices of gamification, the thesis author defines a set of ontological structures to enable the systematic formalization and representation of knowledge to gamify CL scenarios and its theoretical foundation.

\item
The second stage is: the development of computational mechanisms and procedures whereby intelligent tools will provide support in the gamification of CL scenarios to deal with motivational problems in a scripted collaborative learning.
Such support is given by the knowledge formalized in the ontology OntoGaCLeS during the first stage, so that the purpose for the computational mechanisms and procedures in intelligent tools is to use the ontological structures from this ontology to facilitate the tasks of instructional designer and practitioners, especially novice users, in the gamification of CL scenarios.
These ontological structures contain the theoretical justification for the personalization of gamification, and they are used to obtain tailored gamified CL sessions adapted for each situation.
These sessions are known as ontology-based CL sessions, and they are CL scenarios that have been gamified and instantiated at the most concrete level of CL scenarios in which the participants and the content-domain to be learned are well defined and it can be directly run in a learning environment.

\item
The third stage is: the execution of empirical studies to understand the effects of ontology-based gamification on CL scenarios, and then, to validate the ontological engineering approach to gamify CL scenarios as a method to deal with motivational problems in scripted collaborative learning.
This validation has carried out in ontology-based gamified CL sessions obtained by the approach, and it consists in measuring the effectiveness and efficiency of these sessions for dealing with motivational problems.
\end{enumerate}

Regarding to the formalization of knowledge about how to gamify CL scenarios for dealing with motivational problems in a scripted collaborative learning (Stage 1), the research questions answered by this dissertation are:

\begin{description}
\item[RQ1:]the
\emph{Which concepts from theories and practices of gamification should be contemplated to deal with motivational problems in a scripted collaborative learning?}, and
\emph{How should these concepts be applied in the gamification of CL scenarios?}

\item[RQ2:]
\emph{What ontological structures are necessary to represent the concepts identified as relevant in the theories and practices of gamification to deal with motivational problems in scripted collaborative learning?}

\end{description}

Regarding the development of computational mechanisms and procedures whereby intelligent tools will provide support in the gamification of scenarios using the knowledge described in the ontology OntoGaCLeS (Stage 2), the research questions answered by this dissertation are:

\begin{description}
\item[RQ3:]
\emph{What computational mechanisms and procedures are necessary in intelligent tools to give a helpful support in the gamification of CL scenarios?} and
\emph{How can the knowledge encoded in the ontology OntoGaCLeS be used by these mechanisms and procedures to deal with motivational problems in a scripted collaborative learning?}
\end{description}

Regarding to the validation of the ontological engineering approach to gamify CL scenarios as a method to deal with motivational problems in scripted collaborative learning (Stage 3), the research questions answered by this dissertation are:

\begin{description}
\item[RQ4:]
\emph{What are the effectiveness and efficiency of the ontological enginnering approach to gamify CL scenarios to deal with motivational problems in scripted collaborative learning?}
\end{description}

The research objectives pursued to answer the research questions \emph{RQ1} and \emph{RQ2} are:

\begin{description}
\item[RO1:]
To review the scientific literature in order to identify the most relevant concepts from the theories and practices of gamification that should be taking into account to deal with motivational problems in scripted collaborative learning; and

\item[RO2:]
To define the ontological structures to represent the concepts identified as relevant in the theories and practices of gamification to deal with motivational problems in scripted collaborative learning.
\end{description}


In order to answer the research question \emph{RQ3}, the research objectives is:

\begin{description}
\item[RO3:]
To identify and define the computational mechanisms and procedures that must be implemented by intelligent tools to give a helpful support in the gamification of CL scenarios, and how these mechanisms and procedure use the knowledge encoded in the ontology OntoGaCLeS for dealing with the motivational problems in a scripted collaborative learning.
\end{description}

The research objective pursued to answer the research question \emph{RQ4} is:

\begin{description}
\item[RO4:]
to analyze the effects of ontology-based gamified CL sessions on the students’ motivation and learning outcomes for validating the ontology engineering approach to gamify CL scenarios in reference to the effectiveness and efficiency to deal with the motivational problems in a scripted collaborative learning.
\end{description}

It is out of scope in this dissertation to deal with the following objectives:

\begin{itemize}
\item
To compare, validate or judge the theories and practices of gamification.

\item
To create, modify or extend the concepts described in the theories and practices of gamification.

\item
To create a generic and complete representation of all concepts described in the theories and practices of gamification. The thesis author concentrates only on the formalization of the minimal necessary concepts from these practices and theories to deal with the motivational problems in scripted collaborative learning.

\item
To validate the concepts and ontological structures formalized in the ontology OntoGaCLeS using semantic reasoner engines or formal methods based on logic and/or mathematics.
\end{itemize}

\section{Research Methodology}
\label{sec:research-methodology}

As this PhD thesis dissertation is framed in the multidisciplinary field of CSCL with research questions and research objectives oriented to be answered and achieved by theoretical and empirical studies, a mixed research method needs to be employed to conduct this research.
Following the research methodology framework proposed by \citeonline{Glass1995,GlassVesseyRamesh2002}, the mixed research method employed in this PhD thesis research consisted in four iterative phases: informational, propositional, analytical and evaluation.

\begin{description}
\item[Informational phase:]
In this phase, the thesis author identified the research problems and potential solutions based on information gathered from the scientific literature and discussions with experts in fields of CSCL, gamification and ontology engineering.
The results of this phase were an outline of the knowledge involved in this dissertation, the research questions, and the research objectives.
The tasks carried out in this phase correspond to tasks extracted from the scientific (observing the world) and engineering (observing existing solutions) research methods.
These tasks were:

\begin{itemize}
\item
The search, review and analysis of scientific literature regarding to: CSCL, gamification and ontology engineering.
The thesis author performed this literature review emphasizing in the subjects of scripted collaborative learning, gamification of learning and instruction, and ontology-engineering applied to Artificial Intelligence in Education (AIED).

\item
The participation as member of the research group in Applied Computing in Education Laboratory (CAEd-Lab, \emph{Laboratorio de Computação Applicada a Educação e Tecnologias Sociales Avançadas}) at the University of São Paulo.
Particularly, the expertise field in CSCL and Ontologies of this research group has been very important and valuable to conduct the research and the literature reviews.

\item
The participation in several conferences and workshops related to the context and problem domain in which this dissertation is framed.
These conferences and workshop, in chronological order, were:
the III Escola de Ontologias UFAL-USP, 2014 (Workshop);
the 20\textsuperscript{th} International Conference on Collaboration and Technology, CRIWG, 2014 (Conference);
the Summer School on Computers in Education, 2015 (Workshop);
the XXVI Brazilian Symposium on Computers in Education, 2015 (Conference);
the 6\textsuperscript{th} Latin American School for Education, Cognitive and Neural Sciences, 2016 (Workshop); and
the Higher Education for All: International Workshop on Social, Semantic, Adaptive and Gamification techniques and technologies for Distance Learning, 2017 (Workshop).

\item
The participation as visiting research at the Research Center for Service Science at the School of Knowledge Science in the Japan Advanced Institute of Science and Technology (JAIST) has also been significant for the informational phase.
The focus of this research is to study, design and implement knowledge co-creation process in complex service systems.
This research center focuses in the use of ontologies and ontology-engineering as the technology to develop and solve a broad variety of domains/tasks, and their research members have a long history working in the research field of Artificial Intelligence in Education.
Particularly, the expertise of the Prof. Mitsuro Ikeda and Prof. Riichiro Mizoguchi were valuable and important for this phase resulting from their involvement in various research projects about the modeling of knowledge for the students' learning growth, CL process, and instructional design.
\end{itemize}

\item[Propositional phase:]
In this phase, solutions were proposed and formulated using the information gathered in the previous phase.
As results of the propositional phase, ontological structures to represent the necessary concepts to gamify CL scenarios were identified and formalized in the ontology OntoGaCLeS.
Prototypes of computational mechanisms and procedures to be used by intelligent tools to gamify CL scenario were developed for gathering instructional designers' opinions as early feedback of these systems.
The tasks carried out in this phase correspond to activities extracted from the scientific (proposing theories or models) and engineering (proposing and developing solutions) research methods.
These tasks were:

\begin{itemize}
\item
The proposal of ontological structures to represent gamified CL scenarios and ontological models to personalize the gamification of CL scenarios based on player type models and need-based theories of motivation.

\item
The proposal of ontological structures to represent the application of persuasive game design models in gamified CL scenarios and ontological models to apply persuasive game design strategies as a method for dealing with the motivational problems.

\item
The proposal of a computer-based model to support the representation of the learners' growth process and the principle of good balance between the perceived challenges and skills defined in the flow theory.


\item
The definition of a conceptual flow to gamify CL scenarios as a procedure to use the knowledge described in the ontology OntoGaCLeS, and
the definition of a reference architecture based on this flow to build intelligent tools for dealing with motivational problems in a scripted collaborative learning.
\end{itemize}

\item[Analytical phase:]
This phase consists into analyze and explores the solutions formulated in the propositional phase with the purpose to identify whether the proposed solutions are understandable, how them can be deployed into practice, what are the potential problems in understanding and using them, and wether there are any omissions or gaps in these solutions.
The tasks carried out in this phase correspond to activities extracted from the empirical (applying to case studies) and analytical (developing new solutions derived from the results obtained in the case studies) research methods.
These tasks were:

\begin{itemize}
\item
The formalization of an ontological models to personalize the gamification of CL scenarios.
This formalization is a case study to validate in the evaluation phase the ontological structures proposed to systematically formalize ontological models to personalize the gamification of CL scenarios.

\item
The formalization of an ontological model to apply gamification as a persuasive technology in CL scenarios.

\item
The implementation of a computational mechanism (as a proof of concept) in which the knowledge encoding in the ontology is used for setting up the proper player roles and game elements for CL sessions.

\item
The development of an algorithm (as a proof of concept) to apply the principle of good balance between the perceived challenges and skills from the flow theory in the gamification of CL scenarios.

\item
The development of a computational mechanism (as a proof of concept) to apply gamification as persuasive technology in the gamification of CL scenarios.
\end{itemize}


\item[Evaluation phase:]
The focus of this phase is to conduct empirical tests and evaluations for the solutions formulated in the propositional phase and for the findings found in the analytical phase.
In this phase, the data gathered through the tests and evaluations aimed to assess the contributions from different perspectives.
The tasks carried out during this phase correspond to activities from the empirical (validating the solutions) and analytical (analyzing the results obtained from empirical observations) research methods.
These tasks were:

\begin{itemize}
\item
The analytical evaluation of the ontological structures proposed to represent gamified CL scenarios and the ontological models to personalize the gamification of CL scenarios.
This evaluation was carried out by publishing these ontological structures and the ontological models obtained from them as scientific articles in conferences and journals of the fields of CSCL, and Artificial Intelligent in Education. 
These articles, in chronological order, were:
\aspas{\emph{Towards an Ontology for Gamifying Collaborative Learning Scenarios}} published in the 12\textsuperscript{th} International Conference on Intelligent Tutoring Systems, ITS, 2014;
\aspas{\emph{An Ontology Engineering Approach to Gamify Collaborative Learning Scenarios}} published in the 20\textsuperscript{th} International Conference on Collaboration and Technology, CRIWG, 2014; and
\aspas{\emph{Personalization of Gamification in Collaborative Learning Contexts using Ontologies}} published in the journal of IEEE Latin America Transactions, 2015.
During the conferences, important feedbacks to improve the ontological structures were obtained from discussions with the participants of the conferences who shared their expertise in the domain of CSCL and Artificial Intelligent in Education.

\item
The analytical evaluation of the ontological structures proposed to represent the application of persuasive game design models in gamified CL scenarios and the ontological models to apply persuasive game design strategies as a method for dealing with motivational problems in scripted collaborative learning.
This evaluation was carried out by publishing these ontological structures and the ontological models obtained from them in the analytical phase as scientific articles scientific articles in conferences and journals related to the fields of CSCL, and Artificial Intelligent in Education.
These articles, in chronological order, were:
\aspas{\emph{Steps Towards the Gamification of Collaborative Learning Scenarios Supported by Ontologies}} published in the 17\textsuperscript{th} International Conference on Artificial Intelligence in Education, AIED, 2015;
\aspas{\emph{An Ontological Model to Apply Gamification as Persuasive Technology in Collaborative Learning Scenarios}} published in the 26\textsuperscript{th} Brazilian Symposium of Informatics in Education, SBIE, 2015;
\aspas{\emph{Gamification of Collaborative Learning Scenarios: Structuring Persuasive Strategies Using Game Elements and Ontologies}} published in the 1\textsuperscript{st} International Workshop of Social Computing in Digital Education, SOCIALEDU, 2015; and
\aspas{\emph{An Ontology Framework to Apply Gamification in CSCL Scenarios as Persuasive Technology}} published in the Brazilian Journal of Computers in Education, 2016.
During the conferences, important feedbacks to improve the ontological structures were obtained from discussions with the participants of the conferences who shared their expertise in the domain of CSCL and Artificial Intelligent in Education.

\item
The conduction of a pilot empirical study in which, prior to carry out the full-scale empirical studies, the activities, methods, instruments and activities that have been used in the full-scale studies were evaluated to adjust and improve the full-scale study design.
This empirical study has been conducted to assess the effectiveness of \emph{the ontological engineering approach to gamify CL scenarios} for dealing with the motivational problems in scripted collaborative learning.
Such effectiveness was measured by comparing the participants' motivation and learning outcomes in the ontology-based CL sessions against the participants' motivation and learning outcomes in non-gamified CL sessions, and the percentage of participation by groups.

\item
The conduction of two full-scale empirical to evaluate the effectiveness of \emph{the ontological engineering approach to gamify CL scenarios}.
This effectiveness has been measured in the empirical studies by comparing the participants' motivation and learning outcomes in ontology-based gamified CL sessions against the participants' motivation and learning outcomes in non-gamified CL sessions.


\item
The conduction of a full-scale empirical study to evaluate the efficiency of \emph{the ontological engineering approach to gamify CL scenarios} for dealing with motivational problems in scripted collaborative learning.
Such efficiency was measured by comparing the participants' motivation and learning outcomes in ontology-based CL sessions against the participants' motivation and learning outcomes in CL sessions that were gamified without using the support given by the ontology OntoGaCLeS.
\end{itemize}
\end{description}


\section{Thesis Statement and Claimed Contributions}
\label{sec:thesis-statement-and-claimed-contributions}

The thesis statement of this PhD thesis dissertation is that:

\aspas{\emph{For CL activities where the CSCL scripts are used as a method to orchestrate and structure the collaboration among the participants, the ontological engineering approach to gamify CL scenarios, understood from the viewpoint of an instructional designer as the gamification of CL scenarios in which the ontology OntoGaCLeS is used as support to personalize the gamification, constitutes an effective and efficient solution to deal with motivational problems}.}

Related to this thesis statement, the claimed contributions discussed throughout this PhD thesis dissertation are:

\begin{enumerate}
\item 
The identification of relevant concepts from the theories and practices of gamification to deal with motivational problems in scripted collaborative learning (RO1).

\item
Ontological structures to represent: the concepts identified as relevant in theories and practices of gamification for dealing with motivational problems in scripted collaborative learning (RO2).

\begin{enumerate}
\item
Ontological structures to represent: gamified CL scenarios, and ontological models to personalize the gamification of CL scenarios based on player type models and need-based theories of motivation.

\item
Ontological structures to represent: persuasive game design in CL scenarios, and ontological models to apply persuasive game design strategies as a method for dealing with the motivational problems in scripted collaborative learning.
\end{enumerate}

\item 
A computer-based model to support the representation of the learners' growth process and the principle of good balance between challenges and abilities defined in the flow theory.

\item
A conceptual flow to gamify CL scenarios using the knowledge described in the ontology OntoGaCLeS, and a reference architecture based on this flow to build intelligent tools that provide theoretical support for dealing with motivational problems in scripted collaborative learning (RO3).

\item
Empirical evaluations of \emph{the ontological engineering approach to gamify CL scenarios} in which, to validate the effectiveness and efficiency of this approach to deal with motivational problems, the participants' motivation and the learning outcomes in ontology-based gamified CL sessions are compared against the participants' motivation and the learning outcomes in non-gamified CL sessions and in CL sessions gamified without the support given by the ontology (RO4).
\end{enumerate}

\section{Structure of the Dissertation}
\label{sec:structure-of-dissertation}

This PhD thesis dissertation is structured in eight chapters that are described as follow:

\begin{description}

\item[Chapter 1:]
\emph{Introduction}

\item[Chapter 2:]
\emph{General Background and Fundamental Concepts} contains the background related to the research problem addressed in this dissertation.
An overview related to the fields of CSCL and scripted collaborative learning, gamification and ontology engineering are presented in the chapter.
The motivational problems in scripted collaborative learning and the current approaches to deal with these problems are detailed in the chapter.
The concepts that were identified as relevant in the theories and practices to deal with the motivational problems through gamification of CL scenarios are presented in the chapter.

\item[Chapter 3:]
\emph{Ontological Structure to Personalize the Gamification in CL Scenarios} describes the ontological structures formalized in the ontology OntoGaCLeS to represent gamified CL scenarios.
These ontological structures support the personalization of gamification in CL scenarios based on player types models and need-based theories of motivation.
Therefore, the chapter also shows the procedure followed by the thesis author to build an ontological model ontological model to personalize the gamification of CL scenarios.

\item[Chapter 4:]
\emph{Ontological Structures of Persuasive Game Design in CL Scenarios} describes the ontological structures proposed to apply persuasive game design models in CL scenarios.
The chapter also describes the procedure employed by the thesis author to formalize an ontological model to apply gamification as persuasive technology in Cognitive Apprenticeship scenarios.

\item[Chapter 5:]
\emph{A Unify Modeling of Learners' Growth Process and Flow Theory} presents the computational model proposed to unify the modeling of the learners' growth process and the principle of good balance between the perceived challenges and skills described in the flow theory.
This model has been used in the gamification of CL scenarios to define the reward levels given in the CL process as an attempt to maintain the flow states of participants.

\item[Chapter 6:]
\emph{Computer-based Mechanisms and Procedures to Gamify CL Scenarios} describes a flow proposed to gamify CL sessions based on the knowledge described in the ontology OntoGaCLeS.
Based on this flow, a reference architecture by which intelligent tools to provide support in the gamification of CL scenarios for dealing with motivational problems is presented in the chapter.
The chapter also describes the computational mechanisms and procedures developed based on the reference architecture to conduct the evaluation of the ontological engineering approach to gamify CL scenarios.

\item[Chapter 7:]
\emph{Evaluation of the Ontological Engineering Approach to Gamify CL Scenarios} presents the empirical studies carried out in real situations to validate the effectiveness and efficiency of this approach to deal with motivational problems.

\item[Chapter 8:]
\emph{Conclusions and Future Work} summarizes the contributions of this PhD thesis dissertation, and the chapter also discusses possible future research directions. 

\end{description}

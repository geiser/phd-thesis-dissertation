
Tasks in the cognitive scope, such as attention, perception, categorization and memorization are affected by affective states.
Several studies indicated that negative affective states narrow cognitive scope, whereas positive affective states broaden cognitive scope \cite{HicksFriedmanGableDavis2012,GableHarmon-Jones2008,Fredrickson2001,KimchiPalmer1982,Navon1977,Easterbrook1959}.
More specifically, recent studies indicate that, regardless of the positive or negative (emotional) valence of the affective state, the cognitive scope are broaden or narrow according to the motivational intensity of affective states (broaden when it has low motivational intensity, and narrow when it has high motivational intensity) \cite{Harmon-JonesGablePrice2012,DomachowskaHeitmannDeutschGoschkeScherbaumBolte2016,Harmon-JonesGablePrice2013}.
In this sense, in the context of CSCL, there are several studies that observe and highlight the importance of taking into account the affective states as a factor that affects the learning outcomes \cite{DaradoumisArguedasXhafa2013, ZhouChen2014,ReisRodriguezLyraJaquesBittencourtIsotani2015,ReisIsotaniRodriguezLyraJaquesBittencourt2018}.
For example, one study indicated that the lack of affection (defined as beloging and connectedness with group members) is correlated to the students' resistance to work in groups due to the amotivation \cite{SoBrush2008}. 
In other study, avoiding to participate in a collaborative writing was observed as a consequence of the lost of a sense of personal ownership or peer ownership \cite{CaspiBlau2011}.
The study of \citeonline{Gonzalez-IbanezShah2014} identified that, in collaborative information seeking, the performance of participants varies according to their different combinations as affective states related to their mood.
Pair of participants with negative-negative moods significantly performed better than other combinations, and pair of participants with positive-negative moods had lower level of performance.
\citeonline{ReisRodriguezChallcoLyraMarquesJaquesBittencourtIsotani2016} investigated the effects of affective states on the group formation based on personality traits.
This study observed that unsociable characteristic from introverted participants negatively affect their performances in CL activities.
This study also showed that, contrary to the common-sense, the impulsive characteristics of extroverted participants is a non-threat for the group work.





When one looks more closely at the traditional approaches for dealing with the motivational problems, one will notice that these solutions 

motivational problems are the reasons why people do not act or behave in a certain way.
In this sense, an easier solution can be to remove the source of the motivational problems when it is an external factors, but the use of CSCL scripts to orchestrate and 


the traditional approaches does 


While most people would suggest that intrinsic motivation is best, it is not always possible in every situation. In some cases, people simply have no internal desire to engage in an activity. Excessive rewards may be problematic, but when used appropriately, extrinsic motivators can be a useful tool. For example, extrinsic motivation can be used to get people to complete a work task or school assignment in which they have no internal interest.




traditional approaches focus on solving 

by looking at how the traditional approaches deal with motivational problems in scripted collaboration, they focus 

can be used to deal with motivational problems 

use of affective feedback systems and the instructional design models that focus on motivation ,  

whether the motivation arises, i from outside or inside the individual, 

(extrinsic) or inside (intrinsic) the individual

Psychologists have proposed some different ways of thinking about motivation, including one method that involves .


While both types are important, researchers have found that intrinsic motivation and extrinsic motivation can have different effects on behaviors and how people pursue goals. In order to understand how these types of motivation influence human action, it is important to understand what each one is and how it works.



By fostering the use of gamification, Instructional Designers kind of acknowledge that their content is not compelling; otherwise why would they need to gamify it? Gamifying boring content does not make it more interesting or more relevant, it just makes it more interesting to use. What is required is not bells and whistles to extrinsically motivate learners, but more relevant content to foster intrinsic motivation.

Using gamification to make a boring subject matter more appealing, and using extrinsic rewards to compensate for low intrinsic motivation, is likely killing whatever intrinsic motivation to learn is left. So since there is little evidence, if any, that the use of gamification returns any significant performance improvement, the fact is that there is more to lose than there is to gain with gamification.

The question is not to know whether gamification makes content more appealing or more engaging, but whether it makes it more effective. And based on the research to date, there is very little, empirical proof of the instructional effectiveness of gamification, at least in the mid to long term. So instead of trying to make the container more appealing, maybe efforts should be put on creating better, more inspiring, and more relevant content.

%%%%%%%





less self-determined kinds of motivation (amotivation and external regulation) lead to more negative consequences (Martín-Albo et al., 2009, Vallerand, 1997). These relations indicate that students who feel pleasure and satisfaction when performing an academic task may have higher levels of persistence to continue to carry out the task in the future, and vice versa.






 Thereby, by providing tools that increase the awareness of task conflict, the lack of motivation caused by Wiki issues for the CL process are addressed.


the lack of motivation is pointed out as a problem caused by the missing of an effective incentive mechanism.


Thereby, by providing tools that increase the awareness of task conflict, the lack of motivation caused by Wiki issues for the CL process are addressed.



Low level of participation[4], low persistence[5,6], dropout[7], superficial interaction[8].


% ======= %

The students' motivation is one of the most important aspect that should be considered to obtain a well-thought-out designed learning scenario.



Students are unwilling to participate in groups.
Backward students prefer teamwork to work less and take advantage of the work developed by their mates.
In general, students prefer to work individually.
Team-works are more time-consuming for students.
Students can no longer aspire to higher marks when team-working


% ======= %

% =====  %

Malhotra and Galletta (2003) study motivation factors with regard to the implementation of knowledge management systems. Furthermore, they discard the traditional distinction between intrinsic and extrinsic factors as opposites. Intrinsic and extrinsic motivation is seen as part of a continuum where users (and learners) transition from a status of lack of motivation, through a series of externally-driven regulations and incentives, to reach a self-determined level of intrinsic regulations which increases their curiosity and determination to achieve a specific objective.

Malhotra, Y. & Galletta, D. (2003). Role of commitment and motivation in knowledge management systems implementation: Theory, conceptualization, and measurement of antecedents of success. In Proceedings of 36th annual Hawaii international conference on systems sciences (pp. 1–10).

% ===== %

% ===== %



% ===== %


Motivation 


An underlying theory base of good instruction generally involves instructional design considerations for motivation (Bohlin, Milheim, & Viechnicki, 1990). Motivation is 




It's pointed out in different studies 
CSCL scripts with too much structure support may lead to a lack of motivation , 



The aim of our approach was to try to break down some of the barriers to participation, including lack of confidence, lack of experience and the lack of motivation.


%In the following paragraphs, we first discuss some relevant approaches to deal with the motivational problems caused by the scripting collaboration found in the literature that are related to the approaches proposed in this dissertation.%We characterize our collaboration script within a classification framework proposed by Dillenbourg (2002) and introduce the experimental paradigm it has been investigated with. We provide empirical support for our hypothesis from a recent study (see Rummel & Spada, 2005b)



Motivation is a core factor in the learning–teaching process to improve active learning (Pintrich, 2003), because motivation concerns energy, direction, persistence and equifinality – all aspects of activation and intention (Ryan & Deci, 2000a). The literature shows a high diversity in terms and approaches about motivation (Murphy & Alexander, 2000). From these different conceptual models, the self-determination theory can be a theoretical framework very useful to understand motivation within the educational and academic contexts (Deci et al., 1991, Vansteenkiste et al., 2006). 

 Self-determination theory (Deci & Ryan, 1985) emphasizes the importance of the development of internal human resources for personal development and self-regulation of behavior. Self-determination is based in intrinsic motivation, or prototypical manifestation of the human tendency toward learning and creativity, and in self-regulation, which is concerned with how the people assume social values and extrinsic contingencies and progressively transform these into personal values and self-motivation (Ryan & Deci, 2000a).

There are several dimensions of motivation depending on the level of self-determination, ranging through a continuum from more to less self-determination:
1.
Intrinsic motivation refers to doing something because it is inherently interesting or enjoyable; intrinsic motivation is an important phenomenon for educators because it is a natural wellspring of learning and achievement that can be systematically catalyzed or undermined by instructor practices, and because intrinsic motivation produces results in high-quality learning and creativity (Ryan & Deci, 2000b).

2.
Extrinsic motivation via identified regulation – a more self-determined or somewhat internal regulation – implies an option as it occurs when the behavior is considered important for the subject’s goals and values.

3.
Extrinsic motivation via external regulation – a less self-determined or more external regulation – refers to doing something because it leads to a separable outcome – to obtain a reward or to avoid a punishment.

4.
Amotivation, the least self-determined dimension, implies non-regulation and occurs when individuals do not perceive the contingencies between the behavior and its consequences, and behavior has not intrinsic or extrinsic motivators (Ryan & Deci, 2000a).


These authors note that each type of motivation leads to different consequences.
Previous research carried on from this model has shown that the most self-determined forms of motivation (i.e., intrinsic motivation and identified regulation) are more closely associated with positive consequences such as the natural propensities for growth and integration, as well as personal well-being and constructive social development.
On the other hand, the most negative consequences, for instance, a low self-esteem and avoidance behaviors, are linked to lower levels of self-determination, such as amotivation and external regulation.

Moreover, self-determination theory postulates that social and environmental factors affect motivation, facilitating or inhibiting intrinsic motivation and its potential positive consequences. These factors are present in educational contexts, especially in collaborative learning and group active learning where social interaction, sensemaking processes and collective processes in distributing, sharing and interpreting information and knowledge play a central role (Alcover, Gil, & Barrasa, 2004). Research results point out that, rather than focusing on rewards for motivating students’ learning, it is important to focus more on how to facilitate intrinsic motivation (Deci, Kostner, & Ryan, 2001). One of the environmental factors that may be more relevant in educational settings to enhance motivation refers to the learning strategies and teaching methodologies used (Schunk, Pintrich, & Meece, 2008). Therefore, to research the educational methodologies and group and collaborative learning from an intrinsic motivation view is very interesting and relevant in the EHEA context.



%%%%%%%%%%%%%%%%%%%%%%%%%%%%%%%%%%%%%%%%%%%%%%%%%%%%%%

Dornyei (2005: 143) defines demotivation as “specific external forces that reduce or diminish the motivational basis of a behavioral intention or an ongoing action”. Deci and Ryan (1985) uses a similar term “amotivation”, which means “ the relative absence of motivation that is not caused by a lack of initial interest but rather by the individual’s experiencing feelings of incompetence and helplessness when faced with the activity.” Though both of these terms concern lack of motivation, they differ in that amotivation is related to general outcomes expectations that are unrealistic for some reason whereas de-motivation concern specific external causes. A de-motivated learner is someone who was once motivated but has lost his or her commitment /interest for some reason. De-motives are the negative counterparts of motives. Some de-motives can lead to general amotivation regarding the particular activity whereas others may have no effect on amotivation as long as the negative external motives cease to exist. Dornyei points out that de-motivation does not mean that all the positive influences that originally made up the motivational basis of a behavior have been got rid of. It only means that a strong negative factor restrains the present motivation with some other positive motives still remain ready to be activated. For example, a Chinese student may lose his interest in learning English as soon as he passed the CET-4.

2.2 Research on Demotivation

2.2.1 Christophel and Gorham’s study

Christophel and Gorham (1995, 1992) initiated two different investigations of demotivation with both qualitative and quantitative techniques.
The results indicate that most subjects attribute demotivation to what the teacher had done or had been responsible for.

Gorham and Christophel (1992) also summarized a rank of order of the frequency of the various demotives, with first five categories as dissatisfaction with grading and assignments; the teacher being boring, bored, unorganized and unprepared; the dislike of the subject area; the inferior organization of the teaching material and the teacher being unapproachable, self-centered, biased, condescending and insulting.
This rank offers a initiative insight into the true nature of teacher’s role in demotivation.

% ==== %

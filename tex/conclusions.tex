\chapter{Conclusions and Future Work}
\label{chapter:conclusions} 

This chapter presents the conclusions of this PhD thesis dissertation along with its contributions as well as the directions of future work.
These conclusions and future work are indicated according to the three main reasearch topics addressed in this dissertation:
(1) CSCL and scripted collaborative learning,
(2) Gamification, and
(3) Ontologies and ontology engineering.

The publications related to the work presented in this dissertation (four journal papers, three full papers in international conferences and workshops, two short papers in international conferences and workshops, and one full paper in a national conference), and the award received in the 26\textsuperscript{th} Brazilian Symposium on Computer in Education (2015) indicate the relevance of the research topic and the research problem addressed in this thesis.

\section{Conclusions and Contributions}
\label{sec:conclusions-contributions}

Throughout this thesis dissertation, the author has tackled several new challenges around the three stages defined on the ontological engineering approach to gamify CL scenario.

The literature review presented in \autoref{chapter:general-background} has led us to answer the research question \textbf{RQ1}: 
\aspas{\emph{Which concepts from theories and practices of gamification should be contemplated to deal with motivational problems in a scripted collaborative learning?}, and
\emph{How should these concepts be applied in the gamification of CL scenarios?},} by concluding that:

\begin{itemize}
 \item The need-based theories of motivation (e.g. ERG theory, SDT theory) and player type models (e.g. Yee's model, Dodecad player type model) are essential to solve the context-dependency of gamification related to the individual personality traits, preferences, and affective state.
According to this review, need-based theories of motivation allow us to identify the reason why the motivational problems occurs in CL activities in which CSCL scripts are used to orchestrate and structure the CL process.
Player type models define segmentation of participants in groups that share the same preferences and liking for game elements. 
Thus, the need-based theories of motivation associated with player type models give us enough support to known how to personalize the gamification in CL scenarios.

\item To solve the context-dependency of gamification related to the target behavior that is being gamified Persuasive design models (e.g. PSD - Persuasive System Design model) and Persuasive game design models (e.g. Model-driven persuasive game) have been identify as the source valuable of information to setting up the interactions between the introduced game elements and the participants of CL scenario.
These models give us information to link the design of CL process and the game interactions, as well as, they provide support to known how to persuade the participants to behave or act in a certain way. Thus, these models will be used to convice the participants in a scripted collaborative learning to follow the interactions defined by the sequencing mechanism of a CSCL script.

\item Flow theory has been identify in the literature review as the most relevant theory to deal with the persons' affective state in game-like systems, and by the same reason, in the gamification of CL scenario.
Its nine principles/conditions can be applied to intend to maintain the participants in the flow state.
Currently, several game designers and developed indicate that the most important principle/condition of the flow theory is the good balance between the perceived challenge level and the perceived ability to maintain the students in the flow state.
Thus, to provide such support in the gamification of CL scenarios, the thesis author has developed a unify model that
 
\end{itemize}

%%%% GCCC %%%%

The research topic explored in this PhD thesis dissertation is addressed to answer the question:
\aspas{\emph{How to deal with motivational problems in scripted collaborative learning?}}


\aspas{\emph{How can gamification and ontologies be used to deal with motivational problems in scripted collaborative learning?}}

\begin{enumerate}
\item
The first stage is: the formalization of the necessary knowledge about how to gamify CL scenarios for dealing with motivational problems in scripted collaborative learning into an ontology named \textbf{OntoGaCLeS} – \emph{\textbf{Onto}logy to \textbf{Ga}mify \textbf{C}ollaborative \textbf{Le}arning \textbf{S}cenarios}.
This ontology has been developed using ontology engineering in which, by extracting concepts from the theories and practices of gamification, the thesis author defines a set of ontological structures to enable the systematic formalization and representation of knowledge to gamify CL scenarios and its theoretical foundation.

\item
The second stage is: the development of computational mechanisms and procedures whereby intelligent tools will provide support in the gamification of CL scenarios to deal with motivational problems in a scripted collaborative learning.
Such support is given by the knowledge formalized in the ontology OntoGaCLeS during the first stage, so that the purpose for the computational mechanisms and procedures in intelligent tools is to use the ontological structures from this ontology to facilitate the tasks of instructional designer and practitioners, especially novice users, in the gamification of CL scenarios.
These ontological structures contain the theoretical justification for the personalization of gamification, and they are used to obtain tailored gamified CL sessions adapted for each situation.
These sessions are known as ontology-based CL sessions, and they are CL scenarios that have been gamified and instantiated at the most concrete level of CL scenarios in which the participants and the content-domain to be learned are well defined and it can be directly run in a learning environment.

\item
The third stage is: the execution of empirical studies to understand the effects of ontology-based gamification on CL scenarios, and then, to validate the ontological engineering approach to gamify CL scenarios as a method to deal with motivational problems in scripted collaborative learning.
This validation has carried out in ontology-based gamified CL sessions obtained by the approach, and it consists in measuring the effectiveness and efficiency of these sessions for dealing with motivational problems.
\end{enumerate}

Regarding to the formalization of knowledge about how to gamify CL scenarios for dealing with motivational problems in a scripted collaborative learning (Stage 1), the research questions answered by this dissertation are:

\begin{description}
\item[RQ1:]the
\emph{Which concepts from theories and practices of gamification should be contemplated to deal with motivational problems in a scripted collaborative learning?}, and
\emph{How should these concepts be applied in the gamification of CL scenarios?}

\item[RQ2:]
\emph{What ontological structures are necessary to represent the concepts identified as relevant in the theories and practices of gamification to deal with motivational problems in scripted collaborative learning?}

\end{description}

Regarding the development of computational mechanisms and procedures whereby intelligent tools will provide support in the gamification of scenarios using the knowledge described in the ontology OntoGaCLeS (Stage 2), the research questions answered by this dissertation are:

\begin{description}
\item[RQ3:]
\emph{What computational mechanisms and procedures are necessary in intelligent tools to give a helpful support in the gamification of CL scenarios?} and
\emph{How can the knowledge encoded in the ontology OntoGaCLeS be used by these mechanisms and procedures to deal with motivational problems in a scripted collaborative learning?}
\end{description}

Regarding to the validation of the ontological engineering approach to gamify CL scenarios as a method to deal with motivational problems in scripted collaborative learning (Stage 3), the research questions answered by this dissertation are:

\begin{description}
\item[RQ4:]
\emph{What are the effectiveness and efficiency of the ontological enginnering approach to gamify CL scenarios to deal with motivational problems in scripted collaborative learning?}
\end{description}

The research objectives pursued to answer the research questions \emph{RQ1} and \emph{RQ2} are:

\begin{description}
\item[RO1:]
To review the scientific literature in order to identify the most relevant concepts from the theories and practices of gamification that should be taking into account to deal with motivational problems in scripted collaborative learning; and

\item[RO2:]
To define the ontological structures to represent the concepts identified as relevant in the theories and practices of gamification to deal with motivational problems in scripted collaborative learning.
\end{description}


In order to answer the research question \emph{RQ3}, the research objectives is:

\begin{description}
\item[RO3:]
To identify and define the computational mechanisms and procedures that must be implemented by intelligent tools to give a helpful support in the gamification of CL scenarios, and how these mechanisms and procedure use the knowledge encoded in the ontology OntoGaCLeS for dealing with the motivational problems in a scripted collaborative learning.
\end{description}

The research objective pursued to answer the research question \emph{RQ4} is:

\begin{description}
\item[RO4:]
to analyze the effects of ontology-based gamified CL sessions on the students’ motivation and learning outcomes for validating the ontology engineering approach to gamify CL scenarios in reference to the effectiveness and efficiency to deal with the motivational problems in a scripted collaborative learning.
\end{description}


%%%% GCCC %%%

In order to answer the research question \textbf{RQ2}: \aspas{\emph{How can the concepts extracted from the theories and best practices related to gamification, and identified as relevant to deal with the motivation problem caused by the scripted collaboration, be represented as ontological structures?},}

\begin{itemize}
\item
\autoref{chapter:ontogacles-1} has formalized ontological structures to represent concepts extracted from need-based theories and player types models, and thus, to solve the context-dependency of gamification related to the individual personality traits, preferences, and affective state. These ontological structures, encoded into the ontology OntoGaCLeS, were: \emph{individual motivational strategy} (\emph{Y<=I-mot goal}) to represent guidelines extracted from theories and best practices of gamication to motivate a participant to interact with others by using a learning strategy; \emph{individual motivational goal} (\emph{I-mot goal}) to represent what is expected to happen in the participants' motivational stage when an individual motivational strategy is applied in the CL scenario; \emph{player role} to represent the segmentation of participants in groups according to a player type model; and \emph{individual gameplay strategy} (\emph{I-gameplay}) to describe the game elements that are necessary to implement a individual motivational strategy.

\item
\autoref{chapter:ontogacles-2} has formalized the ontological structures to solve the context-dependency of gamification related to persuade the participants to follow the interactions defined by the sequencing mechanism of a CSCL scripts. These ontological structures, encoded into the ontology OntoGaCLeS, were: \emph{gameplay event} to explicitly represent the link between persuasive game design and the CL process; \emph{WAY-knowledge of PGDS} to support the prescriptive representation of persuasive game design and the CL process; \emph{Gameplay Scenario Model} to describe the design rationale of how to apply persuasive game design in the interactions defined by the sequencing mechanism of a CSCL script; \emph{Gamified I\_L event} to represent an interaction in which the persuasive game design has been applied to persuade the participants of a CL scenario to perform an instructional action and a learning action; \emph{CL Game Dynamic} to describe the run-time behaviors of game elements acting to persuade the persuade the participants to follow the interactions defined by the sequencing mechanism of a CSCL script; and \emph{CL Gameplay} to describe the whole CL process in a gamified CL scenario as CL Game Dynamics.

\item
\autoref{chapter:unify-modeling-learner-growth-flow-theory} has proposed a computer-based model to apply the condition of good balance between the perceived ability and challenges in the CL Game play of a gamified CL scenario. Thus, this model known as GMIF model has been proposed to integrate the learner's growth process and the condition of good balance by labeling the LGM model with intervals that indicate the proper challenge levels to maintain the learner's flow state in gamified CL scenarios. An algorithm for labeling the LGM model with a n-scale of challenge levels has been proposed in the chapter, and to demonstrate its usefulness an application to set the proper level of game rewards in gamified CL scenarios has been presented in the chapter.
\end{itemize}


\autoref{chapter:computer-based-mechanisms-procedures} answered the research question \textbf{RQ3}: \aspas{\emph{What computer-based mechanisms and procedure are necessary in intelligent-theory aware systems to give a helpful support in the gamification of CL scenarios? and How can the knowledge encoded in the ontology OntoGaCLeS be used by these mechanisms and procedures for dealing with the motivation problem caused by the scripted collaboration?},} by proposing a conceptual flow to gamify CL sessions as a computer-based procedure to be used by intelligent-theory aware system for extracting the knowledge encoded in the ontology OntoGaCLeS, and thus, to use the proposed ontological structures to provide helpful suggestion that will lead us to obtain ontology-based CL sessions (most concrete level of gamified CL scenarios in which the content-domain and participants are well defined). A reference architecture based on the conceptual flow to gamify CL sessions has also been proposed to build different intelligent theory-aware systems.


Finally, \autoref{chapter:evaluation} answered the research question 
\textbf{RQ4}: \aspas{\emph{What are the effects of ontology-based gamified CL sessions on the participants’ motivation and learning outcomes? and What are the effectiveness and efficiency of these sessions to deal with the motivation problem caused by the scripted collaboration?},} by conducting four empirical studies in which the effectiveness has been demonstrated by significant differences on the participants' motivation and learning outcomes between ontology-based CL sessions and non-gamified CL sessions, and the efficiency has been demonstrated by significant differences on the participants' motivation and learning outcomes between ontology-based CL sessions and CL sessions that were gamified without using the support given by the ontology OntoGaCLeS. The results indicated that the ontological engineering approach to gamify CL scenarios is an effective and efficient method to deal with the motivation problem caused by the scripted collaboration because the ontology-based CL sessions significantly have better participants' perceived choice and intrinsic motivation than non-gamified CL sessions, and because the ontology-based CL sessions significantly have better participants' perceived choice, intrinsic motivation and effort/importance than CL sessions that were gamified without the support given by the ontology OntoGaCLeS. These empirical studies demonstrated that ontology-based CL sessions affect in a proper way the participants' motivation to obtain positive learning outcomes the participants' gains in skill/knowledge as measurement of learning outcomes were not significantly different in non-gamified CL sessions and ont-gamified CL sessions, and the participants' gains in skill/knowledge obtained in ont-gamified CL sessions are better than in CL sessions that were gamified without using the support given by the ontology OntoGaCLeS. The pilot empirical study also demonstrated, as a benefit of the effectiveness to deal with the motivation problem caused by the scripted collaboration, that ontology-based gamified CL sessions had better percentages of participation per groups than non-gamified CL sessions.

\section{Future Research Directions}
\label{sec:future-research-directions}

Besides the research objectives of this PhD thesis dissertation have been achieved, this work has identified a number of open research problems that are summarized next.

Regarding the concepts identified as relevant in theories and practices related to gamification that should be taking into account to deal with the motivation problem caused by the scripted collaboration (RO1) in CL activities in which CSCL scripts are used as method to orchestrate and structure the CL process, future work includes:

\begin{itemize}
\item The identification of relevant concepts in the theory of fun \cite{Koster2004, Lazzaro2009} that can be apply to deal with the motivation problem caused by the scripted collaboration. The research objective in this direction can be to identify the concepts in this theory that influenced changes in the affective state of participants to engender an immersion in the CL process, and thus, to be more likely to achieve the flow state.

\item The identification of relevant concepts in game aesthetics and game art design \cite{NamKimKimLee2016, Dickey2012} that can be apply to deal with the motivation problem caused by the scripted collaboration. Representation used in game, such as concept art, item sprites, icons, and character models, and their functional aspect affect the participants' motivational state, so that their design can be important to persuade the participants to follow the interactions defined by the sequencing mechanism of a CSCL script. The research goal, in this direction, can be the use of these concepts to gamify CL scenarios.

\item The identification of relevant concepts from the theories and best practices related to gamification with the potential to affect the participants' acceptance of CL roles assigned by CSCL scripts. Not only, the lack of choice causes the motivation problem, the acceptance of a CL role may also cause the motivation problem when the participant does not like the functions, goals, duties, and responsibilities that he/she has playing a CL role. Thus, the research objective can be the gamification of CL scenarios to convince the participants to accept the CL roles assigned by the CSCL scripts. 
\end{itemize}

The proposed ontological structures to represent the most relevant concepts from the theories and practices related to the gamification of CL scenarios for dealing with the motivation problem caused by the scripted collaboration (RO2) can be extended according to the following interesting research directions:

\begin{itemize}
\item The formalization of emotional responses evoked by the participants in gamified CL scenarios when he/she interacts with the game elements. So, this formalization will be directly related to extend the ontological structures to represent gamified I\_L events, CL game dynamics and CL gameplay, including the emotional responses as changes of affective states in these ontological structures.
\item The formalization of ontological structures to gamify individual learning activities and/or interaction in individual learning scenarios. Many of the current formalized concepts to be applied in the gamification of CL scenarios can be reused, such as individual motivational goal (\emph{I-mot goal}), individual motivational strategy (\emph{Y<=I-mot goal}), such as individual gameplay strategy (\emph{I-gameplay strategy}).
\end{itemize}

With regard to the development of computer-based mechanisms and procedures that should be used by intelligent theory-award systems to give helpful support in the gamification of CL sessions for dealing with the motivation problem caused by the scripted collaboration (RO3), it is possible to point out the following open problems:

\begin{itemize}
\item The definition of computer-based mechanisms for an automatic interaction analysis in the data gathered by the execution of ontology-based CL sessions. These mechanisms should provide support to recognize under which conditions a ontology-based gamified CL session failed or not. In this direction, a future research work would be the definition of criteria and indicators to automatically identify what game elements and game actions should be changed to deal in better way with the motivation problem caused by the scripted collaboration. These support will help to improve the ontology-based model to personalize the gamification of CL sessions and the ontology-based model to apply gamification as persuasive technology. Here, data mining techniques can be used to analyze the data gathered by the execution of ontology-based gamified CL sessions.

\item The development of a complete intelligent theory-aware authoring environment to make CL sessions more motivating and engaging based on well-grounded theoretical knowledge related to theories and practices related to gamification. At the moment, only a little part of the needed functionalities that should be provided by the support systems defined in the architecture of reference of intelligent theory-aware systems to gamify CL systems have been developed. For example, the algorithm proposed (in \autoref{chapter:computer-based-mechanisms-procedures}) to set player roles and game elements for the participants in CL sessions is being incorporated in the \emph{visual grouping tool}\footnote{Management group formation tool defined as a block module for the Moodle platform - URL: \url{https://github.com/geiser/vgrouping}}. It is done to make this tool into a Player Roles \& Game Elements Decision Support System for the Moodle platform system when this tool will be integrated with \emph{Gamification Plugins for Moodle}. However, to build a complete Intelligent theory-aware authoring environment for the Moodle platform as was proposed in \autoref{chapter:computer-based-mechanisms-procedures} by the reference architecture, it is necessary to build a CL Gameplay Design Support System, and an Interaction Analysis Support System for the Moodle platform.
\end{itemize}

The effectiveness and efficiency of \aspas{\emph{the ontological engineering approach to gamify CL scenarios}} as a method to deal with motivational problems in scripted collaborative learning (RO4) have been evaluated only in the educational context of the course \aspas{Introduction to Computer Science} at the University of São Paulo with a homogeneous population of undergraduate students in the age range from 17 to 25 years old
ndergraduated students 

, additional empirical studies are needed.
This further evaluation should be accomplished in other educational contexts, employing content-domains related to other topics (than cond. structures, loop structures, and recursion) and courses (than Introduction to Computer Science), and with other groups of participants (than undergraduate Brazilian students in the age range from 17 to 25 years old).
This evaluation should also be conducted using other CSCL scripts (than the CSCL scripts inspired by the Cognitive Apprentice theory) to instantiate the CL sessions, and to orchestrate and structure the CL process of the group activities.
Besides to change the educational contexts and the CSCL scripts used in the evaluation of the ontological approach proposed in this dissertation, the further evaluation should be carried out using other ontology-based models to personalize the gamification and to apply gamification as gamification as persuasive technology.

\section{List of Publications and Awards}

\subsubsection*{Journal Papers}

\begin{enumerate}
\item
\aspas{\emph{Personalization of Gamification in Collaborative Learning Contexts using Ontologies}} published as Volume 13, Issue 6, in the journal of IEEE Latin America Transactions, 2015 \cite{ChallcoMoreiraBittencourtMizoguchiIsotani2015}.

\item
\aspas{\emph{Computer-based Systems for Automating Instructional Design of Collaborative Learning Scenarios: A Systematic Literature Review}} published as Volume 11, Issue 4, in the International Journal of Knowledge and Learning - IJKL, 2016
\cite{ChallcoBittencourtIsotani2016}.


\item
\aspas{\emph{An Ontology Framework to Apply Gamification in CSCL Scenarios as Persuasive Technology}} published as Volume 24, Issue 2, in the Brazilian Journal of Computers in Education - RBIE, 2016 \cite{ChallcoMizoguchiIsotani2016}.

\item
\aspas{\emph{Toward A Unified Modeling of Learner's Growth Process and Flow Theory}} published in the International Journal of Educational Technology \& Society, Vol. 19, No. 2, April 2016 \cite{ChallcoAndradeBorgesBittencourtIsotani2016}.
\end{enumerate}

\subsubsection*{Full Papers in International Conferences and Workshops}


\begin{enumerate}
\item
\aspas{\emph{An Ontology Engineering Approach to Gamify Collaborative Learning Scenarios}} published in the 20\textsuperscript{th} International Conference on Collaboration and Technology, CRIWG 2014, held in Santiago, Chile \cite{ChallcoMoreiraMizoguchiIsotani2014}.

\item
\aspas{\emph{Gamification of Collaborative Learning Scenarios: Structuring Persuasive Strategies Using Game Elements and Ontologies}} published in the 1\textsuperscript{st} International Workshop on Social Computing in Digital Education, SocialEdu 2015, held in Stanford, CA, USA \cite{ChallcoMizoguchiBittencourtIsotani2015}.

\item
\aspas{\emph{Using Ontology and Gamification to Improve Students' Participation and Motivation in CSCL}} that will be published as chapter of book \aspas{\emph{First International Workshop on Social, Semantic, Adaptive and Gamification Techniques and Technologies for Distance Learning},} HEFA 2017 \cite{ChallcoMizoguchiIsotani2017}.
\end{enumerate}


\subsubsection*{Short Papers in International Conferences and Workshops}

\begin{enumerate}
\item
\aspas{\emph{Towards an Ontology for Gamifying Collaborative Learning Scenarios}} published in the 12\textsuperscript{th} International Conference on Intelligent Tutoring Systems, ITS 2014, held in Honolulu, HI, USA \cite{ChallcoMoreiraMizoguchiIsotani2014a}.

\item
\aspas{\emph{Steps Towards the Gamification of Collaborative Learning Scenarios Supported by Ontologies}} published in the 17\textsuperscript{th} International Conference on Artificial Intelligence in Education, AIED 2015, held in Madrid, Spain \cite{ChallcoMizoguchiBittencourtIsotani2015a}.
\end{enumerate}


\subsubsection*{Full Papers in National Conferences and Workshops}

\begin{enumerate}
\item
\aspas{\emph{An Ontological Model to Apply Gamification as Persuasive Technology in Collaborative Learning Scenarios}} published in the 26\textsuperscript{th} Brazilian Symposium on Computer in Education, SBIE 2015, held in Maceio, AL, Brazil \cite{ChallcoAndradeOliveiraMizoguchiIsotani2015}.
\end{enumerate}



\subsubsection*{Awards}

\begin{enumerate}
\item
Honored mention in the 26\textsuperscript{th} Brazilian Symposium on Computer in Education, 2015.
\end{enumerate}


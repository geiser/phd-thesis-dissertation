\chapter{Conclusions and Future Work}
\label{chapter:conclusions} 

This chapter presents the conclusions of this PhD thesis dissertation along with its contributions as well as the directions of future work.
These conclusions and future work are indicated according to the three main research topics addressed in this dissertation:
(1) CSCL and scripted collaborative learning,
(2) Gamification, and
(3) Ontologies and ontology engineering.

The publications originated by the research works: four journal papers, three full papers in international conferences and workshops, two short papers in international conferences and workshops, and one full paper in a national conference; as well as, the award received in the 26\textsuperscript{th} Brazilian Symposium on Computer in Education (2015), all them indicate the relevance of the research topic and the research problem addressed in this PhD thesis dissertation.

\section{Conclusions and Contributions}
\label{sec:conclusions-contributions}

Throughout this thesis dissertation, the author has tackled several new challenges around the three stages defined on the ontological engineering approach to gamify CL scenario.

The literature review presented in \autoref{chapter:general-background} has led us to answer the research question \textbf{RQ1}: 
\aspas{\emph{Which concepts from theories and practices of gamification should be contemplated to deal with motivational problems in a scripted collaborative learning?}, and
\emph{How should these concepts be applied in the gamification of CL scenarios?},} by concluding that:

\begin{itemize}
\item The need-based theories of motivation (e.g. ERG theory, SDT theory) and player type models (e.g. Yee's model, Dodecad players type model) are essential to solve the context-dependency of gamification related to the individual personality traits, preferences, and affective state.
According to this review, need-based theories of motivation allow us to identify the reason why the motivational problems occurred in CL activities in which CSCL scripts are used to orchestrate and structure the CL process.
Player type models define segmentation of participants in groups that share the same preferences and liking for game elements. 
Thus, the need-based theories of motivation associated with player type models give us enough support to know how to personalize the gamification in CL scenarios.

\item To solve the context-dependency of gamification related to the target behavior that is being gamified Persuasive design models (e.g. PSD - Persuasive System Design model) and Persuasive game design models (e.g. Model-driven persuasive game) have been identified as the source valuable of information to setting up the interactions between the introduced game elements and the participants of CL scenario.
These models give us information to link the design of CL process and the game interactions, as well as, they provide support to know how to persuade the participants to behave or act in a certain way.
Thus, these models will be used to convince the participants in a scripted collaborative learning to follow the interactions defined by the sequencing mechanism of a CSCL script.

\item Flow theory has been identified in the literature review as the most relevant theory to deal with the persons' affective state in game-like systems, and by the same reason, in the gamification of CL scenario.
Its nine principles/conditions can be applied to intend to maintain the participants in the flow state.
Currently, several game researchers, designers and developers considered that the most important principle/condition of the flow theory is the good balance between the perceived challenge level and the perceived ability to maintain the students in the flow state.
\end{itemize}


To answer the research question \textbf{RQ2}:
\aspas{\emph{What ontological structures are necessary to represent the concepts identified as relevant in the theories and practices of gamification to deal with motivational problems in scripted collaborative learning?},}

\begin{itemize}
\item
\autoref{chapter:ontogacles-1} has formalized ontological structures to represent concepts extracted from the need-based theories of motivation and the player types models, and thus, to solve the context-dependency of gamification that refers to the individual personality traits, preferences, and affective state.
These ontological structures, encoded into the ontology OntoGaCLeS, were:
the \emph{individual motivational strategy} (\emph{Y<=I-mot goal}) to represent guidelines extracted from theories and practices of gamification to motivate a participant to interact with other group member (You) by using a learning strategy;
the \emph{individual motivational goal} (\emph{I-mot goal}) to represent what is contemplated to happen in the participants' motivational stage when an individual motivational strategy is applied in the CL scenario;
the \emph{player role} to represent the segmentation of participants in different types according to a player type typology described by a player type model; and
the \emph{individual gameplay strategy} (\emph{I-gameplay}) to describe the game elements that are necessary to implement an individual motivational strategy.

\item
\autoref{chapter:ontogacles-2} has formalized the ontological structures to solve the context-dependency of gamification that refers to persuade the participants to follow the interactions defined by the sequencing mechanism of a CSCL script in a CL activity.
These ontological structures, encoded into the ontology OntoGaCLeS, were:
the \emph{gameplay event} to explicitly represent the link between persuasive game design and the CL process;
the \emph{WAY-knowledge of PGDS} for supporting the prescriptive representation of persuasive game designs and the CL process;
the \emph{Gameplay Scenario Model} to delineate the design rationale of how to apply persuasive game design in the interactions defined by the sequencing mechanism of a CSCL script;
the \emph{Gamified I\_L event} to represent an interaction in which the persuasive game design has been applied to persuade the participants of a CL scenario to perform an instructional action and a learning action;
the \emph{CL Game Dynamic} to delineate the run-time behaviors of game elements acting to persuade the persuade the participants to follow the interactions defined by the sequencing mechanism of a CSCL script; and
the \emph{CL Gameplay} to describe the whole CL process in a gamified CL scenario as a set of CL Game Dynamics.
\end{itemize}


To answer the research question \textbf{RQ3}: \aspas{\emph{What computational mechanisms and procedures are necessary in intelligent tools to give a helpful support in the gamification of CL scenarios?} and
\emph{How can the knowledge encoded in the ontology OntoGaCLeS be used by these mechanisms and procedures to deal with motivational problems in a scripted collaborative learning?},} 

\begin{itemize}
\item
\autoref{chapter:unify-modeling-learner-growth-flow-theory} has proposed a computational model to apply, from the flow theory, the condition/principle of good balance between the perceived ability and challenges in the CL process.
This model known as GMIF model indicates the proper challenge levels to maintain the learner's flow state in gamified CL scenarios.
An algorithm for labeling the Learner Growth model with a n-scale of challenge levels has been proposed in the chapter, and to demonstrate its usefulness an application to set the proper level of game rewards in gamified CL scenarios has developed and detailed in the chapter.

\item
\autoref{chapter:computer-based-mechanisms-procedures} answered the research question by proposing a conceptual flow to gamify CL sessions as a computational procedure to be used in intelligent tools for extracting the knowledge encoded in the ontology OntoGaCLeS, and thus, to use the proposed ontological structures to provide helpful suggestion that will lead us to obtain ontology-based gamified CL sessions (most concrete level of gamified CL scenarios in which the content-domain and participants are well defined).
A reference architecture based on the conceptual flow to gamify CL sessions has also been proposed to build different intelligent theory-aware tools.
\end{itemize}


Finally, \autoref{chapter:evaluation} answered the research question \textbf{RQ4}:
\aspas{\emph{What are the effectiveness and efficiency of the ontological enginnering approach to gamify CL scenarios to deal with motivational problems in scripted collaborative learning?},}
by conducting four empirical studies in which the effectiveness has been demonstrated by significant differences on the participants' motivation and learning outcomes between ontology-based gamified CL sessions (\emph{ont-gamified} CL sessions) and non-gamified CL sessions.
The efficiency has been demonstrated indicating the significant differences on the participants' motivation and learning outcomes between ont-gamified CL sessions and CL sessions that were gamified using a conventional form to gamify CL sessions - in which all the possible game elements provided by the gamification platform Moodle were used, the gamification was applied by using  one-size-fits-all approach without the personalization of gamification, and the gamification was carried out by the instructional designer \emph{without} using any information given by the ontology OntoGaCLeS (\emph{w/o-gamified} CL sessions).

The result for the evaluation of the ontological engineering approach to gamify CL scenarios indicate that this approach is an effective and efficient method to deal with motivational problems in scripted collaborative learning because
the ont-gamified CL sessions significantly have better participants' perceived choice and intrinsic motivation than non-gamified CL sessions;
and because the ont-gamified CL sessions significantly have better participants' perceived choice, intrinsic motivation and effort/importance than w/o-gamified CL sessions.
These empirical studies demonstrated that ont-gamified CL sessions affect in a proper way the participants' motivation to obtain positive learning outcomes the participants' gains in skill/knowledge as measurements of learning outcomes were not significantly different in non-gamified CL sessions and ont-gamified CL sessions, and the participants' gains in skill/knowledge obtained in ont-gamified CL sessions are better than in w/o-gamified CL sessions.
The pilot empirical study also demonstrated, as a benefit of the effectiveness to deal with the motivation problem caused by the scripted collaboration, that ont-gamified CL sessions should percentages of participation per groups than non-gamified CL sessions.

\section{Future Research Directions}
\label{sec:future-research-directions}

During the development of this PhD research work, the thesis author has identified open research problems that are summarized next.

Regarding the concepts identified as relevant in theories and practices of gamification that should be contemplated to deal with motivational problems in scripted collaborative learning (RO1), future work includes:

\begin{itemize}
\item The identification of relevant concepts in the theory of fun \cite{Koster2004, Lazzaro2009} that can be apply to deal with motivational problems not yet contemplated in scripted collaborative learning.%%% increase here
The research objective in this direction can be to identify the concepts in this theory that influenced changes in the affective state of participants to engender an immersion in the CL process, and thus, to be more likely to achieve the flow state.


\item The identification of relevant concepts in game aesthetics and game art design \cite{NamKimKimLee2016, Dickey2012} that can be applied to deal with motivational problems in scripted collaborative learning.
Representation used in game, such as concept art, item sprites, icons, and character models, and their functional aspect affected the participants' motivational state, so that their design can be important to persuade the participants to follow the interactions defined by the sequencing mechanism of a CSCL script in a CL process.
The research goal, in this direction, can be the use of these concepts to gamify CL scenarios.

\item The identification of relevant concepts from the theories and practices of gamification with the potential to affect the participants' acceptance of CL roles assigned by CSCL scripts.
Not only, the lack of choice can cause motivational problems, the acceptance of a CL role may also the source for motivational problems when the participant does not like the functions, goals, duties, and responsibilities that he/she has playing a CL role.
Thus, the research objective can be the gamification of CL scenarios with the purpose to convince the participants to accept the CL roles assigned by the CSCL scripts.
\end{itemize}

The proposed ontological structures to represent the most relevant concepts from the theories and practices of gamification for dealing with motivational problems in scripted collaborative learning (RO2) can be extended according to the following interesting research directions:

\begin{itemize}
\item The formalization of emotional responses evoked by the participants in gamified CL scenarios when he/she interacts with the game elements.
So, this formalization will be directly related to extend the ontological structures to represent gamified I\_L events, CL game dynamics and CL gameplay, including the emotional responses as changes of affective state in these ontological structures.
\item The formalization of ontological structures to gamify individual learning activities and/or interaction in individual learning scenarios.
Many currently formalized concepts to be applied in the gamification of CL scenarios can be reused, such as individual motivational goal (\emph{I-mot goal}), individual motivational strategy (\emph{Y<=I-mot goal}), such as individual gameplay strategy (\emph{I-gameplay strategy}).
\end{itemize}

Regarding the development of computational mechanisms and procedures that should be used by intelligent tools to give helpful support in the gamification of CL sessions for dealing with motivational problems in scripted collaborative learning (RO3), it is possible to point out the following open problems:

\begin{itemize}
\item The definition of computational mechanisms for an automatic interaction analysis in the data gathered by the execution of ont-gamified CL sessions.
These mechanisms should provide support to recognize under which conditions a ont-gamified CL session failed or not.
In this direction, a future research work would be the definition of criteria and indicators to automatically identify what game elements and game actions should be changed to deal in better way with motivational problems in scripted collaborative learning.
This support will help to improve the ontology-based model to personalize the gamification of CL sessions, and also, the  building of ontology-based model to apply gamification as persuasive technology.
Here, data mining techniques, such as the data-mining procedure proposed by \citeonline{KnutasvanRoyHynninenGranatoKasurinenIkonen2018}, can be used to analyze the data gathered by the execution of ont-gamified CL sessions, and thus, to elaborate new ontology-based model to personalize the gamification, as well as, ontological models to apply gamification as persuasive technology.


\item The development of a complete intelligent theory-aware authoring environment to make CL sessions more motivating and engaging based on well-grounded theoretical knowledge from theories and practices of gamification.
At the moment, only a little part of the needed functionalities that should be provided by the support tools defined in the architecture of reference of intelligent theory-aware tools to gamify CL systems have been developed.
For example, the algorithm proposed (in \autoref{chapter:computer-based-mechanisms-procedures}) to set player roles and game elements for the participants in CL sessions is being incorporated in the \emph{visual grouping tool}\footnote{Management group formation tool defined as a block module for the Moodle platform - URL: \url{https://github.com/geiser/vgrouping}}.
It is done to make this tool into a Player Roles \& Game Elements Decision Support System for the Moodle platform system when this tool will be integrated with \emph{Gamification Plug-ins for Moodle}.
However, to build a complete Intelligent theory-aware authoring environment for the Moodle platform as was proposed in \autoref{chapter:computer-based-mechanisms-procedures} by the reference architecture, it is necessary to build a CL Gameplay Design Support System, and an Interaction Analysis Support System for the Moodle platform.
\end{itemize}

The effectiveness and efficiency of \aspas{\emph{the ontological engineering approach to gamify CL scenarios}} as a method to deal with motivational problems in scripted collaborative learning (RO4) have been evaluated only in the educational context of the course \aspas{Introduction to Computer Science} at the University of São Paulo with a homogeneous population of undergraduate students in the age range from 17 to 25 years old.
As the gamification is too-context dependent, additional empirical studies are needed to validate the ontological models used to personalize the gamification and to apply gamification as persuasive technology.
This further evaluation should be accomplished in other educational contexts, employing content-domains related to other topics (than cond. structures, loop structures, and recursion) and courses (than Introduction to Computer Science), and with other groups of participants (than undergraduate Brazilian students in the age range from 17 to 25 years old).
This evaluation should also be conducted using other CSCL scripts (than the CSCL scripts inspired by the Cognitive Apprentice theory) to instantiate the CL sessions, and to orchestrate and structure the CL process of the group activities.

\section{List of Publications and Awards}

\subsubsection*{Journal Papers}

\begin{enumerate}
\item
\aspas{\emph{Personalization of Gamification in Collaborative Learning Contexts using Ontologies}} published as Volume 13, Issue 6, in the journal of IEEE Latin America Transactions, 2015 \cite{ChallcoMoreiraBittencourtMizoguchiIsotani2015}.

\item
\aspas{\emph{Computer-based Systems for Automating Instructional Design of Collaborative Learning Scenarios: A Systematic Literature Review}} published as Volume 11, Issue 4, in the International Journal of Knowledge and Learning - IJKL, 2016
\cite{ChallcoBittencourtIsotani2016}.


\item
\aspas{\emph{An Ontology Framework to Apply Gamification in CSCL Scenarios as Persuasive Technology}} published as Volume 24, Issue 2, in the Brazilian Journal of Computers in Education - RBIE, 2016 \cite{ChallcoMizoguchiIsotani2016}.

\item
\aspas{\emph{Toward A Unified Modeling of Learner's Growth Process and Flow Theory}} published in the International Journal of Educational Technology \& Society, Vol. 19, No. 2, April 2016 \cite{ChallcoAndradeBorgesBittencourtIsotani2016}.
\end{enumerate}

\subsubsection*{Full Papers in International Conferences and Workshops}


\begin{enumerate}
\item
\aspas{\emph{An Ontology Engineering Approach to Gamify Collaborative Learning Scenarios}} published in the 20\textsuperscript{th} International Conference on Collaboration and Technology, CRIWG 2014, held in Santiago, Chile \cite{ChallcoMoreiraMizoguchiIsotani2014}.

\item
\aspas{\emph{Gamification of Collaborative Learning Scenarios: Structuring Persuasive Strategies Using Game Elements and Ontologies}} published in the 1\textsuperscript{st} International Workshop on Social Computing in Digital Education, SocialEdu 2015, held in Stanford, CA, USA \cite{ChallcoMizoguchiBittencourtIsotani2016}.

\item
\aspas{\emph{Using Ontology and Gamification to Improve Students' Participation and Motivation in CSCL}} that will be published as chapter of book \aspas{\emph{First International Workshop on Social, Semantic, Adaptive and Gamification Techniques and Technologies for Distance Learning},} HEFA 2017 \cite{ChallcoMizoguchiIsotani2018}.
\end{enumerate}


\subsubsection*{Short Papers in International Conferences and Workshops}

\begin{enumerate}
\item
\aspas{\emph{Towards an Ontology for Gamifying Collaborative Learning Scenarios}} published in the 12\textsuperscript{th} International Conference on Intelligent Tutoring Systems, ITS 2014, held in Honolulu, HI, USA \cite{ChallcoMoreiraMizoguchiIsotani2014a}.

\item
\aspas{\emph{Steps Towards the Gamification of Collaborative Learning Scenarios Supported by Ontologies}} published in the 17\textsuperscript{th} International Conference on Artificial Intelligence in Education, AIED 2015, held in Madrid, Spain \cite{ChallcoMizoguchiBittencourtIsotani2015a}.
\end{enumerate}


\subsubsection*{Full Papers in National Conferences and Workshops}

\begin{enumerate}
\item
\aspas{\emph{An Ontological Model to Apply Gamification as Persuasive Technology in Collaborative Learning Scenarios}} published in the 26\textsuperscript{th} Brazilian Symposium on Computer in Education, SBIE 2015, held in Maceio, AL, Brazil \cite{ChallcoAndradeOliveiraMizoguchiIsotani2015}.
\end{enumerate}



\subsubsection*{Awards}

\begin{enumerate}
\item
Honored mention in the 26\textsuperscript{th} Brazilian Symposium on Computer in Education, 2015.
\end{enumerate}


% ------------------------------------------------------------------------
% ------------------------------------------------------------------------
% ICMC: Modelo de Trabalho Acadêmico (tese de doutorado, dissertação de
% mestrado e trabalhos monográficos em geral) em conformidade com 
% ABNT NBR 14724:2011: Informação e documentação - Trabalhos acadêmicos -
% Apresentação
% ------------------------------------------------------------------------
% ------------------------------------------------------------------------

% Opções: 
%   Qualificação          = qualificacao 
%   Curso                 = doutorado/mestrado
%   Situação do trabalho  = pre-defesa/pos-defesa (exceto para qualificação)
%   Versão para impressão = impressao
\documentclass[doutorado, pre-defesa]{packages/icmc}

% ---------------------------------------------------------------------------
% Pacotes Opcionais
% ---------------------------------------------------------------------------
\usepackage{rotating}           % Usado para rotacionar o texto
\usepackage[all,knot,arc,import,poly]{xy}   % Pacote para desenhos gráficos
% Este pacote pode conflitar com outros pacotes gráficos como o ``pictex''
% Então é necessário usar apenas um dos pacotes conflitantes
\newcommand{\VerbL}{0.52\textwidth}
\newcommand{\LatL}{0.42\textwidth}

\usepackage{multirow} % multirow
\usepackage{lscape}
\usepackage{graphicx}
\usepackage{soul}
% ---------------------------------------------------------------------------
\algnewcommand{\LeftComment}[1]{\Statex \(\triangleright\) #1}

\newcolumntype{L}[1]{>{\raggedright\arraybackslash}p{#1}}
\newcolumntype{C}[1]{>{\centering\arraybackslash}p{#1}}
\newcolumntype{R}[1]{>{\raggedleft\arraybackslash}p{#1}}
% ---
% Informações de dados para CAPA e FOLHA DE ROSTO
% ---
% Tanto na capa quanto nas folhas de rosto apenas a primeira letra da primeira palavra (ou nomes próprios) devem estar em letra maiúscula, todas as demais devem ser em letra minúscula.
\tituloEN{Gamification of Collaborative Learning Scenarios: An Ontological Engineering Approach to Deal with the Motivation Problem Caused by Computer-Supported Collaborative Learning Scripts}
\tituloPT{Gamificação de Cenários de Aprendizagem Colaborativa: Uma Abordagem de Engenharia de Ontologias para Lidar com o Problema de Motivação Causado por Scripts de Aprendizagem Colaborativa Suportados por Computador}

\autor[Challco, G. C.]{Geiser Chalco Challco}
\genero{M} % Gênero do autor (M = Masculino / F = Feminino)
\orientador[Orientadora]{Prof. Dr.}{Seiji Isotani}%\coorientador{Prof. Dr.}{Fulano de Tal}
\curso{CCMC}
\data{07}{07}{2018} % Data do depósito
\idioma{EN} % Idioma principal do documento (PT = português / EN = inglês)
% ---


% ---
% RESUMOS
% ---

% resumo em INGLÊS
% conter no máximo 500 palavras
% conter no mínimo 1 e no máximo 5 palavras-chave
%\textoresumo[english]{
%    Content}{Word1, Word2, Word3, Word4, Word5}

% Resumo em PORTUGUÊS
% conter no máximo 500 palavras
% conter no mínimo 1 e no máximo 5 palavras-chave
%\textoresumo[brazil]{
%    Conteúdo}{Word1, Word2, Word3, Word4, Word5, Word6}



% ----------------------------------------------------------
% ELEMENTOS PRÉ-TEXTUAIS
% ----------------------------------------------------------

% Inserir a ficha catalográfica
%\incluifichacatalografica{tex/pre-textual/ficha-catalografica.pdf}

% DEDICATÓRIA / AGRADECIMENTO / EPÍGRAFE
%\textodedicatoria*{tex/pre-textual/dedicatoria}
%\textoagradecimentos*{tex/pre-textual/agradecimentos}
%\textoepigrafe*{tex/pre-textual/epigrafe}

% Inclui a lista de figuras
%\incluilistadefiguras

% Inclui a lista de tabelas
%\incluilistadetabelas

% Inclui a lista de quadros
%\incluilistadequadros

% Inclui a lista de algoritmos
%\incluilistadealgoritmos

% Inclui a lista de códigos
%\incluilistadecodigos

% Inclui a lista de siglas e abreviaturas
%\incluilistadesiglas

% Inclui a lista de símbolos
%\incluilistadesimbolos

% ----
% Início do documento
% ----
\begin{document}
% ----------------------------------------------------------
% ELEMENTOS TEXTUAIS
% ----------------------------------------------------------
\textual

\newcommand{\comando}[1]{\textbf{$\backslash$#1}}



This chapter starts with the context and delimitation of research problem (\autoref{sec:problem-delimitation}). After that, the chapter formulates the research questions and objectives (\autoref{sec:research-question-and-research-objectives}). The research methodology is presented in \autoref{sec:research-methodology}. The thesis statement and contributions are presented in \autoref{sec:thesis-statement-and-claimed-contributions}. The chapter ends with the structure of this dissertation (\autoref{sec:structure-of-dissertation}).

\section{Context and Problem Delimitation}
\label{sec:problem-delimitation}

Over the last two decades or so, with the growing number of technologies that enable people to communicate and work in group activities using computers and Internet, researches and practitioners have developed technology and software applications that facilitate and foster the Collaborative Learning (CL) \cite{LehtinenHakkarainenLipponenRahikainenMuukkonen1999}. Such technology and the research field that studies how to effectively link together the advanced in computer science with the collaborative learning is known as Computer-Supported Collaborative Learning (CSCL), and it has been proved an important to support the learning process of students by cognitive, social and technological reasons \cite{StahlKoschmannSuthers2006}. However, CSCL is only beneficial when there is an adequate design establishing the way in which the collaboration should happened \cite{Dillenbourg2013, Hewitt2005, IsotaniInabaIkedaMizoguchi2009}. Students frequently fail to be engaged in productive learning interactions when they are left to interact in CL activities without any support. Hence, several researchers propose the use of scripts to guide and orchestrate the collaboration among students \cite{AlharbiAthaudaChiong2014}.

Scripted collaboration aims to engage the students in fruitful and meaningful interactions according to a design that has the purpose to attain a set of pedagogical objectives. Thereby, CSCL scripts have been proposed by the community to support the well-thought-out design of the CL scenarios by means of computer-based systems \cite{FischerKollarStegmannWeckerZottmann2013,KobbeWeinbergerDillenbourgHarrerHamalainenHakkinenFischer2007}. These scripts are the technology that describes how the interactions among students will be orchestrated in a group activity to increase the possibility of achieving the pedagogical objectives \cite{WeinbergerErtlFischerMandl2005}. These scripts provide information that facilitates the group formation, the role distribution, and the sequencing of interaction for the participants of a CL activity. Despite of these benefits, there are situations in which the scripts may cause motivation problem. Sometimes, a learner does not want to play the role assigned by the scripts, and he may neglect his personal behavior to get the task completed without effort and, other times, the lack of choice over the sequence of interactions may produce in the students a sense of obligation in complete an unwilling activity \cite{ChallcoMoreiraMizoguchiIsotani2014, Isotani2009}. These issues may negatively influence the students’ motivation, learning attitudes and behaviors, degrade the classroom group dynamics, and result in long-term and widespread negative learning outcomes \cite{ChallcoMoreiraMizoguchiIsotani2014, FaloutElwoodHood2009}

The motivation problem caused by the scripted collaboration makes more difficult the use of CSCL technology over time. In fact, less motivated students prefer to spend more time in other activities rather than to learn and, as consequence, the achievement of expected learning outcomes becomes difficult \cite{Crook2000, SchoorBannert2011}. In this sense, motivating learners in the entire instructional process of CL is important. However, the traditional instructional design practice assumes that the motivation is a simple preliminary step that must happen before the instruction \cite{ChanAhern1999, Keller1987}. This assumption is based in which the good quality of learning materials can keep the students focused during the learning process, but if this process is long, there is a good chance that the students will lose their initial attention. To solve this problem, several approaches, such as the use of affective feedbacks based on emotion-aware \cite{FeidakisDaradoumisCaballeConesa2014,FeidakisCaballeDaradoumisJimenezConesa2014}, peer learning companions \cite{WoolfBurlesonArroyoDragonCooperPicard2009}, and so on, have been proposed to motivated students along the entire instructional process. These solutions assume that the students like the content-domain and/or have the desired to learn, so that students that do not have the desire to learn are not motivated and engage for these approaches.

In the last years, efforts of CSCL community have been directed to finding new innovative solutions that, beside to motivate and engage students during the entire CL process, are not completely tied to the domain-content and desired to learn the domain-content. In this direction, several researchers and practitioners have pointed Gamification as a promising technology to deal with motivation problem in the instructional/learning domain \cite{ChallcoMoreiraMizoguchiIsotani2014, SeabornFels2015, deSousaBorgesDurelliReisIsotani2014}.
Gamification defined \aspas{\emph{as the use of game design elements in non-game contexts}} \cite{DeterdingDixonKhaledNacke2011} aims to increase the students' motivation and engagement by making the learning process more game-like. This is done through the introduction of game elements, such as points, leaderboards, competition, cooperation and so on. These elements are not part of the domain-content, neither they belong to the instructional/learning process, so that they can even motivate students who do not have the desire and/or interest in to learn the content-domain. These game elements are introduced along the entire learning process, so that the benefits of gamification strongly depend on how well these game elements are applied, and how well they are linked with the pedagogical approaches \cite{Kapp2012, KnutasIkonenNikulaPorras2014}.

When CL scenarios are gamified to deal with the motivation problem caused by the scripted collaboration, the author of this thesis hypothesizes that the chances to achieve engagement and educational benefits will be increased whether there is a proper connection between the game elements and the CL process. Nevertheless, developing such well-though-out gamified CL scenario, hereinafter referred to as gamified CL scenarios, is not trivial. The main difficulty to gamify CL scenarios as well as other non-game context is that the gamification is too context dependent \cite{HamariKoivistoSarsa2014, RichardsThompsonGraham2014}. Its effects vary individual to individual, and they depend of many factors such as the individual personality traits, preferences, and current students’ emotions \cite{Nicholson2015, PedroLopesPratesVassilevaIsotani2015} (e.g., a user who likes competition would be more motivated by a leaderboard rather than a user who want to obtain items to customize his/her avatar). Also, the expected effects of the game elements vary according to context and the target behavior that is being gamified \cite{DeterdingBjorkNackeDixonLawley2013, HeeterLeeMedlerMagerko2011} (e.g., gamifying a learning scenario to promote the sign-up of participants is not the same that gamifying an interactive enviroment to maintain the students attention). As consequence of this context-dependency, when a CL scenario is not well gamified, instead to have a positive effect, they may cause a detrimental on the students’ motivation \cite{AndradeMizoguchiIsotani2016}, cheating \cite{NunesBittencourtIsotaniJaques2016}, embarrassment \cite{OhnoYamasakiTokiwa2013}, and lack of credibility on badges \cite{DavisSingh2015}.

Another difficulty to gamify CL scenarios, as well as other non-game contexts, it is the lack of approaches to systematically represent, in an unambiguous way, the gamification knowledge acquired in the last years by researchers and practitioners. This knowledge constituted by theories and best practices related to gamification lacks of a formal and common vocabulary, definitions, and representation to apply gamification. As can be appreciated in the current literature of gamification \cite{DichevaDichevAgreAngelova2015, HamariKoivistoSarsa2014, MoraRieraGonzalezArnedo-Moreno2015, SeabornFels2015}, each author proposes his/her own definitions, classifications and representation to describe the concepts and characteristics about how to gamify a non-game context. This fact hinders the creation of models and/or frameworks that formally represent the gamification and its application by computer-based systems in a common understandable and sharable manner, and to the best of the thesis author’s knowledge, there are no one approaches has been proposed to represent the knowledge about how to gamify CL scenarios to deal with the motivation problem caused by the scripted collaboration.

Due to the variety of students who can participate in CL sessions, the diversity of subjects that can be under study in a CL activity, and the range of different CSCL scripts that can be used to orchestrate the CL process, it is necessary to personalize the gamification, providing a tailored gamified CL scenario for each situation. This task is difficult and time-consuming, so that developing a computational based-support in intelligent-theory aware systems to give assistance with the personalization of gamification is very helpful and necessary. In this direction, in the context of CSCL, one interesting solution has been proposed to gamify CL scenarios using adaptive profiles and machine learning techniques \cite{KnutasIkonenMaggioriniRipamontiPorras2014,KnutasIkonenNikulaPorras2014}. However, this solution is not oriented to deal with the motivation problem caused by the scripted collaboration, its purpose is to increase the communication among the participants in CL scenarios. Furthermore, this solution falls into the category of computer-based mechanisms and procedures that support the gamification, it does not provide a model to share the theoretical knowledge related to gamification obtained by this computer-based mechanism. Solutions based on machine learning to personalize gamification require a lot of data to support the personalization of gamification, and they may have overfitting or underfitting problem with the data. A computer mechanism based only in machine learning techniques to personalize gamification lacks of theoretical-justification to explain why a game element is introduced, and why a certain configuration of game elements increases the motivation participants in a CL scenario.

For the reason exposed above, to deal with the motivation problem caused by the scripted collaboration through the gamification of CL scenarios, a computational support with a common and shareable structure to describe knowledge extracted from the best practices and theories related to gamification is essential to overcome the challenges and difficulties of gamification. In the direction to make explicit the knowledge contained in computer-based mechanisms and procedures, ontologies have been consolidated as the most advanced technology to support the representation of knowledge in a common computer-understandable and sharable manner \cite{AsikriLaassiriKritChaib2016, Devedzic2006, MizoguchiBourdeau2016}. Ontologies constitute an explicit mapping between the target world of interest and its representation with the purpose to describe concepts without ambiguities providing a common way to represent the knowledge \cite{GuarinoOberleStaab2009}. Taking advantages of this commonality, and using the computer interconnection technologies such as Internet, computer-based mechanisms in intelligent systems use ontologies to share understandings and interpretations of target world. In this direction, employing ontologies, some interesting and practical results have been obtained in the formalization and organization of knowledge extracted from different theories and practices related to gamification \cite{DermevalVilelaBittencourtCastroIsotaniBritoSilva2016, KarkarAlJa'amFoufou2016, ZouaqNkambou2010}. However, currently, there is no one ontology that allows the description of fundaments concepts extracted from the best practices and theories related to gamification, and how these concepts are applied in CL scenarios to deal with the motivation problem caused by the scripted collaboration.

Therefore, the general research goal in this PhD thesis dissertation refers to the definition of an ontology to, from a philosophical perspective, systematically formalize the knowledge extracted from the best practices and theories related to gamification, and the definition of computer-based mechanisms that employ this ontology to deal with the motivation problem caused by the scripted collaboration in CL activities where the CSCL scripts are used as a method to orchestrate and structure the collaboration among students.

\section{Research Questions and Research Objectives}
\label{sec:research-question-and-research-objectives}

The overarching research question (\textbf{RQ}) answered in this PhD thesis dissertation is: \aspas{\emph{How can gamification and ontologies be used to deal with the motivation problem caused by the scripted collaboration in CL activities where CSCL scripts are used as a method to orchestrate and structure the collaboration among students?}}

To answer this research question, the author of this thesis proposes the ontological engineering approach to gamify CL scenarios shown in \autoref{fig:ontological-engineering-approach-to-gamify-cl-scenarios}. This approach consists into three major stages described as follows:

\begin{figure}[htb]
 \caption{Ontological engineering approach to gamify CL scenarios}
 \label{fig:ontological-engineering-approach-to-gamify-cl-scenarios}
 \centering
 \includegraphics[width=0.9\textwidth]{images/chap-introduction/ontological-engineering-approach-to-gamify-cl-scenarios.png}
 \fautor
\end{figure}

\begin{enumerate}
\item
The first stage is the formalization of the necessary knowledge about how to gamify CL scenarios for dealing with the motivation problem caused by the scripted collaboration into an ontology named \textbf{OntoGaCLeS} – \emph{\textbf{Onto}logy to \textbf{Ga}mify \textbf{C}ollaborative \textbf{Le}arning \textbf{S}cenarios}. This ontology has been developed using ontology engineering in which, by extracting concepts from the theories and practices related to gamification, the author of this thesis defines a set of ontological structures that enables the systematic formalization and representation of necessary knowledge to gamify CL scenarios.

\item
The second stage is the development of computer-based mechanisms and procedures whereby intelligent theory-aware systems will provide support in the gamification of CL scenarios to deal with the motivation problem caused by the scripting collaboration. Such support is given by the knowledge formalized in the ontology OntoGaCLeS during the first stage, and the purpose of the computer-based mechanisms is to use this knowledge to facilitate the tasks of instructional designer and practitioners, especially novice users, in the gamification of CL scenarios. This knowledge provides theoretical justification for the personalization of gamification and, thus, to obtain tailored gamified CL sessions adapted for each situation. Such sessions are known as ontology-based CL sessions, and they are CL scenarios that have been gamified and instantiated at the most concrete level by detailing the participants and content-domain to be directly run in a learning environment.

\item
The third stage is the validation of the ontological engineering approach to gamify CL scenarios as a method to deal with the motivation problem caused by the scripted collaboration. This validation is carried out in ontology-based gamified CL sessions obtained by the approach, and it consists in measuring the effectiveness and efficiency of these sessions for dealing with the motivation problem caused by the scripted collaboration. The effectiveness and efficiency were measured by comparing the effects on students' motivation and learning outcomes caused by ontology-based CL sessions, non-gamified CL sessions and CL sessions gamified without using the support given by the ontology OntoGaCLeS.
\end{enumerate}

Regarding to the formalization of knowledge about how to gamify CL scenarios for dealing with the motivation problem caused by the scripted collaboration (Stage 1), the research questions answered by this dissertation are:

\begin{description}
\item[RQ1:]
Which concepts from the theories and practices related to gamification should be taking into account to deal with the motivation problem caused by the scripted collaboration? and How should these concepts be applied in the gamification of CL scenarios?

\item[RQ2:]
How can the concepts extracted from the theories and best practices related to gamification, and identified as relevant to deal with the motivation problem caused by the scripted collaboration, be represented as ontological structures?
\end{description}

Regarding to the development of computer-based mechanisms and procedures whereby intelligent theory-aware systems will provide support in the gamified CL scenarios using the knowledge described in the ontology OntoGaCLeS (Stage 2), the research questions answered by this dissertation are:

\begin{description}
\item[RQ3:]
What computer-based mechanisms and procedure are necessary in intelligent-theory aware systems to give a helpful support in the gamification of CL scenarios? and How can the knowledge encoded in the ontology OntoGaCLeS be used by these mechanisms and procedures for dealing with the motivation problem caused by the scripted collaboration?
\end{description}

Regarding to the validation of the ontological engineering approach to gamify CL scenarios as a method to deal with the motivation problem caused by the scripted collaboration (Stage 3), the research questions answered by this dissertation are:

\begin{description}
\item[RQ4:]
What are the effects of ontology-based gamified CL sessions on the students’ motivation and learning outcomes? and What are the effectiveness and efficiency of these sessions to deal with the motivation problem caused by the scripted collaboration?
\end{description}

The research objectives pursued to answer the research questions \emph{RQ1} and \emph{RQ2} are:

\begin{description}
\item[RO1:]
To review the scientific literature in order to identify the most relevant concepts from the theories and practices related to gamification that should be taking into account to deal with the motivation problem caused by the scripted collaboration, and how these concepts be applied in the gamification of CL scenarios; and

\item[RO2:]
To define the necessary ontological structures to represent the concepts identified as relevant in the scientific literature of gamification to deal with the motivation problem caused by the scripted collaboration.
\end{description}


In order to answer the research question \emph{RQ3}, the research objectives is:

\begin{description}
\item[RO3:]
To identify and define the computer-based mechanisms and procedures that must be implemented by intelligent-theory aware systems to give a helpful support in the gamification of CL scenarios, and how these mechanisms and procedure use the knowledge encoded in the ontology OntoGaCLeS for dealing with the motivation problem caused by the scripted collaboration.
\end{description}

The research objective pursued to answer the research question \emph{RQ4} is:

\begin{description}
\item[RO4:]
to analyze the effects of ontology-based gamified CL sessions on the students’ motivation and learning outcomes for the purpose of validating the ontology engineering approach to gamify CL scenarios in reference to the effectiveness and efficiency to deal with the motivation problem caused by the scripted collaboration.
\end{description}

It is out of scope in this dissertation to deal with the following objectives:

\begin{itemize}
\item
To compare, validate or judge the best practices and theories related to gamification.

\item
To create, modify or extend the concepts described in the best practices and theories related to gamification.

\item
To create a generic and complete representation of all concepts described in the practices and theories related to gamification. The author of this thesis only concentrates on the formalization of the minimal necessary concepts from these practices and theories to deal with the motivation problem caused by the scripted collaboration.

\item
To validate the concepts and ontological structures formalized in the ontology OntoGaCLeS using semantic reasoner engines or formal methods based on logic and/or mathematics.
\end{itemize}

\section{Research Methodology}
\label{sec:research-methodology}

As this PhD thesis dissertation is framed in the multidisciplinary field of CSCL with research questions and research objectives oriented to be answered and achieved by theoretical and empirical studies, a mixed research method needs to be employed to conduct this research. Following the research methodology framework proposed by \citeonline{Glass1995,GlassVesseyRamesh2002}, the mixed research method employed in this PhD thesis research consists in four iterative phases: informational, propositional, analytical and evaluation.

\begin{description}
\item[Informational phase:]
In this phase, the research problems and potential solutions were identified based on information gathered from the scientific literature and discussions with experts in fields of CSCL, gamification and ontology engineering. The results of this phase were an outline of the knowledge involved in this dissertation, the research questions, and the research objectives. The tasks carried out in this phase correspond to tasks extracted from the scientific (observing the world) and engineering (observing existing solutions) research methods. These tasks were:

\begin{itemize}
\item
The search, review and analysis of scientific literature regarding to: CSCL, gamification and ontology engineering. This literature review was performed with emphasis in scripted collaboration, gamification of learning and instruction, and ontology-engineering applied to Artificial Intelligence in Education (AIED).

\item
The participation as member of the research group in Applied Computing in Education Laboratory (CAEd-Lab, \emph{Laboratorio de Computação Applicada a Educação e Tecnologias Sociales Avançadas}) at the University of São Paulo. Particularly, the expertise field in CSCL and Ontologies of this research group has been very important and valuable to conduct the research and the literature reviews.

\item
The participation in several conferences and workshops related to the context and problem domain in which this dissertation is framed. These conferences and workshop, in chronological order, were: the III Escola de Ontologias UFAL-USP, 2014 (Workshop); the 20\textsuperscript{th} International Conference on Collaboration and Technology, CRIWG, 2014 (Conference), the Summer School on Computers in Education, 2015 (Workshop); the XXVI Brazilian Symposium on Computers in Education, 2015 (Conference); the 6\textsuperscript{th} Latin American School for Education, Cognitive and Neural Sciences, 2016 (Workshop); and the Higher Education for All: International Workshop on Social, Semantic, Adaptive and Gamification techniques and technologies for Distance Learning, 2017 (Workshop).

\item
The participation as visiting research at the Research Center for Service Science at the School of Knowledge Science in the Japan Advanced Institute of Science and Technology (JAIST) has also been significant for the informational phase. This research center is dedicated to study, design and implementation knowledge co-creation process in complex service systems.  This research center focuses in the use of ontologies and ontology-engineering as the technology to develop and solve a broad variety of domains/tasks, and their research members have a long history working in the research field of Artificial Intelligence in Education. Particularly, the expertise of the Prof. Mitsuro Ikeda and Prof. Riichiro Mizoguchi were valuable and important for this phase due to their involvement in various research projects related to the modeling of knowledge for the students’ learning growth, CL process, and instructional design.
\end{itemize}

\item[Propositional phase:]
In this phase, solutions were proposed and formulated using the information gathered in the previous phase. As results of the propositional phase, constructors of necessary concepts to gamify CL scenarios were identified and proposed as ontological structures in the ontology OntoGaCLeS. Prototypes of computer-based mechanisms and procedures were also developed for gathering practitioner and user opinions as early feedback of these systems. The tasks carried out in this phase correspond to task extracted from the scientific (proposing theories or models) and engineering (proposing and developing solutions) research methods. These tasks were:

\begin{itemize}
\item
The proposal of ontological structures in the ontology OntoGaCLeS to represent gamified CL scenarios and ontological models to personalize the gamification of CL scenarios based on player type models and need-based theories of motivation.

\item
The proposal of ontological structures in the ontology OntoGaCLeS to represent the application of persuasive game design models in gamified CL scenarios and ontological models to apply persuasive game strategies as a method for dealing with the motivation problem caused by the scripted collaboration.

\item
The proposal of a computer-based model to unify the modeling of the learners' growth process and the flow theory based on the principle of good balance between the perceived challenges and skills.


\item
The definition of a conceptual flow to gamify CL scenarios as a computer-based procedure to use the knowledge described in the ontology OntoGaCLeS, and the definition of a reference architecture based on this flow to build computer-based mechanisms that provide support in intelligent-theory aware systems for dealing with the motivation problem caused by the scripted collaboration.
\end{itemize}

\item[Analytical phase:]
This phase consists into analyze and explore the solutions formulated in the propositional phase with the purpose to identify whether the proposed solutions are understandable, how them can be deployed into practice, what are the potential problems in understanding and using them, and wether there are any omissions or gaps in these solutions. The tasks carried out in this phase correspond to task extracted from the empirical (applying to case studies) and analytical (developing new solutions derived from the results obtained in the case studies) research methods. These tasks were:

\begin{itemize}
\item
The formalization of an ontological model to personalize the gamification of CL scenarios based on the Dodecad player type model proposed by \citeonline{Marczewski2015b}, and the formalization of an ontological model to personalize the gamification of Cognitive Apprentice CL scenarios based on the Yee's player type model. These two formalizations were developed as case studies to validate in the evaluation phase the ontological structures proposed to systematically formalize ontological models to personalize the gamification of CL scenarios.

\item
The formalization of an ontological model to apply gamification as a persuasive technology in gamified Cognitive Apprenticeship scenarios employing the persuasive game design strategies defined in the Model-driven persuasive game proposed by \citeonline{Orji2014}.

\item
The implementation of a computer-based mechanism (as a proof of concept) in which the knowledge encoding in the ontology OntoGaCLeS is used for setting up the proper player roles and game elements for CL sessions.

\item
The development of an algorithm (as a proof of concept) to apply the principle of good balance between the perceived challenges and skills from the flow theory in the gamification of CL scenarios.

\item
The development of a computer-based mechanisms (as a proof of concept) to apply gamification as persuasive technology in the gamification of CL scenarios. 
\end{itemize}


\item[Evaluation phase:]
The focus of this phase is to conduct empirical tests and evaluations for the solutions formulated in the propositional phase and for the findings found in the analytical phase. In this phase, the empirical data gathered through the tests and evaluations aim to assess the contributions from different perspectives. The task carried out in this phase correspond to task from the empirical (validating the solutions) and analytical (analyzing the results obtained from empirical observations) research methods. These tasks were:

\begin{itemize}
\item
The analytical evaluation of the ontological structures proposed to represent gamified CL scenarios and the ontological models to personalize the gamification of CL scenarios. This evaluation was carried out by publishing these ontological structures and the ontological models obtained from them in the analytical phase (the ontological model to personalize gamification in CL scenarios based on the Dodecad player type model, and the ontological model to personalize gamification in Cognitive Apprentice CL scenarios based on the Yee's player type model) as scientific articles in conferences and journals related to the fields of CSCL, and Artificial Intelligent in Education. These articles, in chronological order, were: \aspas{\emph{Towards an Ontology for Gamifying Collaborative Learning Scenarios}} published in the 12\textsuperscript{th} International Conference on Intelligent Tutoring Systems, ITS, 2014; \aspas{\emph{An Ontology Engineering Approach to Gamify Collaborative Learning Scenarios}} published in the 20\textsuperscript{th} International Conference on Collaboration and Technology, CRIWG, 2014; and \aspas{\emph{Personalization of Gamification in Collaborative Learning Contexts using Ontologies}} published in the journal of IEEE Latin America Transactions, 2015. During the conferences important feedbacks to improve the ontological structures were obtained from informal discussions with the participants of the conferences who shared their expertise in the domain of CSCL and Artificial Intelligent in Education.

\item
The analytical evaluation of the ontological structures proposed to represent the application of persuasive game design models in gamified CL scenarios and the ontological models to apply persuasive game strategies as a method for dealing with the motivation problem caused by the scripted collaboration. This evaluation was carried out by publishing these ontological structures and the ontological models obtained from them in the analytical phase (the ontological model to apply gamification as a persuasive technology in gamified Cognitive Apprenticeship scenarios employing the persuasive game design strategies defined in the Model-driven persuasive game) as scientific articles scientific articles in conferences and journals related to the fields of CSCL, and Artificial Intelligent in Education. These articles, in chronological order, were: \aspas{\emph{Steps Towards the Gamification of Collaborative Learning Scenarios Supported by Ontologies}} published in the 17\textsuperscript{th} International Conference on Artificial Intelligence in Education, AIED, 2015; \aspas{\emph{An Ontological Model to Apply Gamification as Persuasive Technology in Collaborative Learning Scenarios}} published in the 26\textsuperscript{th} Brazilian Symposium of Informatics in Education, SBIE, 2015; \aspas{\emph{Gamification of Collaborative Learning Scenarios: Structuring Persuasive Strategies Using Game Elements and Ontologies}} published in the 1\textsuperscript{st} International Workshop of Social Computing in Digital Education, SOCIALEDU, 2015; and \aspas{\emph{An Ontology Framework to Apply Gamification in CSCL Scenarios as Persuasive Technology}} published in the Brazilian Journal of Computers in Education, 2016. During the conferences important feedbacks to improve the ontological structures were obtained from informal discussions with the participants of the conferences who shared their expertise in the domain of CSCL and Artificial Intelligent in Education.

\item
The conduction of a pilot empirical study in which, prior to carry out the full-scale empirical studies, the activities, methods, instruments and activities that have been used in the full-scale studies were evaluated to adjust and improve the full-scale study design. This empirical study has been conducted to assess the effectiveness of \emph{the ontological engineering approach to gamify CL scenarios} for dealing with the motivation problem caused by the scripted collaboration. Such effectiveness is measured by comparing the effect of the ontology-based CL sessions obtained by the approach against the effect of non-gamified CL sessions on the participants' intrinsic motivation and learning outcomes, and the percentage of participation by groups. This empirical study was conducted with undergraduate computer science students at the university of São Paulo during the second semester of 2016 in the course of Laboratory of Introduction to Computer Science, and for a CL activity related to the topic of loop structures. In such CL activity, the ontology-based gamified sessions and non-gamified CL sessions have been instantiated using a CSCL script inspired by the cognitive apprenticeship theory as the method to orchestrate and structure the collaboration among the students.

\item
The conduction of a full-scale empirical to evaluate the effectiveness of \emph{the ontological engineering approach to gamify CL scenarios}. This effectiveness has been measured by comparing the effects of ontology-based gamified CL sessions against the effects of non-gamified CL sessions on the participants' intrinsic motivation and learning outcomes. This study was carried out in the course of introduction to computer science with undergraduate computer engineering students at the university of São Paulo during the first semester of 2017. The CL activity in which these CL sessions have been instantiated was related to the topic of condition structures using a CSCL script based on the cognitive apprentice theory to orchestrate and structure the collaboration among the participants.

\item
The conduction of a full-scale empirical study to also evaluate the effectiveness of \emph{the ontological engineering approach to gamify CL scenarios}. However, in this empirical study, the effects of ontology-based gamified CL sessions against the effect of non-gamified CL sessions were compared on the participants' level of motivation instead to compare these effects on the participants' intrinsic motivation. This empirical study was carried out during the first semester of 2017 in the course of Introduction to Computer Science at the university of São Paulo with undergraduate computer engineering students. In this context, a CSCL script inspired by the cognitive apprentice theory was used to structure and orchestrate the collaboration among the students a CL activity related to the the topic of loop structures.

\item
The conduction of a full-scale empirical study to evaluate the efficiency of \emph{the ontological engineering approach to gamify CL scenarios} for dealing with the motivation problem caused by the scripted collaboration. Such efficiency was measured by comparing the effects on the participants intrinsic motivation, level of motivation, and learning outcomes caused by ontology-based CL sessions against the effects caused by CL sessions that have been gamified without using the support given by the ontology OntoGaCLeS. This empirical study was carried out in the course of Introduction to Computer Science at the university of São Paulo during the first semester of 2017. The undergraduate computer engineering students signed up in this course participated in a CL activity related to the topic of recursion in which the collaboration among the students was orchestrated and structured by a CSCL script inspired by the cognitive apprentice theory.
\end{itemize}
\end{description}

\section{Thesis Statement and Claimed Contributions}
\label{sec:thesis-statement-and-claimed-contributions}

The thesis statement of this PhD thesis dissertation is that:

\aspas{\emph{For CL activities where the CSCL scripts are used as a method to orchestrate and structure the collaboration among the participants, the gamification of CL scenarios using the support given by the ontology OntoGaCLeS constitutes an effective and efficient solution to deal with the motivation problem caused by the scripted collaboration because this ontology encodes the necessary theoretical knowledge related to theories and best practices of gamification to perform this task}.}

The claimed contributions are:

\begin{enumerate}
\item 
The identification of most relevant concepts from the theories and practices related to gamification that should be taking into account to deal with the motivation problem caused by the scripted collaboration (RO1).

\item
Ontological structures that represent the concepts identified as relevant in the theories and practices related to gamification for dealing with the motivation problem caused by the scripted collaboration (RO2).

\begin{enumerate}
\item
A set of ontological structures to represent gamified CL scenarios and ontological models to personalize the gamification of CL scenarios based on player types models and need-based theories of motivation.

\item
A set of ontological structures to apply persuasive game design models in gamified CL scenarios and ontological models to apply persuasive game strategies as a method for dealing with the motivation problem caused by the scripted collaboration.

\item 
A unify modeling of learners' growth process and flow theory as a computer-based model to apply the principle of good balance between the perceived challenges and skills for gamified CL scenarios.
\end{enumerate}

\item
A conceptual flow to gamify CL scenarios using the knowledge described in the ontology OntoGaCLeS, and a reference architecture based on this flow to build computer-based mechanisms that provide support in intelligent-theory aware systems for dealing with the motivation problem caused by the scripted collaboration (RO3).

\item
An empirical evaluation of \emph{the ontological engineering approach to gamify CL scenarios} in which, to validate the effectiveness and efficiency of this approach for dealing with the motivation problem caused by the scripted collaboration, the effects of ontology-based gamified CL sessions on students' intrinsic motivation, level of motivation and learning outcomes are compared against the effects caused by the non-gamified CL sessions and CL sessions that have been gamified without using the support given by the ontology OntoGaCLeS (RO4).
\end{enumerate}

\section{Structure of the Dissertation}
\label{sec:structure-of-dissertation}

This PhD thesis dissertation is structured in eight chapters:

\begin{description}

\item[Chapter 1:]
\emph{Introduction}

\item[Chapter 2:]
\emph{General Background and Fundamental Concepts}
contains the background related to the context and research problem addressed in this dissertation. An overview related to the fields of CSCL and scripted collaboration, gamification and ontology engineering are presented in the chapter. The motivation problem caused by the scripted collaboration, and the current approaches to deal with this problem are also detailed in the chapter. The concepts that were identified as relevant in the theories and practices of gamification and their difficulties to apply it in CL scenarios for dealing with the motivation problem caused by the scripted collaboration are presented in the chapter.

\item[Chapter 3:]
\emph{Ontological Structure to Personalize the Gamification in CL Scenarios} describes the ontological structures, that have been proposed by the author of this thesis, and that have been formalized in the ontology OntoGaCLeS, to represent gamified CL scenarios and ontological models to personalize the gamification in CL scenarios based on player types models and need-based theories of motivation. The chapter also shows the procedure followed to build an ontological model ontological model to personalize the gamification of CL scenarios based on the Dodecad player type model. 

\item[Chapter 4:]
\emph{Ontological Structures of Persuasive Game Design in CL Scenarios} describes the ontological structures proposed by the author of this thesis to apply persuasive game design models in gamified CL scenarios and to represent ontological models to apply persuasive game strategies as a method for dealing with the motivation problem caused by the scripted collaboration. The chapter also describes the procedure to formalize an ontological model in which gamification is applied as persuasive technology for gamified Cognitive Apprenticeship scenarios employing the persuasive game design strategies defined in the Model-driven persuasive game proposed by \citeonline{Orji2014}.

\item[Chapter 5:]
\emph{A Unify Modeling of Learners' Growth Process and Flow Theory} presents the computer-based model proposed by the author of this thesis to unify the modeling of the learners' growth process and the flow theory based on the principle of good balance between the perceived challenges and skills. This model has been used in the gamification of CL scenarios through an algorithm in which this principle is used to define the level of rewards. This algorithm as a proof of concept for the computer-based model to unify the modeling of learners' growth process and flow theory is also presented in the chapter.

\item[Chapter 6:]
\emph{Computer-based Mechanisms and Procedures to Gamify CL Scenarios} describes the flow proposed by the author of this thesis to use the knowledge described in the ontology OntoGaCLeS to gamify CL scenarios. The reference architecture based on this flow by which computer-based mechanisms could be built in intelligent-theory aware systems to provide support in the gamification of CL scenarios for dealing with the motivation problem caused by the scripted collaboration is presented in the chapter. The chapter also describes the computer-based mechanisms that has been developed by the author of this thesis using the reference architecture to conduct the evaluation of the ontological engineering approach to gamify CL scenarios.

\item[Chapter 7:]
\emph{Evaluation of the Ontological Engineering Approach to Gamify CL Scenarios} presents the empirical studies that have been carried out in real situations to validate the effectiveness and efficiency of this approach to deal with the motivation problem caused by the scripted collaboration.

\item[Chapter 8:]
\emph{Conclusions and Future Work} summarizes the contributions of this PhD thesis dissertation, and the chapter also discusses possible future research directions.

\end{description}






\chapter{General Background and Fundamental Concepts}
\label{chapter:general-background}
%
\chapter{General Background and Fundamental Concepts}
\label{chapter:general-background}

This chapter presents the general background and fundamental concepts related to the domain problem that is addressed in this thesis. At the first section (\autoref{sec:cscl-and-scripted-collaboration}), an overview of the CSCL field and scripted collaboration is presented to provide a comprehensive and clear understanding about the research context. %This section also describes in detail the motivation problem caused by the scripted collaboration when CL activities are orchestrated and structured by CSCL scripts, as well as, the related works and current computer-based solutions to deal with this problem.
The \autoref{sec:gamification} elaborates an overview of gamification, and the best practices and theories related to this technology. % Furthermore, this section presents related works that use gamification in the context of CSCL and other contexts to deal with the motivation problem are presented in this section.
Finally, the \autoref{sec:ontologies-and-ontology-engineering} presents the fundamentals of ontologies and ontology engineering.% This sections also discusses how ontologies is currently used to support the systematic formalization of theory-based knowledge, and how this formalization, in the field of artificial intelligence in education, is used to overcome problems that are similar to those that must be solved to provide a computational support in the gamification of CL scenarios with theoretical justifications based on the best practices and theories related to gamification.

%%%%%%%%%%%%%%%%%%%%%%%%%%%%%%%%%%%%%%%%%%%%%%%%%%%%%%
\section{CSCL and Scripted Collaboration}
\label{sec:cscl-and-scripted-collaboration}

Although CL has a long history in education, it is not until the early 1990s that the research field dedicated to study how to provide support for the CL through the use of Internet and computational technology had gained attention and strength \cite{StahlKoschmannSuthers2006}. Such research field known as Computer-Supported Collaborative Learning (CSCL) is a multidisciplinary field that combines studies from the cognitive psychology education and from the computer science to effectively enhance the CL process through the use of computational technology \cite{HoppeOgataSoller2007}.

The general aim of CSCL field is to develop technologies to support or create situations in which two or more students learn together through the interaction among them \cite{Dillenbourg1999}. In these situations, the learning outcomes is consequence of students' interactions and how these interactions affect the individual learning for each one of the students. In consequence, to enable a well-though-out design of CL, the CSCL scripts have been proposed by the CSCL community as the technology to facilitate the social and cognitive processes of learning by describing the way in which the learners will interact with each other in a CL scenario \cite{HarrerKobbeMalzahn2007}.

\subsection{CSCL Scripts}
\label{sec:cscl-scripts}

CSCL scripts are the technology that describes how to structure and orchestrate the CL process to attain a set of pedagogical objectives defined by an instructional design \cite{DillenbourgJermann2007}. Such description is provided in the CSCL scripts through prescribed instructions that indicates how to facilitate the social and cognitive processes in group activities \cite{Dillenbourg2002}. These prescribed instructions are defined by instructors, like teachers or instructional designers, as a way to attain a set of learning goals, and they indicate the way in which students should collaborate, they constrain the interactions among the participants, they specify the roles for the participants, they indicate the distribution of task, tools, and resources used in the CL process.

In order to narrow the number of elements used to describe the CSCL scripts, and provide a common and sharable description of CSCL scripts, \citeonline{KobbeWeinbergerDillenbourgHarrerHamalainenHakkinenFischer2007} propose a framework that is currently wide accepted by the community as the common specification to describe the CSCL scripts using natural language. This framework formalizes the CSCL scripts as a set of components and mechanisms illustrate in \autoref{fig:components-and-mechanisms-of-cscl-scripts}.


\begin{figure}[htb]
 \caption{Components and mechanisms of CSCL scripts}
 \label{fig:components-and-mechanisms-of-cscl-scripts}
 \centering
 \includegraphics[width=0.95\textwidth]{images/components-and-mechanisms-of-cscl-scripts}
 \fadaptada{Fischer2007}
\end{figure}

The structural \emph{components} of CSCL scripts are the participants, groups, activities, roles and resources. The component of \emph{participants} is used to describe the participants, such as learners, monitors, and teachers. Although this description can be abstract or concrete and simple or complex, it is often presented in a simple manner with rules that indicate conditions to participate in the CL process. The component of \emph{activities} describes what will be performed by the participants in the CL process to attain the learning goals defined by the instructional designers. The component of \emph{roles} describes the privileges, obligations and expectations of participants in the CL process. The component of \emph{groups} of participants defines through hierarchical structures how the students are grouped according to the participants' characteristics. The component of \emph{resources} describes the learning objects (e.g. content, material, and tools) that can be used by the participants during the CL process.

The \emph{mechanisms} of CSCL scripts are the group formation, component distribution and sequencing. The mechanism of \emph{group formation} consists in the specifications of how the participants will be distributed over the groups. The mechanism of \emph{task distribution} provides the specification about how the components of scripts are distributed over groups using the mapping of groups, activities, roles, and resources. The mechanism of \emph{sequencing} consists in the definition of how the components and groups defined in scripts are distributed over time. In general, this sequencing describes the execution order of activities in the CL process.

\autoref{qua:social-script-framework-kobbe} shows the description of social script \cite{WeinbergerErtlFischerMandl2005} using the framework proposed by \citeonline{KobbeWeinbergerDillenbourgHarrerHamalainenHakkinenFischer2007}. In this example, the CL scenarios orchestrated by the social script foster the acquisition of knowledge through a set of case studies (\emph{resources}) that are analyzed and reviewed by the students groups. The number of students in each group is equal to the number of case studies, and the ideal number is three. In the first step of sequencing, each learner playing the \emph{analysis} role writes down an analysis of case study, and then, he critiques the analyses made by other learners playing the \emph{critics} role. In the second step of sequencing, each learner revises his/her own analysis, taking into consideration the critiques received by the other learners in the case group.

\begin{quadro}[htb]
\caption{Social script describes using the framework proposed by  \citeonline{ KobbeWeinbergerDillenbourgHarrerHamalainenHakkinenFischer2007}}
\label{qua:social-script-framework-kobbe}
\centering
\footnotesize
\begin{tabular}{l p{12cm}}
\toprule
\multicolumn{2}{l}{\textbf{Structural components:}} \\ \midrule
\textbf{Participants:} &
A number of participants that must be divisible by the number of case studies.  \\
\textbf{Groups:} &
Case groups \\
\textbf{Activities:} &
(a) Applying theoretical concepts to the case study and constructing arguments \\
 &
(b) Critiquing initially scaffolder with prompts for eliciting clarification, identifying conflicting views and constructing counter-arguments \\
\textbf{Roles:} &
\emph{Analyst} and \emph{Critic} \\
\textbf{Resources:} &
Case studies (minimal number is three case studies) \\
\toprule
\multicolumn{2}{l}{\textbf{Mechanisms}:} \\ \midrule
\textbf{Group formation:} &
All participants are grouped by the number of case studies. Each participant becomes member of all case groups although with different roles in each. Each participant is the responsible analyst for one case study and critic for all other cases \\
\textbf{Task distribution:} &
Each case group receives one case study, and the roles are distributed in a way that each participant assumes the role of analyst in one case group and the role of critic in all other case groups \\
\textbf{Sequencing:}
& - the analyst writes an analysis of case study. (a) \\
& - wait for all case group analysts to be done, and writes a critique for the analysis of case study. (b) \\
& - wait for all case group critics to be done, and the analyst considers each critique and writes a reply to each. (a) \\
& - wait for all case group analysts to be done each critic in turn reads the reply and writes a second critique. (b) \\
& - wait for all case group critics to be done... the analyst considers all critiques and revises the analysis of case study (a) \\
\bottomrule
\end{tabular}
 \fadaptada{KobbeWeinbergerDillenbourgHarrerHamalainenHakkinenFischer2007}
\end{quadro}


Having the description of CSCL scripts only in natural language does not allow the computers programs to interpret them, and to run a CL scenario following the instructions indicated by the scripts without human intervention. Therefore, to represent the CSCL scripts in a computer readable manner, the IMS-Learning Design\footnote{\url{http://www.imsglobal.org/learningdesign/}} (IMS-LD) specification has been adopted by different tools, such as (web)COLLAGE \cite{Hernandez-LeoVillasclaras-FernandezAsensio-PerezDimitriadisJorrin-AbellanRuiz-RequiesRubia-Avi2006,Villasclaras-FernandezHernandez-LeoAsensio-PerezDimitriadis2013}, CIAN \cite{MolinaRedondoOrtega2012}, LeadFlow4LD \cite{Palomino-RamirezBote-LorenzoAsensio-PerezDimitriadis2008}, NUCLEO \cite{SanchoFuentes-FernandezFernandez-Manjon2008}, CoLearn \cite{StylianakisArapiMoumoutzisChristodoulakis2013}, CeLS \cite{RonenKohen-Vacs2009}, and LAMS \cite{Romero-MorenoOrtegaTroyano2007}, as the language to describe CSCL scripts.
 
Despite the benefits that brings  the use of the IMS-LD specification to represent CSCL scripts, several researchers had indicated that this language is insufficient to fully support the modeling of CSCL scripts \cite{AlharbiAthaudaChiong2014, CaeiroAnidoLlamas2003}. Of course, the purpose of IMS-LD specification is not to provide a full support for describing CSCL scripts in a computer-readable manner, the IMS-LD has been developed as a neutral, generic and flexible educational modeling language to describe a wide range of pedagogies approaches (the teaching strategies, pedagogical goals and their associated activities) \cite{Koper2005}. In this sense, to support the representation of CSCL scripts in a computer-readable manner, a wide variety of extensions on the IMS-LD elements has been proposed in by several researchers \cite{Bote-LorenzoVaquero-GonzalezVega-GorgojoDimitriadisAsensio-PerezGomez-SanchezHernandez-Leo2004, LeoPerezDimitriadis2004, MagnisalisDemetriadis2012, MiaoHoeksemaHoppeHarrer2005, Vega-GorgojoBote-LorenzoGomez-SanchezDimitriadisAsensio-Perez2005}.

Instead, to simply provide a computer-readable representation of CSCL scripts, the work of \citeonline{Isotani2009} proposes the formalization of these scripts in a computer-understandable manner through the use of ontologies. This solution consists in a set of ontological structures that makes the description of CSCL scripts more semantically-rich, allowing the explicit specification of learning goals, purposes, and other relevant information that cannot be represented using the IMS-LD specification, i.e., learning strategies, group goals, interaction patters from learning theories. Providing this formalization in the CL ontology, \citeonline{IsotaniMizoguchiIsotaniCapeliIsotanideAlbuquerqueBittencourtJaques2013} demonstrates that intelligent-theory aware systems can interpret these scripts and provide advice and recommendation to support for the modeling of learners' development \cite{InabaIkedaMizoguchi2003}, the formation of effective groups \cite{IsotaniMizoguchi2008a}, and the instructional design of CL activities \cite{IsotaniMizoguchiIsotaniCapeliIsotanideAlbuquerqueBittencourtJaques2013}.

\subsection{Levels of Abstraction and Granularity of CSCL Scripts}
\label{sec:level-of-abstraction-and-granularity-of-cscl-scripts}

CSCL scripts have different levels of abstraction and granularity in the description of CL scenarios \cite{Dillenbourg2002, DillenbourgJermann2007, Villasclaras-FernandezIsotaniHayashiMizoguchi2009}. This classification of scripts in two dimension, abstraction and granularity, gives them an enormous flexibility to be reused in the instructional design process of CL scenarios, and it also allows the use of multiple scripts to describe different aspects of CL scenario in separated scripts. Whereas the levels of abstraction classify a script according to the completeness of elements described by them (from the most abstract to the most concrete), the levels of granularity classify the scripts according to the aggregation level of elements described by them (from the most coarser grained to the finest grained).

According to \citeonline{DillenbourgJermann2007}, a CSCL script can be classified in one of the four levels of abstraction defined as follows as:

\begin{description}
\item[\emph{Script Schemata}:] are scripts use to describe the core instructional design principles whereby is expected to trigger interactions among participants in the CL process. In this sense, these scripts are defined in a content free didactic form, so that they can be used to describe patterns of CL. Examples of script schemata are the Jigsaw script \cite{Aronson1978, KordakiSiempos2010}, conflict script \cite{WeinbergerErtlFischerMandl2005}, and reciprocal script \cite{King2007}. The jigsaw script describes a CL scenario in which the principle of interaction consists in the grouping and re-grouping of participants with complementary information to share their knowledge. The conflict script describes a CL scenario to group learners with contradictory knowledges or opinions to instigate the discussion. The reciprocal script describes a CL scenario that assigns alternate roles to the students for facilitating questioning and tutoring activities.

\item[\emph{Script Classes}:] are specialization of scripts schemata for a specific learning context. This specialization is not absolute complete, so that script classes are independent in the content-domain and student data. The script classes cover a range of scripts that describe variations of a prototype with particular details related to a specific learning context of a script schemata to facilitate its adoption. These details are, for example, the number of participants, and the king of content (matter) that will be taught. In this sense, a script class is based in a script schemata to describe CL scenarios for a specific learning context. For instance, the Universanté Script \cite{DillenbourgJermann2007} is a script class based on Jigsaw schema that was designed to describe CL scenarios for learning contexts with different thematic groups and participants from different nations.

\item[\emph{Script Instances}:] are scripts in which the content-domain are specified for a particular situation. A script instance is more concrete than a script class, and it has been instantiated from a script schema or class to be reusable more or less by teachers who only need to define participants' data. These scripts are more concrete that script classes, but they are independent in the particularities of students and learning environment.

\item[\emph{Script Sessions}:] are scripts in which the content-domain and participants data are specified to be directly executed in a learning environment. In this sense, these scripts detail the information of participants and content-domain in the most concrete level defining, for example, the students' names and the deadlines of activities. A CL scenario that is described by a script session is known as CL session, and when it is represented in a script session using a computer-readable formalization, it can be directly executed in a learning environment to orchestrate and conduct the CL process.
\end{description}

Different benefits from the use of script schemata and classes as patterns are obtained in the instructional design process of CL scenarios \cite{AlharbiAthaudaChiong2014, ChallcoBittencourtIsotani2016, MiaoHoeksemaHoppeHarrer2005}. During the design/authoring phase, repositories of script schemata and classes facilitate the sharing and reuse of these scripts in distributed learning environments \cite{PrietoAsensio-PerezMunoz-CristobalDimitriadisJorrin-AbellanGomez-Sanchez2013, PrietoTchounikineAsensio-PerezSobreiraDimitriadis2014}. The structures of script schemata and classes are used as templates to create new script schemata and classes \cite{AndreasHarrerUlrchHoppe2007, RonenKohen-Vacs2009}. 

During the instantiation/production phase, script schemata and classes provide advice and recommendation that help the CL practitioners to instantiate these scripts and to obtain CL sessions \cite{MagnisalisDemetriadis2012a, PrietoAsensio-PerezDimitriadisGomez-SanchezMunoz-Cristobal2011,Alario-HoyosBote-LorenzoGomez-SanchezAsensio-PerezVega-GorgojoRuiz-Calleja2013}. Script schemata and classes facilitate the generation of computer-interpretable scripts, they provide information to support the search of applicable learning material and tools for the CL scenario \cite{Bote-LorenzoVaquero-GonzalezVega-GorgojoDimitriadisAsensio-PerezGomez-SanchezHernandez-Leo2004, IsotaniMizoguchi2008a, Vega-GorgojoBote-LorenzoGomez-SanchezDimitriadisAsensio-Perez2005}. The script schemata and classes are also uses to obtain recommendation about how to bind individuals in groups and roles according to the knowledge described in these scripts \cite{IsotaniMizoguchiIsotaniCapeliIsotanideAlbuquerqueBittencourtJaques2013,Villasclaras-FernandezHernandez-GonzaloLeoAsensio-PerezDimitriadisMartinez-Mones2009}.

Regarding to the level of granularity \cite{FischerKollarStegmannWeckerZottmann2013}, the CSCL scripts can be classified in macro-scripts and micro-scripts.

\begin{description}
\item[\emph{Macro-scripts}:] are scripts that basically describe the CL process in a courser-grained level without detailing the specific interactions among participants. A macro-script describes how to attain a set of pedagogical objective indicating the sequencing of individual and group activities that must be follow by participants. Thus, for example, in the Jigsaw macro-script, to promotes the individual accountability and positive interdependence, the sequencing of activities consists in three activities: an individual activity, expert group activity, and jigsaw group activity. In the individual activity, each student studies a particular part of a whole problem. In the expert group, the students of different groups that study the same part of the whole problem meet together for exchanging ideas. At last activity, students of each jigsaw group meet to contribute with their expertise to solve the whole problem.

\item[\emph{Micro-scripts}:] are scripts that describe the CL process in a fine-grained level \cite{WeinbergerFischerStegmann2005}, they indicate, for example, the dialogues that must happen among student to achieve the pedagogical objectives, and they are intended to describe the communication model between participants. Thus, for example, to facilitate the negotiation and elaboration of a domain concepts, Weinberger, Ertl, Fischer, and Mandl  \cite{WeinbergerErtlFischerMandl2005} describe a micro-scripts for online peer discussion using a sequence of sentence openers (e.g. my proposal for an adjustment of the analysis is….) that prompted learners to contribute with the discussion and critique one another's contributions.
\end{description}

As can be noticed above, the macro-scripts and micro-scripts have a hierarchical relationship to describe the CL process of CL activities. The micro-scripts describe the communication process in a CL activity \cite{WeinbergerFischerStegmann2005}, whereas the macro-scripts describe groups, roles, and flow of CL activities \cite{DillenbourgHong2008}. Despite this explicit hierarchical relationship, there are few models and tools in which all the elements of macro-scripts and micro-scripts are combined to support the design of CL scenarios \cite{AlharbiAthaudaChiong2014, ChallcoBittencourtIsotani2016}. \citeonline{Hernandez-LeoVillasclaras-FernandezAsensio-PerezDimitriadisRetalis2006} propose a hierarchical model in which schemata and classes of macro-scripts and micro-scripts are used as templates to generate scripts. In the work of \citeonline{ChallcoGerosaBittencourtIsotani2014}, the hierarchical relationships of macro-scripts and micro-scripts is represented as hierarchical task networks to support the automatic generation of unit of learning.

In the CL ontology \cite{IsotaniInabaIkedaMizoguchi2009}, and therefore in the ontology OntoGaCLeS, the hierarchical relationship of macro-scripts and micro-scripts is not explicitly described as a direct link between macro- and micro-scripts. The hierarchical relationship is implicitly described as part of the conceptualization of events and processes proposed by Galton and Mizoguchi \cite{GaltonMizoguchi2009}. Based on in this conceptualization in which the representation of an event can be constituted by many distinct sub-events to describe a process, the hierarchical relationship of macro- and micro-scripts can be inferred from these events that are explicitly described in the CL ontology and the ontology OntoGaCLeS.

%%%%%%%%%%%%%%%%%%%%%%%%%%%%%%%%%%%%%%%%%%%%%%%%%%%%%%
\section{Gamification}
\label{sec:gamification}

\subsection{Gamification of Learning and Instruction}


\subsection{Flow Theory and Learning and Gamification}


%In the context of education, learners in the flow state frequently experience positive affect and better scores/performances compared with other learners who are in a similar situation but not in the flow state \cite{BaydasKarakusTopuYilmazOzturkGoktas2015, IbanezDiSerioVillaranDelgadoKloos2014, ShernoffCsikszentmihalyiSchneiderShernoff2014}. For example, several empirical studies conducted by D’Mello, Graesser, and colleagues using intelligent tutoring systems (e.g. AutoTutor) have shown strong positive correlations between learning gains, confusion, and flow state \cite{D'MelloGraesser2012}. Another example is the study of Choi et al., where participants used a web-based e-learning system in a program on Enterprise Resource Planning. The data of this study revealed that flow experiences were directly linked with learning outcomes and leaners’ attitude towards e-learning.

%According to previous findings, having an explicit formalization of the minimum and maximum difficulty/challenge levels to maintain the learners' flow is one of the key features needed to promote more effective and robust learning in different scenarios \cite{Esteban-MillatMartinez-LopezHuertas-GarciaMeseguerRodriguez-Ardura2014, FulmerD'MelloStrainGraesser2015 ,LinehanBellordKirmanMorfordRoche2014}. 


%To support the design of better learning scenarios that are pedagogically sound and can keep learners in a flow state, it is essential during the instructional design process to take into account the level of difficulty of learning objects and to link learning objects with theories that describe leaners’ growth. Unfortunately, this task requires specialized knowledge about instructional/learning theories, Flow Theory, and Affect Theory, and the skills to apply this knowledge in an integrated manner in order to select adequate learning objects and design effective learning scenarios that match students’ abilities. 

%To support the design of authoring tools that help instructional designers with the proper selection of levels of challenges that keep the participants in flow, in this paper we propose a framework to integrate the learner’s growth process and Flow Theory through a new theory-based model, named GMIF: Learner’s Growth Model Improved by Flow Theory. This model explains and describes the necessary conditions under which learners are able to learn more effectively based on learning theories, while keeping the ability-challenge balance of tasks defined in the Flow Theory. In particular, the GMIF has been used to create algorithms that help to automatize the selection of proper learning objects for specific learning situations.

%In a collaborative learning scenario, the challenge of designing adequate activities and selecting learning objects is even harder. If the instructional designer selects problems and learning objects that are too difficult (or too easy) for students, it will hinder students’ interactions, demotivate students, and lead students to not want to work in groups over time (Challco, Moreira, Mizoguchi, & Isotani, 2014; Isotani, Inaba, Ikeda, & Mizoguchi, 2009). For instance, consider a scenario where a student (the tutor) interacts with another student (the tutee) to solve a given problem (i.e. a selected learning object). In this situation, the tutor will learn by using his knowledge/skills to demonstrate how to solve a problem and the tutee will learn by following the tutor’s guidance. If the problem is too hard or the sequence of activities is not created to help students to collaborate, the tutor will not have the sufficient skill level or knowledge to solve and guide the tutee in the resolution of the problem. As a result, the learning scenario will cause emotional distress in both tutor and tutee, and the desired learning outcomes will not be achieved. 

%To support the design of better learning scenarios that are pedagogically sound and can keep learners in a flow state, it is essential during the instructional design process to take into account the level of difficulty of learning objects and to link learning objects with theories that describe leaners’ growth. Unfortunately, this task requires specialized knowledge about instructional/learning theories, Flow Theory, and Affect Theory, and the skills to apply this knowledge in an integrated manner in order to select adequate learning objects and design effective learning scenarios that match students’ abilities. 

%To support the design of authoring tools that help instructional designers with the proper selection of levels of difficulty that keep the learners in flow, in this paper we propose a framework to integrate the learner’s growth process and Flow Theory through a new theory-based model, named GMIF: Learner’s Growth Model Improved by Flow Theory. This model explains and describes the necessary conditions under which learners are able to learn more effectively based on learning theories, while keeping the ability-challenge balance of tasks defined in the Flow Theory. In particular, the GMIF has been used to create algorithms that help to automatize the selection of proper learning objects for specific learning situations.

%we present related works on the application of Flow Theory in educational settings. Following that, we present the GMIF, offering a detailed description of our framework that integrates the LGM and Flow Theory. 



%The related work and frameworks presented in the previous paragraph are important for guiding educators and game designers to create better learning situations. Nevertheless, they were not created to automate the process of learning design and do not have the necessary formalization to be implemented and included as a feature in a learning design authoring tool. This means that, if an instructional designer wants to maintain the learner's flow state, he/she will need to do so manually, without any computational support. Such a manual approach is infeasible to be carried out when there is a need to plan personalized sequences of activities for a class of students with different needs, using a database with multiple learning objects (e.g. games, texts, videos, images, etc.) and taking into consideration several pedagogical approaches to support flow experiences. Toward the automation of detecting and using flow state to create better learning experiences, Lee and colleagues provide adaptive learning contents by selecting appropriate problems based on the three-channel flow model \cite{LeeJhengHsiao2014}. They propose an automatic flow detector where the three-channel flow model is built based on features related to affective dimensions (i.e. valence and arousal) and interactions (i.e. mouse click duration, keystroke duration) with learning software.

\newpage
%%%%%%%%%%%%%%%%%%%%%%%%%%%%%%%%%%%%%%%%%%%%%%%%%%%%%%
\section{Ontologies and Ontology Engineering}
\label{sec:ontologies-and-ontology-engineering}

This formalization is achieved through ontology engineering in which the similarities and differences of these concepts are identified to describe their application in the gamification of CL scenarios and the building of gamification model for CL scenarios.



%In Section \ref{sec:relevant-technologies}, we presented the fundamental concepts about ontologies and ontology engineering. In this section, we detail these concepts and the techniques to create ontologies. The most of the concepts and ideas presented in this section come from the tutorial on ontology engineering published in three parts in \cite{mizoguchi2003part, mizoguchi2004tutorial, mizoguchi2004part}.

\subsection{Ontology and Its Elements}
\label{subsec:ontologies}

%For philosophers, ontology (from the Greek "being, that which is," present participle of the verb "be," and "science, study, and theory") is the philosophical study of the nature of being, becoming, existence, or reality, as well as the basic categories of being and their relations \cite{wikipedia2014ontology}. In computer science, \cite{gruber1993a} defines an ontology as an explicit specification of a conceptualization, in which this conceptualization refers to the meaningfulness of concepts and their relationship given the context of the target world. 

%\cite{swartout1999guest} defines an ontology as the basic structure or armature around a knowledge base that can be built. As the knowledge bases are composed by facts of a given domain \cite{hayes1984building}, an ontology can also be defined as a framework in which these facts are represented. For \cite{guarino2009ontology}, an ontology is not a simple representation of concepts and their relations. The ontologies contain restrictions defined through axioms, in which these axioms are formal logical expressions that validate and check the consistency of domain. Finally, \cite{mizoguchi2003part} states that ontologies constitute agreements to achieve the mutual understanding of the target domain in a human and computer understandable manner.

%The definitions of ontologies presented above are the reasons why ontologies have attracted many researches and practitioners to use them in the development of powerful and intelligent computational applications. In these applications, ontologies first provide a common conceptual structure that enables the development of sharable and reusable knowledge-based, and second they facilitate the interoperability of information enabling the merge and integration of data from different sources.

%The common components of ontologies are individuals, classes, attributes, and relations \cite{biemann2005ontology, sugumaran2002ontologies}. Individuals are instances or objects that constitute the basic or ground level of ontologies. The classes are sets, collections, concepts, types of objects, or kinds of things. Attributes are aspects, properties, features, characteristics, or parameters that classes can have. Relations are ways in which classes and individuals can be related to one another.

%According to \cite{mizoguchi2003part}, an ontology is a theory of concepts rather than terms, where it is constituted by the following two elements \cite{isotani2010ontological}:

%\begin{enumerate}
%\item A \emph{set of essential concepts} that result from the articulation of basic knowledge present in a given domain, where the concepts can be represent using a specialized vocabulary.

%\item The \emph{body of knowledge} that describes the given domain using the essential concepts, and it is composed by:

%\begin{itemize}
%\item The hierarchy (class/sub-class) resulting from "\emph{is-a}" relations between concepts;
%\item The definition of important relations between concepts apart from the \emph{is-a} relation (e.g. \emph{part-of}, and \emph{same-as} relations);
%\item The \emph{axiomatization} of semantic constraints between those concepts and relations.
%\end{itemize}
%\end{enumerate}

%Usually, when developing ontologies, large amount of time is spent over the discussing about the terminology to be used (vocabulary) rather than understanding the critical concepts of the domain \cite{isotani2010ontological,mizoguchi2003part}. However, to create a good ontology the definition of concepts is more important, and the labeling of these concepts pass to have less importance. Thus, when there are terminological problems, it is not a bad practice to use a sentence or a provisional term to denote a concept. For this concept-oriented viewpoint process, the quality of an ontology is decided by the knowledge that can be explained by the ontology and the essential properties of concepts that are explicitly represented it \cite{mizoguchi2004tutorial}.

%According to \cite{isotani2010ontological}, to create the body of knowledge of an ontology, besides the definition of concepts and terms to label them, it's more important to make: 
%\begin{enumerate}%[(a)]
%\item a clear distinction between roles and basic concepts,
%\item identify the appropriate use of relations, especially \emph{is-a} and \emph{part-of} relations,
%\item avoid multiple inheritance, and
%\item properly distinguish what an attribute is and what a property is, as well as many other import an decisions that need to be made in order to produce a \emph{good} ontology.
%\end{enumerate}

%According to\cite{mizoguchi2003part, isotani2010ontological}, a \emph{good ontology} is the \emph{more ontological}. By ontological, we mean close to the fundamental conceptualization where the knowledge can be explained and the essential properties of concept are explicitly represented.

\subsection{Types of Ontologies}
\label{subsec:types-otologies}

%According to different characteristic of ontologies, they can be classified in different types, using the level of dependence (upper ontology, task ontology, domain ontology and application ontology) \cite{guarino1997semantic}, and the level of formal representation (lightweight and heavyweight ontologies) \cite{wong2012ontology}. 

\subsubsection{Classification by Level of Dependence}
\label{subsubsec:classification-level-dependence}

%Figure \ref{fig:ontologies-level-dependence} shows the classification of ontologies based on the level of dependence. Thus, the ontologies are classified in:

%\begin{itemize}
%\item \emph{upper ontologies} describe what exist in the world, using very general concepts like space, time, matter, objects, events, actions, among others. The description of concepts in this type of ontology is independently of the problem or domain. These ontologies should always be used in conjunction with other ontologies. Examples of upper ontologies are Standard Upper Ontology (SUO) \cite{pease2002ieee}, Suggested Upper Merged Ontology (SUMO) \cite{pease2002suggested}, and Cyc / OpenCyc \cite{matuszek2006introduction}.

%\item \emph{domain ontologies} and \emph{task ontologies} describe, respectively, the vocabulary related to a generic domain (i.e. vehicles and places) and activity (i.e. repairing and traveling). The domain ontologies define a vocabulary with common terms for reuse and sharing of information for a specific domain. In the task ontologies, the vocabulary is associated with the problem solving, independent of domain.

%\item \emph{application ontologies} describe concepts depending both on the particular domain (domain ontology) and task (task ontology). These concepts often correspond to roles played by domain entities while performing a certain activity with the resolution of a problem. The concepts defined in this type of ontologies are often described by specializations of domain and task ontologies.
%\end{itemize}

%\begin{figure}[thbp]
%\begin{center}
%\includegraphics[width=0.5\textwidth]{ontologies-level-dependence.png}
%\caption[Types of ontologies according to level of dependence]{Types of ontologies according to level of dependence (Adapted from \cite{guarino1997semantic}).}
%\label{fig:ontologies-level-dependence}
%\end{center}
%\end{figure}

\subsubsection{Lightweight Ontologies and Heavyweight Ontologies}
\label{subsubsec:lightweight-ontologies}

%Based on the level of formal representation, ontologies can be classified in lightweight ontologies and heavyweight ontologies. As we can see in the Figure \ref{fig:spectrum-ontologies}, at one extreme, there are lightweight ontologies that consist of terms with little or no specification of the meaning. At the other end of the spectrum, we have heavyweight ontologies that comprise ontologies rigorously formalized by logical theories. As we move along the continuum, the amount of meaning specified and the degree of formality increases, reducing possible ambiguities \cite{uschold2004ontologies}.

%\begin{figure}[thbp]
%\begin{center}
%\includegraphics[width=0.85\textwidth]{spectrum-ontologies.png}
%\caption[The spectrum of lightweight and heavyweight ontologies]{The spectrum of lightweight and heavyweight ontologies (Taken from \cite{wong2012ontology}).}
%\label{fig:spectrum-ontologies}
%\end{center}
%\end{figure}

%The \emph{ligthweight ontologies} are ontologies based on topical hierarchies with lack rigorous conceptual definitions, principled conceptual organization, and label-concept distinctions. As instance of this kind of ontologies, we have terms, glossaries, thesauri, and database schemas. The main purpose of this kind of ontology is to provide a weak categorization of content to improve search engine functionality. Thus, the ligthweight ontologies are broadly used on the Web to categorize a large amount of data, such as available data on Web portals. However, these ontologies tend to be very usage-dependent and user-dependent of applications.

%The \emph{heavyweight ontologies} are much more than just lightweight ontologies. They are ontologies enriched with axioms for semantic interpretation of concepts and relations. Thus, the development of heavyweight ontologies needs a rigorous definition of concepts, an organization of defined concepts based on philosophical principles, a precise and formal semantic definition of relations among concepts, and so on. Heavyweight ontologies are important to create shareable and reusable knowledge bases, because they give more value to concepts represented on them by providing greater semantic precision and ensuring the fidelity and consistency of concepts about a target world.

%In this dissertation, we will develop the ontology OntoGaCLes as a heavyweight ontology, and application ontology for the domain of gamified CL scenarios.

\subsection{Ontology Representation}
\label{subsec:ontology-representation}

%Nowadays, the ontologies can be represented in two ways, one representation is the formal representation that is used for computer consumption, and another representation is the graphical representation for human comprehension. Both representations are relevant for this dissertation and they affect the quality of the representation of ontology.

\subsubsection{Formal Representation}
\label{subsubsec:formal-representation}

%To allow the formal representation for computer consumption, there are many languages that have been proposed using the predicate logics, description logics or frame based languages \cite{mizoguchi2004tutorial}. However, the most popular language and framework to describe ontologies is the Web Ontology Language (OWL) language that is based on the Resource Description Framework (RDF)/RDF-Schema.

%The RDF specification was developed by the World Wide Web Consortium (W3C) for metadata description. It is formally represented in the eXtensible Markup Language (XML) employing triplets that contain a subject node, predicate, and object node ($<$subject, predicate, object$>$). Each node in the triplet can be a web resource (URI reference), a value (literal) or a document identifier (to represent a blank node). A set of triples also can become a node itself, and a property is a semantic relation between nodes (subject and object).

%To represent triplets, the RDF/RDF-Schema specifications define classes, properties, and relationships that can be used to describe these triples as statements about resources. It also includes definition of tags and hierarchical structures (taxonomy) providing the basic elements for the description of ontologies. However, the RDF-Schema has some limitation, especially to support computational reasoning on data available through the internet \cite{patel2005building}. Thus, the OWL specification provides the expressive language to develop ontologies.

%OWL is a language developed and endorsed by the W3C to satisfy the formalism for the Semantic Web (SW). It allows the SW applications to understand and answer queries of agents (people or other programs) by reasoning on Web content by ontological descriptions. OWL was developed based on DAML+OIL \cite{horrocks2002damloil} with a formal specification influenced by description logics, the frames paradigms and the OWL exchange syntax (namely RDF/XML) \cite{horrocks2003shiq}.

%There are three variants of OWL referred as OWL Lite, OWL DL and OWL Full. These three variants allow to achieve a good balance between scalability and expressive power. According to the OWL specification, each variant is an extension of its simpler predecessor. Thus, OWL Lite is used manly for classification hierarchy and simple constraints; OWL DL gives maximum expressiveness retaining computational completeness and decidability; and OWL Full gives maximum expressiveness, however with no computational guarantees, the reasoning process using OWL Full may not be completed in a finite time. Figure \ref{fig:owl-example} shows as example part of an ontology related to bicycle in OWL.

%\begin{figure}[thbp]
%\begin{center}
%\includegraphics[width=0.65\textwidth]{owl-example.png}
%\caption[Part of bicycle ontology represented using OWL]{Part of bicycle ontology represented using OWL (Taken from \cite{isotani2010ontological}).}
%\label{fig:owl-example}
%\end{center}
%\end{figure}

%Since this dissertation will use graphical representation to describe the ontologies, we do not detail the RDF/RDF-Schema and OWL languages. RDF/RDF-Schema and OWL can be automatically generated by graphical ontology editors, such as Prot\'{e}g\'{e} \cite{noy2001creating}, OntoEdit \cite{sure2002ontoedit} and Hozo \cite{kozaki2002hozo}.

%\newpage
\subsubsection{Graphical representation}
\label{subsubsec:graphical-representation}

%Since an ontology is mainly composed by concepts and their relations, the graph is a common representation of ontologies, where the nodes represent concepts and the arrows represent relations between concepts \cite{dieng1998comparison}. Figure \ref{fig:graph-bicycle-ontology} shows an example of an ontology referred to bicycle. In this ontology, the concept of a bicycle is a specialization of vehicle represented using \emph{is-a} relation ($<$bicycle \emph{is-a} vehicle$>$). In ontology engineering, we can say that the class vehicle is a super-class of the class bicycle, and the bicycle is specialized into a sport bicycle and city bicycle. Finally, this Figure shows that the bicycle has attributes: color and weight. And the bicycle using the part-of relation is composed by elements: wheel, handlebar, frame and pedal. In this Figure, the scheme of colors helps the reader identify the relationship between concepts.

%\begin{figure}[thbp]
%\begin{center}
%\includegraphics[width=0.8\textwidth]{graph-bicycle-ontology.png}
%\caption[A simple example of a bicycle ontology]{A simple example of a bicycle ontology (Taken from \cite{isotani2010ontological}).}
%\label{fig:graph-bicycle-ontology}
%\end{center}
%\end{figure}

%Although the representation of ontologies using graphs is the most common, it suffers deficiencies that do not help to capture important elements in an ontology \cite{devedzic2006semantic}, especially when trying to represent roles. Therefore, we decide to use the model of roles propose by \cite{ mizoguchi2007the}, and the Hozo ontology editor \cite{kozaki2002hozo} that is an authoring environment for building ontologies based on the model of roles.

%Hozo editor adequately deals with the differentiating of basic concepts (e.g. human, and artefact) from role concepts (e.g. learner, and reward) based on the model of roles. In this model, to deal with the concept of role, the following three classes are introduced: (a) Role concept - A concept representing a role that depends on a context (e.g. learner role that depends on the school); (b) Basic concept - A concept that does not need other concepts to be defined (e.g. human); and (c) Role holder - An instance of a base concept that is holding the role (e.g. learner). The basic concepts are used as class constraints, and the instances that satisfy the class constraints play the role, becoming role holders. For example as shown in the Figure \ref{fig:learner-role-hozo}, "In a learning environment there is a vacancy for a learner, and a person, whose name is Geiser, fills the position, becoming a learner in the particular environment." The person who plays a role is referred as a role holder. Thus, Geiser becomes a learner in the learning environment by playing the learner role. The top of the figure shows how the concepts around a role are related to each other and in the bottom is shown the representation in Hozo (without the instantiation of the person's name).

%\begin{figure}[thbp]
%\begin{center}
%\includegraphics[width=0.65\textwidth]{learner-role-hozo.png}
%\caption[The learner role represented using Hozo]{The learner role represented using Hozo (Adapted from \cite{isotani2010ontological}).}
%\label{fig:learner-role-hozo}
%\end{center}
%\end{figure}

%Figure \ref{fig:hozo-bicycle-ontology} shows the representation of bicycle ontology using Hozo representation. In this figure, the relations part-of and attribute-of is respectively represented by labels "p/o" and "a/o" that appear in front of each slot. Thus, the frame, pedal, handlebar, and wheels are part of the bicycle. Observe that in the context of bicycle, a wheel (basic concept) can play the role of front wheel or rear wheel (role concepts). Thus, a particular instance of a wheel that plays one of these roles (front wheel role or rear wheel role) is referred to as role holder. In summary, a front wheel is an instance of wheel playing the front wheel role.

%\begin{figure}[thbp]
%\begin{center}
%\includegraphics[width=0.95\textwidth]{hozo-bicycle-ontology.png}
%\caption[Example of bicycle ontology represented using Hozo]{Example of bicycle ontology represented using Hozo (Taken from \cite{isotani2010ontological}).}
%\label{fig:hozo-bicycle-ontology}
%\end{center}
%\end{figure}

\subsection{Ontology Engineering}
\label{subsec:ontology-engineering}

%According to \cite{devedzic2002understanding}, ontology engineering encompasses a set of activities conducted during the conceptualization, design, implementation and deployment of ontologies. It covers topics including philosophy, metaphysics, knowledge representation formalisms, development methodology, knowledge sharing and reuse, knowledge management, business process modeling, common sense knowledge, systematization of domain knowledge, information retrieval from the Internet, standardization, and evaluation.

%The development of ontologies is a time-consuming and difficult task that need knowledge about the target domain, theoretical background on otology research, and the skills to properly define the concepts and develop the body of knowledge. Thus, to facilitate the development of ontologies, there are several methodologies and methods for building ontologies. According to \cite{mizoguchi2004tutorial}, these methods are categorized into a three layer guideline as shown in Figure \ref{fig:three-layer-model}, in which:

%\begin{enumerate}
%\item \textbf{Top-layer} is the coarsest level which specifies the whole building process with standard software development life cycles. The ontological methodologies of top layer have a set of guidelines associated with conventional software development processes and practices.
%\item \textbf{Middle-layer} is the generic constraints and guidelines that specify a set of major steps and their order of execution. In each step of middle layer, the detailed information about the activities to be completed, and the way for each activity should be carried out. 
%\item \textbf{Bottom layer} is the most fine-grain guidelines that enable the construction of concepts hierarchy. It describes guidelines to create explicit semantic structures from identified concepts in the target world.
%\end{enumerate}

%\begin{figure}[thbp]
%\begin{center}
%\includegraphics[width=0.75\textwidth]{three-layer-model.png}
%\caption[Three-layer model of ontology building methodology]{Three-layer model of ontology building methodology (Taken from \cite{mizoguchi2004tutorial}).}
%\label{fig:three-layer-model}
%\end{center}
%\end{figure}

%Most of the existing methodologies are concerned mainly with the top-layer. Some examples are METHONTOLOGY \cite{fernandez1997methontology}, On-To-Knowledge \cite{sure2004knowledge}, and Ushold and King’s methodology \cite{uschold1995towards}. Unfortunately, only a few methodologies deal with the middle and bottom layers.  The main problem of having few methodologies for the development of the middle and bottom layers is the lack of support for novices. Thus, the chances of creating a good ontology at the end of some process decreases condonably without guidelines for ontological development covering all the layers.

%In this sense, \cite{mizoguchi2004tutorial} offers guidelines to support the development of ontologies at the middle and bottom layers using the Activity-First Method for task ontology development \cite{mizoguchi1995ontology}. Thus, in this dissertation, we will utilizes these guidelines to create the ontology OntoGaCLeS. In the rest of this section, we present an overview of guidelines proposed by Mizoguchi in \cite{mizoguchi2004tutorial, mizoguchi2003part, mizoguchi2004part}.

%\subsubsection{Middle Layer Guidelines}
%\label{subsubsec:middle-layer}

%\begin{enumerate}
%\item Identify concepts rather than terms. As ontology is totally independent of terminological problems, one cannot stress the importance of this distinction too much. Since people will be easily trapped by the endless terminological discussion departing from the underlying conceptual structure of the target domain.
%\item Use mixed and flexible strategies of top-down, bottom-up and middle-out. Never stick to only one of the strategies. 
%\item Whenever possible, identify and use top-level ontology in the early phase of the development process to govern the rest of the steps. 
%\item When you deal with a concept, identify its main components, using “part-of” relation as well as its main attributes. You can thus find and extend candidates of concepts to be included in the ontology. 
%\item Definition of axioms should be done after finishing is-a hierarchy building and informal term definition. 
%\item Note that you cannot define any concept completely in theory. Therefore, do not stick to the definition of each term too much. At the best, you only can give necessary conditions of them. Term definition in the early phase can be rough. Detailed definition of a term should be done after you grasp the whole structure of the ontology, that is, after building is-a hierarchy. 
%\item Never try to seriously define a term one by one. Definition of a concept needs sufficient contextual information, which is usually not available in the early phase. Terms are related to each other and could have several meanings, which should be clarified by the context given. 
%\item Arrange and resolve the terminological issues(how to name a concept) at the last step. 

%\item When you find the necessity to define more than one meaning for one term, then you are facing the terminological problem. Each term should correspond to exactly one concept in ontology, since you are not building a dictionary, but a well-organized conceptual structure. Each term is only a label of the concept. You of course can build a dictionary after building ontology.

%\item Put a higher priority on is-a hierarchy construction than term definition. Carefully designed is-a hierarchy gives you a correct context to define a term.
%\item When you get stuck with a term definition, follow either one of the following : 
%\begin{itemize}
%\item Multiple meanings? Then concentrate on meaning one by one. 
%\item Multiple Viewpoints? Make the viewpoint explicit and then try it again 
%\item Check if you are discussing terminology.
%\item Use is-a hierarchy to give enough context. 
%\end{itemize}
%\end{enumerate}

\subsubsection{Bottom Layer Guidelines}
\label{subsubsec:bottom-layer}

%According to \cite{isotani2010ontological}, the following list summarizes the guidelines for the bottom layer:

%\begin{enumerate}
%\item \emph{Identify essential properties} for each concept considered essential in the scope of a given problem. These essential properties facilitate to create more stable concepts and hierarchies during the ontology development process.
%\item \emph{Make correct use of the role-concept} that can be defined as the association of a concept to a particular role within a given context. When developing an ontology, one should carefully distinguish the difference between role-concept, role-holder, and basic concepts. Such differentiation helps to treat multiple meanings as introduced previously in the guidelines for the Middle layer.
%\item \emph{Be careful when using is-a relation} in ontologies is different from the one utilized by object- oriented programming. The is-a relation applies only for classes. Furthermore, for given classes A and B the relation $<$class A is-a class B$>$ is true if and only if the instance set of A is a subset of the instance set of B. Therefore building a relation such as $<$teacher is-a human$>$ is ontologically incorrect, since \emph{teacher} is not an ontologically-valid class because there is no person (no instance of class person) whose intrinsic property is being a teacher. Thus, in ontology, it is inappropriate to model $<$Mizoguchi instance-of Teacher$>$ and $<$Teacher is-a Human$>$ (What happens if Mizoguchi quits his job?).
%\item \emph{Be careful when using part-of relation} such as functional, qualification, spatial and etc. Thus, it needs to be used carefully, especially avoid part-of relation to create class hierarchies such as $<$man part-of human$>$. Such an expression is valid only when you want to deal with man as a subspecies of human.
%\item \emph{Pay attention to the difference between is-a and part-of relations}. Usually, the meaning of is-a and part-of relations is easy to distinguish. The former indicates generic vs. specific relations between classes. The latter is often utilized to specify the composition of a thing (although, part-of can be used in other situations). However, sometimes when developing an ontology, one can encounter difficulties to distinguish their difference. For example, which of the following relation is correct: $<$Dog part-of mammal living in Japan$>$ or $<$Dog is-a mammal living in Japan$>$? People agree that $<$Dog is-a mammal$>$ is correct. However, by adding the \emph{living in Japan} phrase to mammal the distinction is not quite that easy because \emph{mammal living in Japan} seems to represent species, and therefore, $<$\emph{dog-species} part-of  \emph{mammal-species} living in Japan$>$ could be considered. Thus, the distinction between is-a and part-of relation is not always easy.
%\item \emph{Avoid the use of multiple inheritances} Creating multiple inheritances is the source of many problems to keep the consistency of the representation. In particular, propagating the essential properties is a problem since each concept should be recognized and represented by their own essential properties.
%\item \emph{Some boundary between similar concepts can be vague} When developing an ontology some boundaries between similar concepts do not need to be strongly conceptualized. An example is the distinction between the concepts black and white. Since the distinction between them is based on the ambiguous color gray, we cannot give a clear and unique boundary between black and white (although we recognize their existence).
%\item \emph{Create terms if there is no label to represent a concept}. If you cannot find an appropriate term to represent a concept, then you can temporarily label them with some sentences or marks (e.g., concept 1 and concept 2) until you can come up with a good label. When the body of the ontology is good enough to represent the target domain, you can revise the labels of each concept.
%\end{enumerate}








This Chapter presents the formalization of how to deal with the motivation problem caused by the scripted collaboration through the application of concepts identified as relevant in the player types models and needs-based theories of motivation. In the gamification of CL sessions, these concepts are used to solve the context-dependency related to the individual user characteristics, and they provide the essential information to represent the fundamental elements of gamified CL scenarios and the personalization of gamification for each participant in the CL sessions. Since these concepts require an ontology-based formalization of CL scenarios to extend it, and then enable the representation of game elements in them, the chapter starts with the overview of CL ontology (Section 3.1) used by the thesis author as basis to represent gamified CL scenarios. Section 3.2 presents the ontological structures that have been proposed in the \emph{\textbf{Onto}logy to \textbf{Ga}mify \textbf{C}ollaborative \textbf{Le}arning \textbf{S}cenarios} - \textbf{OntoGaCLeS} as a formalization to represent the application of concepts extracted from the player types models and needs-based theories of motivation. To demonstrate the usefulness of this formalization and then to validate it, Section 3.3 show an example of how to build and represent an ontology-based gamification model for CL scenarios using the ontological structures proposed in the Section 3.2. Finally, the Section 3.3 presents the conclusions and remarks of the formalization proposed in this chapter.

Part of the work described in this chapter was published by the thesis author in the following articles:

\begin{itemize}
\item \emph{Towards an Ontology for Gamifying Collaborative Learning Scenarios} \cite{ChallcoMoreiraMizoguchiIsotani2014a},
\item \emph{An Ontology Engineering Approach to Gamify Collaborative Learning Scenarios} \cite{ChallcoMoreiraMizoguchiIsotani2014}, and
\item \emph{Personalization of Gamification in Collaborative Learning Contexts using Ontologies} \cite{ChallcoMoreiraBittencourtMizoguchiIsotani2015}.
\end{itemize}

\section{Overview of Collaborative Learning Ontology}
\label{sec:overview-of-cl-ontology}

The CL ontology has a long history of development through the contributions of many researchers. Initially, the CL ontology was conceived to support the opportunistic group formation \cite{IkedaGoMizoguchi1997} in which the purpose is to identify situations for an individual shifting from individual learning session to CL session. In this first version, the CL ontology was formalized to represent the agreement in negotiation process of the group formation. Such agreement was formalized in the CL ontology as ontological structures to describe the individual and group learning goals. Using these structures, intelligent agents have been developed to help participants to find other group members and to establish a CL session in which them should participate. These agents check the individual and learning goals and learning group goals, and then they initiate a negotiation process to establish an agreement of whom participate in the CL session. This first version of the CL ontology has been demonstrated to be benefit in agent- based systems to form group \cite{InabaOhkuboIkedaMizoguchiToyoda2001, SupnithiInabaIkedaMizoguchi1999}.

In order to provide theoretical and pedagogical justification in the group formation, the CL ontology has been extended to represent CL scenario that compliant with instructional and learning theories \cite{InabaMizoguchi2004,IsotaniMizoguchiIsotaniCapeliIsotanideAlbuquerqueBittencourtJaques2013}. These concepts, such as interaction patterns, group goals, individual goals, CL roles and so on, have been extracted from different theories, and in addition to support the group formation \cite{IsotaniMizoguchi2008}, have been successfully applied in: the modeling of learners' development \cite{InabaIkedaMizoguchi2003} the interaction analysis \cite{InabaOhkuboIkedaMizoguchi2002}, and the design of CL process \cite{IsotaniMizoguchiIsotaniCapeliIsotanideAlbuquerqueBittencourtJaques2013}.

\autoref{fig:concepts-terms-and-relation-in-cl-ontology} shows the terms, concepts and relations defined in the CL ontology. These concepts are defined as follows as:

\begin{description}
 \item[\textbf{I-goal}] is the individual learning goal that represents what the participant in focus (\emph{I}) is expected to acquire, and it is described as a change in his/her learning stage.

 \item[\textbf{I-role}] is the CL role played by the participant in focus (\emph{I}).

 \item[\textbf{You-role}] is the CL role played by the participant (\emph{You}) who is interacting with the participant in focus (\emph{I}).

 \item[\textbf{Y<=I-goal}] is the learning strategy employed by the participant in focus (\emph{I}) to interact with the participant (\emph{You}) in order to achieve his/her individual learning goals (\emph{I-goal}).

 \item[\textbf{W(L)-goal}] is the common learning goal for the group members in the CL scenario.

 \item[\textbf{W(A)-goal}] is the rational arrangement of the group activity used to achieve the common learning goal (\emph{W(L)-goal}) and the individual learning goals (\emph{I-goal}).
\end{description}

\begin{figure}[htb]
 \caption{Concepts, terms and relations defined in the CL Ontology}
 \label{fig:concepts-terms-and-relation-in-cl-ontology}
 \centering
 \includegraphics[width=0.95\textwidth]{images/chap-ontogacles1/concepts-terms-and-relation-in-cl-ontology.png}
 \fdireta{Isotani2009}
\end{figure}

To express the relationship of concepts described above, the CL Ontology employs the ontological structure shown in \autoref{fig:ontological-structure-cl-scenario} to represent CL scenarios. In this ontological structure, a CL scenario is composed by three parts defined as:  the \emph{Group structure benefit} (\emph{W(S)-goal}) to represent the expected benefits of the structured collaboration (i.e. positive interdependence, individual accountability, promotive interactions); the \emph{Learning strategy} (\emph{Y<=I-goal}) to represent the learning strategies employed by the group members in the CL scenario; and (3) the \emph{CL process} to represent the rational arrangement of the group activity (\emph{W(A)-goal}).

\begin{figure}[htb]
 \caption{Ontological structure to represent CL scenarios}
 \label{fig:ontological-structure-cl-scenario}
 \centering
 \includegraphics[width=1\textwidth]{images/chap-ontogacles1/ontological-structure-cl-scenario.png}
 \fdireta{Isotani2009}
\end{figure}

\begin{enumerate} [label=(\alph*)]

\item
The \textbf{Learning strategies} (\emph{Y<=I-goal}) are guidelines that specify how the participants should interact with others to achieve their individual goals. These guidelines help the group members to externalize a desired behavior to play a given CL role more adequately. Therefore, the Learning strategy is represented as an ontological structure composes by: the participant in focus (\emph{I}) who play a CL role (\emph{I-role}), the participant (\emph{You}) who interacts with the participant in focus (\emph{I}) playing a CL role (\emph{You-role}), and the individual learning goals (\emph{I-goal}) that are expected to be achieved by him/her at the end of CL scenario. The desired behaviors to be externalized by the group members  are represented in the ontological structure to represent \emph{Learning strategies} as part of the CL roles (\emph{I-role} and \emph{You-role}) in which these behaviors are expressed as a \emph{behavioral roles} to be played by a participant.

\item
The \textbf{CL role} consists functions, goals, duties and responsibilities that must be taken by the group members to achieve the common and individual learning goals. Thus, the ontological structure to represent a CL role describes: the \emph{necessary condition} and \emph{desired conditions} to play the role, \emph{how to collaborate} for playing the role, and the \emph{benefits for playing the role}. The current \emph{Cognitive/Knowledges states} of group members are used as necessary and desired conditions, the \emph{behaviors} describe \emph{how to collaborate} playing the role, and the expected \emph{benefits for playing the role} are expressed through individual learning goals (\emph{I-goal}).

\item
The \textbf{CL process} is the \emph{rational arrangement of group activity} (\emph{W(A)-goal}) whereby the common and individual learning goals are achieved by the group members. In this arrangement, the \emph{common learning goals} (\emph{W(L)-goal}) is the result of the negotiation process for the group formation, and the \emph{Interaction Pattern} is the sequencing mechanism whereby the participants will achieve their individual learning goals (\emph{I-goal}). This interaction pattern is represented as a set of \emph{necessary} and \emph{desired interactions} in which the interaction for the group members is defined as influential Instructional-Learning events (\emph{Influential I\_L events}).

\item
The \textbf{Influential I\_L event} explicitly represents the interaction among the group members and the benefits from two points of view: from the participants who play the role of instructor, and for the participants who play the role of learner. In the influential I\_L event, a group member performs actions and influences other members in the production of changes in the learning state helping them to achieve their individual learning goals. Therefore, the ontological structure to represent an influential I\_L event is composed by two events: a \emph{learning event} and an \emph{instructional event} in which there are represented the actors as participants of CL scenario playing CL roles and their actions. A participant in these event can act as an \emph{instructor} (participant who does an instructional action) or as a \emph{learner} (participant who does a learning action), and then, through the interactions among instructor, learner and learning objects, the attainment of educational benefits occurs. Thus, the ontological structure to represent an instructional event (\emph{I event}) and a learning event (\emph{L event}) consist in: a learner who plays the roles of \emph{Instructor}, a learner who plays the role of \emph{Learner}, the \emph{Instructional actions}, the \emph{Learning actions}, and the \emph{Benefits for the instructor} and \emph{Benefits for the learner}.
\end{enumerate}

As it was said before, the ontological structures presented in the \autoref{fig:ontological-structure-cl-scenario} are used to represent CL scenarios that compliant with instructional and learning theories. To illustrate this use, \autoref{fig:cognitive-apprenticeship-ontological-structure} shows the representation of CL scenario based on the Cognitive Apprentice theory. According to this theory, the CL activities should incorporate situations that are familiar to those who are using these activities, and these situations must to lead the participants to act and interact acquiring skills in a specific context, and then generalizing these skills to other situations. Therefore, the CL scenarios based on the Cognitive Apprentice theory focuses on supporting a more skilled participant, known as master, to teach a familiar situation for the lesser skilled participants known as apprentices, and then, the lesser skilled participants, known as apprentices, learn by observing the master's behavior and mimic it in other similar situations. Thus, from the viewpoint of the more skilled participant, he/she is supported by the learning strategy \aspas{\emph{learning by guiding}} (a1), his/her role (\emph{I-role}) is the \emph{Master role} with the behavioral role of \emph{Guider}, and his/her individual learning goals are the \emph{development of cognitive} or \emph{meta-cognitive skills} at the levels of \emph{Autonomous stage}. From the viewpoint of a lesser skilled participant, he/she is supported by the learning strategy \aspas{\emph{learning strategy by guiding}} (a2) to interact with the master, his/her role (\emph{I-role}) is the \emph{Apprentice role} with the behavioral role of \emph{Imitator}, and his/her individual goals are the \emph{development of cognitive} and/or \emph{meta-cognitive skills} at the levels of \emph{Cognitive stage} and \emph{Associative stage}.

\begin{figure}[htb]
 \caption{Ontological structure to represent CL scenario based on the Cognitive Apprenticeship theory}
 \label{fig:cognitive-apprenticeship-ontological-structure}
 \centering
 \includegraphics[width=1\textwidth]{images/chap-ontogacles1/cognitive-apprenticeship-ontological-structure.png}
 \fautor
\end{figure}

According to the Cognitive Apprentice theory, the necessary conditions to play the \emph{Master role} (b1) are: \emph{having the knowledge how to use the target cognitive skill}, \emph{having experience in using the target cognitive skill}, and/or \emph{having experience in using the target metacognitive skill}. If an participants adequately plays the role of master, he/she acts \emph{Guiding} to other participant to act or think similarly, and as consequence, he/she is benefited with the \emph{development of cognitive or metacognitive skill} to the \emph{Autonomous stage}. To play the \emph{Apprentice role} (b2), the Cognitive Apprenticeship theory indicates that the participants without any knowledge or experience in how to use the target skill should play the role of apprentice. Thus, there is not necessary conditions in the ontological structure shown in \autoref{fig:cognitive-apprenticeship-ontological-structure} (b2), and the desired conditions are \emph{not having the knowledge how to use target metacognitive or cognitive skill} and/or \emph{not having experience in using the target metacognitive or cognitive skill}. When a participant adequately play the \emph{Apprentice role}, he/she acts as \emph{Imitating} the behavior of the master participant and obtaining the benefits in the \emph{development of metacognitive or cognitive skill} at the levels of \emph{Cognitive stage} or \emph{Associative stage}.

When the two learning strategies, \emph{Learning by Guiding} and \emph{Learning by Apprenticeship}, are simultaneously employed to structure the interactions in the CL scenario, a positive synergy is created producing a \emph{spread of skills} across the group members. This arrangement is formalized in the ontological structure as shown in \autoref{fig:cognitive-apprenticeship-ontological-structure} (c), where the \emph{CL process} is a \emph{Cognitive Apprenticeship type CL session}, the \emph{Common goal} of this session is \emph{Spread of a skill}, and the \emph{Teaching-Learning Process} as an \emph{Interaction Pattern} is the sequencing mechanism defined by the CSCL scripts inspired by the Cognitive Apprenticeship theory shown in \autoref{fig:cognitive-apprenticeship-cscl-script}. To avoid all the details in the formalization of each \emph{Influential I\_L event} defined in the interaction pattern, only the structure to represent interaction \aspas{\emph{Setting up learning context type CA}} is shown with more detail in \autoref{fig:cognitive-apprenticeship-ontological-structure} (d). As shown in this formalization, in Instructional event \aspas{\emph{Giving Information}} (\emph{I event}), the participant who plays the role of \emph{Master} acts as an \emph{Instructor} doing the instructional action \aspas{\emph{Explain}} to achieve benefits in the \emph{Development of his/her metacognitive skill} at the level of \emph{Autonomous stage}. In the Learning event \aspas{\emph{Receiving information}} (\emph{L event}), the participant who acts playing the role of \emph{Learner} performs the learning action \aspas{\emph{Identify the context}} to obtain the benefits on the \emph{development of cognitive skill} at the level of \emph{Rough-Cognitive stage}.


\section{Modeling Gamified CL Scenarios}
\label{sec:modeling-gamified-cl-scenarios}

The concepts, terms and relations shown in \autoref{fig:concepts-terms-and-relation-in-gamified-cl-scenarios} have been formalized in the ontology OntoGaCLeS to represent a gamified CL scenario. These elements are independent from any theory and practice, and they are described as follows as:

\begin{description}
\item[\textbf{Y<=I-mot goal}]
is the \emph{individual motivational strategy} used to enhance the learning strategy (\emph{Y<=I-goal}) employed by a participant in focus (\emph{I}).

\item[\textbf{I-mot goal}]
is the \emph{individual motivational goal} for a participant in focus (\emph{I}), and it represents what is expected to happen in his/her motivational stage when an individual motivational strategy (\emph{Y<=I-mot goal}) is applied in the CL scenario to enhance the learning strategy (\emph{Y<=I-goal}) employed by the group members.

\item[\textbf{I-player role}]
is the \emph{player role} for a participant in focus (\emph{I}).

\item[\textbf{You-player role}]
is the \emph{player role} for a participant (\emph{You}) who interacts with the participant in focus (\emph{I}).

\item[\textbf{I-gameplay}]
is the \emph{individual gameplay strategy} for the participant in focus (\emph{I}), and it defines the implementation of individual motivational strategy (\emph{Y<=I-mot goal}) when this strategy is extracted from game design model or gamification models.
\end{description}

\begin{figure}[htb]
 \caption{Concepts, terms and relations defined in the ontology to represent gamified CL scenarios}
 \label{fig:concepts-terms-and-relation-in-gamified-cl-scenarios}
 \centering
 \includegraphics[width=1\textwidth]{images/chap-ontogacles1/concepts-terms-and-relation-in-gamified-cl-scenarios.png}
 \fautor
\end{figure}

In the following subsection, a detailed description related to the formalization of the concepts, terms and relation briefly described above are detailed.

\subsection{Individual Motivational Goal (I-mot goal)}
\label{subsec:individual-motivational-goal}

The \emph{individual motivational goal} (\emph{I-mot goal}) has been formalized in the ontology OntoGaCLeS to represent the reason why is necessary to apply an individual motivational strategy in a CL scenario. Thus, for a participant (\emph{I}) of CL scenario, the individual motivational goal (\emph{I-mot goal}) represents what it is expected to happen in his/her motivational stage when a motivational strategy is applied in the CL scenario to enhance the learning strategy employed by him/her to interact with others. In this sense, the individual motivational goal describes the motivational stages that must be reached by a person to be motivated to interact with other group members.

\autoref{fig:ontological-structure-i-mot-goal} shows the ontological structure that has been formalized in the ontology OntoGaCLeS to represent an individual motivational goal (\emph{I-mot goal}), where: the \emph{initial stage} and \emph{goal stage} are stages used to represent the expected change in the motivational stage of the person in focus (\emph{I}).

\begin{figure}[htb]
 \caption{Ontological structures to represent individual motivational goal (\emph{I-mot goal}). At the bottom, the \aspas{\emph{Satisfaction of psychological need}} (left) and the \aspas{\emph{Internalization of motivation}} (right)}
 \label{fig:ontological-structure-i-mot-goal}
 \centering
 \includegraphics[width=1\textwidth]{images/chap-ontogacles1/ontological-structure-i-mot-goal.png}
 \fautor
\end{figure}

Two types of individual motivational goals have been currently formalized in the ontology OntoGaCLeS to describe the individual motivational goals (\emph{I-mot goal}) related to gamification as individual motivational strategy. The former know as \emph{Satisfaction of psychological needs} has been formalized based on the conceptualization of motivation as internal psychological process to satisfy human needs \cite{PritchardAshwood2008}, and the latter know as \emph{Internalization of motivation} has been formalized based on the form in which an individual regulates his own choices to behave and act \cite{DeciRyan2010}. \autoref{fig:ontological-structure-i-mot-goal} shows the current representation of these two type of individual motivational goals. The initial and goal stages related to the \emph{Internalization of motivation} are defined by the self-determination stage, whereas the initial and goal stages for the \emph{Satisfaction of psychological need} are defined by the \emph{psychological need stages}. In the articles  \cite{ChallcoMoreiraBittencourtMizoguchiIsotani2015, ChallcoMoreiraMizoguchiIsotani2014, ChallcoMoreiraMizoguchiIsotani2014a}, the author of this thesis has been used the term \aspas{\emph{Phychological need}} to refer the term \aspas{\emph{Psychological need stage}} used here, and the term \aspas{\emph{Without need}} to refer the group of terms described here as \aspas{\emph{\$1 need satisfied}} where \$1 is substitute for a psychological need (e.g. \emph{Mastery need satisfied}, \emph{Autonomy need satisfied}).

As it was mentioned before in the \autoref{sec:motivation-problem}, motivation is an internal psychological process associated with the three general components of arousal, direction and intensity dimension in which the arousal component is caused by needs (also called \emph{wants} or \emph{desires}). Such needs cause that a person behaves and acts to satisfy the needs \cite{MitchellDaniels2003}. Consequently, motivation describes how a person chooses to allocate time and energy to different behavior and actions to maximize the satisfaction of his own needs \cite{PritchardAshwood2008}. It means that, in a CL session, the motivation problem for the participant in focus (\emph{I}) occurs when he/she believes that the scripted collaboration will not lead him/her to satisfy his/her individual needs. Hence, the motivational strategy is introduced in the CL scenario to change this perception. Thus, the individual motivational goals (\emph{I-mot goal}) for the person in focus (\emph{I}) are the satisfaction of needs. More specifically, for the gamification of CL scenarios, the individual motivational goal is the \emph{Satisfaction of psychological needs} because game elements do not satisfy all human needs, they can only satisfy part of these needs that are referred by the thesis author as \emph{psychological needs}. The psychological needs are the human needs classified in the groups of relatedness and growth needs according to the ERG (Existence, Relatedness and Growth) theory \cite{Alderfer1972}.

\autoref{fig:ontological-structure-satisfaction-psychological-need} shows the ontological structures that had been formalized in the ontology OntoGaCLeS to represent the \emph{Satisfaction of psychological need} related to gamification as an individual motivational strategy. These ontological structures represent the satisfaction of innate psychological needs, and they comprise what is intended to evoke in minds of players by the majority of experts in gamification when a CL scenario is been gamified \cite{MoraRieraGonzalezArnedo-Moreno2015, SeabornFels2015}. According to the SDT theory, the well-being of an individual is reached when the psychological needs of autonomy, competence and relatedness are satisfied by performing a behavior \cite{DeciRyan1985, DeciRyan2010}, and according to the Dan Pink's theory \cite{Pink2011}, a person is motivate and engage in a cognitive, decision-making, creative or higher-order thinking task whether it is given with autonomy, mastery and purpose.

\begin{figure}[htb]
 \caption[Ontological structure to represent Satisfaction of psychological need]{Ontological structures to represent \aspas{\emph{Satisfaction of psychological need}} for gamification as motivational strategy. At the top right, the ontological structure to represent \aspas{\emph{Satisfaction of autonomy}.}}
 \label{fig:ontological-structure-satisfaction-psychological-need}
 \centering
 \includegraphics[width=1\textwidth]{images/chap-ontogacles1/ontological-structure-satisfaction-psychological-need.png}
 \fautor
\end{figure}

At the top right of \autoref{fig:ontological-structure-satisfaction-psychological-need}, the ontological structure to represent the \emph{Satisfaction of autonomy} is detailed in which, based on a unipolar scale of satisfaction from unsatisfied to satisfied needs, the roles of initial and goal stages are played by the \emph{Autonomy unsatisfied} and the \emph{Autonomy satisfied}, respectively. Employing the same unipolar scale, the satisfaction of four psychological needs have been formalized in the ontology OntoGaCLeS, and they are detailed in the \autoref{appendix:i-mot-goal}.

The \emph{internalization of motivation} is the process by which \aspas{\emph{values, attitudes, or regulatory structures, such that the external regulation of a behavior is transformed into an internal regulation, so no longer requires the presence of an external contingency}} \cite{GagneDeci2005}. In this sense, the internalization of motivation in relation to the satisfaction of needs refers to the change from a non-free choice to a free choice of needs that will be satisfied by oneself. According to the SDT theory \cite{DeciRyan1985, RyanDeci2000}, this change happens from the extrinsic motivation to intrinsic motivation when motivation is changed from a non-self-determined form (\emph{non-freely choice}) to a self-determined form (\emph{freely choice by oneself}). Here, instead to consider only as positive the creation of intrinsic motivation while the extrinsic motivation is treated as undesirable when a CL session is being gamified, the extrinsic motivators employed by the game elements must be used as an attempt to transform the current stage of students' motivation from amotivation and extrinsic motivation into intrinsic motivation.

\autoref{fig:ontological-structure-internalization-motivation} shows the ontological structures formalized in the ontology OntoGaCLeS to represent the \emph{Internalization of motivation} as individual motivational goal for a person in focus (\emph{I}) when gamification is applied as an individual motivational strategy. According to the SDT theory \cite{DeciRyan2010, DeciVansteenkiste2004}, this internalization happens on a continuum ranging of stages from \emph{amotivation} (not internalized behave) into \emph{external motivation} (not at all internalized behave) to \emph{introjected motivation} (partially internalized behave) to \emph{identified motivation} (fully internalized behave) to \emph{intrinsic motivation} (automatically internalized behave. Thus, this range of \emph{motivation stages} has been used to represent the initial and goal stages in the ontological structures to represent the internalizations of motivation as shown in \autoref{fig:ontological-structure-internalization-motivation}.

\begin{figure}[htb]
 \caption[Ontological structures to represent Internalization of motivation]{Ontological structures to represent \aspas{\emph{Internalization of motivation}} as individual motivational goal. At the top right, the structure to represent \aspas{\emph{Internalization from amotivation to intrinsic motion}.}}
 \label{fig:ontological-structure-internalization-motivation}
 \centering
 \includegraphics[width=1\textwidth]{images/chap-ontogacles1/ontological-structure-internalization-motivation.png}
 \fautor
\end{figure}

In the gamification of CL scenarios, the goal stage is always the \emph{intrinsic motivation stage} because the game elements are introduced in the CL scenario as an attempt to transform the current motivation stage, from amotivation or extrinsic motivation stage into intrinsic motivation stage (\emph{Internalization to intrinsic motivation}). Thus, for example, in the formalization of \emph{Internalization from amotivation to instrinsic motivation}, the initial stage is \emph{Amotivation stage} and the goal stage is \emph{Intrinsic motivation stage} as shown at the top right of \autoref{fig:ontological-structure-internalization-motivation}. Based on the range of motivation stage described by the SDT theory, we have formalized sixteen internalizations of motivation in the ontology OntoGaCLeS that are detailed in \autoref{appendix:i-mot-goal}.


\subsection{Player Role}
\label{subsec:player-role}

The identification of homogeneous people groups that differ from other groups in a significant way is essential to define the personalization in any system. In game design, this segmentation of the people is established by player types models in which player typologies categorize the players in different groups according to their geographic location \cite{BenJuddChrisAvelloneHideoKojimaKeijiInafune2016, ChakrabortyNorcioVeerAndreMillerRegelsberger2015}, their demographic situation \cite{GreenbergSherryLachlanLucasHolmstrom2010, Shaw2012}, their psychographic characteristic \cite{Tseng2011, Yee2006}, and their behavioral characteristics \cite{Bartle2004, Lazzaro2009}. These player type models aim to help the game development teams to identify the features that make a game fun, enjoyable and/or desirable for a particular audience. However, the player type models cannot be extrapolated to others types of games for which they were not intended, and the information provided by them cannot be directly used in the gamification of CL scenarios. Thus, the concept of \emph{Player role} has been defined in the formalization of gamified CL scenarios to describe player types.

The \emph{Player role} is the role that describes the functionality, responsibilities and requirements whereby a particular group of participants becomes players in a gamified CL scenario. This segmentation is based on individual characteristics that establish a segmentation of participants using necessary and desired conditions. In this sense, the \emph{Player role} has been formalized in the ontology OntoGaCLeS as the ontological structure shown in \autoref{fig:ontological-structure-player-role}. This structure defines the conditions that must be satisfied by a participant in the CL scenario to play the player role as: \emph{necessary condition} and \emph{desired condition}. Thus, a participant of CL scenario cannot play a player role if he/she does not fulfill the necessary conditions, and when the participant fulfills the necessary and desired conditions has more probability to obtain the expected \emph{benefits for playing the role}.

\begin{figure}[htb]
 \caption[Ontological structures to represent Player role]{Ontological structures to represent \aspas{\emph{Player role}} (At the top). At the bottom, the ontological structure to represent the player role of \aspas{\emph{Dreamer role}.}}
 \label{fig:ontological-structure-player-role}
 \centering
 \includegraphics[width=1\textwidth]{images/chap-ontogacles1/ontological-structure-player-role.png}
 \fautor
\end{figure}

The necessary and desire conditions of a \emph{Player role} is represented using: motivation states, psychological need states, and individual personality trait states. \autoref{appendix:tree-overview-states} shows a tree overview for these states that have been currently formalized in the ontology OntoGaCLeS to represent the conditions in the \emph{Player roles}.

\begin{itemize}
\item
The \emph{Motivation state} is an internal state that describes the temporal attitudinal state of a person in relation to his/her desire to be a participant in the CL session. These stages can be \emph{Not motivated} and \emph{Motivated}. The state of motivated is also divided in two types: \aspas{\emph{Intrinsic motivated}} and \aspas{\emph{Extrinsic motivated}} \cite{DeciRyan2010}. It is important to notice here that the motivation state is not as same as motivation stage. Although both concepts represent changes in relation to the motivation, the motivation state represents a specific part of the whole process of being motivated, whereas the motivation stage represents an interval in the whole process of being motivated.

\item
The \emph{Psychological need state} is used to represent the current psychological needs of a person in which the states for each one of the psychological needs are formalized through the representation of pair states: \aspas{\emph{Having need of \$1}} and \aspas{\emph{Not having need of \$1}} where \aspas{\emph{\$1}} is replaced by the name of the need that is being described as prerequisite. For instance, to represent the states related to the psychological need of competence, the states \aspas{\emph{Having need of competence}} and \aspas{\emph{Not having need of competence}} have been formalized in the ontology OntoGaCLeS.

\item
The \emph{Ind, personality trait state} describes states related to individual personality traits, such as introversion, extraversion, openness to experience, and conscientiousness. The individual personality trait states describe the characteristic that make a person unique by indicating his habitual patterns of thought, emotion and behavior for different situations \cite{MatthewsDearyWhiteman2003}. These states express whether a participant either has or does not have the individual personality trait. These states also express the personality traits in a rating of dimensions, e.g. extraversion vs introversion. Currently, the ontology OntoGaCLeS represents the individual personality traits states related to: the big five personality traits \cite{CostaMacCrae1992}, the MBTI personality traits \cite{Briggs1976}, the game-playing style preferences described in the Bartle's player type model \cite{Bartle2004}, and the game-playing liking preferences described in the Yee's motivation components \cite{Yee2006} (\autoref{ appendix:tree-overview-states}).
\end{itemize}

The ontological structure to represent \emph{Player role} shown in \autoref{fig:ontological-structure-player-role} also includes the information about: how the participant with the player role is expected to interact with the game elements (\emph{how to interact}), and the expected benefits for playing the player role (\emph{benefits for playing the role}). The \emph{behavior} indicates the possible manners in which a participant should interact playing the player role to attain the expected \emph{benefits for playing the role}. These benefits are represented as \emph{individual motivational goals} (\emph{I-mot goal}).

At the bottom of \autoref{fig:ontological-structure-player-role}, the \emph{Creator role} is shown as example of the formalization of a player role using the ontological structures proposed in this section. According to this structure, participants who have a greater liking for customization-components rather than for other game components are classified as creators. This segmentation is represented by the necessary condition of \emph{having a non-negative liking for customization-components}, and the desired conditions of \emph{having a positive liking for customization-components}, \emph{having a non-positive liking for achievement-component}, \emph{having a negative liking for achievement-component}, \emph{having a non-positive liking for social-component}, and \emph{having a negative liking for social-component}. The desired conditions related to the behavioral characteristics are: \emph{having preference for interacting on the system} and \emph{having need of autonomy}. The expected behaviors to obtain the benefits for playing the creator role are: \emph{Creating}, \emph{Tweaking}, \emph{Building}, \emph{Customizing}, \emph{Transforming}, \emph{Adapting}, \emph{Inventing} or \emph{Crafting}. As consequence of this behave, it is expected that the participants attain the \emph{Satisfaction of autonomy} and the \emph{Internalization to intrinsic motivation} (\emph{I-mot goal}).

In the ontology OntoGaCLeS, based on the information extracted from five different player type models, twenty-six players roles have been formalized and represented using the ontological structure proposed in this section. These player roles, their conditions, expected behaviors and benefits for the person who plays the role are detailed in the \autoref{appendix:player-role}.

\subsection{Individual Motivational Strategy (Y<=I-mot goal)}

In the context of CL scenarios, an \emph{individual motivational strategy} is defined by the thesis author as a set of guidelines defined to motivate a participant to interact with other group member using a learning strategy. These guidelines are independent of any technology, so that the individual motivational strategy describes what motivates a participant to act and behave in certain way. For example, consider the following guidelines extracted from the Model-driven Persuasive Game in which:

\begin{citation}
\aspas{cooperation is only a significant motivator of behaviour change for achievers and socializers, with ($\beta$=.15) and ($\beta$=.17) respectively. This is in line with the gaming style of socializers, who enjoy helping others. Achievers would also prefer to cooperate because they are inherently more altruistic ... achievers do often co-operate with one another, usually to perform some difficult collective goal, and from these shared experiences can grow deep, enduring friendships which may surpass in intensity those commonly found among individuals other groups.} \citeonline{Orji2014}.
\end{citation}

When these two guidelines are applied in a CL scenario by providing a situation in which the participants must cooperate to achieve a group goal (e.g. obtain a especial reward based on the collective performance of group members), these guidelines becomes a individual motivational strategy that could be applied to motivate participant who fall in the category of socializer or achiever because they are motivated by the desired to accomplish the group goal and the desired to help others, respectively.

For the formalization of individual motivational strategies whose guidelines are extracted from game design models or gamification models, the ontological structure shown in \autoref{fig:ontological-structure-individual-motivational-strategy} has been proposed by the thesis author. According to this structure, an \emph{individual motivational strategy} (\emph{Y<=I-mot goal}) describes:

\begin{description}
\item[\textbf{I-player role}]
as the participant in focus (\emph{I}) as a \emph{player role holder} who is motivated by the motivational strategy.

\item[\textbf{You-player role}]


\item[\textbf{I-mot goal (I)}]

\end{description}

the participant are \emph{player roles}, and the reasons why the participants are motivated by this strategy are expressed as \emph{individual motivational goal}. According to this structure, an \emph{individual motivational strategy} (\emph{Y<=I-mot goal}) describes the following elements:

\begin{description}
\item[\textbf{I-player role}] 
to indicate the \emph{player role holder} for the participant in focus (\emph{I}), and the \emph{behavioral roles} whereby this participant is motivated to interact with other participant (\emph{You}) using a learning strategy (\emph{Y<=I-goal}).


\item[\textbf{You-player role}]


\item[\textbf{I-mot goal}]

\end{description} 

ontology OntoGaCLeS


 that is extracted from game design model or gamification models, their 



 cooperative situation in which a participant who plays the socializer or achiever role must to help others, and the achiever 





 that motivates a participant in focus (\emph{I}) to use the learning strategy (\emph{Y<=I-goal}) during his/her interaction with other group member (\emph{You}), this guideline becomes an individual motivational strategy.



As is shown in the example , these guidelines are extracted from game design or gamification models, they consist in the description about how players types are motivated to perform behaviors in a specific situation, and the reason why they are motivated to perform these behaviors. For example, in 


In a CL scenario, for the participant in focus (\emph{I}) who interacts with other participant (\emph{You}) employing a learning strategy (\emph{Y<=I-goal}), an \emph{Individual motivational strategy} (\emph{Y<=I-mot goal}) is defined as the strategy that, independently of any technology, causes effects on motivation to interact with other group members using the learning strategy (\emph{Y<=I-goal}).


As we see in  example presented above, the individual motivational strategies extracted from game design models consist in a 

 In the first example, the behavior to motivate socializers is helping others in a cooperative situation because they like to help others. In the second example, the guidelines indicates that, in cooperative situation, achievers are motivated to collaborate with other participant when there is a difficult collective goal to be achieved. Thus, the ontological structure that has been formalized in the ontology OntoGaCLeS to represent individual motivational strategies extracted from game design models and gamification models is shown in \autoref{fig:ontological-structure-individual-motivation-strategy}, and it consists in:

\begin{itemize}
\item


\item
The player role (\emph{You-player role}) for the participant (\emph{You}) who interacts with the participant in focus (\emph{I}), and the \emph{behavioral roles} through which this participant is motivate to support the interaction of participant in focus (\emph{I}) using a learning strategy (\emph{Y<=I-goal}).

\item
The individual motivational goals (\emph{I-mot goal (I)}) whereby the participant in focus (\emph{I}) is motivated to interact with the other participant (\emph{You}) employing a learning strategy (\emph{Y<=I-goal}). In this sense, the individual motivational goals “\emph{I-mot goal (I)}” describe the reasons why the guidelines from a game design model or/and gamification model are applied as individual motivation strategy to enhance the learning strategy (\emph{Y<=I-goal}) employed by the participant in focus (\emph{I}) to interact with other participant (\emph{You}).
\end{itemize}

\begin{figure}[htb]
 \caption[Ontological structure to represent Individual motivational strategy]{Ontological structure to represent \aspas{\emph{Individual motivational strategy}} (at the left). At the right-top, the motivational strategy \aspas{\emph{Gamifying for Consumer and Dodecad Achiever}.} At the right-bottom, the motivational strategy \aspas{\emph{Gamifying by COOP}.}}
 \label{fig:ontological-structure-individual-motivational-strategy}
 \centering
 \includegraphics[width=1\textwidth]{images/chap-ontogacles1/ontological-structure-individual-motivational-strategy.png}
 \fautor
\end{figure}

To exemplify the formalization of the individual motivational strategies using the ontological structure proposed in this section, \autoref{fig:ontological-structure-individual-motivational-strategy} shows two examples in which the attribute \aspas{\emph{based on}} indicates the game design models and/or gamification model in which the motivational strategy (\emph{Y<=I-mot goal}) is based. The individual motivational strategy shown at the top-right of \autoref{fig:ontological-structure-individual-motivational-strategy} is known as \emph{Gamifying for Consumer and Dodecad Achiever}, and it has been formalized based on the guidelines of Dodecad model and 5 Groups of fun framework proposed by \citeonline{Marczewski2015a, 2015b}. According to these guidelines, the Consumers and Achievers are motivated by the need to obtain rewards that demonstrate for others their accomplishments in the domain of something. Hence, the behavioral roles of \emph{Accomplisher} and \emph{Social-comparer} are roles whereby a participant in focus (\emph{I}) playing the \emph{Consumer role} is motivated to interact with other participant (\emph{You}) who plays the \emph{Achieve role}. The \emph{Satisfaction of mastery} and the \emph{Internalization from extrinsic to intrinsic motivation} have been formalized as individual motivational goals for this individual motivational strategy because a consumer has the desired to demonstrate his/her mastery, and because the motivational strategy applied is in the CL scenario to turn the current extrinsic motivation of consumer into intrinsic motivation.

At the bottom-right of \autoref{fig:ontological-structure-individual-motivational-stratagy}, it is shown the ontological structure to represent the motivational strategy \aspas{\emph{Gamifying by COOP}.} This formalization has been based on the guidelines of Model-driven persuasive game \cite{OrjiVassilevaMandryk2014} in which, the cooperation is indicated as a significant motivator for socializers and achievers who enjoy helping others and cooperate with others in order to accomplish a difficult collective goal. Thereby, the behavioral roles for the participant in focus (\emph{I}) who play the \emph{BrainHex Socializer role} or \emph{Brainhex Achiever role} are either the \emph{Helper} and the \emph{Accomplisher}. Thus, the participant in focus (\emph{I}) will be to motivated to interact with other participant (\emph{You}) by his/her desire to accomplish the \emph{Satisfaction of relatedness} or the \emph{Satisfaction of competence}. As consequence of the cooperation, it is expected changes in the motivational state of participant in focus (\emph{I}) from the amotivation or extrinsic motivated state to the intrinsic motivated state (\emph{Internalization to intrinsic motivation}).

\autoref{appendix:ind-motivational-strategy} shows the individual motivational strategies for gamification currently defined in the ontology OntoGaCLeS, their player roles, their behavioral roles, and their individual motivational goals.


\chapter[Ontological Structures of Persuasive Game Design in CL Scenarios]{Ontological Structures of Persuasive Game Design in Collaborative Learning Scenarios}
\label{chapter:ontogacles-2}

In the previous chapter, ontological structures have been formalized in the ontology OntoGaCLeS to represent the personalization of gamification in CL scenarios based on player type models. These ontological structures have been proposed to support the definition of player roles and the selection of game elements for each participant in a CL scenario. However, to deal with the motivation problem caused by the scripted collaboration, it is also necessary to provide support for the design of CL gameplay. This design consists into setting up the selected game elements to persuade the participants to follow the interactions defined by a CSCL script in which the CL process of CL scenario has been based. To accomplish this, gamification as Persuasive Game Design (PGD) should be linked to the design of CL process.

This chapter present the ontological structures proposed by the author of this PhD thesis dissertation to represent the connection between PGD and the design of CL process in CL scenarios. This connection intends to solve the context-dependency of gamification related to the non-game context and target behaviors being gamified with the purpose to deal with the motivation problem caused by the scripted collaboration. Thus, the first section (\autoref{sec:modeling-game-non-game-worlds}) presents a nested-structure proposed to identify things that belong to the game world and non-game world. Having this clearly separation, the formalization of PGD as ontological structures is presented in \autoref{sec:modeling-persuasive-game-design-gamification-cl-scenarios}. Then, the ontological structures proposed to represent the CL gameplay based on PGD are presented in \autoref{sec:modeling-cl-gameplay-persuasive-game-design}. To demonstrate the usefulness of these ontological structures, \autoref{sec:formalizing-ontological-model-apply-gamification-persuasive-technology} shows the formalization of an ontological model to apply gamification as persuasive technology in Cognitive Apprenticeship scenarios. Finally, \autoref{sec:ontogacles2-concluding-remarks} presents the concluding remarks of this chapter.
 
Part of the work described in this chapter was published by the author of this PhD thesis dissertation in the scientific articles:

\begin{itemize}
\item
\aspas{\emph{Steps Towards the Gamification of Collaborative Learning Scenarios Supported by Ontologies}} published in the 17\textsuperscript{th} International Conference on Artificial Intelligence in Education, AIED 2015, held in Madrid, Spain \cite{ChallcoMizoguchiBittencourtIsotani2015a}.

\item
\aspas{\emph{An Ontological Model to Apply Gamification as Persuasive Technology in Collaborative Learning Scenarios}} published in the 26\textsuperscript{th} Brazilian Symposium on Computer in Education, SBIE 2015, held in Maceió, AL, Brazil \cite{ChallcoAndradeOliveiraMizoguchiIsotani2015}.

\item
\aspas{\emph{Gamification of Collaborative Learning Scenarios: Structuring Persuasive Strategies Using Game Elements and Ontologies}} published in the 1\textsuperscript{st} International Workshop on Social Computing in Digital Education, SocialEdu 2015, held in Stanford, CA, USA \cite{ChallcoMizoguchiBittencourtIsotani2015}.

\item
\aspas{\emph{An Ontology Framework to Apply Gamification in CSCL Scenarios as Persuasive Technology}} published as Volume 24, Issue 2, in the Brazilian Journal of Computers in Education - RBIE, 2016 \cite{ChallcoMizoguchiIsotani2016}.
\end{itemize}

%%%%%%%%%%%%%%%%%%%%%%%%%%%%%%%%%%%%%%%%%%%%%%%%%%
\section{Modeling Game and Non-game Worlds}
\label{sec:modeling-game-non-game-worlds}

One of the main difficulties to formally represent the gamification in a computer understandable manner is the lack of a clearly separation between game world and non-game world. As was mentioned at the \autoref{chapter:general-background}, a game is a problem-solving activity approached with playful attitude\footnote{A gameful attitude is defined here as a playful attitude in which the intrinsic motivation is a necessary condition to achieve this attitude, but the immersion and enjoyment are desirable conditions}\cite{Schell2008}, and a non-game context is being gamified with the intention to make it more game-like \cite{Werbach2014}. Therefore, to make the interactions defined by a CSCL script more game-liking in a gamified CL scenario, the gamification process consists into add game elements in the environment in which the actions of participants will take place, and to define how these game elements will interact with the participants during the CL process. In this gamification process, gamification models and/or frameworks are used to explain the game design process using a theoretical foundation in game design models. The game elements and their interaction with the students produce and/or induce changes in the psychological state of participants, and these changes are theoretically justified through theories/models of motivation and human behavior.

Based on this description of gamification process, a nested-structure sees adequate to enable a systematic separation of things between  the game world and non-game world.

the interactions between the participants and game elements are formalized as events, 

 \autoref{fig:nested-structure-game-nongame-worlds} shows this model in which the nested-structure classify the events in two types: the non-game events and the game events. The non-game events describe the activities/actions in the CL process that have the potential to be gamified, and the game events describe the activities/action of game elements to make the activities/actions described in the non-game events more game-like. The theoretical justification in this nested-structure for the gamification process are given as follows: \emph{gamification models and/or frameworks} explain the \emph{game design process} used to introduce and to define \emph{game events} whereby the non-game situation becomes more game-like; the reasons why these \emph{game events} had been introduced in the non-game situation is explained by \emph{game design models}; and the changes produced and/or induced by the game events are explained by \emph{theories/models of motivation and human behavior}. 

\begin{figure}[!htb]
 \caption{Nested structure of non-game world, game world and gamification world}
 \label{fig:nested-structure-game-nongame-worlds}
 \centering
 \includegraphics[width=0.55\textwidth]{images/chap-ontogacles2/nested-structure-game-nongame-worlds.png}
 \fautor
\end{figure}

Employing the nested-structure of non-game world, game world and gamification world (\autoref{fig:nested-structure-game-nongame-worlds}), the concepts in the ontology OntoGaCLeS related to the game events and non-game events have been classified in the \aspas{\emph{is-a}} hierarchy structure of class shown in \autoref{fig:is-a-hierarchy-structure-of-classes}. This structure categorizes any concept of ontology as a sub-type of classes: \emph{Gamification world}, \emph{Game world}, Non-game world, Common world, and Theory/Model. The classes defined under the categories of common, non-game, game and gamification worlds are concepts for things in their respective worlds, and the concepts formalized as sub-type of \emph{Theory/Model} define the theoretical foundation and justification of gamification, and game design.

\begin{figure}[!htb]
 \caption{\aspas{\emph{is-a}} hierarchy structure of classes to represent concepts in the ontology OntoGaCLeS}
 \label{fig:is-a-hierarchy-structure-of-classes}
 \centering
 \includegraphics[width=0.45\textwidth]{images/chap-ontogacles2/is-a-hierarchy-structure-of-classes.png}
 \fautor
\end{figure}
\newpage

\emph{Gamification world} is the class of all things that depend of the gamification world to exist. In this sense, a concept is formalized as sub-type of \emph{Gamification world} whether it represents something that needs of gamification world to be described. For instance, the \emph{Gamification goal/purpose} is a concept formalized as sub-type of Gamification world to describe the goals and/or purposes of a gamification model and/or framework (e.g. \emph{avoiding dropout}, \emph{reducing weariness}). The basic concepts defined as sub-types of \emph{Gamification world} for the gamification of CL scenarios are: \emph{Gamified CL session}, \emph{Motivational strategy}  (\emph{Y<=I-mot goal}) by gamification, \emph{Player role}, and \emph{Individual gameplay strategy} (\emph{I-gameplay strategy}).

\emph{Game world} is the class of all things that depend of the game world to exist. Concepts formalized as sub-types of \emph{Game world} require only elements defined in the games to be described. The basic concept defined as sub-types of \emph{Game world} to gamify CL scenarios is: \emph{Game element}. \emph{Non-game world} is the class of all things that does not need concepts from the \emph{Gamification world} or \emph{Game world} to exist. The non-game world is divided in the sub-types: \emph{Learning world}, \emph{Instructional world}, \emph{World of cognition}, \emph{ID-ISD world}, \emph{CL world}, and \emph{World of motivation and human behavior}. Basic concepts defined as one of these world only need things from its respectively world to exist. Thus, for instance, the concepts formalized as sub-type of \emph{World of motivation and human behavior} represent things that only need elements from motivation and human behavior to exist, so that the basic concepts related to the gamification of CL scenarios formalized as sub-types of \emph{World of motivation and human behavior} are: Individual motivational goal (\emph{I-mot goal}), \emph{Motivation stage}, and \emph{Human need stage}.

\emph{Common world} is the class of anything used to represent things that require concepts of other worlds to be formalized. These concepts are common to the other worlds, and they have been taxonomically classified taking as base the classification defined in the upper-level ontology \textbf{YAMATO} – \emph{\textbf{Y}et \textbf{A}nother \textbf{M}ore \textbf{A}dvanced \textbf{T}op-level \textbf{O}ntology} \cite{Mizoguchi2010}. The basic concepts in the \emph{Common world} to represent persuasive game design are the concepts of: (i) \emph{action}, (ii) \emph{entity} (e.g. \emph{object}, \emph{agent}), (iii) \emph{state}, and (iv) \emph{event}. These concepts, their sub-types, and their ontological structures have been formalized following the formalization proposed by Galton and Mizoguchi in the article \aspas{\emph{The Water Falls but the Waterfall Does Not Fall: New Perspectives on Objects, Processes and Events}} \cite{GaltonMizoguchi2009}. According to these definitions, there is a mutual dependency between processes and entities whereby no one process (\emph{action}) can exist without an entity (\emph{agent} or \emph{object}) to enact it, and an entity is what it is as consequence of its processes. Therefore, an entity has properties knows as \emph{states} that change over time when processes are enacted by the object. An \emph{event} is then defined as integration of entities, actions, and states in a particular context to describe a fixed chunk of any process in which the participants of process are the agents and objects.

\begin{figure}[!htb]
 \caption{Ontological structures to represent events}
 \label{fig:ontological-structures-event}
 \centering
 \includegraphics[width=1\textwidth]{images/chap-ontogacles2/ontological-structures-event.png}
 \fautor
\end{figure}

\autoref{fig:ontological-structures-event} shows the formalization of events as ontological structures in the ontology OntoGaCLeS. As it shown in this formalization, the class event is classified in \emph{ordinal event} and \emph{instant event} in which the ordinal event is constituted by a process (e.g. \emph{action}, \emph{behavior}), the participants in the events are entities, and the ordinal event has instantaneous events as starting and ending event to delimit the chunk of processes that compose the event. Finally, the \emph{ordinal event} is classified in \emph{Game event} and \emph{Non-game event} as shown in the \aspas{\emph{is-a}} hierarchy of classes (\autoref{fig:is-a-hierarchy-structure-of-classes}). The composed events in the \aspas{\emph{is-a}} hierarchy structure of classes are defined as subtype of \emph{event}, and they are: \emph{I\_L event}, \emph{Gameplay event}, \emph{Gamified Instructional event}, \emph{Gamified Learning event}, and \emph{Gamified I\_L event}.

%%%%%%%%%%%%%%%%%%%%%%%%%%%%%%%%%%%%%%%%%%%%%%%%%%
\section[Modeling Persuasive Game Design for CL Scenarios as Ontological Structures]{Modeling Persuasive Game Design for Collaborative Learning Scenarios as Ontological Structures}
\label{sec:modeling-persuasive-game-design-gamification-cl-scenarios}

\emph{Persuasive Game Design} (PGD) is defined by the author of this PhD thesis dissertation as \aspas{\emph{the game design for the purpose to change peoples’ attitudes and behaviors through persuasion and social influence without using coercion and/or deception}.} In this sense, to represent the PGD as ontological structures, it is necessary an ontology-based formalization of \emph{game design} because PGD is conceptualized as a game design that is embedded in a persuasive design.

As was explained in the previous section, the game design models are used to define the game events whereby the changes in the users’ states are produced or induced in a non-game events, and these changes are explained by theories/models of motivation a human behavior. Therefore, the game design consist into establish the relation between non-game event and game event based theoretical justification extracted from game design models and theories of motivation and human behavior. This game design when the theoretical justification is extracted from PGD models is the PGD, and it has been formalized in the ontology OntoGaCLeS by means of two ontological structures. The former ontological structure is a descriptive form, known as \emph{persuasive gameplay event}, and it is detailed in \autoref{subsec:persuasive-gameplay-event}. The latter ontological structure is a prescriptive form, known as \emph{WAY-knowledge of PGD}, and it is detailed in \autoref{subsec:way-knowledge-of-persuasive-game-design}.

Employing the ontological structures to represent PGD, the concept of \aspas{\emph{Persuasive Game Design CL Scenario Model}} has been proposed to represent the knowledge of how to apply PGD in the CL process of a CL scenario. This model has the purpose to represent the knowledge about how to persuade the learners to follow the interactions defined by a CSCL script. The formalization of this model as ontological structures is presented in \autoref{subsec:persuasive-game-design-cl-scenario-model}.

\subsection{Persuasive Gameplay Event}
\label{subsec:persuasive-gameplay-event}

The relation between game events and non-game events is explicitly represented in the ontology OntoGaCLeS under the concept of \emph{Gameplay event}. This concept describes, in an explicit way, what happens in the non-game world and the game world when a user performs the processes (actions) defined in the non-game event.



A persuasive gameplay event is a descriptive and explicit description of the relation between game events and a non-game event in which the doer of the non-game event has been persuaded and/or social influenced by the game events.

the gameplay event defines the interactions between the game elements and the participants of non-game process. 



 Thus, the persuasive gameplay event is formalized through the ontological structures shown in \autoref{fig:ontological-structures-persuasive-gameplay-events}, where the \emph{Gameplay event} (at the top of figure) represents any interaction that would occur between the participants and game elements. In the gameplay event, the \emph{Game event} describes actions performed by an \emph{agent} that becomes \emph{Game agent}, an \emph{action} of this agent becomes \emph{Game action}, the \emph{participant} who interacts with the game agent becomes \emph{Player}, and if there is an object produced as consequence of \emph{Game action}, it becomes a \emph{Game component}. Employing this formalization, let us to identify the elements for a gameplay event that represents the interaction of \aspas{\emph{a user obtaining points by making a comment},} such gameplay (shown at the top of \autoref{fig:ontological-structures-example-persuasive-gameplay-events}) is defined as a game event \aspas{\emph{Give points event}} in which the \emph{Point-system} becomes a \emph{Game agent} that performs the \emph{Game action} of \emph{Give points}, and the \emph{Points} given by this system becomes a \emph{Game component}. The \emph{Non-game event} in this gameplay event is \emph{Comment event} in which the action performed by the participant is \emph{Comment}.

\begin{figure}[!htb]
 \caption{Ontological structures to represent events}
 \label{fig:ontological-structures-persuasive-gameplay-events}
 \centering
 %\includegraphics[width=1\textwidth]{images/chap-ontogacles2/ontological-structures-persuasive-gameplay-events.png}
 \fautor
\end{figure}

According to the rules defined in the system, the interaction between the participants and game elements produces changes in the system communicated to the participants. These changes could influence 



\begin{figure}[!htb]
 \caption{Ontological structures to represent events}
 \label{fig:ontological-structures-example-persuasive-gameplay-events}
 \centering
% \includegraphics[width=1\textwidth]{images/chap-ontogacles2/ontological-structures-example-persuasive-gameplay-events.png}
 \fautor
\end{figure}







\aspas{\emph{C}} shown ,  for instance, the 


In this interaction, as shown in ontological structure to represent \emph{Gameplay event} (at the top of \autoref{fig:ontological-structures-persuasive-gameplay-event}), the \emph{Game event} 









 Therefore

This interaction represented as the ontological structure shown at the top of \autoref{fig:ontological-structures-persuasive-gameplay-event}



Thus, in the ontological structure to represent a \emph{Gameplay event},  to produce t changes as 

are produced by an \emph{agent} 


Thus,  In the \emph{Game event}, 


As was mentioned before, gameplay of a gamified CL scenario is defined by the way in which the interactions between the participants and the game elements could occur. When a participant interacts with the game elements, the rules defined in the gamified CL scenario process his/her inputs causing changes in the game elements, and these modifications are communicated to the participant. These rules and changes are related to individual motivational goals that must be achieved by the participants, so that each participant has his/her own strategy to interact with the gamified CL scenario to achieve these goals. 


Gameplay event is a prescriptive description of a PGD as a chunk of process between the game elements and the participants of a non-game context being gamified.

In the gamification of CL scenarios, this chuck of process has the purpose to persuade the participants to perform an interaction defined in the sequencing mechanism of CSCL script. 

definition of interactions in this process consists 

Gamification defines a gameplay process to motivate and engage the participants in a non-game process; and the gameplay event is a chunk of this process.

Thus, the gameplay event defines the interactions between the game elements and the participants of non-game process. The relation between game events and non-game events is explicitly represented in the ontology OntoGaCLeS under the concept of Gameplay event. This concept describes, in an explicit way, what happens in the non-game world and the game world when a user performs the processes (actions) defined in the non-game event.


\newpage
\subsection{WAY-knowledge of PGD}
\label{subsec:way-knowledge-of-persuasive-game-design}

 is a prescriptive description of this relation.

\newpage
\subsection{Persuasive Game Design CL Scenario Model}
\label{subsec:persuasive-game-design-cl-scenario-model}






%%%%%%%%%%%%%%%%%%%%%%%%%%%%%%%%%%%%%%%%%%%%%%%%%%
\section[Modeling CL Gameplay Based on Persuasive Game Design]{Modeling Collaborative Learning Gameplay Based on Persuasive Game Design}
\label{sec:modeling-cl-gameplay-persuasive-game-design}


%%%%%%%%%%%%%%%%%%%%%%%%%%%%%%%%%%%%%%%%%%%%%%%%%%
\section[Formalizing an Ontological Model to Apply Gamification as a Persuasive Technology in CL Scenarios]{Formalizing an Ontological Model to Apply Gamification as a Persuasive Technology in Collaborative Learning Scenarios}
\label{sec:formalizing-ontological-model-apply-gamification-persuasive-technology}



 an ontological model to apply gamification as a  employing the persuasive game design strategies defined in the Model-driven persuasive game proposed by \citeonline{Orji2014}.
 
 
%%%%%%%%%%%%%%%%%%%%%%%%%%%%%%%%%%%%%%%%%%%%%%%%%%
\section{Concluding Remarks}
\label{sec:ontogacles2-concluding-remarks} 

 

\chapter[A Unify Modeling of Learner's Growth Process and Flow Theory]{A Unify Modeling of Learner's Growth Process and Flow Theory}
\label{chapter:unify-modeling-learner-growth-flow-theory}

In the learning process, the affective state of students plays an essential role that influences several mechanisms of rational thinking and learning \cite{D'Mello2012, Picard2000, ReisRodriguezLyraJaquesBittencourtIsotani2015}. Students with negative affective states (e.g. boredom) during the learning process are, in general, significantly more likely to obtain inadequate learning outcomes, because they often are not motivated and are not engaged in the learning process \cite{CraigGraesserSullinsGholson2004, ShernoffCsikszentmihalyiSchneiderShernoff2014}. In this sense, to motivate a student so that he/she participates in a learning scenario with complete immersion, it is necessary that his/her affective state provide an optimal experience. This affective state is denominated flow, and it is a mental state of operation characterized by a feeling of energized focus, full involvement, and success in the task being performed \cite{Csikszentmihalyi2008}. 

To define a gamified CL scenario with game elements that favor and maintain the participants in the flow state during the CL process, it is necessary to have understanding about the influence of these game elements in the affective state of participants. One condition for attaining and maintaining the flow state is the good balance between the perceived challenges of the tasks that will be carried out, and the participant’s own perceived abilities to accomplish these tasks. A task that is perceived too challenging or one that is not challenging enough may lead to anxiety or boredom, and when a person perceives that he/she does not have enough ability or he/she has too ability to carry out the task, he/she would be anxious or bored. Thus, a model known as GMIF model: \aspas{\emph{Learner’s \textbf{G}rowth \textbf{M}odel \textbf{I}mproved by \textbf{F}low Theory}} to integrate the learner's growth process and the third condition of good balance between the perceived challenges and ability is presented in this chapter. 

This chapter is organized as follows: The first section provides details about the Learner’s Growth Model (LGM model) and the three-channel flow model (\autoref{sec:lgm-model-three-channel-flow-model}). Then, the GMIF model is presented in \autoref{sec:integrating-learners-growth-model-flow-theory}. To demonstrate the usefulness of the GMIF model, \autoref{sec:application-giving-rewards-by-gimf-model} illustrates how this model can be used to establish the game rewards that will be given to the participants in a gamified CL scenario to maintain them in the flow state. Finally, \autoref{sec:model-gmif-concluding-remarks} presents the concluding remarks.

Part of the work described in this chapter was published by the author of this PhD thesis dissertation in the scientific article:

\begin{itemize}
\item
\aspas{\emph{Toward A Unified Modeling of Learner's Growth Process and Flow Theory}} published in the International Journal of Educational Technology \& Society, Vol. 19, No. 2, April 2016 \cite{ChallcoAndradeBorgesBittencourtIsotani2016}.
\end{itemize}

%%%%%%%%%%%%%%%%%%%%%%%%%%%%%%%%%%%%%%%%%%%%%%%%%%

\section{Learner’s Growth Model and Three-channel Flow Model}
\label{sec:lgm-model-three-channel-flow-model}

\subsection{Learner’s Growth Model}
\label{subsec:learners-growth-model}

Based on learning theories, the \aspas{\emph{\textbf{L}earners \textbf{G}rowth \textbf{M}odel}} (LGM model) is a graph that represents the learning process of a student as stages of skill development and knowledge acquisition as a directed graph \cite{InabaIkedaMizoguchi2003, IsotaniMizoguchi2006}. The learner’s growth process is represented as paths on the graph that allow for the representation of the relationships between learning strategies and their educational benefits.

As shown in \autoref{fig:lgm-model}, the LGM model has twenty states that are the result of the number of stages related to skill development multiplied by the number of stages related to knowledge acquisition. In the graph, the stages of skill development (nothing, rough cognitive, explanatory cognitive, associative, and autonomous) are represented in the lower-left triangle, while the stages of knowledge acquisition (nothing, accretion, tuning, and restructuring) are represented in the upper-right triangle. In skill development, the cognitive stage (rough, and explanatory) involves an initial encoding of a target skill that allows the learner to present the desired behavior or, at least, some rough approximation thereof; the associative stage is the improvement of the desired skill through practice; and the autonomous stage involves gradual continued improvement in the performance of the skill \cite{Anderson1982}. During knowledge acquisition, the accretion stage incorporates the addition and interpretation of new information in terms of pre-existent knowledge; the tuning stage involves coming to understand the knowledge through its application in a specific situation; and the restructuring stage comprises a process in which the relationship of the acquired knowledge is considered and the existent knowledge structure is rebuilt \cite{RumelhartNorman1976}.

The arrows in the LGM model shown in \autoref{fig:lgm-model} represent possible transitions between stages, and the form $s(x,y)$ on the top of each vertex is the simplified form of representing a stage, where the symbol \aspas{$x$} represents the current stage of skill development, and the symbol \aspas{$y$} represents the current stage of knowledge acquisition. For instance, the transition $s(0,0) \to s(2,0)$ means the possible transition from the stage $s(0,0)$ where a learner does not have any knowledge or skills to the associative stage $s(2,0)$ of skill development.

 \begin{figure}[htb]
 \caption{Learner’s Growth Model (LGM model)}
 \label{fig:lgm-model}
 \centering
 \includegraphics[width=1\textwidth]{images/chap-model-gmif/lgm-model.png}
 \fadaptada{InabaIkedaMizoguchi2003}
\end{figure}

One of the most interesting uses of this model is the representation of transitions in the skill development and knowledge acquisition stages of participants in CL scenarios based on the learning strategies employed by them and the benefits that different learning theories offer. \autoref{fig:lgm-model-cognitiveapprenticeship} shows the representation for the transition of stages in the development of skill and acquisition of knowledge involved in a CL scenario based on the cognitive apprentice theory in which the black arrows imply the application of the learning strategies that facilitates the learner’s growth process.

 \begin{figure}[htb]
 \caption[Transitions in the LGM model for cognitive apprenticeship scenarios]{Transitions in the LGM model for cognitive apprenticeship scenarios. On the left side, stages in the learning by apprentice strategy for participants who play the apprentice role. On the right side, stages in the learning by guiding strategy for participants who play the master role}
 \label{fig:lgm-model-cognitiveapprenticeship}
 \centering
 \includegraphics[width=1\textwidth]{images/chap-model-gmif/lgm-model-cognitiveapprenticeship.png}
 \fadaptada{IsotaniMizoguchiInabaIkeda2010}
\end{figure}

On the left side of \autoref{fig:lgm-model-cognitiveapprenticeship} is shown the transition of stages for the apprenticeship learning strategy, where the transition of stages in the LGM model represent the growing in cognitive skills from $s(0,y)$:\emph{nothing} into the $s(3,y)$:\emph{associative stage} through the $s(1,y)$:\emph{rough-cognitive stage} and the $s(2,y)$:\emph{explanatory-cognitive stage}. These transitions in the skill development are transitions carried out by participants who play the apprentice role. On the right side of \autoref{fig:lgm-model-cognitiveapprenticeship} is shown the transitions of stages described by the learning strategy \aspas{\emph{learning by guiding}} in the LGM model. According to this learning strategy, the participant who plays the master role grows in his/her cognitive skill from the $s(3,y)$:\emph{associative stage} into the $s(4,y)$:\emph{autonomous stage}.

With the use of the LGM model, any learning strategy or educational best practice can be explicitly described as a path on the graph, facilitating the understanding, visualization and utilization of the model \cite{IsotaniMizoguchiInabaIkeda2010}.

\subsection{Three-channel Flow Model}
\label{subsec:three-channel-flow-model}

\emph{Csikszentmihalyi’s flow theory} constitutes an important theory regarding to affective states of people during activities that require active work, such as discussions, exercises, and group activities \cite{Csikszentmihalyi2014,SnyderLopezPedrotti2010}. This theory has been applied in several fields, including game design, commerce, and education. The key concept of this theory is the \aspas{\emph{The Zone Flow}} as a situation in which a person is so engaged and focused on a particular task that he/she is completely immersed in it. According to the flow theory, to achieve the flow state, the following conditions must be satisfied:

\begin{itemize}
\item Clear goals in which the expectations and rules are clearly discernable
\item Direct and immediate feedback in which the successes and failures of the tasks are apparent, so that behavior can be adjusted as needed
\item Good balance between the perceived ability and challenge
\end{itemize}

One of the conditions given above is that the flow state only occurs if there is a good balance between the perceived challenges of the task at hand and the learner’s own perceived ability to solve it. This means that the definition of an appropriate challenge (i.e. level of difficulty) is fundamental to design situations that promotes a flow state \cite{LinehanBellordKirmanMorfordRoche2014}. Thus, Csikszentmihalyi proposes the three-channel flow model \cite{Csikszentmihalyi2008} shown in \autoref{fig:three-channel-flow-model}, in which both anxiety and boredom drive persons to frustration. When a task is too difficult to be solved, it causes anxiety because it is perceived as too challenging or because the person’s ability level is not sufficient to solve the task. In the same way, when a task is too easy it causes boredom because it is not challenging enough, or because the person’s ability level is too high for the task.

 \begin{figure}[htb]
 \caption{Affective states in terms of perceived ability level and challenge level, according to the three-channel flow model}
 \label{fig:three-channel-flow-model}
 \centering
 \includegraphics[width=0.5\textwidth]{images/chap-model-gmif/three-channel-flow-model.png}
 \fadaptada{Csikszentmihalyi2008}
\end{figure}

The three-channel flow model has been frequently used to build instruments and tools for the detection of the flow state \cite{KortReillyPicard2001,PearceAinleyHoward2005,Esteban-MillatMartinez-LopezHuertas-GarciaMeseguerRodriguez-Ardura2014a,LeeJhengHsiao2014}. More recently, in the context of computer education and instructional technology, studies have attempted to analyze and modeling the flow state in order: (a) to evaluate the participants’ interactions with learning objects; (b) to personalize educational activities (e.g. lessons); and (c) to develop better learning content. In the context of game-based learning, a framework to support the integration of games as learning activities is proposed by \citeonline{delBlancoTorrenteMarchioriMartinez-OrtizMoreno-GerFernandez-Manjon2012}. To do so, they identified key aspects about the mechanisms that facilitate the use of pedagogical approaches with games to keep students in the flow state. Then, they proposed a workflow to integrate games into the learning process. As a result, this workflow can be used to create guidelines for helping instructional designers the use (and reuse) of games in the learning process. Although this work provides some initial support for creating better learning experiences using game in the learning process, if the games themselves do not have the qualities and attributes necessary to maintain student engagement, the flow experiences will not occur. Considering this problem, \citeonline{KiiliLainemadeFreitasArnab2014} proposed a framework for analyzing and designing educational games based on the flow theory. This framework describes several dimensions of flow experience as well as meaning factors that affect the design of game-based learning activities.

Despite the broad use of the three-channel flow model in educational contexts and its use in game-based learning, to the best of the knowledge for the author of this dissertation, there is not a computational model based on the three-channel flow model that provides support to create CL scenarios that maintain the flow state in the participants while offering theoretical justifications regarding the learner’s growth as an indicator for the perceived ability level.
In particular, there is no computational help to define the appropriate levels of challenges for the game elements of a gamified CL scenario.

\section{Integrating the Learner’s Growth Model and the Three-channel Flow Model}
\label{sec:integrating-learners-growth-model-flow-theory}

The perceived challenge and ability level balance of flow theory can be determined as the current stage of the participant in the LGM model, and the challenge level to maintain the learner in the flow state. Thus, to integrate the representation of the learner’s growth process and the condition of good balance between the perceived challenge and ability, the \emph{Learner’s \textbf{G}rowth \textbf{M}odel \textbf{I}mproved by \textbf{F}low Theory}, hereinafter referred to as GMIF model, has been proposed as a LGM model in which the arrows $s(x_{1},y_{1}) \to s(x_{2},y_{2})$ are labeling with the form $[z_{min}; z_{max}]$ to indicate the \emph{minimum challenge level} ($z_{min}$) and the \emph{maximum challenge level} ($z_{max}$) that are necessary to maintain the learner's flow.

Before to present the algorithm proposed to create a GMIF model with a n-scale of challenge level (\emph{n-scale GMIF model}), a five-scale GMIF model is presented to introduce and detail the elements involved in the building of a GMIF model. After that, the algorithm to create a n-scale GIMF model is presented, and also, the benefits and application of GMIF model in the learning design are detailed.

\subsection{Five-scale GMIF Model}
\label{subsec:five-scale-gmif-model}

In the three-channel flow model (detailed in \autoref{subsec:three-channel-flow-model}), the levels of perceived challenge and ability are used as indicators to identify the current person's affective state in zones of anxiety, flow, and boredom. These two indicators are represented as axes in the three-channel flow model to depict situations where a learner are anxious, bored or in a flow state. These situations could be represented as a rectangular regions in the plane defined by the division of the perceived challenge and ability axes. Thus, to build a GMIF model, two three-channel flow models with the division of $5 \times 5$ rectangular regions are obtained by dividing the axes into five parts. Then, the transitions of the skill development defined by the LGM model are set to the ability axis using a uniform distribution in the first three-channel flow model to define a five-scale three-channel flow model of skill development stages and challenge levels. In the second three-channel flow model, the transitions of the knowledge acquisition defined by the LGM model are set to the ability axis using also a uniform distribution to define a five-scale three-channel flow model of knowledge acquisition stages and challenge levels.

\begin{figure}[htb]
 \caption{Five-scale three-channel flow model of skill development}
 \label{fig:five-scale-three-channel-flow-model-skill-development}
 \centering
 \includegraphics[width=0.65\textwidth]{images/chap-model-gmif/five-scale-three-channel-flow-model-skill-development.png}
 \fautor
\end{figure}

\autoref{fig:five-scale-three-channel-flow-model-skill-development} shows the five-scale three-channel flow model of skill development stages and challenge levels. In this model, the five-scale challenge levels are: 0:\emph{lowest}, 1:\emph{low}, 2:\emph{medium}, 3:\emph{high}, 4:\emph{highest}. The transitions in the skill development are: $s(0,y) \to s(1,y)$: from \emph{nothing} to \emph{rough-cognitive stage}; $s(1,y) \to s(2,y)$: from \emph{rough-cognitive stage} to \emph{explanatory-cognitive stage}; $s(2,y) \to s(3,y)$: from \emph{explanatory-cognitive stage} to \emph{associative stage}; $s(3,y) \to s(4,y)$: from \emph{associative stage} to \emph{autonomous stage}. According to this model, the label sequence of minimum and maximum challenge levels for maintaining the learner’s flow is $s_{1}=\{[0;0], [1;2], [2;3], [4;4]\}$ in which the first element \aspas{$[0;0]$} extracted from region $(1,1)$ means that, during the transition: $s(0,y) \to s(1,y)$, the proper level of challenge to maintain the learner’s flow is 0:\emph{lowest}. The second element \aspas{$[1;2]$} extracted from regions $(2,2)$ and $(3,3)$ means that, during the transition $s(1,y) \to s(2,y)$, the proper level of challenge to maintain the learner’s flow is in the range of 1:\emph{low} to 2:\emph{medium}. The third element \aspas{$[2;3]$} means that, during the transition $s(2,y) \to s(3,y)$ extracted from region $(3,3)$ and $(4,4)$, the proper level of challenge to maintain the learner’s flow is in the range of 2:\emph{medium} to 3:\emph{high}. Finally, the fourth element \aspas{$[4;4]$} extracted from region $(5,5)$ means that, during the transition $s(3,y) \to s(4,y)$, the proper level of challenge is 4:\emph{highest}.
 
By employing the transitions $s(x,0) \to s(x,1) \to s(x,2) \to s(x,3)$ of knowledge acquisition ($s(x,0) \to s(x,1)$: from \emph{nothing} to \emph{accretion stage}; $s(x,1) \to s(x,2)$: from the \emph{accretion stage} to \emph{tuning stage}; $s(x,2) \to s(x,3)$: from the \emph{tuning stage} to \emph{restructuring stage}), the five-scale three-channel flow model shown in \autoref{fig:five-scale-three-channel-flow-model-knowledge-acquistion} has been obtained to represent the relation of knowledge acquisition stages and challenge levels. In this space, labels of minimum and maximum challenge levels for maintaining the learner’s flow is defined by the sequence $s_{2} = \{[0;0], [1;3], [4;4]\}$ in which the first element \aspas{$[0;0]$} extracted from the region $(1,1)$ means that means that, during the transition $s(x,0) \to s(x,1)$, the level of challenge should be 0:\emph{lowest} to maintain the learner’s flow. The second element \aspas{$[1;3]$} extracted from regions $(1,1)$, $(2,2)$ and $(3,3)$ means that, during the transition $s(x,1) \to s(x,2)$, the proper level of challenge to maintain the learner’s flow is in the range of challenge levels 1:\emph{low}, 2:\emph{medium} and 3:\emph{high}. Finally, the proper level of challenge during the transition $s(x,2) \to s(x,3)$ is 4:\emph{highest}.

 \begin{figure}[htb]
 \caption{Five-scale three-channel flow model of knowledge acquisition}
 \label{fig:five-scale-three-channel-flow-model-knowledge-acquistion}
 \centering
 \includegraphics[width=0.65\textwidth]{images/chap-model-gmif/five-scale-three-channel-flow-model-knowledge-acquistion.png}
 \fautor
\end{figure}

To obtain the five-scale GMIF model, the relationship between the transitions of stages in the skill development and knowledge acquisition and the challenge levels should be clearly understood from the two five-scale three-channel flow models shown in \autoref{fig:five-scale-three-channel-flow-model-skill-development} and \autoref{fig:five-scale-three-channel-flow-model-knowledge-acquistion}. With this knowledge, it is possible to design CL scenarios that (i) favor the maintenance of a flow state for students; and (ii) help them to achieve desired educational goals (i.e. acquisition of knowledge or development of skills). To accomplish these objectives, the label sequences ($s_{1}$ and $s_{2}$) to maintain the learner's flow identified from \autoref{fig:five-scale-three-channel-flow-model-skill-development} and \autoref{fig:five-scale-three-channel-flow-model-knowledge-acquistion}, which enables us to understand when a participant are in flow state (by making the correlation between knowledge and skills with a five-scale of challenge level), to adequately label each transition (i.e. $s(x,y) \to s(x',y')$) between states in the LGM with a tuple \aspas{$[z_{min}; z_{max}]$,} where $z_{min}$ refers to the minimum challenge level necessary of to be considered interesting and not too easy, and $z_{max}$ refers to the maximum challenge level possible to be considered challenging but not too difficult.

To define the values of $z_{min}$ and $z_{max}$ in the labels \aspas{$[z_{min}; z_{max}]$} of a transition $s(x,y) \to s(x',y')$, the sequences $s_{1}$ and $s_{2}$ are used to define the proper levels of challenge for the transitions related to skill development and knowledge acquisition, respectively. Thus, when a transition $s(x, y) \to s(x', y)$ is related to the skill development, the sequence $s_{1}$ extracted from the model shown in \autoref{fig:five-scale-three-channel-flow-model-skill-development} is used to label this transition. For example, to develop skill from the \emph{explanatory-cognitive stage} to the \emph{associate stage}, the transition $s(2,y) \to s(3,y)$ is labeled in the LGM graph as $[2;3]$ by looking at where this transition is located in the flow area of \autoref{fig:five-scale-three-channel-flow-model-skill-development}. In this particular situation, the label $[2;3]$ means that, to maintain the learner’s flow, the level of challenge for an element in the learning scenario should be selected in the range of 2:\emph{medium} to 3:\emph{high}. Following the same procedure, the transitions related to knowledge acquisition are used to label the transitions $s(x,y) \to s(x,y')$ defined by the sequence $s_{2}$.

\autoref{fig:five-scale-gmif-model} shows the five-scale GMIF model that results from labeling the LGM with a scale of five levels of challenge. 

 \begin{figure}[htb]
 \caption{Five-scale learner's growth model improved by the flow theory}
 \label{fig:five-scale-gmif-model}
 \centering
 \includegraphics[width=1\textwidth]{images/chap-model-gmif/five-scale-gmif-model.png}
 \fautor
\end{figure}

\newpage
\subsection{Algorithm for Building a n-scale GMIF Model}
\label{subsec:pseudo-algorithm-n-scale-gmifs}

\autoref{algorithm:build-n-gimf-model} shows the algorithm proposed to build a GMIF model with a n-scale of challenge levels (\emph{n-scale GMIF model}), where the expected difference for the levels of challenge in the flow area is passed as the param \aspas{\emph{delta}} (as a second argument that has the default value zero). In the algorithm, the variable GMIF contains the labels for the transitions $t = s(x,y) \to s(x’,y’)$ of the LGM model, and each label is represented as the form $[z_{min}; z_{max}]$, which indicates the minimum $z_{min}$ and maximum $z_{max}$ levels of challenge that is necessary to maintain a participant is the flow state.

In the algorithm, the flow regions for the \aspas{\emph{transitions in skill development stages} vs \emph{challenge level}} and the \aspas{\emph{transitions in knowledge development stages} vs \emph{challenge level}} are obtained by the function \aspas{\emph{get\_flow\_region}} (lines 2-3), where the first parameter is the number of transitions for skill development or knowledge acquisition, and the second parameter is the n-scale of space for the challenge level, and the third parameter is the expected difference for levels of challenge.

\begin{algoritmo}
\caption{Algorithm to build a $n$-scale GMIF model}
\label{algorithm:build-n-gimf-model}
\begin{algorithmic}[1]\small
\Procedure{build\_GIMF}{n\_scale = 5, delta = 0}
  \State skill\_flow $\gets$ get\_flow\_region(4, n\_scale, delta)
  \State knowledge\_flow $\gets$ get\_flow\_region(3, n\_scale, delta)
  \ForAll{$t = (x,y) \to (x',y)$ in LGM model}
    \State GIMF[t] $\gets$ $\cup_{i=x}^{x'-1}$skill\_flow[i]
  \EndFor
  \ForAll{$t = (x,y) \to (x,y')$ in LGM model}
    \State GIMF[t] $\gets$  knowledge\_flow[y]
  \EndFor
\EndProcedure
\end{algorithmic}
\end{algoritmo}

Because the transition in the skill development stages includes flexibility that allows to increase the skill stage without following all the transitions between stages, transitions $s(x,y) \to s(x',y)$ are labeled in all levels of challenge that are defined in intermediate transitions as shown in the lines (4-6) of \autoref{algorithm:build-n-gimf-model}. For example, it is possible to go from 0:\emph{nothing} to 3:\emph{associative stage} without moving through the intermediate stages 1:\emph{rough-cognitive stage} and 2:\emph{explanatory-cognitive stage}; thus, the transition $s(0,y) \to s(3,y)$ is labeled with the union of challenge levels defined in the transitions $s(0,y) \to s(1,y)$, $s(1,y) \to s(2,y)$ and $s(2,y) \to s(3,y)$. In the case of transitions related to the knowledge acquisition, the transition of stages is completed step-by-step without skipping any of the stages; thus, the transition $s(x,y) \to s(x,y')$ is labeled by setting the corresponding levels of challenge for the transitions of knowledge acquisition at shown in lines (7-9) of \autoref{algorithm:build-n-gimf-model}.

\autoref{algorithm:get-flow-region} details the algorithm for the function \aspas{\emph{get\_flow\_region}.} This function calculates the flow region in the n-scale three-channel flow models, where the flow region is represented as an array of size $m$ (number of transitions for skill development or for knowledge acquisition) in which each $i$-th element contains the levels of challenge for the transition from the $i$-th stage to the next stage ($i+1$ stage). For an instance of five levels of challenge and three transitions of knowledge acquisition (shown in \autoref{fig:five-scale-three-channel-flow-model-knowledge-acquistion}), the flow region as a result of the algorithm is a sequence $s = \{[0;0], [1;3], [4;4]\}$, where the first element \aspas{$[0;0]$} indicates the level of challenge as 0:\emph{lowest} for the transition $s(x,0) \to s(x,1)$.

\begin{algoritmo}
\caption{Algorithm to obtain a flow region in $m$ transitions with $n$ challenges}
\label{algorithm:get-flow-region}
\begin{algorithmic}[1]\small
\Function{get\_flow\_region}{$m$, $n\_challenges = 5$, $delta = 0$}
  \State $n \gets n\_challenges$%\LeftComment{normalize the number of challenge levels}
  \If{($n\_challenges > m$) and is.odd($n\_challenges$)}
    \State $n \gets n-1$
  \EndIf
  \State $distr \gets$ initialize\_array($m$, $\lfloor s/m \rfloor$)%\LeftComment{sets number of challenge levels for each transition}
  \State $rest \gets s - m \lfloor s/m \rfloor$
  \If{($rest > 0$)}
    \State $inv\_sigma \gets (n-rest)/2$
    \For{$i \gets 0$ to $rest-1$}
      \State $distr[inv\_sigma+i] \gets distr[inv\_sigma+i]+1$
    \EndFor
  \EndIf
  \State $flow[0].min \gets 0$%\LeftComment{make labels for flow region}
  \State $flow[0].max \gets distr[0] - 1$
  \For{$i \gets 1$ to $m-1$}
    \State $flow[i].min \gets flow[i-1].max + 1$
    \State $flow[i].max \gets flow[i-1].max + distr[i]$
    \If{($n\_challenges > m$) and is.odd($n\_challenges$)}
      \If{is.odd($m$) and ($i = \lfloor m/2 \rfloor$)}
        \State $flow[i].max \gets flow[i].max + 1$
      \EndIf
      \If{is.even($m$)}
        \If{$i = \lfloor m/2 \rfloor - 1$}
          \State $flow[i].max \gets flow[i] + 1$
        \EndIf
        \If{$i = \lfloor m/2 \rfloor$}
          \State $flow[i].min \gets flow[i] - 1$
        \EndIf
      \EndIf
  \EndIf
  \EndFor
  \ForAll{$r$ in $flow$}
    \If{($r.max = -1$)}
      \State $r.min \gets -1$
    \Else
      \State $r.min \gets r.min - delta$
      \State $r.max \gets r.max + delta$
    \EndIf
  \EndFor
  \State \Return $flow$
\EndFunction
\end{algorithmic}
\end{algoritmo}

The function \aspas{\emph{get\_flow\_region}} described as the \autoref{algorithm:get-flow-region} is summarized in a narrative form as:
Calculates the number of levels that should be distributed for each transition of stage (lines 2-13). These values are calculated through a uniform distribution that tries to maintain the same number of levels in all stages. The stages located in the same distance of the mean stage should have the same number of levels. For example, the distribution of eight levels of challenge in five transitions is defined as the array $s = \{1,2,2,2,1\}$, where the second, third, and fourth transitions are set with two levels, it is $s(1) = s(2) = s(3) = 2$, whereas the first and fifth transitions are set with one level, it is $s(0) = s(4) = 1$. Finally, the transition located in the third transition is set with two levels, it is $s(2) = 2$. The steps that calculate these levels are as follows:

\begin{itemize}
\item
The normalization for the number of challenges. This is done to avoid the non-uniform distribution that happens when this number is odd and it is greater than the number of transitions. For example, the distribution of nine levels among four transitions only can be done by setting one transition with three levels, and setting the rest of transitions with two levels. Therefore, the normalization for the levels of challenge is done by reducing the number of levels by one (lines 2-5). In the previous example, the distribution of nine levels into four transitions can be defined as the array $s = \{2, 2, 3, 2\}$ before the normalization, and the distribution of these nine levels after the normalization is defined as the array $s = \{2, 2, 2, 2\}$.
\item
After the normalization, the minimum number of challenge level for each stage is defined by the function \aspas{\emph{initialize\_array}} (line 6), which initializes an array of size $m$ with the value. The remaining levels of challenge (line 7) are distributed according to the position \aspas{$inv\_sigma$} (lines 10-12). The value \aspas{$inv\_sigma$} is the result of dividing the number of free spaces after the distribution of the remaining challenge levels by two (line 11).
\end{itemize}

After determining the number of challenge levels that will be distributed for each transition (\emph{distr}), the next step is to set the labels for the flow region that has no expected difference in the levels of challenge (lines 14-32). Thus, the process to define these labels consists in:

\begin{itemize}
\item To set the flow region for the first transition through the definition of the minimum challenge level with value zero (line 14), and the definition of the maximum challenge level with the number of challenge levels decreased by one (line 15).
\item Setting the flow region for the rest of the transitions (lines 16-32). The minimum level of challenge is defined as the maximum challenge level of the previous transition increased by one (line 17), and the maximum challenge level is defined as the maximum challenge level of the previous transition increased by the number of challenge levels (line 18). For cases in which the normalization of levels has been done, the following two rules must be applied:

\begin{itemize}
\item If the number of transitions is odd, then the maximum challenge level is increased by one in the mid-transition (lines 20-22). Thus, the flow region for nine levels of challenge in five transitions is defined as the array $s = \{[0;0], [1;2], [3;5], [6;7], [8;8]\}$.
\item If the number of transitions is even, then there are two mid-transitions: the first mid-transition is located in the position $\lfloor m/2 \rfloor - 1$, and the second mid-transition is located in the position $\lfloor m/2 \rfloor$. Next, the maximum challenge level is increased by one in the first mid-transition (lines 24-26). Finally, the minimum challenge level is decreased by one for the second mid-transition (lines 27-29). Thus, the flow region for nine challenge levels in four transitions is defined as the array $s = \{[0;1], [2;4], [4;6], [7;8]\}$.
\end{itemize}
\end{itemize}

Finally, the expected difference in level of challenge, defined as the parameter delta, is used to decrease and increase the minimum and maximum levels of challenge for each transition in the flow region (lines 33-40).

\subsection{Benefits and Application of GMIF Model}

Several factors must be considered during the learning design process, such as learning goals, pedagogical preferences, intervention timing, type of feedback, students’ needs, available resources, and so on. The work of \citeonline{KoedingerBoothKlahr2013} estimates that there is a poll of 330 (205 trillion) instructional choices that could be considered when designing a learning activity. Unfortunately, most designers and educators do not have enough knowledge/skills to cope with this huge number of instructional choices and select those choices that are the best fit for a particular situation. To provide help for the instructional designers in this process, the n-scale GMIF model provides an appropriate integration of instructional design with learning theories, models of learner’s growth, and the three-channel flow model, in order to reduce the complexity of the learning design task. Specifically, the GIMF model can be used to foster flow experiences in theory-based learning scenarios.

\subsubsection*{Foster Flow Experiences in Theory-Based Learning Scenarios}

To get students into the flow state and produce optimal learning experiences, one should initially consider: 

\begin{itemize}
\item The student's initial stage and learning objectives (as final stage) in terms of knowledge acquisition and skills development \cite{Anderson1982,RumelhartNorman1976};
\item The learning path to be follow by the student based on theoretical justifications \cite{IsotaniMizoguchiInabaIkeda2010,Romiszowski1981}; and 
\item The definition of the challenge level based on the three-channel flow model. Here it is necessary to select the necessary challenge level to keep the student in the flow state \cite{Csikszentmihalyi2014,D'Mello2012}. 
\end{itemize}

The GMIF model has been developed to support these steps. In the first step, the GMIF model provides a standard to describe and represent learning objectives as well as the learner’s stage. Thus, the problem of sharing learning designs among people and computers is reduced. Accordingly, an instructional designer can indicate the initial stage of the student and select his/her learning objectives. Both correspond to stages in the GMIF model. After that, the designer can check manually or automatically (using learning design authoring tools) which learning strategy based on instructional/learning theories provides an adequate learning path that supports a learner in achieving the desired goals. In this situation, the GMIF model offers a visual representation as a sequence of arrows on the GMIF model that represent learning strategies and how they support the learner’s growth process. Finally, to provide a flow experiences, the designer needs to define the level of challenge that is needed to maintain the student in the flow state. In this regard, the GMIF model will indicate the level of challenge that should be considered when creating tasks to alter the state of the student while keeping him/her motivated.

%%%%%%%%%%%%%%%%%%%%%%%%%%%%%%%%%%%%%%%%%%%%%%%%%%
\section[Application of GIMF Model for the Definition of Game Rewards]{Application of GIMF Model for the Definition of Game Rewards in Gamified CL Scenarios}
\label{sec:application-giving-rewards-by-gimf-model}

With the GMIF model detailed above, we can develop different functions in authoring tools of learning scenarios. A useful function developed by the author of this dissertation is the searching of proper learning objects that will favor and maintain the learner’s flow in the learning scenario \cite{ChallcoAndradeBorgesBittencourtIsotani2016}. Thus, in this function, an instructional designer firstly set the initial and goal stages of a student in a learning scenario using the graphical representation of the GMIF model. Next, each label for a difficulty level in the transition from the initial stage to the goal stage is used as a constraint to search learning objects from different repositories. 

To demonstrate the usefulness of the GIMF model in the gamification of CL scenarios, the definition of game rewards to be promised and given by game agents in gamified instructional and learning events is presented here as an application in which ontological structures to represent gamified I\_L events are used as information source. For accomplish this task, the instructional designer first set the initial and goal stages in the graphical representation of GIMF model using the information provided by the individual goal (\emph{I-goal}) in the instructional and learning event. Then, the learning path from the initial stage to the goal stage is identified as the learning strategy employed by the participant, and the labels of challenge levels are calculated for the arrows in the learning path according to the number of challenges/levels that could have a game component. Finally, these labels can be used as constraints to set the game reward to be promised or given by the game agent to keep the participant in the flow state.

\begin{figure}[htb]
 \caption{Application of the GMIF model to set the game points to be given in the gamified instructional event \aspas{\emph{Gamified Checking}} of the \emph{Gamified Cognitive Apprenticeship Scenario for Master/Yee Achiever and Apprentice/Yee Achiever}}
 \label{fig:set-game-reward-gamified-checking}
 \centering
 \includegraphics[width=1\textwidth]{images/chap-model-gmif/set-game-reward-gamified-checking.png}
 \fautor
\end{figure}

For the instance shown in \autoref{fig:set-game-reward-gamified-checking}, where the five-scale GMIF model has been applied to set the game points to be given by the point system as consequence of instructional event \aspas{\emph{Checking}} in a gamified CL scenario based on the cognitive apprenticeship theory with the Yee's achiever player role assigned for the master and apprentice role holder - \emph{Gamified Cognitive Apprenticeship Scenario for Master/Yee Achiever and Apprentice/Yee Achiever}. In this situation, the instructional designer set the initial stage for the \emph{Master/Yee Achiever} role holder as $s(3,1)$ - associative stage for skill development and accretion for knowledge acquisition - and the goal stage as $s(4,1)$ - autonomous stage for skill development and accretion for knowledge acquisition. Thus, the learning path in the GIMF model is defined by the learning strategy \aspas{\emph{Learning by Guiding},} and the proper level of challenges that will favor and maintain the \emph{Master/Yee Achiever} in the flow state is defined by the label \aspas{$[4;4]$} that indicate a 4:\emph{highest} challenge level in the five-scale three-channel flow model of skill development stages. Having this flow region, the proper reward to be given in the \emph{Game consequence event} by \emph{Game Point system (individual)} for the \emph{Master/Yee Achiever} role holder is \emph{+1600 points} (as \emph{Game component}).

\begin{figure}[htb]
 \caption{Application of the GMIF model to set the game points to be given in the gamified learning event \aspas{\emph{Gamified Being Checked}} of the \emph{Gamified Cognitive Apprenticeship Scenario for Master/Yee Achiever and Apprentice/Yee Achiever}}
 \label{fig:set-game-reward-gamified-being-checked}
 \centering
 \includegraphics[width=1\textwidth]{images/chap-model-gmif/set-game-reward-gamified-being-checked.png}
 \fautor
\end{figure}

\autoref{fig:set-game-reward-gamified-being-checked} shows the application of GMIF model to set the game rewards in the gamified learning event \aspas{\emph{Gamified Being Checked}.} In this example, the learning path identified for the \emph{Apprentice/Yee Achiever} from the initial stage $s(0,0)$ - nothing for skill development and knowledge acquisition -  to the goal stage $s(3,0)$ - associative stage for skill development and nothing for knowledge acquisition - is based on the learning strategy \aspas{\emph{Learning by Apprenticeship}.} By the application of five-scale GMIF model, the label 
\aspas{$[1;2]$} in the transition $s(1,0) \to s(2,0)$ indicates that the proper challenges levels of 1:\emph{low} and 2:\emph{easy} are necessary to maintain the \emph{Apprentice/Yee Achiever} role holder in the flow state. These challenge levels in the five-scale three-channel flow model of skill development stages correspond to the rewards of \emph{+200 points} or \emph{+400 points} as the game rewards that will be given by the point-system in the game consequence event when the expected benefit for the \emph{Apprentice/Yee Achiver} role holder is the \emph{Development of Cognitive Skill (Exploratory-Cognitive stage)}. The label \aspas{$[3;3]$} in the transition $s(2,0) \to s(3,0)$ of GIMF model indicates that the challenge level to maintain the learner's flow state is 4:\emph{high}. This challenge level corresponds to the game reward \aspas{\emph{+800 points}} as the reward to be given by the point-system to maintain him/her in the flow state during the game consequence event when the expected benefit for the \emph{Apprentice/Yee Achiever} role holder is the \emph{Development of Cognitive Skill (Associative stage)}.

%%%%%%%%%%%%%%%%%%%%%%%%%%%%%%%%%%%%%%%%%%%%%%%%%%
\section{Concluding Remarks}
\label{sec:model-gmif-concluding-remarks} 

Balancing the challenge level of elements in learning scenarios according to the current learner's ability favors the learner's flow state in those scenario. This balancing incorporates the flow theory in the instructional/learning design process by means of a theory-based model that integrates the learner's growth process and the three-channel flow model. This new model, called GMIF model (\emph{Learner's \textbf{G}rowth \textbf{M}odel \textbf{I}mproved by \textbf{F}low Theory}), has been developed by labeling the LGM model (\emph{\textbf{L}earner's \textbf{G}rowth \textbf{M}odel}) with intervals that indicate the proper challenge levels to maintain the learner's flow state in the learning scenario.

An algorithm for labeling the LGM model with a n-scale of challenge levels, and then obtains the n-scale GMIF model, has also been proposed in this chapter. To demonstrate the usefulness of the n-scale GIMF model, an application to set the proper level of game rewards in gamified CL scenarios has been presented. This application has been illustrated providing support to define the points given by a point-system as game consequence events in gamified instructional and learning events. This algorithm and the n-scale GIMF model can be used in computer-based mechanisms and procedures to support the gamification of CL scenarios that favor the learner's flow. Furthermore, empirical studies were conducted to validate the application of GIMF model in the evaluation of the ontological engineering approach to gamify CL scenarios.


\chapter[Computer-based Mechanisms and Procedures to Gamify CL Scenarios]{Computer-based Mechanisms and Procedures to Gamify Collaborative Learning Scenarios}
\label{chapter:computer-based-mechanisms-procedures}

\begin{figure}[htb]
 \caption{Conceptual flow to gamify CL scenarios}
 \label{fig:conceptual-flow-gamify-cl-scenarios}
 \centering
% \includegraphics[width=1\textwidth]{images/chap-evaluation/graphical-empirical-studies.png}
 \fautor
\end{figure}


\chapter[Evaluation of the Ontological Engineering Approach to Gamify CL Scenarios]{Evaluation of the Ontological Engineering Approach to Gamify CL Scenarios}
\label{chapter:evaluation}

This Chapter undertakes the evaluation of the ontological engineering approach to gamify CL scenarios proposed in this dissertation. To demonstrate the effectiveness and efficiency of this approach in dealing with the motivation problem, four empirical studies, one pilot and tree full-scale empirical studies, were conducted at the University of São Paulo with computer science and computer engineering undergraduate students who participated in CL sessions ....

 empirical studies, as reported here, investigate the effects of these sessions on the students’ motivation and learning outcomes to demonstrate the effectiveness and efficiency of these sessions to deal with the motivation problem caused by the scripted collaboration.


This Chapter starts by presenting the formulation of the empirical studies in the \autoref{sec:formulation-empirical-studies} where it is detailed the scoping, hypothesis, subjects, instruments and data collection procedure of the empirical studies. Then, the \autoref{sec:pilot-study}, \autoref{sec:first-study}, \autoref{sec:second-study}, and \autoref{sec:third-study} present the results obtained in the four empirical studies. \autoref{sec:interpretation-implications} discuss the interpretation and the implications of the results obtained in the empirical studies in reference to the ontological engineering approach proposed in this PhD thesis. Finally, \autoref{sec:evaluation-conclusion} presents the conclusions and remarks of evaluation.

\section{Formulation of the Empirical Studies}
\label{sec:formulation-empirical-studies}

For the instructional designers and practitioners, the ontological engineering approach to gamify CL scenarios aims to give a structured guidance on how to gamify CL sessions for dealing with the motivation problem caused by scripted collaboration. With this guidance given by computer-based mechanisms in semantic-web intelligent theory-aware systems that use the knowledge described in the ontology OntoGaCLeS, the instructional designers and practitioners obtain CL sessions known as \aspas{ontology-based gamified CL sessions} (\emph{ont-gamified} CL sessions). In this sense, the ont-gamified CL sessions are the final products obtained by the ontological engineering approach, so that to demonstrate the  of the ontology en , it is necessary to investigate the effects of these sessions on the students motivation and learnings outcomes, 

and their is a correlation 

 and  to demonstrate their effectiveness and efficiency in to 




 to deal with the motivation problem, .

. The ontology-based CL sessions are considered the final product the , 

 empirical studies, as reported here, investigate the effects of these sessions on the students’ motivation and learning outcomes to demonstrate the effectiveness and efficiency of these sessions to deal with the motivation problem caused by the scripted collaboration.

\subsection{Scoping}

Following the template suggested by Wohlin et al. (2012) , the scoping of the empirical studies conducted as empirical evaluation in this dissertation  is to:

Analyze “the effects of ontology-based gamified CL sessions on the students’ motivation and learning outcomes” for the purpose of “validating the ontology engineering approach to gamify CL scenarios” with respect to “their effectiveness and efficiency to deal with the motivation problem caused by the scripted collaboration” from the point of view of the “instructional designers and practitioners who would like to know the benefits of these sessions” in the context of “CL activities in which the collaboration among participants is orchestrated by CSCL scripts.”



that are considered as final product of the ontological engineering approach 

 in which the game elements and their setting up are based on theories and practices related to gamification. 


As consequence, 


game elements are introduced and setting-up in the CL sessions to 

, , 

With this guidance , the instructional designers and practitioners gamify CL sessions, 

. , in which the game 




  As consequence of , the CL sessions are gamified by 

,  are the final products 

final users will 


These CL session, considered also as product of the 

henceforward 


a 

This support is given by 
for dealing with the motivation problem caused by the scripted collaboration. 

Employing the ontology-based support, the instructional designers and practitioners gamify CL sessions that are known as , and the 

\subsection{Scoping}



Considering this scoping, the evaluation of the ontological engineering approach to gamify CL scenarios had been organized in four empirical studies shown in \autoref{fig:graphical-empirical-studies}.

\begin{figure}[htb]
 \caption{Graphical representation of the empirical studies to evaluate the ontological engineering approach}
 \label{fig:graphical-empirical-studies}
 \centering
% \includegraphics[width=0.95\textwidth]{images/graphical-empirical-studies.png}
 \fautor
\end{figure}

The graphical representation for each empirical studies shown in \autoref{fig:graphical-empirical-studies} is an adaptation of the Evaluand-oriented Responsive Evaluation Model (CSCL-EREM) diagram \cite{Jorrin-AbellanStakeMartinez-Mone2009}. This diagram, in the original version, is an artifact that summarizes the characteristics that should be taken into account by researchers to conduct a CSCL evaluation. The diagram employed here is a simplified version of CSCL-EREM diagram that shows only the relevant aspects of evaluating the \emph{ontological engineering approach to gamify CL scenarios} by means of empirical studies. The context of empirical studies are described at the left- and right-upper sides of diagrams, where: the right-upper side indicates the environments in which the experiment is executed, and the left-upper side 

\subsection{Planning}

\subsubsection{Design}

\subsubsection{Analysis Procedure}


\subsubsection{Evaluation of validity}




, and similitudes and differences among these studies are summarized in \autoref{tab:summary-empirical-studies}


, and the right-upper side indicates the environment in which the empirical study is executed


In the diagrams, t

in which the learning environment where the  is .


The lower side of each circle constitutes 

 part indicates the context in which 

, so that the original version consists in the representation of relevant aspects of a Cevaluation  three facets of he

within the three facets of the model, 

% Please add the following required packages to your document preamble:
% \usepackage{longtable}
% Note: It may be necessary to compile the document several times to get a multi-page table to line up properly
\begin{longtable}[c]{lllll}
\caption{Summary of the empirical studies involved in the evaluation of the ontological engineering approach}
\label{tab:summary-empirical-studies}\\
                                                                          & Pilot study                                                                                                                                                                                                        & First Study                                                                                                                                                                       & Second Study                                                                                                                                                                            & Third Study                                                                                                                                                                                                                                                        \\
\endfirsthead
%
\multicolumn{5}{c}%
{{\bfseries Table \thetable\ continued from previous page}} \\
                                                                          & Pilot study                                                                                                                                                                                                        & First Study                                                                                                                                                                       & Second Study                                                                                                                                                                            & Third Study                                                                                                                                                                                                                                                        \\
\endhead
%
Scope                                                                     &                                                                                                                                                                                                                    &                                                                                                                                                                                   &                                                                                                                                                                                         &                                                                                                                                                                                                                                                                    \\
- Object of study                                                         & \multicolumn{4}{l}{Effects of ontology-based gamified CL sessions on the students' motivation and learning outcomes}                                                                                                                                                                                                                                                                                                                                                                                                                                                                                                                                                                                                                                                                                                                                                  \\
- Purpose                                                                 & \multicolumn{4}{l}{To validate the ontology engineering approach to gamify CL scenarios}                                                                                                                                                                                                                                                                                                                                                                                                                                                                                                                                                                                                                                                                                                                                                                              \\
- Quality focus                                                           & \multicolumn{3}{l}{Effectiveness to deal with the motivation problem caused by the scripted collaboration}                                                                                                                                                                                                                                                                                                                                                                                                                                                                                       & Efficiency to deal with the motivation problem caused by the scripted collaboration                                                                                                                                                                                \\
- Perspective                                                             & \multicolumn{4}{l}{Instructional designers and practitioners who would like to know the benefits of the ontology-based CL sessions}                                                                                                                                                                                                                                                                                                                                                                                                                                                                                                                                                                                                                                                                                                                                   \\
Context                                                                   & \multicolumn{4}{l}{\begin{tabular}[c]{@{}l@{}}CL activities in which the collaboration among participants is orchestrated and structured by CSCL scripts\\ - 04 specifics real situations in the course of Introduction to Computer Science at the University of São Paulo\end{tabular}}                                                                                                                                                                                                                                                                                                                                                                                                                                                                                                                                                                              \\
- Content-domain:                                                         & Loop structures                                                                                                                                                                                                    & Conditional structures                                                                                                                                                            & Loop structures                                                                                                                                                                         & Recursion                                                                                                                                                                                                                                                          \\
- Members per group:                                                      & 4 and 5 students                                                                                                                                                                                                   & 2 students                                                                                                                                                                        & 2 and 3 students                                                                                                                                                                        & 3 students                                                                                                                                                                                                                                                         \\
- Duration:                                                               & 1 week                                                                                                                                                                                                             & 1 week                                                                                                                                                                            & 1 week                                                                                                                                                                                  & 2 week                                                                                                                                                                                                                                                             \\
Hypothesis formulation:                                                   & \begin{tabular}[c]{@{}l@{}}- Hnull: There is not significant difference\\ - Hnull: There is not significant difference in the learning outcomes\\ - Hnull: Percentage of dropout of the CL activities\end{tabular} & \begin{tabular}[c]{@{}l@{}}- Hnull: There is not significant difference intrinsic motivation\\ - Hnull: There is not significant difference in the learning outcomes\end{tabular} & \begin{tabular}[c]{@{}l@{}}- Hnull: There is not significant difference in the level of motivation\\ - Hnull: There is not significant difference in the learning outcomes\end{tabular} & \begin{tabular}[c]{@{}l@{}}- Hnull: There is not significant difference in the intrinsic motivation\\ - Hnull: There is not significant difference in the level of motivation\\ - Hnull: There is not significant difference in the learning outcomes\end{tabular} \\
Variables:                                                                & \begin{tabular}[c]{@{}l@{}}- Intrinsic Motivation (Enjoyment/)\\ - Gain Score\\ \\ -\end{tabular}                                                                                                                  & \begin{tabular}[c]{@{}l@{}}- Intrinsic Motivation (Enjoyment/)\\ - Gain Score\end{tabular}                                                                                        &                                                                                                                                                                                         &                                                                                                                                                                                                                                                                    \\
\begin{tabular}[c]{@{}l@{}}Subjects:\\ (convenient sampling)\end{tabular} & \begin{tabular}[c]{@{}l@{}}Undergraduate Computer Science Students\\ Signed Up in the first bacheler degree\end{tabular}                                                                                           & \multicolumn{3}{l}{Undergraduate Engineering Computer Students}                                                                                                                                                                                                                                                                                                                                                                                                                                                                                                                                                                                  \\
Experiment Design:                                                        &                                                                                                                                                                                                                    &                                                                                                                                                                                   &                                                                                                                                                                                         &                                                                                                                                                                                                                                                                    \\
Instrumentation:                                                          &                                                                                                                                                                                                                    &                                                                                                                                                                                   &                                                                                                                                                                                         &                                                                                                                                                                                                                                                                   
\end{longtable}


The mathematical analysis methods and procedure used in the data analysis are detailed in \autoref{sec:evaluation-analysis-procedure} .

\subsection{Design}

\subsection{Subjects}

\subsection{Objects}

\subsection{Instrumentation}

\subsection{Data Collection Procedure}

In order to find out whether the \emph{ont-gamified CL sessions} affects the students' motivation 


\subsection{Data Analysis Procedure}
\label{sec:evaluation-analysis-procedure}

Prior to perform any statistical analysis related to the motivation of participants in non-gamified, ont-gamified and w/o-gamified CL sessions, outliers refer as careless responses were removed from the self-reported data gathered by motivation surveys. Outliers identified as extreme values in these data were replaced by trimmed minimum and maximum values using the Winsorization method. The detection and treatment of careless responses and extreme values is detailed in \autoref{sec:outliers-motivation}.

To measure the students' motivation towards their participation in the CL sessions, the Rating Scale Model (RSM) was used as a psychometric instrument for analyzing the self-reported data collected by the motivation surveys. The RSM is a model based on the Item Response Theory (IRT) in which the latent trait being measured by the items is estimated in function of rating scale data \cite{George2005}. In this sense, the RMS is appropriate for the data gathered through the motivation surveys in which all the items have Likert scale response format \cite{van2013handbook}. \autoref{appendix:rsm-motivation} details the measurements of students' motivation obtained using the RSM. After having these measurements, two-way ANOVA tests had been carried out to compare the effects of different CL sessions on the students' motivation. The results on these tests had been calculated employing variations between the types of CL sessions, as well as variations between the CL roles played by the participants in these sessions.

To investigate the impact of different types of CL sessions on the learning outcomes, the gains in knowledge/skills of participants had been estimated by the Nominal Item Response Model \cite{DavidThissenLiCaiRDarrellBock2010}, and by stacking the pre-test and post-test data \cite{Wright2003}. \autoref{appendix:stacking-nominal-model} details this stacked data analysis employing the Nominal Item Response Model. After having these results, two-way ANOVA tests had carried out by comparing the learning outcomes in the different type of CL sessions and the CL roles played by the participants.

After to carried out the ANOVA tests by evaluating whether there is no significant differences on the students' motivation and learning outcomes, Pearson correlation tests had been carried out to find out whether the effects of different types of CL sessions on the students' motivation and learning outcomes are linked.

\subsection{Evaluation of validity}



After to remove careless responses and to winsorize extreme values, the validation of motivation surveys was performed using confirmatory factorial analysis (CFA) and reliability analyses for all the responses collected in each empirical study.

The validation of assumptions for the ANOVA tests and the results obtained by these tests are detailed in \autoref{appendix:anova-motivation}.

 The validation of assumptions for the ANOVA tests and the results obtained by these tests are detailed in \autoref{appendix:anova-learning-outcomes}.

This appendix also details the validation of assumptions in these measurements. 

\section{Pilot Empirical Study}
\label{sec:pilot-study}

\subsection{Execution}

\subsubsection{Hypotheis testing}

Scoping;
Selection Variables
Selection of subjects
Formulation of Hypothesis
Instrument:

%%%%%%%%%%%%%%%%%%%%%%%%%%%%%%%%%%%%%%%%%%%%%%%%

\section{Pilot Empirical Study: Data Analysis and Results}
\label{sec:data-analysis-pilot-study}

Two-way between-subjects ANOVA tests were conducted to compare the effects of ontology-based gamified CL sessions (\emph{ont-gamified}) and non-gamified CL sessions (\emph{non-gamfied}) on the participants' intrinsic motivation, perceived choice, pressure/tension and effort/importance. The interaction effects between these two types of CL sessions and the CL roles, \emph{Master} and \emph{Apprentice}, are evaluated in these tests. \autoref{tab:two-way-intrinsic-motivation-pilot-study}  shows the results in which there are statistically significant difference at the $0.05$ level for the participants' intrinsic motivation and perceived choice. The effect on the intrinsic motivation for the type of CL session yielded an $F$ ratio of $F(1,26) = 4.702$, $p = 0.039$. In relation to the perceived choice, the effect for the type of CL session yielded an $F$ ratio of $F(1,26) = 6.980$, $p = 0.014$, indicating significant differences between non-gamified CL sessions ($mean = -0.389$ \emph{logit}, and $SE = 0.273$) and ontology-based gamified CL sessions ($mean = 0.305$ \emph{logit}, and $SE = 0.206$).

%latex.default(result_df, caption = paste("Summary of two-way ANOVA results",     in_title), size = "small", longtable = T, ctable = F, landscape = F,     rowlabel = "", where = "!htbp", file = filename, append = T)%
\setlongtables{\small
\begin{longtable}{lrrrrl}\caption{Two-way ANOVA results for the latent trait estimates of intrinsic motivation, interest/enjoyment, perceived choice, pressure/tension and effort/importance in the pilot empirical study} \tabularnewline
\hline\hline
\multicolumn{1}{l}{}&\multicolumn{1}{c}{Sum Sq}&\multicolumn{1}{c}{Df}&\multicolumn{1}{c}{F value}&\multicolumn{1}{c}{Pr(\textgreater F)}&\multicolumn{1}{c}{Sig}\tabularnewline
\hline
\endfirsthead\caption[]{\em (continued)} \tabularnewline
\hline
\multicolumn{1}{l}{}&\multicolumn{1}{c}{Sum Sq}&\multicolumn{1}{c}{Df}&\multicolumn{1}{c}{F value}&\multicolumn{1}{c}{Pr(\textgreater F)}&\multicolumn{1}{c}{Sig}\tabularnewline
\hline
\endhead
\hline
\multicolumn{6}{r}{\tiny Signif. codes:  0 \aspas{**} 0.01 \aspas{*} 0.05}
\endfoot
\label{tab:two-way-intrinsic-motivation-pilot-study}
%Intrinsic Motivation:(Intercept)&$ 0.042$&$ 1$&$0.082$&$0.777$&\tabularnewline
Intrinsic Motivation:Type&$ 2.397$&$ 1$&$4.702$&$0.039$&*\tabularnewline
Intrinsic Motivation:CLRole&$ 0.456$&$ 1$&$0.894$&$0.353$&\tabularnewline
Intrinsic Motivation:Type:CLRole&$ 0.080$&$ 1$&$0.156$&$0.696$&\tabularnewline
%Intrinsic Motivation:Residuals&$13.254$&$26$&$$&$$&\tabularnewline

%Interest/Enjoyment:(Intercept)&$ 0.483$&$ 1$&$0.208$&$0.652$&$$\tabularnewline
Interest/Enjoyment:Type&$ 8.107$&$ 1$&$3.495$&$0.073$&$$\tabularnewline
Interest/Enjoyment:CLRole&$ 1.879$&$ 1$&$0.810$&$0.376$&$$\tabularnewline
Interest/Enjoyment:Type:CLRole&$ 0.599$&$ 1$&$0.258$&$0.616$&$$\tabularnewline
%Interest/Enjoyment:Residuals&$60.314$&$26$&$$&$$&$$\tabularnewline

%Perceived Choice:(Intercept)&$ 0.103$&$ 1$&$0.195$&$0.663$&\tabularnewline
Perceived Choice:Type&$ 3.675$&$ 1$&$6.980$&$0.014$&*\tabularnewline
Perceived Choice:CLRole&$ 0.066$&$ 1$&$0.126$&$0.725$&\tabularnewline
Perceived Choice:Type:CLRole&$ 1.050$&$ 1$&$1.994$&$0.170$&\tabularnewline
%Perceived Choice:Residuals&$13.689$&$26$&$$&$$&\tabularnewline

%Pressure/Tension:(Intercept)&$ 0.017$&$ 1$&$0.036$&$0.850$&$$\tabularnewline
Pressure/Tension:Type&$ 1.125$&$ 1$&$2.472$&$0.128$&$$\tabularnewline
Pressure/Tension:CLRole&$ 0.027$&$ 1$&$0.060$&$0.809$&$$\tabularnewline
Pressure/Tension:Type:CLRole&$ 0.012$&$ 1$&$0.026$&$0.874$&$$\tabularnewline
%Pressure/Tension:Residuals&$11.838$&$26$&$$&$$&$$\tabularnewline

%Effort/Importance:(Intercept)&$ 0.010$&$ 1$&$0.019$&$0.892$&$$\tabularnewline
Effort/Importance:Type&$ 0.335$&$ 1$&$0.645$&$0.429$&$$\tabularnewline
Effort/Importance:CLRole&$ 0.068$&$ 1$&$0.130$&$0.721$&$$\tabularnewline
Effort/Importance:Type:CLRole&$ 0.273$&$ 1$&$0.525$&$0.475$&$$\tabularnewline
%Effort/Importance:Residuals&$13.516$&$26$&$$&$$&$$\tabularnewline
\hline
\end{longtable}}

Tukey post-hoc comparisons had been run to confirm where the significant differences occurred between the types of CL sessions and the CL roles. \autoref{tab:post-hoc-intrinsic-motivation-pilot-study} summarizes the descriptive statistics and the results of post-hoc comparisons. According to these results, the intrinsic motivation of students who participated in ontology-based gamified CL sessions ($lsmean=0.419$ \emph{logit}, and $SE = 0.206$) is greater than the intrinsic motivation of students who participated in non-gamified CL sessions ($lsmean=-0.322$ \emph{logit}, and $SE = 0.273$) with a p-adj. value of $0.013$ and Hedges' $g=0.956$ large effect size. The interest/enjoyment ($lsmean=0.848$ \emph{logit}, and $SE = 0.440$) of participants in ontology-based gamified CL sessions  is greater than the interest/enjoyment ($lsmean=-0.515$ \emph{logit}, and $SE = 0.582$) of participants in non-gamified CL sessions with a p-adj. value of $0.039$ and Hedges' $g=0.780$ medium effect size. In ontology-based gamified CL sessions, the perceived choice ($lsmean=0.382$ \emph{logit}, and $SE = 0.209$) is significantly greater than the perceived choice in non-gamified CL sessions ($lsmean=-0.918$ \emph{logit}, and $SE = 0.347$) with a p-adj. value of $0.031$ and Hedges' $g=0.814$ large effect size.

%latex.default(post_hoc_df, caption = paste("Descriptive statistics and Tukey post-hoc test results",     in_title), size = "small", longtable = T, ctable = F, landscape = T,     rowlabel = "", where = "!htbp", file = filename, append = T)%
\setlongtables\begin{landscape}{\scriptsize
\begin{longtable}{lrrrrrrrrrrrll}\caption{Descriptive statistics and Tukey post-hoc test results for the latent trait estimates of intrinsic motivation, interest/enjoyment, perceived choice, pressure/tension and effort/importance in the pilot empirical study} \tabularnewline
\hline\hline
\multicolumn{1}{l}{}&\multicolumn{1}{c}{N}&\multicolumn{1}{c}{mean}&\multicolumn{1}{c}{lsmean}&\multicolumn{1}{c}{SE}&\multicolumn{1}{c}{df}&\multicolumn{1}{c}{lwr.CI}&\multicolumn{1}{c}{upr.CI}&\multicolumn{1}{c}{t.ratio}&\multicolumn{1}{c}{p.value}&\multicolumn{1}{c}{p-adj.}&\multicolumn{1}{c}{g}&\multicolumn{1}{c}{sig}&\multicolumn{1}{c}{mag}\tabularnewline
\hline
\endfirsthead\caption[]{\em (continued)} \tabularnewline
\hline
\multicolumn{1}{l}{}&\multicolumn{1}{c}{N}&\multicolumn{1}{c}{mean}&\multicolumn{1}{c}{lsmean}&\multicolumn{1}{c}{SE}&\multicolumn{1}{c}{df}&\multicolumn{1}{c}{lwr.CI}&\multicolumn{1}{c}{upr.CI}&\multicolumn{1}{c}{t.ratio}&\multicolumn{1}{c}{p.value}&\multicolumn{1}{c}{p-adj.}&\multicolumn{1}{c}{g}&\multicolumn{1}{c}{sig}&\multicolumn{1}{c}{mag}\tabularnewline
\hline
\endhead
\hline
\multicolumn{14}{r}{\tiny Signif. codes:  0 \aspas{**} 0.01 \aspas{*} 0.05} 
\endfoot
\label{tab:post-hoc-intrinsic-motivation-pilot-study}
Intrinsic Motivation:non-gamified&$14$&$-0.389$&$-0.322$&$0.273$&$26$&$-0.882$&$ 0.239$&$$&$$&$$&$$&&\tabularnewline
Intrinsic Motivation:ont-gamified&$16$&$ 0.305$&$ 0.419$&$0.206$&$26$&$-0.004$&$ 0.843$&$$&$$&$$&$$&&\tabularnewline
Intrinsic Motivation:non-gamified - ont-gamified&$30$&$-0.694$&$-0.741$&$0.342$&$$&$-1.231$&$-0.157$&$-2.168$&$0.039$&$0.013$&$-0.956$&*&large\tabularnewline
Intrinsic Motivation:non-gamified.Apprentice&$12$&$-0.416$&$-0.416$&$0.206$&$26$&$-0.839$&$ 0.008$&$$&$$&$$&$$&&\tabularnewline
Intrinsic Motivation:ont-gamified.Apprentice&$12$&$ 0.190$&$ 0.190$&$0.206$&$26$&$-0.233$&$ 0.614$&$$&$$&$$&$$&&\tabularnewline
Intrinsic Motivation:non-gamified.Apprentice - ont-gamified.Apprentice&$24$&$-0.606$&$-0.606$&$0.291$&$$&$-1.406$&$ 0.194$&$-2.079$&$0.048$&$0.186$&$-0.818$&&\tabularnewline
Intrinsic Motivation:non-gamified.Master&$ 2$&$-0.228$&$-0.228$&$0.505$&$26$&$-1.266$&$ 0.810$&$$&$$&$$&$$&&\tabularnewline
Intrinsic Motivation:ont-gamified.Master&$ 4$&$ 0.649$&$ 0.649$&$0.357$&$26$&$-0.085$&$ 1.382$&$$&$$&$$&$$&&\tabularnewline
Intrinsic Motivation:non-gamified.Master - ont-gamified.Master&$ 6$&$-0.876$&$-0.876$&$0.618$&$$&$-2.573$&$ 0.820$&$-1.417$&$0.168$&$0.500$&$-0.991$&&\tabularnewline
\hline

Interest/Enjoyment:non-gamified&$14$&$-0.617$&$-0.515$&$0.582$&$26$&$-1.711$&$ 0.680$&$$&$$&$$&$$&&\tabularnewline
Interest/Enjoyment:ont-gamified&$16$&$ 0.591$&$ 0.848$&$0.440$&$26$&$-0.056$&$ 1.752$&$$&$$&$$&$$&&\tabularnewline
Interest/Enjoyment:non-gamified - ont-gamified&$30$&$-1.208$&$-1.363$&$0.729$&$$&$-2.354$&$-0.063$&$-1.869$&$0.073$&$0.039$&$-0.780$&*&medium\tabularnewline
Interest/Enjoyment:non-gamified.Apprentice&$12$&$-0.658$&$-0.658$&$0.440$&$26$&$-1.562$&$ 0.246$&$$&$$&$$&$$&&\tabularnewline
Interest/Enjoyment:ont-gamified.Apprentice&$12$&$ 0.335$&$ 0.335$&$0.440$&$26$&$-0.569$&$ 1.238$&$$&$$&$$&$$&&\tabularnewline
Interest/Enjoyment:non-gamified.Apprentice - ont-gamified.Apprentice&$24$&$-0.993$&$-0.993$&$0.622$&$$&$-2.698$&$ 0.713$&$-1.596$&$0.123$&$0.398$&$-0.612$&&\tabularnewline
Interest/Enjoyment:non-gamified.Master&$ 2$&$-0.372$&$-0.372$&$1.077$&$26$&$-2.586$&$ 1.841$&$$&$$&$$&$$&&\tabularnewline
Interest/Enjoyment:ont-gamified.Master&$ 4$&$ 1.361$&$ 1.361$&$0.762$&$26$&$-0.204$&$ 2.927$&$$&$$&$$&$$&&\tabularnewline
Interest/Enjoyment:non-gamified.Master - ont-gamified.Master&$ 6$&$-1.734$&$-1.734$&$1.319$&$$&$-5.352$&$ 1.885$&$-1.314$&$0.200$&$0.562$&$-1.100$&&\tabularnewline
\hline

Perceived Choice:non-gamified&$14$&$-0.316$&$-0.535$&$0.277$&$26$&$-1.105$&$ 0.034$&$$&$$&$$&$$&&\tabularnewline
Perceived Choice:ont-gamified&$16$&$ 0.290$&$ 0.382$&$0.209$&$26$&$-0.048$&$ 0.813$&$$&$$&$$&$$&&\tabularnewline
Perceived Choice:non-gamified - ont-gamified&$30$&$-0.607$&$-0.918$&$0.347$&$$&$-1.152$&$-0.061$&$-2.642$&$0.014$&$0.031$&$-0.814$&*&large\tabularnewline
Perceived Choice:non-gamified.Apprentice&$12$&$-0.229$&$-0.229$&$0.209$&$26$&$-0.659$&$ 0.202$&$$&$$&$$&$$&&\tabularnewline
Perceived Choice:ont-gamified.Apprentice&$12$&$ 0.199$&$ 0.199$&$0.209$&$26$&$-0.232$&$ 0.629$&$$&$$&$$&$$&&\tabularnewline
Perceived Choice:non-gamified.Apprentice - ont-gamified.Apprentice&$24$&$-0.427$&$-0.427$&$0.296$&$$&$-1.240$&$ 0.386$&$-1.442$&$0.161$&$0.486$&$-0.557$&&\tabularnewline
Perceived Choice:non-gamified.Master&$ 2$&$-0.842$&$-0.842$&$0.513$&$26$&$-1.897$&$ 0.212$&$$&$$&$$&$$&&\tabularnewline
Perceived Choice:ont-gamified.Master&$ 4$&$ 0.566$&$ 0.566$&$0.363$&$26$&$-0.180$&$ 1.312$&$$&$$&$$&$$&&\tabularnewline
Perceived Choice:non-gamified.Master - ont-gamified.Master&$ 6$&$-1.408$&$-1.408$&$0.628$&$$&$-3.132$&$ 0.316$&$-2.241$&$0.034$&$0.139$&$-1.767$&&\tabularnewline
\newpage

Pressure/Tension:non-gamified&$14$&$ 0.238$&$ 0.285$&$0.258$&$26$&$-0.245$&$0.814$&$$&$$&$$&$$&$$&$$\tabularnewline
Pressure/Tension:ont-gamified&$16$&$-0.230$&$-0.223$&$0.195$&$26$&$-0.624$&$0.177$&$$&$$&$$&$$&$$&$$\tabularnewline
Pressure/Tension:non-gamified - ont-gamified&$30$&$ 0.468$&$ 0.508$&$0.323$&$$&$-0.040$&$0.976$&$1.572$&$0.128$&$0.069$&$0.699$&$$&$$\tabularnewline
Pressure/Tension:non-gamified.Apprentice&$12$&$ 0.219$&$ 0.219$&$0.195$&$26$&$-0.181$&$0.620$&$$&$$&$$&$$&$$&$$\tabularnewline
Pressure/Tension:ont-gamified.Apprentice&$12$&$-0.237$&$-0.237$&$0.195$&$26$&$-0.637$&$0.164$&$$&$$&$$&$$&$$&$$\tabularnewline
Pressure/Tension:non-gamified.Apprentice - ont-gamified.Apprentice&$24$&$ 0.456$&$ 0.456$&$0.275$&$$&$-0.300$&$1.212$&$1.656$&$0.110$&$0.366$&$0.624$&$$&$$\tabularnewline
Pressure/Tension:non-gamified.Master&$ 2$&$ 0.350$&$ 0.350$&$0.477$&$26$&$-0.631$&$1.331$&$$&$$&$$&$$&$$&$$\tabularnewline
Pressure/Tension:ont-gamified.Master&$ 4$&$-0.210$&$-0.210$&$0.337$&$26$&$-0.903$&$0.484$&$$&$$&$$&$$&$$&$$\tabularnewline
Pressure/Tension:non-gamified.Master - ont-gamified.Master&$ 6$&$ 0.560$&$ 0.560$&$0.584$&$$&$-1.044$&$2.163$&$0.958$&$0.347$&$0.774$&$0.950$&$$&$$\tabularnewline
\hline

Effort/Importance:non-gamified&$14$&$ 0.028$&$ 0.162$&$0.275$&$26$&$-0.404$&$0.728$&$$&$$&$$&$$&$$&$$\tabularnewline
Effort/Importance:ont-gamified&$16$&$-0.084$&$-0.115$&$0.208$&$26$&$-0.543$&$0.313$&$$&$$&$$&$$&$$&$$\tabularnewline
Effort/Importance:non-gamified - ont-gamified&$30$&$ 0.112$&$ 0.277$&$0.345$&$$&$-0.430$&$0.654$&$0.803$&$0.429$&$0.674$&$0.155$&$$&$$\tabularnewline
Effort/Importance:non-gamified.Apprentice&$12$&$-0.025$&$-0.025$&$0.208$&$26$&$-0.453$&$0.403$&$$&$$&$$&$$&$$&$$\tabularnewline
Effort/Importance:ont-gamified.Apprentice&$12$&$-0.052$&$-0.052$&$0.208$&$26$&$-0.480$&$0.376$&$$&$$&$$&$$&$$&$$\tabularnewline
Effort/Importance:non-gamified.Apprentice - ont-gamified.Apprentice&$24$&$ 0.027$&$ 0.027$&$0.294$&$$&$-0.780$&$0.835$&$0.092$&$0.927$&$1.000$&$0.037$&$$&$$\tabularnewline
Effort/Importance:non-gamified.Master&$ 2$&$ 0.349$&$ 0.349$&$0.510$&$26$&$-0.699$&$1.397$&$$&$$&$$&$$&$$&$$\tabularnewline
Effort/Importance:ont-gamified.Master&$ 4$&$-0.178$&$-0.178$&$0.361$&$26$&$-0.919$&$0.563$&$$&$$&$$&$$&$$&$$\tabularnewline
Effort/Importance:non-gamified.Master - ont-gamified.Master&$ 6$&$ 0.527$&$ 0.527$&$0.624$&$$&$-1.186$&$2.240$&$0.844$&$0.406$&$0.833$&$0.545$&$$&$$\tabularnewline
\hline
\end{longtable}}\end{landscape}


%%%%%%%%%%%%%%%%%%%%%%%%%%%%%%%%%%%%%%%%%%%%%%%%
\section{First Empirical Study: Data Analysis and Results}
\label{sec:first-study}


%\begin{longtable}{lrrrrl}\caption{Two-way ANOVA results for the latent trait estimates of Intrinsic Motivation, Interest/Enjoyment, Perceived Choice, Pressure/Tension and Effort/Importance in the pilot empirical study} \tabularnewline



%latex.default(result_df, caption = paste("Summary of two-way ANOVA results",     in_title), size = "small", longtable = T, ctable = F, landscape = F,     rowlabel = "", where = "!htbp", file = filename, append = T)%
\setlongtables{\small
\begin{longtable}{lrrrrl}\caption{Two-way ANOVA for the latent trait estimates of Intrinsic Motivation, in the first empirical study} \tabularnewline
\hline\hline
\multicolumn{1}{l}{}&\multicolumn{1}{c}{Sum Sq}&\multicolumn{1}{c}{Df}&\multicolumn{1}{c}{F value}&\multicolumn{1}{c}{Pr(\textgreater F)}&\multicolumn{1}{c}{Sig}\tabularnewline
\hline
\endfirsthead\caption[]{\em (continued)} \tabularnewline
\hline
\multicolumn{1}{l}{}&\multicolumn{1}{c}{Sum Sq}&\multicolumn{1}{c}{Df}&\multicolumn{1}{c}{F value}&\multicolumn{1}{c}{Pr(\textgreater F)}&\multicolumn{1}{c}{Sig}\tabularnewline
\hline
\endhead
\hline
\multicolumn{6}{r}{\tiny Signif. codes:  0 \aspas{**} 0.01 \aspas{*} 0.05}
\endfoot
\label{result}
%Intrinsic Motivation:(Intercept)&$ 0.579$&$ 1$&$1.520$&$0.223$&\tabularnewline
Intrinsic Motivation:Type&$ 3.087$&$ 1$&$8.103$&$0.006$&**\tabularnewline
Intrinsic Motivation:CLRole&$ 0.064$&$ 1$&$0.168$&$0.683$&\tabularnewline
Intrinsic Motivation:Type:CLRole&$ 0.020$&$ 1$&$0.053$&$0.819$&\tabularnewline
%Intrinsic Motivation:Residuals&$21.333$&$56$&$$&$$&\tabularnewline

%Interest/Enjoyment:(Intercept)&$ 0.421$&$ 1$&$0.419$&$0.520$&$$\tabularnewline
Interest/Enjoyment:Type&$ 0.946$&$ 1$&$0.942$&$0.336$&$$\tabularnewline
Interest/Enjoyment:CLRole&$ 0.268$&$ 1$&$0.267$&$0.607$&$$\tabularnewline
Interest/Enjoyment:Type:CLRole&$ 2.003$&$ 1$&$1.993$&$0.164$&$$\tabularnewline
%Interest/Enjoyment:Residuals&$52.258$&$52$&$$&$$&$$\tabularnewline

%Perceived Choice:(Intercept)&$ 0.016$&$ 1$&$0.020$&$0.889$&\tabularnewline
Perceived Choice:Type&$ 6.371$&$ 1$&$7.885$&$0.007$&**\tabularnewline
Perceived Choice:CLRole&$ 0.229$&$ 1$&$0.283$&$0.597$&\tabularnewline
Perceived Choice:Type:CLRole&$ 0.050$&$ 1$&$0.061$&$0.805$&\tabularnewline
%Perceived Choice:Residuals&$45.244$&$56$&$$&$$&\tabularnewline

%Pressure/Tension:(Intercept)&$ 5.506$&$ 1$&$5.333$&$0.025$&\tabularnewline
Pressure/Tension:Type&$ 2.428$&$ 1$&$2.352$&$0.131$&\tabularnewline
Pressure/Tension:CLRole&$ 0.011$&$ 1$&$0.011$&$0.918$&\tabularnewline
Pressure/Tension:Type:CLRole&$ 6.351$&$ 1$&$6.151$&$0.016$&*\tabularnewline
%Pressure/Tension:Residuals&$57.815$&$56$&$$&$$&\tabularnewline

%Effort/Importance:(Intercept)&$ 0.078$&$ 1$&$0.086$&$0.770$&\tabularnewline
Effort/Importance:Type&$ 3.861$&$ 1$&$4.303$&$0.043$&*\tabularnewline
Effort/Importance:CLRole&$ 2.374$&$ 1$&$2.646$&$0.109$&\tabularnewline
Effort/Importance:Type:CLRole&$ 0.799$&$ 1$&$0.891$&$0.349$&\tabularnewline
%Effort/Importance:Residuals&$50.244$&$56$&$$&$$&\tabularnewline

\hline
\end{longtable}}




%latex.default(post_hoc_df, caption = paste("Descriptive statistics and Tukey post-hoc test results",     in_title), size = "small", longtable = T, ctable = F, landscape = T,     rowlabel = "", where = "!htbp", file = filename, append = T)%
\setlongtables\begin{landscape}{\scriptsize
\begin{longtable}{lrrrrrrrrrrrll}\caption{Descriptive statistics and Tukey post-hoc test results  for the latent trait estimates of Intrinsic Motivation in the first empirical study} \tabularnewline
\hline\hline
\multicolumn{1}{l}{}&\multicolumn{1}{c}{N}&\multicolumn{1}{c}{mean}&\multicolumn{1}{c}{lsmean}&\multicolumn{1}{c}{SE}&\multicolumn{1}{c}{df}&\multicolumn{1}{c}{lwr.CI}&\multicolumn{1}{c}{upr.CI}&\multicolumn{1}{c}{t.ratio}&\multicolumn{1}{c}{p.value}&\multicolumn{1}{c}{p-adj.}&\multicolumn{1}{c}{g}&\multicolumn{1}{c}{sig}&\multicolumn{1}{c}{mag}\tabularnewline
\hline
\endfirsthead\caption[]{\em (continued)} \tabularnewline
\hline
\multicolumn{1}{l}{}&\multicolumn{1}{c}{N}&\multicolumn{1}{c}{mean}&\multicolumn{1}{c}{lsmean}&\multicolumn{1}{c}{SE}&\multicolumn{1}{c}{df}&\multicolumn{1}{c}{lwr.CI}&\multicolumn{1}{c}{upr.CI}&\multicolumn{1}{c}{t.ratio}&\multicolumn{1}{c}{p.value}&\multicolumn{1}{c}{p-adj.}&\multicolumn{1}{c}{g}&\multicolumn{1}{c}{sig}&\multicolumn{1}{c}{mag}\tabularnewline
\hline
\endhead
\hline
\multicolumn{14}{r}{\tiny Signif. codes:  0 \aspas{**} 0.01 \aspas{*} 0.05}
\endfoot
\label{post}
Intrinsic Motivation:non-gamified&$30$&$-0.329$&$-0.325$&$0.113$&$56$&$-0.552$&$-0.099$&$$&$$&$$&$$&&\tabularnewline
Intrinsic Motivation:ont-gamified&$30$&$ 0.129$&$ 0.129$&$0.113$&$56$&$-0.097$&$ 0.354$&$$&$$&$$&$$&&\tabularnewline
Intrinsic Motivation:non-gamified - ont-gamified&$60$&$-0.458$&$-0.454$&$0.160$&$$&$-0.777$&$-0.138$&$-2.847$&$0.006$&$0.006$&$-0.743$&**&medium\tabularnewline
Intrinsic Motivation:non-gamified.Apprentice&$14$&$-0.274$&$-0.274$&$0.165$&$56$&$-0.605$&$ 0.056$&$$&$$&$$&$$&&\tabularnewline
Intrinsic Motivation:ont-gamified.Apprentice&$15$&$ 0.143$&$ 0.143$&$0.159$&$56$&$-0.176$&$ 0.462$&$$&$$&$$&$$&&\tabularnewline
Intrinsic Motivation:non-gamified.Apprentice - ont-gamified.Apprentice&$29$&$-0.418$&$-0.418$&$0.229$&$$&$-1.025$&$ 0.190$&$-1.820$&$0.074$&$0.275$&$-0.583$&&\tabularnewline
Intrinsic Motivation:non-gamified.Master&$16$&$-0.376$&$-0.376$&$0.154$&$56$&$-0.686$&$-0.067$&$$&$$&$$&$$&&\tabularnewline
Intrinsic Motivation:ont-gamified.Master&$15$&$ 0.114$&$ 0.114$&$0.159$&$56$&$-0.205$&$ 0.434$&$$&$$&$$&$$&&\tabularnewline
Intrinsic Motivation:non-gamified.Master - ont-gamified.Master&$31$&$-0.491$&$-0.491$&$0.222$&$$&$-1.078$&$ 0.097$&$-2.212$&$0.031$&$0.132$&$-0.896$&&\tabularnewline
\hline

Interest/Enjoyment:non-gamified&$28$&$-0.217$&$-0.217$&$0.189$&$52$&$-0.597$&$0.163$&$$&$$&$$&$$&$$&$$\tabularnewline
Interest/Enjoyment:ont-gamified&$28$&$ 0.062$&$ 0.043$&$0.190$&$52$&$-0.338$&$0.424$&$$&$$&$$&$$&$$&$$\tabularnewline
Interest/Enjoyment:non-gamified - ont-gamified&$56$&$-0.279$&$-0.260$&$0.268$&$$&$-0.816$&$0.259$&$-0.970$&$0.336$&$0.303$&$-0.274$&$$&$$\tabularnewline
Interest/Enjoyment:non-gamified.Apprentice&$14$&$-0.097$&$-0.097$&$0.268$&$52$&$-0.635$&$0.441$&$$&$$&$$&$$&$$&$$\tabularnewline
Interest/Enjoyment:ont-gamified.Apprentice&$13$&$-0.215$&$-0.215$&$0.278$&$52$&$-0.773$&$0.343$&$$&$$&$$&$$&$$&$$\tabularnewline
Interest/Enjoyment:non-gamified.Apprentice - ont-gamified.Apprentice&$27$&$ 0.118$&$ 0.118$&$0.386$&$$&$-0.906$&$1.143$&$ 0.307$&$0.760$&$0.990$&$ 0.106$&$$&$$\tabularnewline
Interest/Enjoyment:non-gamified.Master&$14$&$-0.337$&$-0.337$&$0.268$&$52$&$-0.875$&$0.201$&$$&$$&$$&$$&$$&$$\tabularnewline
Interest/Enjoyment:ont-gamified.Master&$15$&$ 0.302$&$ 0.302$&$0.259$&$52$&$-0.217$&$0.821$&$$&$$&$$&$$&$$&$$\tabularnewline
Interest/Enjoyment:non-gamified.Master - ont-gamified.Master&$29$&$-0.639$&$-0.639$&$0.373$&$$&$-1.628$&$0.350$&$-1.715$&$0.092$&$0.326$&$-0.678$&$$&$$\tabularnewline
\hline

Perceived Choice:non-gamified&$30$&$-0.340$&$-0.343$&$0.164$&$56$&$-0.672$&$-0.013$&$$&$$&$$&$$&&\tabularnewline
Perceived Choice:ont-gamified&$30$&$ 0.310$&$ 0.310$&$0.164$&$56$&$-0.019$&$ 0.639$&$$&$$&$$&$$&&\tabularnewline
Perceived Choice:non-gamified - ont-gamified&$60$&$-0.650$&$-0.652$&$0.232$&$$&$-1.115$&$-0.185$&$-2.808$&$0.007$&$0.007$&$-0.724$&**&medium\tabularnewline
Perceived Choice:non-gamified.Apprentice&$14$&$-0.376$&$-0.376$&$0.240$&$56$&$-0.857$&$ 0.106$&$$&$$&$$&$$&&\tabularnewline
Perceived Choice:ont-gamified.Apprentice&$15$&$ 0.219$&$ 0.219$&$0.232$&$56$&$-0.246$&$ 0.684$&$$&$$&$$&$$&&\tabularnewline
Perceived Choice:non-gamified.Apprentice - ont-gamified.Apprentice&$29$&$-0.595$&$-0.595$&$0.334$&$$&$-1.479$&$ 0.290$&$-1.781$&$0.080$&$0.293$&$-0.617$&&\tabularnewline
Perceived Choice:non-gamified.Master&$16$&$-0.310$&$-0.310$&$0.225$&$56$&$-0.760$&$ 0.141$&$$&$$&$$&$$&&\tabularnewline
Perceived Choice:ont-gamified.Master&$15$&$ 0.401$&$ 0.401$&$0.232$&$56$&$-0.064$&$ 0.865$&$$&$$&$$&$$&&\tabularnewline
Perceived Choice:non-gamified.Master - ont-gamified.Master&$31$&$-0.710$&$-0.710$&$0.323$&$$&$-1.565$&$ 0.145$&$-2.198$&$0.032$&$0.136$&$-0.802$&&\tabularnewline
\newpage

Pressure/Tension:non-gamified&$30$&$ 0.484$&$ 0.505$&$0.186$&$56$&$ 0.132$&$0.877$&$$&$$&$$&$$&&\tabularnewline
Pressure/Tension:ont-gamified&$30$&$ 0.102$&$ 0.102$&$0.186$&$56$&$-0.270$&$0.473$&$$&$$&$$&$$&&\tabularnewline
Pressure/Tension:non-gamified - ont-gamified&$60$&$ 0.382$&$ 0.403$&$0.263$&$$&$-0.144$&$0.908$&$ 1.534$&$0.131$&$0.151$&$ 0.358$&&\tabularnewline
Pressure/Tension:non-gamified.Apprentice&$14$&$ 0.817$&$ 0.817$&$0.272$&$56$&$ 0.273$&$1.361$&$$&$$&$$&$$&&\tabularnewline
Pressure/Tension:ont-gamified.Apprentice&$15$&$-0.237$&$-0.237$&$0.262$&$56$&$-0.763$&$0.288$&$$&$$&$$&$$&&\tabularnewline
Pressure/Tension:non-gamified.Apprentice - ont-gamified.Apprentice&$29$&$ 1.054$&$ 1.054$&$0.378$&$$&$ 0.054$&$2.054$&$ 2.792$&$0.007$&$0.035$&$ 0.964$&*&large\tabularnewline
Pressure/Tension:non-gamified.Master&$16$&$ 0.193$&$ 0.193$&$0.254$&$56$&$-0.316$&$0.701$&$$&$$&$$&$$&&\tabularnewline
Pressure/Tension:ont-gamified.Master&$15$&$ 0.441$&$ 0.441$&$0.262$&$56$&$-0.084$&$0.967$&$$&$$&$$&$$&&\tabularnewline
Pressure/Tension:non-gamified.Master - ont-gamified.Master&$31$&$-0.249$&$-0.249$&$0.365$&$$&$-1.216$&$0.718$&$-0.681$&$0.499$&$0.904$&$-0.249$&&\tabularnewline
\hline

Effort/Importance:non-gamified&$30$&$-0.311$&$-0.290$&$0.173$&$56$&$-0.637$&$ 0.057$&$$&$$&$$&$$&&\tabularnewline
Effort/Importance:ont-gamified&$30$&$ 0.218$&$ 0.218$&$0.173$&$56$&$-0.128$&$ 0.564$&$$&$$&$$&$$&&\tabularnewline
Effort/Importance:non-gamified - ont-gamified&$60$&$-0.529$&$-0.508$&$0.245$&$$&$-1.019$&$-0.039$&$-2.074$&$0.043$&$0.035$&$-0.544$&*&medium\tabularnewline
Effort/Importance:non-gamified.Apprentice&$14$&$ 0.025$&$ 0.025$&$0.253$&$56$&$-0.482$&$ 0.532$&$$&$$&$$&$$&&\tabularnewline
Effort/Importance:ont-gamified.Apprentice&$15$&$ 0.302$&$ 0.302$&$0.245$&$56$&$-0.188$&$ 0.792$&$$&$$&$$&$$&&\tabularnewline
Effort/Importance:non-gamified.Apprentice - ont-gamified.Apprentice&$29$&$-0.277$&$-0.277$&$0.352$&$$&$-1.209$&$ 0.655$&$-0.786$&$0.435$&$0.860$&$-0.260$&&\tabularnewline
Effort/Importance:non-gamified.Master&$16$&$-0.605$&$-0.605$&$0.237$&$56$&$-1.079$&$-0.130$&$$&$$&$$&$$&&\tabularnewline
Effort/Importance:ont-gamified.Master&$15$&$ 0.134$&$ 0.134$&$0.245$&$56$&$-0.356$&$ 0.624$&$$&$$&$$&$$&&\tabularnewline
Effort/Importance:non-gamified.Master - ont-gamified.Master&$31$&$-0.739$&$-0.739$&$0.340$&$$&$-1.640$&$ 0.162$&$-2.171$&$0.034$&$0.144$&$-0.839$&&\tabularnewline
\hline
\end{longtable}}\end{landscape}




%%%%%%%%%%%%%%%%%%%%%%%%%%%%%%%%%%%%%%%%%%%%%%%%
\section{Data Analysis and Results: Second Empirical Study}
\label{sec:second-study}



%latex.default(result_df, caption = paste("Summary of two-way ANOVA results",     in_title), size = "small", longtable = T, ctable = F, landscape = F,     rowlabel = "", where = "!htbp", file = filename, append = T)%
\setlongtables{\small
\begin{longtable}{lrrrrr}\caption{Two-way ANOVA results  for the latent trait estimates of Level of Motivation in the second empirical study} \tabularnewline
\hline\hline
\multicolumn{1}{l}{}&\multicolumn{1}{c}{Sum Sq}&\multicolumn{1}{c}{Df}&\multicolumn{1}{c}{F value}&\multicolumn{1}{c}{Pr(\textgreater F)}&\multicolumn{1}{c}{Sig}\tabularnewline
\hline
\endfirsthead\caption[]{\em (continued)} \tabularnewline
\hline
\multicolumn{1}{l}{}&\multicolumn{1}{c}{Sum Sq}&\multicolumn{1}{c}{Df}&\multicolumn{1}{c}{F value}&\multicolumn{1}{c}{Pr(\textgreater F)}&\multicolumn{1}{c}{Sig}\tabularnewline
\hline
\endhead
\hline
\multicolumn{6}{r}{\tiny Signif. codes:  0 \aspas{**} 0.01 \aspas{*} 0.05}
\endfoot
\label{result}
%Level of Motivation:(Intercept)&$ 0.003$&$ 1$&$0.005$&$0.947$&$$\tabularnewline
Level of Motivation:Type&$ 0.261$&$ 1$&$0.345$&$0.559$&$$\tabularnewline
Level of Motivation:CLRole&$ 0.011$&$ 1$&$0.015$&$0.903$&$$\tabularnewline
Level of Motivation:Type:CLRole&$ 0.770$&$ 1$&$1.022$&$0.317$&$$\tabularnewline
%Level of Motivation:Residuals&$40.727$&$54$&$$&$$&$$\tabularnewline

%Attention:;(Intercept)&$ 0.028$&$ 1$&$0.018$&$0.893$&$$\tabularnewline
Attention:;Type&$ 1.605$&$ 1$&$1.031$&$0.314$&$$\tabularnewline
Attention:;CLRole&$ 0.006$&$ 1$&$0.004$&$0.951$&$$\tabularnewline
Attention:;Type:CLRole&$ 2.002$&$ 1$&$1.286$&$0.262$&$$\tabularnewline
%Attention:;Residuals&$84.055$&$54$&$$&$$&$$\tabularnewline

%Relevance:(Intercept)&$ 0.002$&$ 1$&$0.005$&$0.946$&$$\tabularnewline
Relevance:Type&$ 0.057$&$ 1$&$0.116$&$0.735$&$$\tabularnewline
Relevance:CLRole&$ 0.153$&$ 1$&$0.310$&$0.580$&$$\tabularnewline
Relevance:Type:CLRole&$ 0.016$&$ 1$&$0.033$&$0.856$&$$\tabularnewline
%Relevance:Residuals&$26.662$&$54$&$$&$$&$$\tabularnewline

%Satisfaction:(Intercept)&$ 0.002$&$ 1$&$0.001$&$0.969$&$$\tabularnewline
Satisfaction:Type&$ 0.084$&$ 1$&$0.056$&$0.814$&$$\tabularnewline
Satisfaction:CLRole&$ 0.325$&$ 1$&$0.216$&$0.644$&$$\tabularnewline
Satisfaction:Type:CLRole&$ 1.311$&$ 1$&$0.871$&$0.355$&$$\tabularnewline
%Satisfaction:Residuals&$79.806$&$53$&$$&$$&$$\tabularnewline
\hline
\end{longtable}}



%latex.default(post_hoc_df, caption = paste("Descriptive statistics and Tukey post-hoc test results",     in_title), size = "small", longtable = T, ctable = F, landscape = T,     rowlabel = "", where = "!htbp", file = filename, append = T)%
\setlongtables\begin{landscape}{\scriptsize
\begin{longtable}{lrrrrrrrrrrrll}\caption{Descriptive statistics and Tukey post-hoc test results  for the latent trait estimates of Level of Motivation in the second empirical study} \tabularnewline
\hline\hline
\multicolumn{1}{l}{}&\multicolumn{1}{c}{N}&\multicolumn{1}{c}{mean}&\multicolumn{1}{c}{lsmean}&\multicolumn{1}{c}{SE}&\multicolumn{1}{c}{df}&\multicolumn{1}{c}{lwr.CI}&\multicolumn{1}{c}{upr.CI}&\multicolumn{1}{c}{t.ratio}&\multicolumn{1}{c}{p.value}&\multicolumn{1}{c}{p-adj.}&\multicolumn{1}{c}{g}&\multicolumn{1}{c}{sig}&\multicolumn{1}{c}{mag}\tabularnewline
\hline
\endfirsthead\caption[]{\em (continued)} \tabularnewline
\hline
\multicolumn{1}{l}{}&\multicolumn{1}{c}{N}&\multicolumn{1}{c}{mean}&\multicolumn{1}{c}{lsmean}&\multicolumn{1}{c}{SE}&\multicolumn{1}{c}{df}&\multicolumn{1}{c}{lwr.CI}&\multicolumn{1}{c}{upr.CI}&\multicolumn{1}{c}{t.ratio}&\multicolumn{1}{c}{p.value}&\multicolumn{1}{c}{p-adj.}&\multicolumn{1}{c}{g}&\multicolumn{1}{c}{sig}&\multicolumn{1}{c}{mag}\tabularnewline
\hline
\endhead
\hline
\multicolumn{14}{r}{\tiny Signif. codes:  0 \aspas{**} 0.01 \aspas{*} 0.05}
\endfoot
\label{post}
Level of Motivation:non-gamified&$34$&$-0.113$&$-0.064$&$0.159$&$54$&$-0.383$&$0.255$&$$&$$&$$&$$&$$&$$\tabularnewline
Level of Motivation:ont-gamified&$24$&$ 0.117$&$ 0.081$&$0.188$&$54$&$-0.296$&$0.458$&$$&$$&$$&$$&$$&$$\tabularnewline
Level of Motivation:non-gamified - ont-gamified&$58$&$-0.231$&$-0.145$&$0.246$&$$&$-0.695$&$0.234$&$-0.588$&$0.559$&$0.324$&$-0.264$&$$&$$\tabularnewline
Level of Motivation:non-gamified.Apprentice&$23$&$-0.204$&$-0.204$&$0.181$&$54$&$-0.567$&$0.159$&$$&$$&$$&$$&$$&$$\tabularnewline
Level of Motivation:ont-gamified.Apprentice&$16$&$ 0.190$&$ 0.190$&$0.217$&$54$&$-0.245$&$0.625$&$$&$$&$$&$$&$$&$$\tabularnewline
Level of Motivation:non-gamified.Apprentice - ont-gamified.Apprentice&$39$&$-0.394$&$-0.394$&$0.283$&$$&$-1.143$&$0.356$&$-1.393$&$0.169$&$0.509$&$-0.451$&$$&$$\tabularnewline
Level of Motivation:non-gamified.Master&$11$&$ 0.075$&$ 0.075$&$0.262$&$54$&$-0.450$&$0.600$&$$&$$&$$&$$&$$&$$\tabularnewline
Level of Motivation:ont-gamified.Master&$ 8$&$-0.029$&$-0.029$&$0.307$&$54$&$-0.644$&$0.587$&$$&$$&$$&$$&$$&$$\tabularnewline
Level of Motivation:non-gamified.Master - ont-gamified.Master&$19$&$ 0.104$&$ 0.104$&$0.404$&$$&$-0.966$&$1.174$&$ 0.258$&$0.797$&$0.994$&$ 0.111$&$$&$$\tabularnewline
\hline

Attention:non-gamified&$34$&$-0.278$&$-0.204$&$0.229$&$54$&$-0.662$&$0.255$&$$&$$&$$&$$&$$&$$\tabularnewline
Attention:ont-gamified&$24$&$ 0.219$&$ 0.156$&$0.270$&$54$&$-0.386$&$0.697$&$$&$$&$$&$$&$$&$$\tabularnewline
Attention:non-gamified - ont-gamified&$58$&$-0.497$&$-0.359$&$0.354$&$$&$-1.164$&$0.170$&$-1.015$&$0.314$&$0.141$&$-0.396$&$$&$$\tabularnewline
Attention:;non-gamified.Apprentice&$23$&$-0.415$&$-0.415$&$0.260$&$54$&$-0.937$&$0.106$&$$&$$&$$&$$&$$&$$\tabularnewline
Attention:ont-gamified.Apprentice&$16$&$ 0.346$&$ 0.346$&$0.312$&$54$&$-0.280$&$0.971$&$$&$$&$$&$$&$$&$$\tabularnewline
Attention:non-gamified.Apprentice - ont-gamified.Apprentice&$39$&$-0.761$&$-0.761$&$0.406$&$$&$-1.837$&$0.316$&$-1.873$&$0.066$&$0.252$&$-0.601$&$$&$$\tabularnewline
Attention:non-gamified.Master&$11$&$ 0.008$&$ 0.008$&$0.376$&$54$&$-0.746$&$0.762$&$$&$$&$$&$$&$$&$$\tabularnewline
Attention:ont-gamified.Master&$ 8$&$-0.034$&$-0.034$&$0.441$&$54$&$-0.918$&$0.850$&$$&$$&$$&$$&$$&$$\tabularnewline
Attention:non-gamified.Master - ont-gamified.Master&$19$&$ 0.042$&$ 0.042$&$0.580$&$$&$-1.495$&$1.579$&$ 0.072$&$0.943$&$1.000$&$ 0.032$&$$&$$\tabularnewline
\hline

Relevance:non-gamified&$34$&$-0.053$&$-0.027$&$0.129$&$54$&$-0.285$&$0.231$&$$&$$&$$&$$&$$&$$\tabularnewline
Relevance:ont-gamified&$24$&$ 0.028$&$ 0.041$&$0.152$&$54$&$-0.264$&$0.346$&$$&$$&$$&$$&$$&$$\tabularnewline
Relevance:non-gamified - ont-gamified&$58$&$-0.081$&$-0.068$&$0.199$&$$&$-0.457$&$0.294$&$-0.340$&$0.735$&$0.666$&$-0.116$&$$&$$\tabularnewline
Relevance:non-gamified.Apprentice&$23$&$-0.101$&$-0.101$&$0.147$&$54$&$-0.395$&$0.193$&$$&$$&$$&$$&$$&$$\tabularnewline
Relevance:ont-gamified.Apprentice&$16$&$ 0.003$&$ 0.003$&$0.176$&$54$&$-0.349$&$0.356$&$$&$$&$$&$$&$$&$$\tabularnewline
Relevance:non-gamified.Apprentice - ont-gamified.Apprentice&$39$&$-0.104$&$-0.104$&$0.229$&$$&$-0.711$&$0.502$&$-0.455$&$0.651$&$0.968$&$-0.130$&$$&$$\tabularnewline
Relevance:non-gamified.Master&$11$&$ 0.047$&$ 0.047$&$0.212$&$54$&$-0.378$&$0.471$&$$&$$&$$&$$&$$&$$\tabularnewline
Relevance:ont-gamified.Master&$ 8$&$ 0.078$&$ 0.078$&$0.248$&$54$&$-0.420$&$0.576$&$$&$$&$$&$$&$$&$$\tabularnewline
Relevance:non-gamified.Master - ont-gamified.Master&$19$&$-0.031$&$-0.031$&$0.327$&$$&$-0.897$&$0.834$&$-0.096$&$0.924$&$1.000$&$-0.064$&$$&$$\tabularnewline
\newpage

Satisfaction:non-gamified&$33$&$-0.047$&$ 0.034$&$0.227$&$53$&$-0.420$&$0.489$&$$&$$&$$&$$&$$&$$\tabularnewline
Satisfaction:ont-gamified&$24$&$-0.021$&$-0.048$&$0.266$&$53$&$-0.581$&$0.485$&$$&$$&$$&$$&$$&$$\tabularnewline
Satisfaction:non-gamified - ont-gamified&$57$&$-0.026$&$ 0.082$&$0.349$&$$&$-0.687$&$0.634$&$ 0.236$&$0.814$&$0.937$&$-0.021$&$$&$$\tabularnewline
Satisfaction:non-gamified.Apprentice&$22$&$-0.210$&$-0.210$&$0.262$&$53$&$-0.734$&$0.315$&$$&$$&$$&$$&$$&$$\tabularnewline
Satisfaction:ont-gamified.Apprentice&$16$&$ 0.034$&$ 0.034$&$0.307$&$53$&$-0.581$&$0.649$&$$&$$&$$&$$&$$&$$\tabularnewline
Satisfaction:non-gamified.Apprentice - ont-gamified.Apprentice&$38$&$-0.243$&$-0.243$&$0.403$&$$&$-1.313$&$0.826$&$-0.604$&$0.549$&$0.930$&$-0.188$&$$&$$\tabularnewline
Satisfaction:non-gamified.Master&$11$&$ 0.278$&$ 0.278$&$0.370$&$53$&$-0.464$&$1.021$&$$&$$&$$&$$&$$&$$\tabularnewline
Satisfaction:ont-gamified.Master&$ 8$&$-0.130$&$-0.130$&$0.434$&$53$&$-1.000$&$0.740$&$$&$$&$$&$$&$$&$$\tabularnewline
Satisfaction:non-gamified.Master - ont-gamified.Master&$19$&$ 0.408$&$ 0.408$&$0.570$&$$&$-1.104$&$1.921$&$ 0.716$&$0.477$&$0.890$&$ 0.344$&$$&$$\tabularnewline
\hline
\end{longtable}}\end{landscape} 




%%%%%%%%%%%%%%%%%%%%%%%%%%%%%%%%%%%%%%%%%%%%%%%%
\section{Data Analysis and Results: Third Empirical Study}
\label{sec:third-study}



%latex.default(result_df, caption = paste("Summary of two-way ANOVA results",     in_title), size = "small", longtable = T, ctable = F, landscape = F,     rowlabel = "", where = "!htbp", file = filename, append = T)%
\setlongtables{\small
\begin{longtable}{lrrrrl}\caption{Two-way ANOVA results  for the latent trait estimates of Intrinsic Motivation in the third empirical study} \tabularnewline
\hline\hline
\multicolumn{1}{l}{}&\multicolumn{1}{c}{Sum Sq}&\multicolumn{1}{c}{Df}&\multicolumn{1}{c}{F value}&\multicolumn{1}{c}{Pr(\textgreater F)}&\multicolumn{1}{c}{Sig}\tabularnewline
\hline
\endfirsthead\caption[]{\em (continued)} \tabularnewline
\hline
\multicolumn{1}{l}{}&\multicolumn{1}{c}{Sum Sq}&\multicolumn{1}{c}{Df}&\multicolumn{1}{c}{F value}&\multicolumn{1}{c}{Pr(\textgreater F)}&\multicolumn{1}{c}{Sig}\tabularnewline
\hline
\endhead
\hline
\multicolumn{6}{r}{\tiny Signif. codes:  0 \aspas{**} 0.01 \aspas{*} 0.05}
\endfoot
\label{result}
%Intrinsic Motivation:(Intercept)&$ 0.561$&$ 1$&$1.508$&$0.226$&\tabularnewline
Intrinsic Motivation:Type&$ 2.533$&$ 1$&$6.812$&$0.012$&*\tabularnewline
Intrinsic Motivation:CLRole&$ 1.031$&$ 1$&$2.772$&$0.103$&\tabularnewline
Intrinsic Motivation:Type:CLRole&$ 0.468$&$ 1$&$1.258$&$0.268$&\tabularnewline
%Intrinsic Motivation:Residuals&$17.481$&$47$&$$&$$&\tabularnewline
%Interest/Enjoyment:(Intercept)&$ 0.836$&$ 1$&$0.610$&$0.439$&$$\tabularnewline
Interest/Enjoyment:Type&$ 2.149$&$ 1$&$1.568$&$0.217$&$$\tabularnewline
Interest/Enjoyment:CLRole&$ 2.230$&$ 1$&$1.627$&$0.208$&$$\tabularnewline
Interest/Enjoyment:Type:CLRole&$ 2.804$&$ 1$&$2.046$&$0.159$&$$\tabularnewline
%Interest/Enjoyment:Residuals&$64.426$&$47$&$$&$$&$$\tabularnewline
%Perceived Choice:(Intercept)&$ 0.160$&$ 1$&$0.120$&$0.731$&\tabularnewline
Perceived Choice:Type&$11.013$&$ 1$&$8.236$&$0.006$&**\tabularnewline
Perceived Choice:CLRole&$ 0.465$&$ 1$&$0.348$&$0.558$&\tabularnewline
Perceived Choice:Type:CLRole&$ 1.641$&$ 1$&$1.227$&$0.274$&\tabularnewline
%Perceived Choice:Residuals&$62.848$&$47$&$$&$$&\tabularnewline
%Pressure/Tension:(Intercept)&$ 0.869$&$ 1$&$ 1.690$&$0.200$&\tabularnewline
Pressure/Tension:Type&$ 0.018$&$ 1$&$ 0.035$&$0.853$&\tabularnewline
Pressure/Tension:CLRole&$ 8.430$&$ 1$&$16.389$&$0.000$&**\tabularnewline
Pressure/Tension:Type:CLRole&$ 0.020$&$ 1$&$ 0.039$&$0.845$&\tabularnewline
%Pressure/Tension:Residuals&$24.175$&$47$&$$&$$&\tabularnewline
%Effort/Importance:(Intercept)&$ 0.331$&$ 1$&$0.328$&$0.570$&\tabularnewline
Effort/Importance:Type&$ 7.319$&$ 1$&$7.258$&$0.010$&**\tabularnewline
Effort/Importance:CLRole&$ 1.290$&$ 1$&$1.279$&$0.264$&\tabularnewline
Effort/Importance:Type:CLRole&$ 0.022$&$ 1$&$0.022$&$0.883$&\tabularnewline
%Effort/Importance:Residuals&$47.391$&$47$&$$&$$&\tabularnewline
%Level of Motivation:(Intercept)&$ 0.093$&$ 1$&$0.272$&$0.604$&$$\tabularnewline
Level of Motivation:Type&$ 0.175$&$ 1$&$0.513$&$0.477$&$$\tabularnewline
Level of Motivation:CLRole&$ 0.056$&$ 1$&$0.164$&$0.688$&$$\tabularnewline
Level of Motivation:Type:CLRole&$ 0.045$&$ 1$&$0.131$&$0.719$&$$\tabularnewline
%Level of Motivation:Residuals&$15.976$&$47$&$$&$$&$$\tabularnewline
%Attention:(Intercept)&$ 0.032$&$ 1$&$0.022$&$0.882$&$$\tabularnewline
Attention:Type&$ 0.028$&$ 1$&$0.020$&$0.889$&$$\tabularnewline
Attention:CLRole&$ 0.365$&$ 1$&$0.254$&$0.617$&$$\tabularnewline
Attention:Type:CLRole&$ 0.246$&$ 1$&$0.171$&$0.681$&$$\tabularnewline
%Attention:Residuals&$67.644$&$47$&$$&$$&$$\tabularnewline
%Relevance:(Intercept)&$ 0.222$&$ 1$&$0.505$&$0.481$&$$\tabularnewline
Relevance:Type&$ 1.559$&$ 1$&$3.541$&$0.066$&$$\tabularnewline
Relevance:CLRole&$ 1.299$&$ 1$&$2.950$&$0.092$&$$\tabularnewline
Relevance:Type:CLRole&$ 0.791$&$ 1$&$1.797$&$0.187$&$$\tabularnewline
%Relevance:Residuals&$20.698$&$47$&$$&$$&$$\tabularnewline
%Satisfaction:(Intercept)&$  0.002$&$ 1$&$0.001$&$0.976$&$$\tabularnewline
Satisfaction:Type&$  4.424$&$ 1$&$1.661$&$0.204$&$$\tabularnewline
Satisfaction:CLRole&$  1.741$&$ 1$&$0.654$&$0.423$&$$\tabularnewline
Satisfaction:Type:CLRole&$  0.438$&$ 1$&$0.165$&$0.687$&$$\tabularnewline
%Satisfaction:Residuals&$125.193$&$47$&$$&$$&$$\tabularnewline
\hline
\end{longtable}} 


%latex.default(post_hoc_df, caption = paste("Descriptive statistics and Tukey post-hoc test results",     in_title), size = "small", longtable = T, ctable = F, landscape = T,     rowlabel = "", where = "!htbp", file = filename, append = T)%
\setlongtables\begin{landscape}{\scriptsize
\begin{longtable}{lrrrrrrrrrrrll}\caption{Descriptive statistics and Tukey post-hoc test results  for the latent trait estimates of Intrinsic Motivation in the third empirical study} \tabularnewline
\hline\hline
\multicolumn{1}{l}{}&\multicolumn{1}{c}{N}&\multicolumn{1}{c}{mean}&\multicolumn{1}{c}{lsmean}&\multicolumn{1}{c}{SE}&\multicolumn{1}{c}{df}&\multicolumn{1}{c}{lwr.CI}&\multicolumn{1}{c}{upr.CI}&\multicolumn{1}{c}{t.ratio}&\multicolumn{1}{c}{p.value}&\multicolumn{1}{c}{p-adj.}&\multicolumn{1}{c}{g}&\multicolumn{1}{c}{sig}&\multicolumn{1}{c}{mag}\tabularnewline
\hline
\endfirsthead\caption[]{\em (continued)} \tabularnewline
\hline
\multicolumn{1}{l}{}&\multicolumn{1}{c}{N}&\multicolumn{1}{c}{mean}&\multicolumn{1}{c}{lsmean}&\multicolumn{1}{c}{SE}&\multicolumn{1}{c}{df}&\multicolumn{1}{c}{lwr.CI}&\multicolumn{1}{c}{upr.CI}&\multicolumn{1}{c}{t.ratio}&\multicolumn{1}{c}{p.value}&\multicolumn{1}{c}{p-adj.}&\multicolumn{1}{c}{g}&\multicolumn{1}{c}{sig}&\multicolumn{1}{c}{mag}\tabularnewline
\hline
\endhead
\hline
\multicolumn{14}{r}{\tiny Signif. codes:  0 \aspas{**} 0.01 \aspas{*} 0.05}
\endfoot
\label{post}
Intrinsic Motivation:ont-gamified&$24$&$ 0.286$&$ 0.349$&$0.129$&$47$&$ 0.090$&$0.607$&$$&$$&$$&$$&&\tabularnewline
Intrinsic Motivation:w/o-gamified&$27$&$-0.146$&$-0.126$&$0.129$&$47$&$-0.384$&$0.133$&$$&$$&$$&$$&&\tabularnewline
Intrinsic Motivation:ont-gamified - w/o-gamified&$51$&$ 0.431$&$ 0.474$&$0.182$&$$&$ 0.087$&$0.775$&$2.610$&$0.012$&$0.015$&$0.682$&*&medium\tabularnewline
Intrinsic Motivation:ont-gamified.Apprentice&$15$&$ 0.096$&$ 0.096$&$0.157$&$47$&$-0.221$&$0.412$&$$&$$&$$&$$&&\tabularnewline
Intrinsic Motivation:w/o-gamified.Apprentice&$19$&$-0.175$&$-0.175$&$0.140$&$47$&$-0.456$&$0.106$&$$&$$&$$&$$&&\tabularnewline
Intrinsic Motivation:ont-gamified.Apprentice - w/o-gamified.Apprentice&$34$&$ 0.271$&$ 0.271$&$0.211$&$$&$-0.291$&$0.832$&$1.284$&$0.205$&$0.577$&$0.431$&&\tabularnewline
Intrinsic Motivation:ont-gamified.Master&$ 9$&$ 0.602$&$ 0.602$&$0.203$&$47$&$ 0.193$&$1.011$&$$&$$&$$&$$&&\tabularnewline
Intrinsic Motivation:w/o-gamified.Master&$ 8$&$-0.076$&$-0.076$&$0.216$&$47$&$-0.510$&$0.358$&$$&$$&$$&$$&&\tabularnewline
Intrinsic Motivation:ont-gamified.Master - w/o-gamified.Master&$17$&$ 0.678$&$ 0.678$&$0.296$&$$&$-0.111$&$1.468$&$2.289$&$0.027$&$0.115$&$1.065$&&\tabularnewline
\hline

Interest/Enjoyment:ont-gamified&$24$&$ 0.237$&$ 0.355$&$0.247$&$47$&$-0.142$&$0.851$&$$&$$&$$&$$&$$&$$\tabularnewline
Interest/Enjoyment:w/o-gamified&$27$&$-0.071$&$-0.082$&$0.247$&$47$&$-0.579$&$0.414$&$$&$$&$$&$$&$$&$$\tabularnewline
Interest/Enjoyment:ont-gamified - w/o-gamified&$51$&$ 0.308$&$ 0.437$&$0.349$&$$&$-0.353$&$0.969$&$ 1.252$&$0.217$&$0.353$&$ 0.255$&$$&$$\tabularnewline
Interest/Enjoyment:ont-gamified.Apprentice&$15$&$-0.117$&$-0.117$&$0.302$&$47$&$-0.725$&$0.491$&$$&$$&$$&$$&$$&$$\tabularnewline
Interest/Enjoyment:w/o-gamified.Apprentice&$19$&$-0.055$&$-0.055$&$0.269$&$47$&$-0.596$&$0.485$&$$&$$&$$&$$&$$&$$\tabularnewline
Interest/Enjoyment:ont-gamified.Apprentice - w/o-gamified.Apprentice&$34$&$-0.062$&$-0.062$&$0.404$&$$&$-1.139$&$1.015$&$-0.154$&$0.879$&$0.999$&$-0.049$&$$&$$\tabularnewline
Interest/Enjoyment:ont-gamified.Master&$ 9$&$ 0.827$&$ 0.827$&$0.390$&$47$&$ 0.042$&$1.612$&$$&$$&$$&$$&$$&$$\tabularnewline
Interest/Enjoyment:w/o-gamified.Master&$ 8$&$-0.109$&$-0.109$&$0.414$&$47$&$-0.942$&$0.724$&$$&$$&$$&$$&$$&$$\tabularnewline
Interest/Enjoyment:ont-gamified.Master - w/o-gamified.Master&$17$&$ 0.936$&$ 0.936$&$0.569$&$$&$-0.579$&$2.451$&$ 1.645$&$0.107$&$0.364$&$ 0.862$&$$&$$\tabularnewline
\hline

Perceived Choice:ont-gamified&$24$&$ 0.481$&$ 0.554$&$0.244$&$47$&$ 0.064$&$1.045$&$$&$$&$$&$$&&\tabularnewline
Perceived Choice:w/o-gamified&$27$&$-0.399$&$-0.435$&$0.244$&$47$&$-0.925$&$0.055$&$$&$$&$$&$$&&\tabularnewline
Perceived Choice:ont-gamified - w/o-gamified&$51$&$ 0.880$&$ 0.989$&$0.345$&$$&$ 0.227$&$1.532$&$2.870$&$0.006$&$0.009$&$0.752$&**&medium\tabularnewline
Perceived Choice:ont-gamified.Apprentice&$15$&$ 0.262$&$ 0.262$&$0.299$&$47$&$-0.339$&$0.862$&$$&$$&$$&$$&&\tabularnewline
Perceived Choice:w/o-gamified.Apprentice&$19$&$-0.346$&$-0.346$&$0.265$&$47$&$-0.879$&$0.188$&$$&$$&$$&$$&&\tabularnewline
Perceived Choice:ont-gamified.Apprentice - w/o-gamified.Apprentice&$34$&$ 0.607$&$ 0.607$&$0.399$&$$&$-0.456$&$1.671$&$1.521$&$0.135$&$0.434$&$0.482$&&\tabularnewline
Perceived Choice:ont-gamified.Master&$ 9$&$ 0.847$&$ 0.847$&$0.385$&$47$&$ 0.071$&$1.622$&$$&$$&$$&$$&&\tabularnewline
Perceived Choice:w/o-gamified.Master&$ 8$&$-0.524$&$-0.524$&$0.409$&$47$&$-1.347$&$0.298$&$$&$$&$$&$$&&\tabularnewline
Perceived Choice:ont-gamified.Master - w/o-gamified.Master&$17$&$ 1.371$&$ 1.371$&$0.562$&$$&$-0.125$&$2.868$&$2.440$&$0.019$&$0.083$&$1.324$&&\tabularnewline
\newpage

Pressure/Tension:ont-gamified&$24$&$-0.006$&$-0.119$&$0.151$&$47$&$-0.423$&$ 0.185$&$$&$$&$$&$$&$$&$$\tabularnewline
Pressure/Tension:w/o-gamified&$27$&$ 0.009$&$-0.159$&$0.151$&$47$&$-0.463$&$ 0.145$&$$&$$&$$&$$&$$&$$\tabularnewline
Pressure/Tension:ont-gamified - w/o-gamified&$51$&$-0.014$&$ 0.040$&$0.214$&$$&$-0.419$&$ 0.390$&$ 0.187$&$0.853$&$0.943$&$-0.017$&$$&$$\tabularnewline
Pressure/Tension:ont-gamified.Apprentice&$15$&$ 0.335$&$ 0.335$&$0.185$&$47$&$-0.038$&$ 0.707$&$$&$$&$$&$$&$$&$$\tabularnewline
Pressure/Tension:w/o-gamified.Apprentice&$19$&$ 0.253$&$ 0.253$&$0.165$&$47$&$-0.078$&$ 0.584$&$$&$$&$$&$$&$$&$$\tabularnewline
Pressure/Tension:ont-gamified.Apprentice - w/o-gamified.Apprentice&$34$&$ 0.082$&$ 0.082$&$0.248$&$$&$-0.578$&$ 0.742$&$ 0.331$&$0.742$&$0.987$&$ 0.119$&$$&$$\tabularnewline
Pressure/Tension:ont-gamified.Master&$ 9$&$-0.573$&$-0.573$&$0.239$&$47$&$-1.054$&$-0.092$&$$&$$&$$&$$&$$&$$\tabularnewline
Pressure/Tension:w/o-gamified.Master&$ 8$&$-0.571$&$-0.571$&$0.254$&$47$&$-1.081$&$-0.061$&$$&$$&$$&$$&$$&$$\tabularnewline
Pressure/Tension:ont-gamified.Master - w/o-gamified.Master&$17$&$-0.002$&$-0.002$&$0.348$&$$&$-0.930$&$ 0.926$&$-0.006$&$0.995$&$1.000$&$-0.002$&$$&$$\tabularnewline
\hline

Effort/Importance:ont-gamified&$24$&$ 0.365$&$ 0.318$&$0.212$&$47$&$-0.108$&$ 0.743$&$$&$$&$$&$$&&\tabularnewline
Effort/Importance:w/o-gamified&$27$&$-0.429$&$-0.489$&$0.212$&$47$&$-0.915$&$-0.063$&$$&$$&$$&$$&&\tabularnewline
Effort/Importance:ont-gamified - w/o-gamified&$51$&$ 0.794$&$ 0.806$&$0.299$&$$&$ 0.228$&$ 1.361$&$2.694$&$0.010$&$0.007$&$0.785$&**&medium\tabularnewline
Effort/Importance:ont-gamified.Apprentice&$15$&$ 0.509$&$ 0.509$&$0.259$&$47$&$-0.013$&$ 1.031$&$$&$$&$$&$$&&\tabularnewline
Effort/Importance:w/o-gamified.Apprentice&$19$&$-0.342$&$-0.342$&$0.230$&$47$&$-0.805$&$ 0.122$&$$&$$&$$&$$&&\tabularnewline
Effort/Importance:ont-gamified.Apprentice - w/o-gamified.Apprentice&$34$&$ 0.851$&$ 0.851$&$0.347$&$$&$-0.073$&$ 1.775$&$2.453$&$0.018$&$0.081$&$0.835$&&\tabularnewline
Effort/Importance:ont-gamified.Master&$ 9$&$ 0.126$&$ 0.126$&$0.335$&$47$&$-0.547$&$ 0.799$&$$&$$&$$&$$&&\tabularnewline
Effort/Importance:w/o-gamified.Master&$ 8$&$-0.636$&$-0.636$&$0.355$&$47$&$-1.350$&$ 0.078$&$$&$$&$$&$$&&\tabularnewline
Effort/Importance:ont-gamified.Master - w/o-gamified.Master&$17$&$ 0.762$&$ 0.762$&$0.488$&$$&$-0.538$&$ 2.062$&$1.562$&$0.125$&$0.410$&$0.707$&&\tabularnewline
\hline

Level of Motivation:ont-gamified&$24$&$ 0.091$&$ 0.108$&$0.123$&$47$&$-0.140$&$0.355$&$$&$$&$$&$$&$$&$$\tabularnewline
Level of Motivation:w/o-gamified&$27$&$-0.018$&$-0.017$&$0.123$&$47$&$-0.264$&$0.230$&$$&$$&$$&$$&$$&$$\tabularnewline
Level of Motivation:ont-gamified - w/o-gamified&$51$&$ 0.109$&$ 0.125$&$0.174$&$$&$-0.220$&$0.438$&$0.717$&$0.477$&$0.507$&$0.188$&$$&$$\tabularnewline
Level of Motivation:ont-gamified.Apprentice&$15$&$ 0.041$&$ 0.041$&$0.151$&$47$&$-0.262$&$0.344$&$$&$$&$$&$$&$$&$$\tabularnewline
Level of Motivation:w/o-gamified.Apprentice&$19$&$-0.021$&$-0.021$&$0.134$&$47$&$-0.290$&$0.248$&$$&$$&$$&$$&$$&$$\tabularnewline
Level of Motivation:ont-gamified.Apprentice - w/o-gamified.Apprentice&$34$&$ 0.062$&$ 0.062$&$0.201$&$$&$-0.475$&$0.598$&$0.306$&$0.761$&$0.990$&$0.097$&$$&$$\tabularnewline
Level of Motivation:ont-gamified.Master&$ 9$&$ 0.174$&$ 0.174$&$0.194$&$47$&$-0.217$&$0.565$&$$&$$&$$&$$&$$&$$\tabularnewline
Level of Motivation:w/o-gamified.Master&$ 8$&$-0.013$&$-0.013$&$0.206$&$47$&$-0.428$&$0.401$&$$&$$&$$&$$&$$&$$\tabularnewline
Level of Motivation:ont-gamified.Master - w/o-gamified.Master&$17$&$ 0.187$&$ 0.187$&$0.283$&$$&$-0.567$&$0.942$&$0.662$&$0.511$&$0.911$&$0.365$&$$&$$\tabularnewline
\newpage

Attention:ont-gamified&$24$&$-0.039$&$ 0.002$&$0.253$&$47$&$-0.507$&$0.511$&$$&$$&$$&$$&$$&$$\tabularnewline
Attention:w/o-gamified&$27$&$ 0.045$&$ 0.052$&$0.253$&$47$&$-0.457$&$0.560$&$$&$$&$$&$$&$$&$$\tabularnewline
Attention:ont-gamified - w/o-gamified&$51$&$-0.085$&$-0.050$&$0.358$&$$&$-0.762$&$0.592$&$-0.140$&$0.889$&$0.803$&$-0.071$&$$&$$\tabularnewline
Attention:ont-gamified.Apprentice&$15$&$-0.162$&$-0.162$&$0.310$&$47$&$-0.786$&$0.461$&$$&$$&$$&$$&$$&$$\tabularnewline
Attention:w/o-gamified.Apprentice&$19$&$ 0.036$&$ 0.036$&$0.275$&$47$&$-0.518$&$0.589$&$$&$$&$$&$$&$$&$$\tabularnewline
Attention:ont-gamified.Apprentice - w/o-gamified.Apprentice&$34$&$-0.198$&$-0.198$&$0.414$&$$&$-1.302$&$0.906$&$-0.478$&$0.635$&$0.964$&$-0.147$&$$&$$\tabularnewline
Attention:ont-gamified.Master&$ 9$&$ 0.166$&$ 0.166$&$0.400$&$47$&$-0.639$&$0.970$&$$&$$&$$&$$&$$&$$\tabularnewline
Attention:w/o-gamified.Master&$ 8$&$ 0.068$&$ 0.068$&$0.424$&$47$&$-0.785$&$0.921$&$$&$$&$$&$$&$$&$$\tabularnewline
Attention:ont-gamified.Master - w/o-gamified.Master&$17$&$ 0.098$&$ 0.098$&$0.583$&$$&$-1.455$&$1.650$&$ 0.168$&$0.867$&$0.998$&$ 0.104$&$$&$$\tabularnewline
\hline

Relevance:ont-gamified&$24$&$ 0.181$&$ 0.256$&$0.140$&$47$&$-0.025$&$0.538$&$$&$$&$$&$$&$$&$$\tabularnewline
Relevance:w/o-gamified&$27$&$-0.131$&$-0.116$&$0.140$&$47$&$-0.397$&$0.165$&$$&$$&$$&$$&$$&$$\tabularnewline
Relevance:ont-gamified - w/o-gamified&$51$&$ 0.312$&$ 0.372$&$0.198$&$$&$-0.063$&$0.686$&$1.882$&$0.066$&$0.101$&$0.450$&$$&$$\tabularnewline
Relevance:ont-gamified.Apprentice&$15$&$-0.046$&$-0.046$&$0.171$&$47$&$-0.391$&$0.299$&$$&$$&$$&$$&$$&$$\tabularnewline
Relevance:w/o-gamified.Apprentice&$19$&$-0.153$&$-0.153$&$0.152$&$47$&$-0.459$&$0.153$&$$&$$&$$&$$&$$&$$\tabularnewline
Relevance:ont-gamified.Apprentice - w/o-gamified.Apprentice&$34$&$ 0.107$&$ 0.107$&$0.229$&$$&$-0.503$&$0.718$&$0.467$&$0.643$&$0.966$&$0.150$&$$&$$\tabularnewline
Relevance:ont-gamified.Master&$ 9$&$ 0.559$&$ 0.559$&$0.221$&$47$&$ 0.114$&$1.004$&$$&$$&$$&$$&$$&$$\tabularnewline
Relevance:w/o-gamified.Master&$ 8$&$-0.079$&$-0.079$&$0.235$&$47$&$-0.551$&$0.393$&$$&$$&$$&$$&$$&$$\tabularnewline
Relevance:ont-gamified.Master - w/o-gamified.Master&$17$&$ 0.637$&$ 0.637$&$0.322$&$$&$-0.221$&$1.496$&$1.977$&$0.054$&$0.211$&$1.039$&$$&$$\tabularnewline
\hline


Satisfaction:ont-gamified&$24$&$ 0.380$&$ 0.306$&$0.344$&$47$&$-0.386$&$0.998$&$$&$$&$$&$$&$$&$$\tabularnewline
Satisfaction:w/o-gamified&$27$&$-0.281$&$-0.321$&$0.344$&$47$&$-1.013$&$0.371$&$$&$$&$$&$$&$$&$$\tabularnewline
Satisfaction:ont-gamified - w/o-gamified&$51$&$ 0.661$&$ 0.627$&$0.486$&$$&$-0.260$&$1.582$&$1.289$&$0.204$&$0.156$&$0.404$&$$&$$\tabularnewline
Satisfaction:ont-gamified.Apprentice&$15$&$ 0.602$&$ 0.602$&$0.421$&$47$&$-0.246$&$1.449$&$$&$$&$$&$$&$$&$$\tabularnewline
Satisfaction:w/o-gamified.Apprentice&$19$&$-0.223$&$-0.223$&$0.374$&$47$&$-0.976$&$0.530$&$$&$$&$$&$$&$$&$$\tabularnewline
Satisfaction:ont-gamified.Apprentice - w/o-gamified.Apprentice&$34$&$ 0.824$&$ 0.824$&$0.564$&$$&$-0.677$&$2.326$&$1.462$&$0.150$&$0.468$&$0.472$&$$&$$\tabularnewline
Satisfaction:ont-gamified.Master&$ 9$&$ 0.011$&$ 0.011$&$0.544$&$47$&$-1.084$&$1.105$&$$&$$&$$&$$&$$&$$\tabularnewline
Satisfaction:w/o-gamified.Master&$ 8$&$-0.419$&$-0.419$&$0.577$&$47$&$-1.580$&$0.742$&$$&$$&$$&$$&$$&$$\tabularnewline
Satisfaction:ont-gamified.Master - w/o-gamified.Master&$17$&$ 0.430$&$ 0.430$&$0.793$&$$&$-1.683$&$2.542$&$0.542$&$0.591$&$0.948$&$0.279$&$$&$$\tabularnewline
\hline

\end{longtable}}\end{landscape} 



%%%%%%%%%%%%%%%%%%%%%%%%%%%%%%%%%%%%%%%%%%%%%%%%
\section{Interpretation and Implications}
\label{sec:interpretation-implications} 


%%%%%%%%%%%%%%%%%%%%%%%%%%%%%%%%%%%%%%%%%%%%%%%%
\section{Conclusions and Remarks}
\label{sec:evaluation-conclusion}


%Once the construct of interest has been determined, it is important to conduct a literature review to identify if a previously validated questionnaire exists. A validated questionnaire refers to a questionnaire/scale that has been developed to be administered among the intended respondents. The validation processes should have been completed using a representative sample, demonstrating adequate reliability and validity. Examples of necessary validation processes can be found in the validation section of this paper. If no existing questionnaires are available, or none that are determined to be appropriate, it is appropriate to construct a new questionnaire. If a questionnaire exists, but only in a different language, the task is to translate and validate the questionnaire in the new language. This process consists in the validation of the meaning and appropriateness of the items




% ---
% Finaliza a parte no bookmark do PDF, para que se inicie o bookmark na raiz
% ---
\bookmarksetup{startatroot}% 
% ---

% ----------------------------------------------------------
% ELEMENTOS PÓS-TEXTUAIS
% ----------------------------------------------------------
\postextual

% ----------------------------------------------------------
% Referências bibliográficas
% ----------------------------------------------------------
\bibliography{references}

% ---------------------------------------------------------------------
% GLOSSÁRIO
% ---------------------------------------------------------------------

% Arquivo que contém as definições que vão aparecer no glossário
%\input{tex/glossario}
% Comando para incluir todas as definições do arquivo glossario.tex
%\glsaddall
% Impressão do glossário
%\printglossaries

% ----------------------------------------------------------
% Apêndices
% ----------------------------------------------------------

% ---
% Inicia os apêndices
% ---
\begin{apendicesenv}
 \chapter{Ontology OntoGaCLeS: Concepts, Terms and Ontological Structures}
\label{appendix:ontology-ontogacles}

\section{Tree Overview of States}
\label{sec:ontogacles:tree-overview-states}

\subsection{Motivation States}
\includegraphics[width=0.5\textwidth]{images/appendix/tree-overview-motivation-states.png} \newpage

\newpage
\subsection{Psychological Need States}
\includegraphics[width=0.75\textwidth]{images/appendix/tree-overview-need-states.png} \newpage

\newpage
\subsection{Individual Personality Trait States}
\includegraphics[width=0.75\textwidth]{images/appendix/tree-overview-ind-personality-trait-states.png}

%%%%%%%%%%%%%%%%%%%%%%%%%%%%%%%%%%%%%%%%%%%%%%%%%%%
\newpage
\section{Individual Motivational Goal (I-mot goal)}
\label{sec:ontogacles:i-mot-goal}
\includegraphics[width=1\textwidth]{images/appendix/ontological-structure-i-mot-goal.png}

%%%%%%%%%%%%%%%%%%%%%%%%%%%%%%%%%%%%%%%%%%%%%%%%%%%
\newpage
\section{Player Role}
\label{sec:ontogacles:player-role}

\subsection{Player Roles based on the Bartle Model}
\label{sec:player-roles-based-bartle}
\includegraphics[width=0.8\textwidth]{images/appendix/player-roles-based-bartle.png}

\newpage
\subsection{Player Roles Based on the Borges Model}
\label{sec:player-roles-based-borges}
\includegraphics[width=1\textwidth]{images/appendix/player-roles-based-borges.png}

\newpage
\subsection{Player Roles Based on the Yee Model}
\label{sec:player-roles-based-yee}
\includegraphics[width=1\textwidth]{images/appendix/player-roles-based-yee.png}

\newpage
\subsection{Player Roles Based on the Dodecad Model}
\label{sec:player-roles-based-dodecad}
\includegraphics[width=1\textwidth]{images/appendix/player-roles-based-dodecad.png}

\newpage
\subsection{Player Roles Based on the BrainHex Model}
\label{sec:player-roles-based-brainhex}
\includegraphics[width=1\textwidth]{images/appendix/player-roles-based-brainhex.png}

%%%%%%%%%%%%%%%%%%%%%%%%%%%%%%%%%%%%%%%%%%%%%%%%%%%
\newpage
\section{Individual Motivational Strategy (Y<=I-mot goal)}
\label{sec:ontogacles:individual-motivational-strategy}
\includegraphics[width=0.92\textwidth]{images/appendix/ontological-structure-individual-motivational-strategy.png}

%%%%%%%%%%%%%%%%%%%%%%%%%%%%%%%%%%%%%%%%%%%%%%%%%%%
\newpage
\section{Individual Gameplay Strategy (I-gameplay strategy)}
\label{sec:ontogacles:individual-gameplay-strategy}
\includegraphics[width=1\textwidth]{images/appendix/ontological-structure-individual-gameplay-strategy.png}


 
\chapter{Outliers Detection and Treatment in Data Gathered from Motivation Surveys}
\label{appendix:outliers}

%In this PhD dissertation, a set of statistical parametric procedures like as ANOVA, IRTs models and correlation test has been used to conduct the data analyses. These procedures are highly sensitive to outliers by which this appendix details how the thesis author dealt with the outliers in the data collected over the empirical studies. The detection and treatment of outliers in the self-reported data gathered by motivation surveys is detailed in \autoref{sec:outliers-motivation}, where these surveys correspond to questionnaires based on the Intrinsic Motivation Inventory (IMI) and Instructional Materials Motivation Survey (IMMS). The treatment of outliers in the data related to the learning outcomes is presented in \autoref{sec:outliers-learning-outcomes}, where the data correspond to data collected over the empirical studies by means of assessments during the pre-test and post-test phases.
%The procedure of outliers detection and their treatments in the ANOVA is detailed in \autoref{sec:outliers-anova-motivation}.

%%%%%%%%%%%%%%%%%%%%%%%%%%%%%%%%%%%%%%%%%%%%%%%%

%\section{Treatment of Outliers in the Self-reported Data Gathered by Motivation Surveys}
%\label{sec:outliers-motivation}

% may be exaggerated; respondents may be too embarrassed to reveal private details; various biases may affect the results, like social desirability bias

Prior to any statistical analysis related to participants' motivation in the empirical studies, outliers identified as careless responses have been drop out from the data collected by motivational surveys (\autoref{subsec:removing-careless-motivation-survey}). These outliers correspond to incorrectly entered data by the students in the adapted Portuguese version of Intrinsic Motivation Inventory (IMI) and the Instructional Materials Motivation Survey (IMMS). After to remove the careless responses, outliers identified as extreme values are replaced to the trimmed minimum and maximum values by the Winsorization method (\autoref{subsec:winsorizing-motivation-survey}).

\section{Removing Careless Responses}
\label{subsec:removing-careless-motivation-survey}

The questionnaires of the adapted Portuguese IMI have 24 items, so that a careless response is defined as a response in which the length of uninterrupted identical values for the items is greater than 12 (half of the items). For the data collected by the questionnaires of the adapted Portuguese IMMS, a careless response is defined as a response in which the length of uninterrupted identical values is greater than half of the items (12 items).

\subsection{Intrinsic Motivation Inventory Data}

\autoref{tab:careless-IMI-pilot-study} shows the two careless responses identified and removed from the IMI data collected over the pilot empirical study. These two careless responses corresponds to participants with user IDs 10119 and 10133; and they were identified in 32 responses collected from computer science undergraduate students by a web-based questionnaire of the adapted Portuguese version of IMI (shown in \autoref{annex:IMI-pilot-study}).

%latex.default(as.data.frame(dd$get_data()), rowname = NULL, caption = paste("Summary of careless responses",     in_title), size = "small", longtable = T, ctable = F, landscape = F,     rowlabel = "", where = "!htbp", file = filename, append = T)%
\setlongtables{\small
\begin{longtable}{lllllllllll}
\caption{Summary of careless responses in the IMI data collected over the pilot empirical study}
\label{tab:careless-IMI-pilot-study} \tabularnewline
\hline\hline
\multicolumn{1}{c}{@@}&\multicolumn{1}{c}{UserID}&\multicolumn{1}{c}{Item01}&\multicolumn{1}{c}{Item02}&\multicolumn{1}{c}{Item03}&\multicolumn{1}{c}{Item04}&\multicolumn{1}{c}{Item05}&\multicolumn{1}{c}{Item06}&\multicolumn{1}{c}{Item07}&\multicolumn{1}{c}{Item08}&\multicolumn{1}{c}{...}\tabularnewline
\hline
\endfirsthead\caption[]{\em (continued)} \tabularnewline
\hline
\multicolumn{1}{c}{@@}&\multicolumn{1}{c}{UserID}&\multicolumn{1}{c}{Item01}&\multicolumn{1}{c}{Item02}&\multicolumn{1}{c}{Item03}&\multicolumn{1}{c}{Item04}&\multicolumn{1}{c}{Item05}&\multicolumn{1}{c}{Item06}&\multicolumn{1}{c}{Item07}&\multicolumn{1}{c}{Item08}&\multicolumn{1}{c}{...}\tabularnewline
\hline
\endhead
\hline
\endfoot
\label{as.data.frame}
&10116&3&7&4&4&2&6&4&4&...\tabularnewline
---&10119&4&4&4&4&4&4&4&4&...\tabularnewline
&10120&7&3&7&7&6&2&1&1&...\tabularnewline
...&...&...&...&...&...&...&...&...&...&...\tabularnewline
&10132&1&7&7&6&4&4&3&4&...\tabularnewline
---&10133&4&4&4&4&4&4&4&4&...\tabularnewline
&10134&2&6&4&4&3&3&4&6&...\tabularnewline
...&...&...&...&...&...&...&...&...&...&...\tabularnewline
\hline
\end{longtable}}

\autoref{tab:careless-IMI-study01} shows the careless responses identified and removed from the data collected over the first empirical study. These two careless responses correspond to participants with user IDs 10229 and 10241, and they were identified in a set of 62 responses collected from computer engineer undergraduate students by means of the paper-based questionnaire version of IMI (shown in \autoref{annex:IMI-first-study}).

%latex.default(as.data.frame(dd$get_data()), rowname = NULL, caption = paste("Summary of careless responses",     in_title), size = "small", longtable = T, ctable = F, landscape = F,     rowlabel = "", where = "!htbp", file = filename, append = T)%
\setlongtables{\small
\begin{longtable}{llllllllll}
\caption{Summary of careless responses in the IMI data collected over the first empirical study}
\label{tab:careless-IMI-study01} \tabularnewline
\hline\hline
\multicolumn{1}{c}{@@}&\multicolumn{1}{c}{UserID}&\multicolumn{1}{c}{Item01}&\multicolumn{1}{c}{Item02}&\multicolumn{1}{c}{Item03}&\multicolumn{1}{c}{Item04}&\multicolumn{1}{c}{Item05}&\multicolumn{1}{c}{Item06}&\multicolumn{1}{c}{Item07}&\multicolumn{1}{c}{...}\tabularnewline
\hline
\endfirsthead\caption[]{\em (continued)} \tabularnewline
\hline
\multicolumn{1}{c}{@@}&\multicolumn{1}{c}{UserID}&\multicolumn{1}{c}{Item01}&\multicolumn{1}{c}{Item02}&\multicolumn{1}{c}{Item03}&\multicolumn{1}{c}{Item04}&\multicolumn{1}{c}{Item05}&\multicolumn{1}{c}{Item06}&\multicolumn{1}{c}{Item07}&\multicolumn{1}{c}{...}\tabularnewline
\hline
\endhead
\hline
\endfoot
\label{as.data.frame}
...&...&...&...&...&...&...&...&...&...\tabularnewline
&10213&7&4&2&7&4&1&4&...\tabularnewline
---&10229&4&4&4&4&4&4&4&...\tabularnewline
---&10241&1&1&1&1&1&1&1&...\tabularnewline
\hline
\end{longtable}}

\autoref{tab:careless-IMI-study03} shows the four careless responses identified and removed from the IMI data collected over the third empirical study. These careless responses correspond to participants with user IDs 10178, 10196, 10211 and 10240. These four careless responses were identified in 55 responses collected from computer engineer undergraduate students by means of the web-based questionnaire version of IMI (shown in \autoref{annex:IMI-IMMS-third-study}).

%latex.default(as.data.frame(dd$get_data()), rowname = NULL, caption = paste("Summary of careless responses",     in_title), size = "small", longtable = T, ctable = F, landscape = F,     rowlabel = "", where = "!htbp", file = filename, append = T)%
\setlongtables{\small
\begin{longtable}{llllllll}
\caption{Summary of careless responses in the IMI data collected over the third empirical study}
\label{tab:careless-IMI-study03} \tabularnewline
\hline\hline
\multicolumn{1}{c}{@@}&\multicolumn{1}{c}{UserID}&\multicolumn{1}{c}{Item01}&\multicolumn{1}{c}{Item02}&\multicolumn{1}{c}{Item03}&\multicolumn{1}{c}{Item04}&\multicolumn{1}{c}{Item05}&\multicolumn{1}{c}{...}\tabularnewline
\hline
\endfirsthead\caption[]{\em (continued)} \tabularnewline
\hline
\multicolumn{1}{c}{@@}&\multicolumn{1}{c}{UserID}&\multicolumn{1}{c}{Item01}&\multicolumn{1}{c}{Item02}&\multicolumn{1}{c}{Item03}&\multicolumn{1}{c}{Item04}&\multicolumn{1}{c}{Item05}&\multicolumn{1}{c}{...}\tabularnewline
\hline
\endhead
\hline
\endfoot
\label{as.data.frame}
...&...&...&...&...&...&...&...\tabularnewline
&10176&3&6&4&4&5&...\tabularnewline
---&10178&4&4&4&4&4&...\tabularnewline
&10179&6&5&6&6&2&...\tabularnewline
...&...&...&...&...&...&...&...\tabularnewline
&10193&1&1&1&1&2&...\tabularnewline
---&10196&4&4&4&4&4&...\tabularnewline
&10197&4&4&4&4&4&...\tabularnewline
...&...&...&...&...&...&...&...\tabularnewline
&10210&1&7&7&4&1&...\tabularnewline
---&10211&4&4&4&4&4&...\tabularnewline
&10213&1&7&7&7&3&...\tabularnewline
...&...&...&...&...&...&...&...\tabularnewline
&10238&3&5&5&5&4&...\tabularnewline
---&10240&4&4&4&4&4&...\tabularnewline
\hline
\end{longtable}}

\subsection{Instructional Materials Motivation Survey Data}

No one careless response has been identified in 58 responses collected over the second empirical study by means of the paper-based questionnaire of the adapted Portuguese IMMS (\autoref{annex:IMMS-second-study}). \autoref{tab:careless-IMMS-study03} shows the three careless responses identified and removed in the IMMS data collected over the third empirical study. These careless responses correspond to participants with user IDs 10196, 10211 and 10240; and they were identified in 55 responses collected from computer engineering undergraduate students by means of the web-based questionnaire version of IMMS (shown in \autoref{annex:IMI-IMMS-third-study}).

%latex.default(as.data.frame(dd$get_data()), rowname = NULL, caption = paste("Summary of careless responses",     in_title), size = "small", longtable = T, ctable = F, landscape = F,     rowlabel = "", where = "!htbp", file = filename, append = T)%
\setlongtables{\small
\begin{longtable}{llllllll}
\caption{Summary of careless responses in the IMMS data collected over the third empirical study}
\label{tab:careless-IMMS-study03} \tabularnewline
\hline\hline
\multicolumn{1}{c}{@@}&\multicolumn{1}{c}{UserID}&\multicolumn{1}{c}{Item01}&\multicolumn{1}{c}{Item02}&\multicolumn{1}{c}{Item03}&\multicolumn{1}{c}{Item04}&\multicolumn{1}{c}{Item06}&\multicolumn{1}{c}{...}\tabularnewline
\hline
\endfirsthead\caption[]{\em (continued)} \tabularnewline
\hline
\multicolumn{1}{c}{@@}&\multicolumn{1}{c}{UserID}&\multicolumn{1}{c}{Item01}&\multicolumn{1}{c}{Item02}&\multicolumn{1}{c}{Item03}&\multicolumn{1}{c}{Item04}&\multicolumn{1}{c}{Item06}&\multicolumn{1}{c}{...}\tabularnewline
\hline
\endhead
\hline
\endfoot
\label{as.data.frame}
...&...&...&...&...&...&...&...\tabularnewline
&10193&1&1&1&2&1&...\tabularnewline
---&10196&4&4&4&4&4&...\tabularnewline
&10197&7&5&3&7&5&...\tabularnewline
...&...&...&...&...&...&...&...\tabularnewline
&10210&1&1&1&1&1&...\tabularnewline
---&10211&4&4&4&4&4&...\tabularnewline
&10213&5&7&7&5&7&...\tabularnewline
...&...&...&...&...&...&...&...\tabularnewline
&10238&4&5&5&4&5&...\tabularnewline
---&10240&4&4&4&4&4&...\tabularnewline
\hline
\end{longtable}}


\section{Winsorizing Extreme Values}
\label{subsec:winsorizing-motivation-survey}

In surveys, a extreme value is an outliers that happens when a participant has an extreme response style score \cite{Lavrakas2008}. These tendency of some participants to answer surveys indicating extreme lower and upper values generates representative outliers that cannot simply removed from the data for the statistical analyses. Thus, to reduce the impact of extreme values in the surveys, by transforming the extreme values into a specified percentile of the data, the data collected by the motivation surveys had been Winsorized. Winsorization is a method that shrink extreme values to the border of the main part of the data, and it had been carried out with the robustHD package version 0.5 \cite{Alfons2016} in R software version 3.4.3 \cite{RCoreTeam2017}.

\subsection{Intrinsic Motivation Inventory Data}

\autoref{tab:winsorized-data-IMI} show the responses identified as extreme values in the data collected by means of the adapted Portuguese IMI over the empirical studies. This table also shows how these extreme values had been transformed into the trimmed minimum and maximum values by the Winsorization method for the validation of the adapted Portuguese IMI.

\subsection{Instructional Materials Motivation Survey Data}

\autoref{tab:winsorized-data-IMMS} presents the responses identified as extreme values in the data collected over the empirical studies by means of the adapted Portuguese version of IMMS. This table also shows the changes in these extreme values by the Winsorization method for the validation of the adapted Portuguese IMMS.

%latex.default(as.data.frame(dd$get_data()), rowname = NULL, caption = paste("Summary of winzorized responses",     in_title), size = "scriptsize", longtable = T, ctable = F,     landscape = T, rowlabel = "", where = "!htbp", file = filename,     append = T)%
\setlongtables\begin{landscape}{\fontsize{7}{7}\selectfont
\begin{longtable}{lllllllllllllllllllllll}\caption{Summary of Winsorized responses for the validation of adapted Portuguese IMI}
\label{tab:winsorized-data-IMI} \tabularnewline
\hline\hline
\multicolumn{1}{c}{@@}&\multicolumn{1}{c}{Study}&\multicolumn{1}{c}{UserID}&\multicolumn{1}{c}{Item01}&\multicolumn{1}{c}{Item02}&\multicolumn{1}{c}{Item03}&\multicolumn{1}{c}{Item04}&\multicolumn{1}{c}{Item05}&\multicolumn{1}{c}{Item06}&\multicolumn{1}{c}{Item07}&\multicolumn{1}{c}{Item08}&\multicolumn{1}{c}{Item09}&\multicolumn{1}{c}{...}&\multicolumn{1}{c}{Item12}&\multicolumn{1}{c}{Item13}&\multicolumn{1}{c}{Item14}&\multicolumn{1}{c}{Item15}&\multicolumn{1}{c}{Item16}&\multicolumn{1}{c}{Item17}&\multicolumn{1}{c}{Item18}&\multicolumn{1}{c}{Item19}&\multicolumn{1}{c}{Item20}&\multicolumn{1}{c}{Item21}\tabularnewline
\hline
\endfirsthead\caption[]{\em (continued)} \tabularnewline
\hline
\multicolumn{1}{c}{@@}&\multicolumn{1}{c}{Study}&\multicolumn{1}{c}{UserID}&\multicolumn{1}{c}{Item01}&\multicolumn{1}{c}{Item02}&\multicolumn{1}{c}{Item03}&\multicolumn{1}{c}{Item04}&\multicolumn{1}{c}{Item05}&\multicolumn{1}{c}{Item06}&\multicolumn{1}{c}{Item07}&\multicolumn{1}{c}{Item08}&\multicolumn{1}{c}{Item09}&\multicolumn{1}{c}{...}&\multicolumn{1}{c}{Item12}&\multicolumn{1}{c}{Item13}&\multicolumn{1}{c}{Item14}&\multicolumn{1}{c}{Item15}&\multicolumn{1}{c}{Item16}&\multicolumn{1}{c}{Item17}&\multicolumn{1}{c}{Item18}&\multicolumn{1}{c}{Item19}&\multicolumn{1}{c}{Item20}&\multicolumn{1}{c}{Item21}\tabularnewline
\hline
\endhead
\hline
\endfoot
\label{as.data.frame}
...&...&...&...&...&...&...&...&...&...&...&...&...&...&...&...&...&...&...&...&...&...&...\tabularnewline
&pilot&10126&5&2&5&7&6&2&5&1&6&...&6&1&2&2&1&2&1&2&2&6\tabularnewline
-\textgreater &pilot&10127&4&2&5&7&6&3&5&4&6&...&6&6-\textgreater 5&2&3&4&4&2&4&1&6\tabularnewline
-\textgreater &pilot&10128&1&4&1-\textgreater 2&1-\textgreater 3&7&1&7-\textgreater 6&7-\textgreater 6&1&...&1&7-\textgreater 5&1&1&1&1&1&7&7-\textgreater 6&1\tabularnewline
-\textgreater &pilot&10129&1&7&7&4&2&4&2&4&2&...&1&5&7-\textgreater 5&5&5&4&7-\textgreater 5&4&2&3\tabularnewline
&pilot&10130&4&3&6&5&4&5&2&5&5&...&4&1&4&4&5&4&5&5&3&5\tabularnewline
-\textgreater &pilot&10131&4&5&6&7&4&7&1&2&5&...&7&3&1&7&2&7-\textgreater 6&2&7&1&7\tabularnewline
-\textgreater &pilot&10132&1&7&7&6&4&4&3&4&1&...&1&3&2&3&2&2&2&7&7-\textgreater 6&1\tabularnewline
&pilot&10134&2&6&4&4&3&3&4&6&2&...&2&4&3&5&3&5&3&4&4&2\tabularnewline
&pilot&10135&5&4&5&7&7&1&5&6&5&...&5&2&4&2&4&2&4&2&2&5\tabularnewline
-\textgreater &pilot&10136&1&7&3&1-\textgreater 3&1&6&6&7-\textgreater 6&1&...&1&5&1&5&3&7-\textgreater 6&7-\textgreater 5&4&6&1\tabularnewline
&pilot&10137&5&1&5&4&5&1&1&1&5&...&6&1&1&4&1&1&1&3&1&6\tabularnewline
...&...&...&...&...&...&...&...&...&...&...&...&...&...&...&...&...&...&...&...&...&...&...\tabularnewline
&pilot&10139&1&4&5&7&7&4&5&1&6&...&3&4&4&7&3&1&1&1&1&5\tabularnewline
-\textgreater &pilot&10140&1&4&3&2-\textgreater 3&4&4&4&4&4&...&1&5&6-\textgreater 5&4&4&4&5&4&5&4\tabularnewline
&pilot&10141&2&4&5&4&4&6&3&4&5&...&3&5&4&4&3&4&1&5&3&5\tabularnewline
...&...&...&...&...&...&...&...&...&...&...&...&...&...&...&...&...&...&...&...&...&...&...\tabularnewline
&pilot&10143&4&4&6&6&5&3&2&2&4&...&4&2&2&4&3&2&2&2&2&5\tabularnewline
-\textgreater &pilot&10145&1&7&7&4&1&4&1&4&4&...&4&1&4&7&4&4&4&4&7-\textgreater 6&3\tabularnewline
&pilot&10146&5&3&6&6&6&3&2&2&5&...&5&1&3&4&3&3&3&3&3&5\tabularnewline
&pilot&10148&4&3&6&4&4&1&1&1&4&...&4&2&2&4&5&3&5&4&1&6\tabularnewline
-\textgreater &pilot&10149&4&3&7&1-\textgreater 3&1&7&5&7-\textgreater 6&1&...&1&1&7-\textgreater 5&6&7-\textgreater 5&6&5&7&7-\textgreater 6&7\tabularnewline
-\textgreater &pilot&10152&7&3&7&6&7&4&7-\textgreater 6&1&7&...&7&5&3&2&2&2&2&7&2&7\tabularnewline
&pilot&10153&5&4&5&5&7&3&2&2&5&...&5&5&2&6&2&5&2&4&4&3\tabularnewline
-\textgreater &pilot&10154&3&6&5&4&2&6&3&5&4&...&2&5&2&6&6-\textgreater 5&6&3&4&7-\textgreater 6&1\tabularnewline
&pilot&10158&7&3&7&7&4&4&1&1&6&...&7&1&1&4&1&1&1&2&2&6\tabularnewline
...&...&...&...&...&...&...&...&...&...&...&...&...&...&...&...&...&...&...&...&...&...&...\tabularnewline
&first&10171&6&2&3&7&7&2&1&1&6&...&6&1&1&3&1&2&1&2&1&5\tabularnewline
-\textgreater &first&10172&6&4&5&7&3&2&4&3&6&...&5&6-\textgreater 5&1&2&1&2&1&2&3&5\tabularnewline
&first&10174&4&4&6&7&4&4&1&3&5&...&4&1&1&4&1&3&1&2&3&4\tabularnewline
...&...&...&...&...&...&...&...&...&...&...&...&...&...&...&...&...&...&...&...&...&...&...\tabularnewline
&first&10201&5&2&7&7&7&3&1&2&7&...&7&1&3&2&2&2&1&2&1&7\tabularnewline
-\textgreater &first&10202&4&7&7&7&4&4&1&5&4&...&4&1&7-\textgreater 5&4&7-\textgreater 5&4&7-\textgreater 5&4&4&4\tabularnewline
&first&10203&7&1&5&7&7&1&5&1&7&...&7&2&1&1&2&1&1&1&1&1\tabularnewline
...&...&...&...&...&...&...&...&...&...&...&...&...&...&...&...&...&...&...&...&...&...&...\tabularnewline
&first&10208&6&4&7&7&7&4&1&1&6&...&7&1&1&2&4&1&1&2&2&7\tabularnewline
-\textgreater &first&10209&4&7&7&7&4&7&1&7-\textgreater 6&3&...&3&1&1&7&1&7-\textgreater 6&1&4&5&3\tabularnewline
-\textgreater &first&10210&4&6&6&6&3&5&1&5&5&...&4&1&4&6&4&4&6-\textgreater 5&5&5&2\tabularnewline
-\textgreater &first&10211&7&7&3&3&3&6&6&3&7&...&5&6-\textgreater 5&2&6&2&3&3&2&2&6\tabularnewline
&first&10212&5&3&6&6&7&3&2&1&6&...&6&2&2&1&4&2&2&1&1&7\tabularnewline
\newpage
...&...&...&...&...&...&...&...&...&...&...&...&...&...&...&...&...&...&...&...&...&...&...\tabularnewline
&first&10221&4&3&6&6&6&4&2&1&5&...&4&1&1&4&1&1&1&7&4&5\tabularnewline
-\textgreater &first&10222&1&7&4&5&2&6&4&7-\textgreater 6&1&...&1&2&3&7&3&6&4&6&6&1\tabularnewline
&first&10223&6&2&6&5&6&2&3&1&6&...&6&3&1&2&2&1&1&6&3&7\tabularnewline
-\textgreater &first&10224&6&4&4&7&7&4&4&7-\textgreater 6&7&...&7&2&1&7&1&7-\textgreater 6&3&5&3&7\tabularnewline
&first&10226&2&6&6&6&2&4&1&5&4&...&4&1&1&6&1&4&3&5&2&4\tabularnewline
...&...&...&...&...&...&...&...&...&...&...&...&...&...&...&...&...&...&...&...&...&...&...\tabularnewline
&first&10238&4&4&5&7&3&5&2&4&3&...&4&2&4&7&4&5&5&3&2&4\tabularnewline
-\textgreater &first&10240&4&6&5&6&6&7&4&5&3&...&4&3&1&7&1&7-\textgreater 6&1&7&7-\textgreater 6&2\tabularnewline
-\textgreater &third&10169&7&4&5&7&5&2&5&2&7&...&7&5&7-\textgreater 5&4&7-\textgreater 5&2&6-\textgreater 5&4&2&7\tabularnewline
&third&10170&3&2&4&6&4&1&3&3&4&...&4&3&1&3&1&2&1&2&2&6\tabularnewline
&third&10171&5&2&7&7&6&2&1&1&7&...&5&1&2&2&2&2&1&1&2&6\tabularnewline
-\textgreater &third&10172&4&5&4&6&5&5&5&5&5&...&5&5&4&5&6-\textgreater 5&6&5&5&5&7\tabularnewline
-\textgreater &third&10174&3&4&4&4&5&3&2&4&4&...&6&5&4&4&5&4&6-\textgreater 5&6&5&6\tabularnewline
&third&10175&4&1&4&5&7&1&1&1&4&...&4&1&1&3&1&1&1&1&1&4\tabularnewline
-\textgreater &third&10176&3&6&4&4&5&5&5&4&4&...&4&4&4&4&6-\textgreater 5&4&6-\textgreater 5&5&5&4\tabularnewline
&third&10179&6&5&6&6&2&5&1&4&4&...&4&2&1&7&1&6&2&4&4&3\tabularnewline
-\textgreater &third&10181&2&5&1-\textgreater 2&2-\textgreater 3&3&5&4&3&3&...&2&7-\textgreater 5&4&5&1&1&5&5&4&2\tabularnewline
-\textgreater &third&10183&3&2&2&2-\textgreater 3&2&3&3&5&3&...&2&4&4&4&4&4&5&4&4&3\tabularnewline
&third&10184&4&3&3&5&3&4&4&4&3&...&4&4&4&4&3&5&4&4&4&4\tabularnewline
...&...&...&...&...&...&...&...&...&...&...&...&...&...&...&...&...&...&...&...&...&...&...\tabularnewline
&third&10189&1&5&5&4&1&4&3&4&2&...&2&1&1&5&1&5&2&6&5&4\tabularnewline
-\textgreater &third&10190&2&3&4&5&5&5&5&4&4&...&4&3&5&7&7-\textgreater 5&7-\textgreater 6&6-\textgreater 5&4&5&3\tabularnewline
&third&10191&1&5&4&4&2&4&4&5&2&...&2&2&1&5&1&5&1&6&4&3\tabularnewline
&third&10192&4&3&7&6&4&4&5&4&5&...&1&1&4&3&4&2&4&4&3&4\tabularnewline
-\textgreater &third&10193&1&1&1-\textgreater 2&1-\textgreater 3&2&1&2&1&1&...&2&1&1&1&1&1&2&1&1&1\tabularnewline
&third&10197&4&4&4&4&4&2&3&4&2&...&2&3&3&2&3&3&5&3&3&5\tabularnewline
...&...&...&...&...&...&...&...&...&...&...&...&...&...&...&...&...&...&...&...&...&...&...\tabularnewline
&third&10203&5&1&7&7&4&1&1&1&4&...&4&1&2&4&5&1&1&1&5&3\tabularnewline
-\textgreater &third&10204&1&7&4&4&1&7&1&7-\textgreater 6&1&...&1&4&7-\textgreater 5&7&7-\textgreater 5&7-\textgreater 6&7-\textgreater 5&7&4&1\tabularnewline
&third&10206&3&3&5&5&5&2&2&2&5&...&4&2&3&3&2&2&3&2&3&4\tabularnewline
&third&10208&3&4&5&3&5&4&2&2&4&...&4&2&2&4&2&2&2&5&5&4\tabularnewline
-\textgreater &third&10209&2&7&3&3&1&7&3&7-\textgreater 6&2&...&2&2&2&7&1&7-\textgreater 6&1&5&6&2\tabularnewline
-\textgreater &third&10210&1&7&7&4&1&7&1&7-\textgreater 6&1&...&1&1&7-\textgreater 5&7&7-\textgreater 5&7-\textgreater 6&7-\textgreater 5&7&7-\textgreater 6&1\tabularnewline
-\textgreater &third&10213&1&7&7&7&3&7&1&7-\textgreater 6&4&...&4&1&1&7&7-\textgreater 5&7-\textgreater 6&7-\textgreater 5&7&1&4\tabularnewline
&third&10214&1&6&7&4&1&7&1&6&1&...&1&1&5&6&1&2&4&5&6&2\tabularnewline
...&...&...&...&...&...&...&...&...&...&...&...&...&...&...&...&...&...&...&...&...&...&...\tabularnewline
&third&10218&4&4&4&4&3&5&4&4&4&...&4&4&4&4&5&5&4&3&3&4\tabularnewline
-\textgreater &third&10219&4&1&1-\textgreater 2&1-\textgreater 3&1&4&7-\textgreater 6&7-\textgreater 6&1&...&2&5&1&7&1&7-\textgreater 6&1&7&7-\textgreater 6&1\tabularnewline
&third&10220&1&1&4&7&7&4&1&1&4&...&4&1&1&1&1&1&1&1&4&4\tabularnewline
\newpage
-\textgreater &third&10221&2&6&5&3&2&5&3&6&3&...&3&2&4&4&2&4&6-\textgreater 5&4&6&3\tabularnewline
&third&10223&5&4&5&5&4&3&4&4&5&...&5&4&4&4&4&4&2&4&3&6\tabularnewline
&third&10224&4&4&5&5&3&4&5&4&4&...&4&4&4&5&5&5&5&4&3&4\tabularnewline
-\textgreater &third&10226&1&7&1-\textgreater 2&1-\textgreater 3&1&7&7-\textgreater 6&7-\textgreater 6&1&...&1&7-\textgreater 5&7-\textgreater 5&7&7-\textgreater 5&7-\textgreater 6&7-\textgreater 5&4&4&4\tabularnewline
&third&10227&5&3&4&5&5&3&3&3&3&...&5&5&5&4&4&4&3&3&3&6\tabularnewline
-\textgreater &third&10228&1&1&1-\textgreater 2&1-\textgreater 3&1&1&1&2&1&...&2&2&2&2&3&2&2&1&2&2\tabularnewline
-\textgreater &third&10230&4&3&3&6&4&3&2&5&5&...&5&3&5&5&4&4&6-\textgreater 5&6&3&5\tabularnewline
-\textgreater &third&10231&6&2&1-\textgreater 2&6&7&2&4&2&4&...&5&3&1&1&1&1&1&2&4&4\tabularnewline
&third&10232&4&3&5&5&4&4&1&4&4&...&5&1&2&3&2&3&2&1&2&5\tabularnewline
-\textgreater &third&10234&1&7&1-\textgreater 2&5&3&6&6&6&2&...&2&6-\textgreater 5&4&6&2&6&2&6&5&2\tabularnewline
-\textgreater &third&10237&4&3&4&5&5&3&4&4&5&...&6&2&6-\textgreater 5&6&5&3&4&4&4&3\tabularnewline
&third&10238&3&5&5&5&4&5&2&4&4&...&4&1&5&7&5&5&5&4&4&4\tabularnewline
\hline
\end{longtable}}\end{landscape}


%latex.default(as.data.frame(dd$get_data()), rowname = NULL, caption = paste("Summary of Winsorized responses",     in_title), size = "scriptsize", longtable = T, ctable = F,     landscape = T, rowlabel = "", where = "!htbp", file = filename,     append = T)%
\setlongtables\begin{landscape}{\fontsize{7}{7}\selectfont
\begin{longtable}{lllllllllllllllllllll}\caption{Summary of Winsorized responses for the validation of adapted Portuguese IMMS}
\label{tab:winsorized-data-IMMS} \tabularnewline
\hline\hline
\multicolumn{1}{c}{@@}&\multicolumn{1}{c}{Study}&\multicolumn{1}{c}{UserID}&\multicolumn{1}{c}{Item01}&\multicolumn{1}{c}{Item02}&\multicolumn{1}{c}{Item03}&\multicolumn{1}{c}{Item04}&\multicolumn{1}{c}{Item06}&\multicolumn{1}{c}{Item07}&\multicolumn{1}{c}{Item08}&\multicolumn{1}{c}{Item09}&\multicolumn{1}{c}{Item10}&\multicolumn{1}{c}{Item11}&\multicolumn{1}{c}{...}&\multicolumn{1}{c}{Item20}&\multicolumn{1}{c}{Item21}&\multicolumn{1}{c}{Item22}&\multicolumn{1}{c}{Item23}&\multicolumn{1}{c}{Item24}&\multicolumn{1}{c}{Item25}&\multicolumn{1}{c}{Item26}\tabularnewline
\hline
\endfirsthead\caption[]{\em (continued)} \tabularnewline
\hline
\multicolumn{1}{c}{@@}&\multicolumn{1}{c}{Study}&\multicolumn{1}{c}{UserID}&\multicolumn{1}{c}{Item01}&\multicolumn{1}{c}{Item02}&\multicolumn{1}{c}{Item03}&\multicolumn{1}{c}{Item04}&\multicolumn{1}{c}{Item06}&\multicolumn{1}{c}{Item07}&\multicolumn{1}{c}{Item08}&\multicolumn{1}{c}{Item09}&\multicolumn{1}{c}{Item10}&\multicolumn{1}{c}{Item11}&\multicolumn{1}{c}{...}&\multicolumn{1}{c}{Item20}&\multicolumn{1}{c}{Item21}&\multicolumn{1}{c}{Item22}&\multicolumn{1}{c}{Item23}&\multicolumn{1}{c}{Item24}&\multicolumn{1}{c}{Item25}&\multicolumn{1}{c}{Item26}\tabularnewline
\hline
\endhead
\hline
\endfoot
\label{as.data.frame}
-\textgreater &second&10169&2&2&1-\textgreater 2&2&4&2&5&1&6&2&...&2&3&6&4&3&7&2\tabularnewline
&second&10170&4&6&6&5&4&6&4&4&2&5&...&3&2&3&2&6&6&6\tabularnewline
...&...&...&...&...&...&...&...&...&...&...&...&...&...&...&...&...&...&...&...&...\tabularnewline
&second&10185&4&4&4&5&4&3&2&3&3&4&...&3&4&2&2&4&4&4\tabularnewline
-\textgreater &second&10186&1&5&1-\textgreater 2&4&7&4&4&4&4&3&...&4&3&4&1&7&7&6\tabularnewline
&second&10187&6&5&5&4&4&4&5&5&5&4&...&5&6&6&5&5&5&6\tabularnewline
&second&10188&4&4&5&5&5&4&1&4&2&4&...&4&4&4&3&5&4&4\tabularnewline
-\textgreater &second&10189&1&2&2&1&1-\textgreater 2&1&7-\textgreater 6&1&7-\textgreater 6&1&...&2&7-\textgreater 6&2&7-\textgreater 6&1-\textgreater 2&3&1\tabularnewline
-\textgreater &second&10190&3&5&7&4&4&5&1&4&6&6&...&1&5&5&7-\textgreater 6&7&5&2\tabularnewline
&second&10191&6&7&7&6&6&7&2&4&1&5&...&6&2&4&1&7&7&7\tabularnewline
...&...&...&...&...&...&...&...&...&...&...&...&...&...&...&...&...&...&...&...&...\tabularnewline
&second&10193&3&2&4&3&5&4&1&3&4&1&...&3&3&2&5&4&3&4\tabularnewline
-\textgreater &second&10196&4&7&3&1&7&4&4&4&7-\textgreater 6&4&...&4&5&4&3&2&4&5\tabularnewline
&second&10197&7&5&5&5&5&4&4&4&1&3&...&4&3&3&5&4&6&3\tabularnewline
...&...&...&...&...&...&...&...&...&...&...&...&...&...&...&...&...&...&...&...&...\tabularnewline
&second&10208&4&5&4&3&5&3&3&4&6&5&...&4&4&4&3&5&4&5\tabularnewline
-\textgreater &second&10209&1&1-\textgreater 2&7&1&4&2&1&4&6&1&...&1&6&1&1&1-\textgreater 2&4&4\tabularnewline
&second&10210&4&4&4&4&6&2&2&2&3&4&...&3&6&6&3&3&3&3\tabularnewline
-\textgreater &second&10211&1&1-\textgreater 2&1-\textgreater 2&1&1-\textgreater 2&1&1&2&5&3&...&1&4&1&4&4&1-\textgreater 2&1\tabularnewline
&second&10212&2&4&6&4&4&3&2&3&4&5&...&3&3&2&3&4&5&4\tabularnewline
...&...&...&...&...&...&...&...&...&...&...&...&...&...&...&...&...&...&...&...&...\tabularnewline
&second&10218&4&7&3&2&7&4&3&4&2&5&...&2&3&4&1&7&7&7\tabularnewline
-\textgreater &second&10219&1&2&4&1&1-\textgreater 2&1&4&4&4&3&...&3&6&4&5&2&3&4\tabularnewline
&second&10220&1&7&5&1&7&7&1&7&1&7&...&4&1&1&4&7&7&7\tabularnewline
&second&10221&4&4&6&2&4&6&2&4&6&6&...&4&5&5&5&4&6&4\tabularnewline
-\textgreater &second&10223&1&1-\textgreater 2&4&1&2&4&4&2&5&4&...&3&6&2&4&2&2&2\tabularnewline
&second&10224&4&7&7&7&7&5&1&4&1&4&...&5&2&6&2&6&6&5\tabularnewline
-\textgreater &second&10226&6&7&1-\textgreater 2&4&5&4&3&4&1&6&...&5&1&4&6&6&6&4\tabularnewline
&second&10227&7&7&6&5&6&6&2&6&2&6&...&6&2&4&2&6&6&6\tabularnewline
...&...&...&...&...&...&...&...&...&...&...&...&...&...&...&...&...&...&...&...&...\tabularnewline
&second&10231&3&3&6&4&6&5&1&1&4&3&...&2&5&4&4&6&5&4\tabularnewline
-\textgreater &second&10232&5&5&5&5&6&5&2&5&2&6&...&5&1&6&1&7&1-\textgreater 2&6\tabularnewline
&second&10233&2&4&6&3&2&1&4&1&5&4&...&7&5&5&4&6&4&3\tabularnewline
...&...&...&...&...&...&...&...&...&...&...&...&...&...&...&...&...&...&...&...&...\tabularnewline
&second&10240&2&2&4&2&2&4&4&2&5&4&...&2&5&2&4&3&2&2\tabularnewline
-\textgreater &second&10242&1&1-\textgreater 2&1-\textgreater 2&1&5&1&4&1&7-\textgreater 6&1&...&1&6&3&3&7&4&4\tabularnewline
&third&10169&7&7&5&7&7&7&3&7&2&6&...&7&1&7&1&7&7&7\tabularnewline
...&...&...&...&...&...&...&...&...&...&...&...&...&...&...&...&...&...&...&...&...\tabularnewline
&third&10188&3&3&4&4&4&3&2&2&1&2&...&3&2&2&3&4&4&4\tabularnewline
-\textgreater &third&10189&1&3&1-\textgreater 2&2&4&1&1&1&3&3&...&1&6&1&5&2&4&3\tabularnewline
\newpage
-\textgreater &third&10190&3&6&5&6&6&5&4&4&5&3&...&5&5&5&4&1-\textgreater 2&3&3\tabularnewline
&third&10191&4&4&3&2&4&2&6&2&3&3&...&2&3&1&2&3&3&5\tabularnewline
&third&10192&4&4&5&4&4&4&4&4&4&4&...&7&1&7&4&6&5&6\tabularnewline
-\textgreater &third&10193&1&1-\textgreater 2&1-\textgreater 2&2&1-\textgreater 2&2&1&1&2&1&...&1&1&1&1&1-\textgreater 2&1-\textgreater 2&1\tabularnewline
&third&10197&7&5&3&7&5&4&4&5&1&4&...&4&5&1&3&5&4&3\tabularnewline
...&...&...&...&...&...&...&...&...&...&...&...&...&...&...&...&...&...&...&...&...\tabularnewline
&third&10203&4&7&4&4&7&7&3&6&4&4&...&4&4&2&1&5&4&5\tabularnewline
-\textgreater &third&10204&1&1-\textgreater 2&4&1&1-\textgreater 2&1&4&1&7-\textgreater 6&1&...&1&1&1&7-\textgreater 6&1-\textgreater 2&4&1\tabularnewline
&third&10206&4&5&4&4&6&4&3&4&3&5&...&4&3&4&3&4&6&5\tabularnewline
&third&10208&2&4&3&2&4&2&3&4&6&3&...&3&5&3&3&4&3&4\tabularnewline
-\textgreater &third&10209&2&4&1-\textgreater 2&2&4&1&1&4&6&4&...&3&4&1&5&4&4&4\tabularnewline
-\textgreater &third&10210&1&1-\textgreater 2&1-\textgreater 2&1&1-\textgreater 2&1&7-\textgreater 6&1&7-\textgreater 6&1&...&1&7-\textgreater 6&4&1&1-\textgreater 2&4&1\tabularnewline
&third&10213&5&7&7&5&7&7&1&3&1&4&...&2&2&5&5&7&4&4\tabularnewline
-\textgreater &third&10214&3&2&2&1&1-\textgreater 2&1&3&1&5&3&...&2&5&1&6&2&2&2\tabularnewline
&third&10215&5&2&2&4&2&1&1&5&1&4&...&1&1&1&1&2&4&3\tabularnewline
...&...&...&...&...&...&...&...&...&...&...&...&...&...&...&...&...&...&...&...&...\tabularnewline
&third&10220&4&7&4&1&7&1&1&4&1&4&...&1&1&4&4&4&7&1\tabularnewline
-\textgreater &third&10221&2&2&4&4&3&3&7-\textgreater 6&3&4&4&...&2&6&3&4&4&3&2\tabularnewline
&third&10223&4&4&3&3&4&4&4&4&3&4&...&4&2&5&3&5&7&5\tabularnewline
&third&10224&5&5&5&4&5&4&3&4&3&5&...&4&4&4&4&4&5&4\tabularnewline
-\textgreater &third&10226&4&4&3&3&2&1&5&1&4&4&...&4&4&4&5&1-\textgreater 2&5&2\tabularnewline
&third&10227&5&4&5&5&4&5&2&5&2&5&...&6&4&6&4&5&6&6\tabularnewline
-\textgreater &third&10228&1&1-\textgreater 2&1-\textgreater 2&1&1-\textgreater 2&1&1&1&2&1&...&2&7-\textgreater 6&1&2&2&2&3\tabularnewline
&third&10230&4&5&5&3&2&2&3&6&4&7&...&5&3&2&5&3&4&4\tabularnewline
...&...&...&...&...&...&...&...&...&...&...&...&...&...&...&...&...&...&...&...&...\tabularnewline
&third&10232&4&6&5&4&5&4&1&4&2&6&...&4&2&7&2&6&7&4\tabularnewline
-\textgreater &third&10234&6&2&1-\textgreater 2&2&6&2&2&1&1&1&...&6&4&1&6&5&4&2\tabularnewline
&third&10237&3&5&3&4&2&3&2&4&4&5&...&4&4&2&4&5&5&4\tabularnewline
...&...&...&...&...&...&...&...&...&...&...&...&...&...&...&...&...&...&...&...&...\tabularnewline
\hline
\end{longtable}}\end{landscape}




 \chapter{Validation of Motivation Surveys}
\label{appendix:validation-motivation-surveys}

This appendix details the validation process on the motivation surveys employed in the empirical studies of this PhD dissertation.
These instruments are the Intrinsic Motivation Inventory (IMI), and the Instructional Materials Motivation Survey (IMMS).
Both instruments have been adapted and translated from their original English versions into Portuguese by the thesis author to measure the students' motivation regarding to their participation in CL sessions.
Thus, the validation and reliability analysis presented here ensure that the translated items are psycho-metrically sound.
The procedure for the validation and reliability tests is presented in \autoref{sec:validation-motivation-surveys-process}, and the results of this procedure is detailed in \autoref{sec:validation-motivation-surveys-results}.

\section{Validation Procedure}
\label{sec:validation-motivation-surveys-process}

\subsection{Participants}

The collected data to conduct the validation and reliability tests of the motivation surveys come from 103 undergraduate Brazilian students who were enrolled in the first year of bachelor degree programs in computer science and computer engineering at the University of São Paulo. 37 of these participants were students signed up on the course \aspas{Introduction to Computer Science} for the second semester of 2016 (September-December), and 66 of them were students signed up on the course for the first semester of 2017 (March-July). These participants were in the age range from 18 to 25 years old, sharing similar social-economy status and culture.

\subsection{Instruments}

\subsubsection*{Intrinsic Motivation Inventory (IMI)}

The IMI is a psychometric instrument in which the Self-Determinant Theory (SDT) has been used as theoretical fundament to define seven scales (Interest/Enjoyment, Perceived Choice, Perceived Competence, Pressure/Tension, Effort/Importance, Value/Usefulness, and Relatedness) to measure the intrinsic motivation of participants towards a target activity \cite{MonteiroMataPeixotoMonteiroMataPeixoto2015, RyanDeci2000}. According to the authors of this instrument, no all the scales are needed to measure the intrinsic motivation, the scales can be selected according to the situation, removing those that are redundant and those that are not in accordance to the situation. In the adapted Portuguese IMI, four scales have been selected by the thesis author to measure the intrinsic motivation of Brazilian students towards their participation in CL sessions. These sub-scales are: the Interest/Enjoyment, Perceived Choice, Pressure/Tension, and Effort/Importance.

The Interest/Enjoyment is the self-report direct measure of intrinsic motivation whereby the items related to this scale has been included in the adapted Portuguese IMI. The Perceived Choice and Perceived Competence are both scales defined as positive predictors of the intrinsic motivation, so that the items related to the Perceived Competence had been removed from the instrument, and items related to the Perceived Choice have been selected as the only positive predictor. Furthermore, the scale of the perceived choice has been selected to measure the intrinsic motivation because the thesis author hypothesizes that the scripted collaboration increases the feeling of obligation in the participants. Items related to the Pressure/Tension have been included in the adapted Portuguese IMI as the negative predictor of intrinsic motivation. Items related to the scale of Effort/Importance has been included in the adapted Portuguese IMI to measure the internalization of motivation. Items related to the scale of Relatedness have not been included in the adapted Portuguese IMI because this scale intends to measure the feeling to be connected to others participants in target activity where the goal of activity is obtain interpersonal relationships.

Three questionnaires of the adapted Portuguese IMI had been used to collect the students' motivation data over the empirical studies. These questionnaires in the paper-based version (Annex \ref{annex:IMI-first-study}) and web-based version (Annex \ref{annex:IMI-pilot-study} and Annex \ref{annex:IMI-IMMS-third-study}) comprised 24 items, with all the items scored on a 7-point Likert scale using the ranging from 1 (\emph{not at all}) to 7 (\emph{very true}).

\subsubsection*{Instructional Materials Motivation Survey (IMMS)}

The IMMS is the psychometric instrument developed by \citeonline{Keller2009} to assess the students' motivational attitude towards instructional materials or courses. This instrument has been developed in correspondence with the ACRS model, thereby the scales of of Attention, Relevance, Confidence and Satisfaction (ARCS) are used to measure the reaction of students to instructional materials or course, and this reaction is then considered a self-report measure to the students' motivational attitude.

Instead to use the 36 items defined in the original version of IMMS, the adapted Portuguese IMMS has been defined using only 25 item. 11-items related to the scale of \emph{C: Confidence} have been removed from the instrument, because the scales of \emph{C: Confidence} and \emph{PC: Perceived Choice} measure the self-regulation of an individual. Removing the scale of Confidence in the adapted Portuguese IMMS avoid an overloading of work for the participants when they were requested to answered the questionnaires. Furthermore, the author of the original version of IMMS indicates that each one of the four scales defined in the IMMS could be used and scored independently \cite{Keller2009}. Thus, in the adapted Portuguese IMMS, the students' motivational attitude towards the CL sessions had been measured as the \emph{LM: Level of Motivation}, a measure that consists in the scales of \emph{A: Attention}, \emph{R: Relevance}, and \emph{S: Satisfaction}. 

Two questionnaires of the adapted Portuguese IMMS had been used to collect the students' motivation data over the empirical studies. These questionnaires in the paper-based version (Annex \ref{annex:IMMS-second-study}) and web-based version (Annex \ref{annex:IMI-IMMS-third-study} had been scored on a 7-point Likert scale using the ranging from 1 (\emph{not at all}) to 7 (\emph{very true}).

\subsection{Data Collection Procedure}

Web-based questionnaires of the motivation surveys were used to collect the responses through the Moodle platform during the pilot and third empirical studies, and paper-based questionnaires of these surveys were used at the classroom to collect the responses during the first empirical study. During the pilot study, 32 responses to the adapted Portuguese IMI were collected from the 37 computer science students by means of a web-based questionnaire (detailed in Annex \ref{annex:IMI-pilot-study}). During the first empirical study, 62 responses to the adapted Portuguese IMI were collected from the 66 computer engineering students by means of a paper-based questionnaire (detailed in Annex \ref{annex:IMI-first-study}). During the second empirical study, 58 responses to the adapted Portuguese IMMS were collected from the 66 computer engineering students by means a paper-based questionnaire (detailed in Annex \ref{annex:IMMS-second-study}). During the third empirical study, 55 responses to the adapted Portuguese IMI and the adapted Portuguese IMMS were collected by means of a web-based questionnaire (detailed in Annex \ref{annex:IMI-IMMS-third-study}).

\subsection{Data Analysis}

Although the common statistical advice to perform the validation of surveys indicates a minimum sample size of 300 observations \cite{Kline1986}, recent simulations demonstrated that the validation process is possible with small samples under certain circumstances \cite{GuadagnoliVelicer1988, RouquetteFalissard2011, Yurdugul2008}. According to these studies, to conduct the validation of surveys and the reliability tests of them with small samples, the items must sufficiently correspond to the scale for which they are intended, and the Cronbach’s alpha coefficient ($\alpha$) must be stable in this small sample. Thus, the correspondence of items and scales had been validated by a factorial analysis using \emph{varimax} rotation, and the items with a component loading less than 0.40 and those with a cross-loading value less than 0.20 had been removed from the instrument. The stability of Cronbach’s alpha ($\alpha$) had been evaluated using the cut-off values defined by \citeonline{Yurdugul2008}. According to these cut-off values, if the sample size is between 30 and 50 observations, and the level of the first eigenvalue is less than 6, the Cronbach’s alpha ($\alpha$) is not stable; if the sample size is between 50 and 100, and the level of the first eigenvalue is between 3 and 6, the Cronbach’s alpha ($\alpha$) is stable, but an informed decision must be conducted by reviewing the literature and/or consulting with specialists to confirm the number of scales; and if the sample is between 100 and 300 observation and the level of the first eigenvalue is between 1 to 3, the Cronbach’s alpha ($\alpha$) is stable but a informed decision should be conducted to define the number of scales.

After to verify the correspondence of the items with the scales and to ensure the stability of Cronbach’s alpha ($\alpha$), the structure of the items in the motivation surveys had been evaluated with a Confirmatory Factor Analysis (CFA) by testing three different models: multidimensional, second order and bi-factor models. To select the model that best fits for the collected data, the CFA had been carried out using the diagonally weighted least squares (WLSMV) estimator. The WLSMV estimator is a estimator specifically designed for small samples with ordinal data (such the 7-point Likert scale used in the IMI and IMMS), and it makes no distributional assumptions about the observed variables \cite{Brown2014,Li2016,RhemtullaBrosseau-LiardSavalei2012}. The result of CFA is a set of goodness fit indices used to select the model that best fits for the collected data. These indices were: Chi-square (${\chi}^2$), Adjusted Goodness of Fit Index (AGFI), the Tucker-Lewis Index (TLI), the Comparative Fit Index (CFI) and the Root Mean Square Error of Approximation (RMSEA). As the ${\chi}^2$ is highly sensitive to the sample size \cite{HuBentler1999}, this indicator was only be used in the case that the others indicators do not significantly differ in relation with the others models. In this case, the model that best fits with the collected data is the model with smaller Chi-square (${\chi}^2$). Values between 0.90 to 0.95 were considered acceptable thresholds for the indices of AGFI, TLI and CFI; and values higher than 0.96 were considered good fit. The RMSEA obtained by the CFA had been a scaled value of the RMSEA, so that it was considered acceptable when the value was 0.10s and good when the value was less than 0.10. After to select the model that best fits for the collected data, separate reliability tests had been conducted in the global sample and the samples obtained in each empirical study to evaluate the consistency of the motivation surveys. In these tests, values in the Cronbach’s alpha ($\alpha$) greater than 0.70 were considered as acceptable, and values above 0.80 were considered as highly reliable.

The CFA and reliability tests had been carried out in R software version 3.4.3 \cite{RCoreTeam2017} using the lavaan package version 0.5 \cite{Rosseel2012} for the CFA, and the psych package version 1.7.8 \cite{Revelle2017} for the reliability tests. The R scripts for the validation of the adapted Portuguese IMI and the adapted Portuguese IMMS are available, with the data files, at the URL: \url{https://geiser.github.io/phd-thesis-evaluation/} 

\section{Results}
\label{sec:validation-motivation-surveys-results}

Prior to the data analysis detailed above, the outliers identified as careless responses had been removed from the data, and the outliers identified as extreme values had been treated using the winsorization method. The detection and treatment of these outliers is detailed in \autoref{sec:outliers-motivation}. After to remove the careless responses, the global sample size employed to obtain the results presented here were 141 observations and 110 observations to validate the adapted Portuguese IMI and IMMS, respectively. To validate the Portuguese adapted IMI, the data consisted in 30 observations from the pilot study, 60 observations from the first empirical study, and 51 observations from the third empirical study. To validate the Portuguese adapted IMMS, the data consisted in 58 observations from the empirical study and 52 observations from the third empirical study.

\subsection{Factorial Structure of the Adapted Portuguese IMI}

\autoref{fig:imi-models-cfa} shows the multidimensional, second-order factor and bi-factor models that had been tested in the CFA to validate the factorial structure of the adapted Portuguese IMI. The construction of these models had been conducted according to the criteria defined in the validation procedure by removing items that had loaded with a value less than 0.4, and also, by removing items that had cross-loading less than 0.2. This construction ensures that the items correspond to the scale for which they are intended, and that the Cronbach’s $\alpha$ is stable.

\begin{figure}[htb!]
 \caption{Models tested in the CFA to validate the factorial structure of the adapted Portuguese IMI}
 \label{fig:imi-models-cfa}
 \centering
 \begin{tabular}{ccc}
 \includegraphics[width=0.25\textwidth]{images/appendix/imi-multidimensional-model.png} &
 \includegraphics[width=0.25\textwidth]{images/appendix/imi-second-order-factor-model} &
 \includegraphics[width=0.25\textwidth]{images/appendix/imi-bi-factor-model.png} \\
 {\scriptsize multidimensional model} & {\scriptsize second-order factor model} & {\scriptsize bi-factor model} \\
 \end{tabular}
 \fautor
\end{figure}

As result of the construction of these models, the Item 19 - \aspas{\emph{Achei que a atividade seria chata}} as translated version of \aspas{I thought this was a boring activity} - was removed from the factorial structure because it loads in  the scale of \emph{PC: Perceived Choice} for which it does not have concordance. The Item 04 - \aspas{\emph{Para mim foi importante realizar bem a atividade}} as a translated version of \aspas{It was important to me to do well at this task} - was also removed from the factorial structure because it does not load in the scale of \emph{EI: Effort/Importance} for which it was intended, and because it loads in the scale of \emph{IE: Interest/Enjoyment} where it lacks of concordance. Instead to load in the scale of \emph{PT: Pressure/Tension}, the Item01 - \aspas{\emph{Foi muito descontraido realizar a ativide}} as the translated version of \aspas{I was very relaxed in doing the activity} - loaded in the the scale of \emph{IE: Interest/Enjoyment} because the word \aspas{\emph{descontraido}} was understood by the participants in the sense of enjoyment rather than the pressure. Thus, the Item 01 has been used as an item to measure the Interest/Enjoyment rather than to measure the Pressure/Tension.

\autoref{tab:imi-goodness-fit} shows the goodness fit statistics for the models tested in the validation of the adapted Portuguese IMI. The results presented in this table indicate that all the models have adequate goodness fit indices for all the samples (the global sample, and the data collected over the pilot, first and third empirical studies). The bi-factor model had not converged for the data collected over the third empirical study, and the second-order factor model had partially converged for those data. According to this results, the model that best fits the global sample is the second-order factor model with ${\chi}^2 = 63.27$ that outperforms the multidimensional model (${\chi}^2 = 80.08$) and the bi-factor model (${\chi}^2 = 86.28$). The AGFI index for the multidimensional model and the second-order factor model are better that the AGFI index for the bi-factor model. In relation to the TLI and CFI indices, the the second-order factor model with $TLI = 0.90$ and $CFI = 0.82$ outperforms the multidimensional model ($TLI = 0.89$ and $CFI = 0.76$), and the bi-factor model ($TLI = 0.84$ and $CFI = 0.72$). The RMSEA of all models are acceptable for a robust estimation with a good value for the lower limit in the confidence interval.

%latex.default(round(fitMeasures_df, 2), caption = paste("Goodness of fit statistics",     in_title), size = "scriptsize", longtable = T, ctable = F,     landscape = F, rowlabel = "", where = "!htbp", file = filename,     append = T)%
\setlongtables{\footnotesize
\begin{longtable}{lrrrrrrrr}
\caption{Goodness of fit statistics in the validation of the adapted Portuguese IMI}
\tabularnewline
\hline\hline
\multicolumn{1}{l}{}&\multicolumn{1}{c}{df}&\multicolumn{1}{c}{${\chi}^2$}&\multicolumn{1}{c}{AGFI}&\multicolumn{1}{c}{TLI}&\multicolumn{1}{c}{CFI}&\multicolumn{1}{c}{RMSEA}&\multicolumn{1}{c}{CI.lwr}&\multicolumn{1}{c}{CI.upr}\tabularnewline
\hline
\endfirsthead\caption[]{\em (continued)} \tabularnewline
\hline
\multicolumn{1}{l}{}&\multicolumn{1}{c}{df}&\multicolumn{1}{c}{${\chi}^2$}&\multicolumn{1}{c}{AGFI}&\multicolumn{1}{c}{TLI}&\multicolumn{1}{c}{CFI}&\multicolumn{1}{c}{RMSEA}&\multicolumn{1}{c}{CI.lwr}&\multicolumn{1}{c}{CI.upr}\tabularnewline
\hline
\endhead
\hline
\multicolumn{9}{r}{\tiny df: degree of freedom; AGFI: Adjusted Goodness of Fit Index; CFI: Comparative Fit Index; TLI: Tucker-Lewis Index;}\tabularnewline \multicolumn{9}{r}{\tiny RMSEA: Root Mean Square Error of Approximation}
\endfoot
\label{tab:imi-goodness-fit}
Global sample: Multidimensional model&$ 26.59$&$80.08$&$0.99$&$0.89$&$0.76$&$0.12$&$0.10$&$0.14$\tabularnewline
Global sample: Second-order factor model&$ 23.35$&$63.27$&$0.99$&$0.90$&$0.82$&$0.11$&$0.09$&$0.13$\tabularnewline
Global sample: Bi-factor model&$ 22.70$&$86.28$&$0.98$&$0.84$&$0.72$&$0.14$&$0.12$&$0.16$\tabularnewline \hline
Pilot study: Multidimensional model&$  8.25$&$14.30$&$0.96$&$0.83$&$0.86$&$0.16$&$0.11$&$0.21$\tabularnewline
Pilot study: Second-order factor model&$  7.88$&$14.06$&$0.96$&$0.82$&$0.85$&$0.16$&$0.11$&$0.21$\tabularnewline
Pilot study: Bi-factor model&$  9.90$&$18.39$&$0.97$&$0.80$&$0.80$&$0.17$&$0.11$&$0.23$\tabularnewline \hline
First study: Multidimensional model&$ 18.93$&$25.80$&$0.99$&$0.92$&$0.87$&$0.08$&$0.02$&$0.12$\tabularnewline
First study: Second-order factor model&$ 17.92$&$26.26$&$0.98$&$0.90$&$0.84$&$0.09$&$0.05$&$0.13$\tabularnewline
First study: Bi-factor model&$ 17.83$&$34.68$&$0.98$&$0.80$&$0.68$&$0.13$&$0.09$&$0.16$\tabularnewline \hline
Third study: Multidimensional model&$ 16.43$&$30.22$&$0.98$&$0.85$&$0.76$&$0.13$&$0.09$&$0.17$\tabularnewline
Third study: Second-order factor model&$131.00$&$$&$0.97$&$$&$$&$$&$0.00$&$0.00$\tabularnewline
Third study: Bi-factor model&$$&$$&$$&$$&$$&$$&$$&$$\tabularnewline
\hline
\end{longtable}
}

In relation to the data collected in each empirical study, the goodness of fit statistics in the validation of the adapted Portuguese IMI (shown in \autoref{tab:imi-goodness-fit}) have slight differences. For the data collected over the pilot study, the second-order factor model with ${\chi}^2=14.06$ fits better than the multidimensional model and the bi-factor model but the difference is not significant. For the data collected over the first empirical study, the multidimensional model with ${\chi}^2=25.80$ outperforms the bi-factor model and the multidimensional model. For the data collected over the third empirical study, the multidimensional model with ${\chi}^2=30.22$ is the only model that had converged in the simulation.

%latex.default(fa_summary_df, caption = paste("Summary of exploratory and confirmatory factor analysis for the",     in_title), size = "small", longtable = T, ctable = F, landscape = F,     rowlabel = "", where = "!htbp", file = filename, append = T)%
\setlongtables{\footnotesize
\begin{longtable}{lrrrr}\caption{Summary of the factor analysis for the adapted Portuguese IMI} \tabularnewline
\hline\hline
\multicolumn{1}{l}{}&\multicolumn{1}{c}{MR1}&\multicolumn{1}{c}{MR3}&\multicolumn{1}{c}{MR2}&\multicolumn{1}{c}{MR4}\tabularnewline
\hline
\endfirsthead\caption[]{\em (continued)} \tabularnewline
\hline
\multicolumn{1}{l}{}&\multicolumn{1}{c}{MR1}&\multicolumn{1}{c}{MR3}&\multicolumn{1}{c}{MR2}&\multicolumn{1}{c}{MR4}\tabularnewline
\hline
\endhead
\hline
\multicolumn{5}{r}{\tiny CFI: 0.822; TLI: 0.904; df: 23.354; ${\chi}^2$: 63.271; RMSEA: 0.11 [0.09, 0.132];}
\endfoot
\label{tab:imi-factor-loading}
\emph{IE: Interest/Enjoyment} & & & & \tabularnewline
Item22: \emph{Achei a atividade muito agradável} &$ 0.837$&$-0.237$&$-0.111$&$-0.073$\tabularnewline
Item09: \emph{Gostei muito de fazer a atividade} &$ 0.828$&$-0.256$&$-0.168$&$-0.106$\tabularnewline
Item12: \emph{A atividade foi divertida} &$ 0.827$&$-0.218$&$-0.157$&$-0.092$\tabularnewline
Item24: \emph{Enquanto estava fazendo a atividade, refleti ...} &$ 0.787$&$ 0.024$&$-0.060$&$-0.188$\tabularnewline
Item21: \emph{Descreveria a atividade como muito interessante} &$ 0.772$&$-0.210$&$ 0.052$&$-0.093$\tabularnewline
Item01: \emph{Foi muito descontraído realizar a atividade} &$ 0.691$&$-0.234$&$-0.216$&$-0.012$\tabularnewline \hline
\emph{PC: Perceived Choice} & & & & \tabularnewline
Item17: \emph{Fiz a atividade porque eu não tinha outra escolha} &$-0.168$&$ 0.802$&$ 0.246$&$ 0.184$\tabularnewline
Item15: \emph{Fiz a atividade porque eu tinha que fazer} &$-0.132$&$ 0.721$&$ 0.070$&$ 0.053$\tabularnewline
Item06: \emph{Realmente não tive escolha para realizar ...} &$-0.108$&$ 0.748$&$ 0.133$&$ 0.012$\tabularnewline
Item02: \emph{Senti como se eu tivesse sido obrigado ...} &$-0.270$&$ 0.707$&$ 0.167$&$-0.020$\tabularnewline
Item08: \emph{Senti que não fiz a atividade por vontade ...} &$-0.360$&$ 0.651$&$ 0.214$&$ 0.240$\tabularnewline \hline
\emph{PT: Pressure/Tension} & & & & \tabularnewline
Item16: \emph{Eu me senti ansioso enquanto trabalhava ...} &$ 0.040$&$ 0.197$&$ 0.839$&$-0.056$\tabularnewline
Item14: \emph{Eu me senti muito tenso ao realizar a atividade} &$-0.121$&$ 0.245$&$ 0.788$&$ 0.110$\tabularnewline
Item18: \emph{Seti-me pressionado enquanto fazia a atividade} &$-0.157$&$ 0.386$&$ 0.739$&$ 0.089$\tabularnewline
Item11: \emph{Não me senti nervoso ao realizar a atividade} &$ 0.365$&$ 0.043$&$-0.636$&$ 0.037$\tabularnewline \hline
\emph{EI: Effort/Importance} & & & & \tabularnewline
Item13: \emph{Não me esforcei muito para realizar bem atividade} &$-0.030$&$ 0.184$&$ 0.185$&$ 0.708$\tabularnewline
Item03: \emph{Me esforcei muito na realização da atividade} &$ 0.276$&$ 0.041$&$ 0.194$&$-0.650$\tabularnewline
Item07: \emph{Não coloquei muita energia (esforço) na atividade} &$-0.062$&$ 0.076$&$ 0.031$&$ 0.691$\tabularnewline \hline
SS loadings&$ 4.280$&$ 3.206$&$ 2.624$&$ 1.589$\tabularnewline
Cumulative Var&$ 0.238$&$ 0.416$&$ 0.562$&$ 0.650$\tabularnewline
Proportion Explained&$ 0.366$&$ 0.274$&$ 0.224$&$ 0.136$\tabularnewline
\hline
\end{longtable}}

\autoref{tab:imi-factor-loading} shows the summary of the factor analysis conducted with the global sample for the adapted Portuguese IMI. The factor loadings, eigenvalues, cumulative variance and proportion explained by the items indicates the emergence of four factors: Interest/Enjoyment (F1), Perceived Choice (F2), Pressure/Tension (F3), and Effort/Importance (F4). The items in the first factor (F1: Interest/Enjoyment) have strong primary loadings with values greater than 0.6, and the majority of proportion (36\%) is explained  by the first factor. These results are similars to the findings obtained in previous validation of the IMI conducted by \citeonline{McAuleyDuncanTammen1989, MarklandHardy1997, MonteiroMataPeixotoMonteiroMataPeixoto2015}. According to the cut-off value defined by \citeonline{Yurdugul2008}, the first eigenvalue has a level of $4.2$ indicating stability in the Cronbach’s $\alpha$ for a sample size ($N=141$) between 100 to 300 observation.

\subsection{Reliability Tests of the Adapted Portuguese IMI}

The overall and internal consistency of the adapted Portuguese IMI had been evaluated by reliability tests in the global sample, and in the data collected over each empirical study (the pilot study, and the first and third studies). \autoref{tab:reliability-IMI} shows the results of the reliability tests in which the Cronbach's alpha ($\alpha$) for the Intrinsic Motivation have good overall consistency for the global sample and the data collected in each empirical study with values greater than 0.80. The Cronbach's alpha ($\alpha$) in the scales of \emph{IE: Interest/Enjoyment}, \emph{PC: Perceived Choice}, \emph{PT: Pressure/Tension} indicate good consistency and high reliability for all the samples with values greater than $0.70$ and $0.80$. The Cronbach's alpha ($\alpha$) in the scale of \emph{EI: Effort/Importance} indicate an acceptable consistency for the global sample and the data collected over the third empirical study. Although the Cronbach's alpha ($\alpha$) in the scale of \emph{EI: Effort/Importance} have values less than 0.70 for the data collected over the pilot and first studies, these values ($\alpha_{pilot} = 0.699$ and $\alpha_{third} = 0.692$) are consider acceptable because they are close to $0.70$.

Separate reliability tests had also been conducted in the adapted Portuguese IMI for the collected data in each empirical study and by dividing this data into: responses from students who participated in non-gamified CL sessions (\emph{non-gamified}), responses from students who participated in ontology-based CL sessions (\emph{ont-gamified}), and responses from students who participated in CL sessions that had been gamified without using ontologies (\emph{w/o-gamified}). \autoref{tab:reliability-IMI-samples} shows the results of these reliability tests. For the data collected over the pilot study where the groups of responses had been divided into ont-gamified CL sessions and non-gamified CL sessions, the results of reliability tests indicate, in the majority of scales and groups, highly consistent with good (Cronbach's $\alpha$ in $0.80$s) and excellent (Cronbach's $\alpha$ in $0.90$s) internal consistency. The Cronbach's $\alpha$ indicates only questionable internal consistency for the \aspas{\emph{ont-gamified}} group in the scale of \emph{PT: Pressure/Tension} with a Cronbach's alpha $\alpha = 0.608$. In the scale of \emph{EI: Effort/Importance}, the reliability test for the \aspas{\emph{non-gamified}} group indicates a Cronbach's $\alpha = 0.690$ that is a value close to the threshold of 0.70 by which its internal consistency is consider acceptable.

For the data collected over the first empirical study where the groups of responses had been divided into ont-gamified CL sessions and w/o-gamified CL sessions, the results of reliability tests indicates good and excellent internal consistency in all the scales and groups, the only exception occurs for the \aspas{\emph{ont-gamified}} group in the scale of \emph{EI: Effort/Importance} that indicates a questionable consistency with a Cronbach's $\alpha = 0.632$. For the data collected over the third empirical study where the groups of responses had been divided into ont-gamified CL sessions and w/o-gamified CL sessions, the results of reliability tests shows highly internal reliability in all the scales and groups. Only, the result in the group \aspas{\emph{ont-gamified}} for the scale of \emph{EI: Effort/Importance} indicates a poor internal consistency with a Cronbach's $\alpha = 0.580$.

%latex.default
\setlongtables{\small
\begin{longtable}{lrrrr}\caption{Result of reliability analysis for the adapted Portuguese IMI} \tabularnewline
\hline\hline
\multicolumn{1}{l}{Cronbach's alpha ($\alpha$)}&\multicolumn{1}{c}{Global}&\multicolumn{1}{c}{Pilot Study}&\multicolumn{1}{c}{First Study}&\multicolumn{1}{c}{Third Study}\tabularnewline
\hline
\endfirsthead\caption[]{\em (continued)} \tabularnewline
\hline
\multicolumn{1}{l}{Cronbach's alpha ($\alpha$)}&\multicolumn{1}{c}{Global}&\multicolumn{1}{c}{Pilot Study}&\multicolumn{1}{c}{First Study}&\multicolumn{1}{c}{Third Study}\tabularnewline
\hline
\endhead
\hline
\endfoot
\label{tab:reliability-IMI}
\emph{Intrinsic Motivation}&$0.894$&$0.890$&$0.865$&$0.850$\tabularnewline
\emph{IE: Interest/Enjoyment}&$0.926$&$0.944$&$0.895$&$0.917$\tabularnewline
\emph{PC: Perceived Choice}&$0.882$&$0.813$&$0.876$&$0.905$\tabularnewline
\emph{PT: Pressure/Tension}&$0.861$&$0.770$&$0.835$&$0.848$\tabularnewline
\emph{EI: Effort/Importance}&$0.724$&$0.699$&$0.692$&$0.783$\tabularnewline
\hline
\end{longtable}}

%latex.default
\setlongtables{\small
\begin{longtable}{lrrrr}\caption{Results of reliability tests in the adapted Portuguese IMI for each empirical study} \tabularnewline
\hline\hline
\multicolumn{1}{l}{Cronbach's alpha ($\alpha$)}&\multicolumn{1}{c}{Global}&\multicolumn{1}{c}{\emph{non-gamified}}&\multicolumn{1}{c}{\emph{ont-gamified}}&\multicolumn{1}{c}{\emph{w/o-gamified}}\tabularnewline
\hline
\endfirsthead\caption[]{\em (continued)} \tabularnewline
\hline
\multicolumn{1}{l}{Cronbach's alpha ($\alpha$)}&\multicolumn{1}{c}{Global}&\multicolumn{1}{c}{\emph{non-gamified}}&\multicolumn{1}{c}{\emph{ont-gamified}}&\multicolumn{1}{c}{\emph{w/o-gamified}}\tabularnewline
\hline
\endhead
\hline
\endfoot
\label{tab:reliability-IMI-samples} 
%& & & & \tabularnewline
\emph{Pilot study}: Intrinsic Motivation&$0.890$&$0.896$&$0.850$& \tabularnewline
\emph{Pilot study}: Interest/Enjoyment&$0.944$&$0.931$&$0.947$& \tabularnewline
\emph{Pilot study}: Perceived Choice&$0.813$&$0.811$&$0.759$& \tabularnewline
\emph{Pilot study}: Pressure/Tension&$0.770$&$0.833$&$0.608$& \tabularnewline
\emph{Pilot study}: Effort/Importance&$0.699$&$0.690$&$0.704$& \tabularnewline \hline
%\emph{First study}& & & & \tabularnewline
\emph{First study}: Intrinsic Motivation&$0.865$&$0.859$&$0.830$& \tabularnewline
\emph{First study}: Interest/Enjoyment&$0.895$&$0.886$&$0.894$& \tabularnewline
\emph{First study}: Perceived Choice&$0.876$&$0.862$&$0.871$& \tabularnewline
\emph{First study}: Pressure/Tension&$0.835$&$0.860$&$0.811$& \tabularnewline
\emph{First study}: Effort/Importance&$0.692$&$0.710$&$0.632$& \tabularnewline
\hline
%\emph{Third study}& & & & \tabularnewline
\emph{Third study}: Intrinsic Motivation&$0.850$& &$0.782$&$0.875$\tabularnewline
\emph{Third study}: Interest/Enjoyment&$0.917$& &$0.929$&$0.906$\tabularnewline
\emph{Third study}: Perceived Choice&$0.905$& &$0.883$&$0.908$\tabularnewline
\emph{Third study}: Pressure/Tension&$0.848$& &$0.823$&$0.879$\tabularnewline
\emph{Third study}: Effort/Importance&$0.783$& &$0.580$&$0.878$\tabularnewline
\end{longtable}}

%%%%%%%%%%%%%%%%%%%%%%%%% continue from here %%%%%%%%%%%%%%%%%%%%%%%%%%%%%

\subsection{Factorial Structure of the Adapted Portuguese IMMS}

\autoref{fig:imms-models-cfa} shows the multidimensional, second-order factor and bi-factor models that had been tested in the CFA to validate the factorial structure of the adapted Portuguese IMMS. The construction of these models had been conducted according to the criteria defined in the validation procedure by removing items that had loaded with a value less than 0.4, and also, by removing items that had cross-loading less than 0.2. This construction ensures that the items correspond to the scale for which they are intended, and that the Cronbach’s $\alpha$ is stable.

\begin{figure}[htb]
 \caption{Models tested in the CFA to validate the factorial structure of the adapted Portuguese IMMS}
 \label{fig:imms-models-cfa}
 \centering
 \begin{tabular}{ccc}
 \includegraphics[width=0.25\textwidth]{images/appendix/imms-multidimensional-model.png} &
 \includegraphics[width=0.25\textwidth]{images/appendix/imms-second-order-factor-model} &
 \includegraphics[width=0.25\textwidth]{images/appendix/imms-bi-factor-model.png} \\
 {\scriptsize multidimensional model} & {\scriptsize second-order factor model} & {\scriptsize bi-factor model} \\
 \end{tabular}
 \fautor
\end{figure}
\newpage

Instead to load in the scale of \emph{A: attention}, the Items 08, 10 and 21 loaded in the scale of \emph{R: Relevance}. The Item 08 - \aspas{\emph{A atividade foi muito abstrata que foi difícil manter minha atenção}} as an adapted and translated version of \aspas{The lesson was so abstract that it was hard to keep my attention on it} - was understood by the participants in the sense of abstraction rather than keeping attention, thereby this item has more concordance with the scale of \emph{R: Relevance}. The Item 10 - \aspas{O ambiente em que foi executada a atividade pareceu sem graça e desagradável} as an adapted and translated version of \aspas{The pages of this lesson looked dry and unappealing} - was understood in the sense of quality of the CL session rather than keeping focus, thereby this item lacks of concordance with the scale of \emph{A: Attention}. The Item 21 - \aspas{\emph{O ambiente e as tarefas da atividade foram chatos ou entediantes}} as an adapted and translated version of \aspas{The style of writing was boring} - was understood by the participants in the sense of quality rather than feeling bored, thereby this item is correlated with the scale of \emph{R: Relevance}. The Item 13 - \aspas{\emph{A atividade teve coisas que estimularam minha curiosidade}} as an adapted and translated version of \aspas{The lesson had things that stimulated my curiosity} - and the Item 17 - \aspas{\emph{Aprendi algumas coisas que foram surpreendentes e/ou inesperadas}} as an adapted version of \aspas{Aprendi algumas coisas que foram surpreendentes e/ou inesperadas} - were understood by the participants in the sense of feeling comfortable rather than playing close attention, thereby these both items loaded in the scale of \emph{S: Satisfaction} rather than loaded in the scale of \emph{A: attention}.

\autoref{tab:imms-goodness-fit} shows the goodness fit statistics for the models tested in the validation of the adapted Portuguese IMMS. The results presented in this table indicate that all the models have adequate goodness fit indices for all the samples (the global sample, and the samples obtained over the second and third empirical studies). Based on these results, the model that best fits the global sample is the bi-factor model with ${\chi}^2 = 22.29$ that outperforms the multidimensional model (${\chi}^2 = 26.39$), and the second-order factor model (${\chi}^2 = 26.39$). The AGFI index has the same value in the multidimensional and second-order model, and these indices are outperformed by the bi-factor model with $TLI = 0.99$ and $CFI = 0.97$. The RMSEA of all models indicates good fit with values less than 0.08.

%latex.default(round(fitMeasures_df, 2), caption = paste("Goodness of fit statistics",     in_title), size = "scriptsize", longtable = T, ctable = F,     landscape = F, rowlabel = "", where = "!htbp", file = filename,     append = T)%
\setlongtables{\footnotesize
\begin{longtable}{lrrrrrrrr}
\caption{Goodness of fit statistics in the validation of the adapted Portuguese IMMS}
\tabularnewline
\hline\hline
\multicolumn{1}{l}{}&\multicolumn{1}{c}{df}&\multicolumn{1}{c}{${\chi}^2$}&\multicolumn{1}{c}{AGFI}&\multicolumn{1}{c}{TLI}&\multicolumn{1}{c}{CFI}&\multicolumn{1}{c}{RMSEA}&\multicolumn{1}{c}{CI.lwr}&\multicolumn{1}{c}{CI.upr}\tabularnewline
\hline
\endfirsthead\caption[]{\em (continued)} \tabularnewline
\hline
\multicolumn{1}{l}{}&\multicolumn{1}{c}{df}&\multicolumn{1}{c}{${\chi}^2$}&\multicolumn{1}{c}{AGFI}&\multicolumn{1}{c}{TLI}&\multicolumn{1}{c}{CFI}&\multicolumn{1}{c}{RMSEA}&\multicolumn{1}{c}{CI.lwr}&\multicolumn{1}{c}{CI.upr}\tabularnewline
\hline
\endhead
\hline
\multicolumn{9}{r}{\tiny df: degree of freedom; AGFI: Adjusted Goodness of Fit Index; CFI: Comparative Fit Index; TLI: Tucker-Lewis Index;}\tabularnewline \multicolumn{9}{r}{\tiny RMSEA: Root Mean Square Error of Approximation}
\endfoot
\label{tab:imms-goodness-fit}
Global sample: Multidimensional model&$19.07$&$26.39$&$1.00$&$0.98$&$0.93$&$0.06$&$ 0$&$0.11$\tabularnewline
Global sample: Second-order factor model&$19.07$&$26.39$&$1.00$&$0.98$&$0.93$&$0.06$&$ 0$&$0.11$\tabularnewline
Global sample: Bi-factor model&$18.62$&$22.29$&$1.00$&$0.99$&$0.97$&$0.04$&$ 0$&$0.10$\tabularnewline \hline
Second study: Multidimensional model&$12.04$&$13.65$&$1.00$&$0.99$&$0.97$&$0.05$&$ 0$&$0.14$\tabularnewline
Second study: Second-order model&$12.04$&$13.65$&$1.00$&$0.99$&$0.97$&$0.05$&$ 0$&$0.14$\tabularnewline
Second study: Bi-factor model&$11.51$&$12.14$&$1.00$&$1.00$&$0.99$&$0.03$&$ 0$&$0.14$\tabularnewline \hline
Third study: Multidimensional model&$12.65$&$13.83$&$0.99$&$0.99$&$0.97$&$0.04$&$ 0$&$0.13$\tabularnewline
Third study: Second-order factor model&$12.65$&$13.83$&$0.99$&$0.99$&$0.97$&$0.04$&$ 0$&$0.13$\tabularnewline
Third study: Bi-factor model&$14.08$&$16.55$&$0.99$&$0.97$&$0.95$&$0.06$&$ 0$&$0.14$\tabularnewline
\hline
\end{longtable}
}

In relation to the data collected in each empirical study, the goodness of fit statistics (shown in \autoref{tab:imms-goodness-fit}) for the validation of the adapted Portuguese IMMS have slight differences. For the data collected over the second empirical study, the bi-factor model with ${\chi}^2=12.14$ fits better than the multidimensional model and the second-order factor model. For the data collected over the third empirical study, the multidimensional model and the second-order factor model with ${\chi}^2=13.83$ outperform the bi-factor model (${\chi}^2=16.555$), but there are not difference in the AGFI index. With the data collected over the third empirical study, the multidimensional model and second-order factor model with $TLI=0.99$ and $CFI=0.97$ outperform the bi-factor model ($TLI = 0.97$ and $CFI = 0.95$).

\autoref{tab:imms-factor-loading} shows the summary of the factor analysis conducted with the global sample for the adapted Portuguese IMMS. The factor loadings, eigenvalues, cumulative variance and proportion explained by the items indicates the emergence of tree factors: Attention (F1), Relevance (F2), and Satisfaction (F3). The items in the first factor (F1: Attention) have strong primary loadings with values greater than 0.6, and the majority of proportion (50\%) is explained by the first factor. These results are similars to the findings obtained in previous validation of the IMMS conducted by \citeonline{LoorbachPetersKarremanSteehouder2015, CookBeckmanThomasThompson2009, HuangHew2016}. According to the cut-off value defined by \citeonline{Yurdugul2008}, the first eigenvalue has a level of $3.9$ indicating stability in the Cronbach’s $\alpha$ for a sample size ($N=110$) between 100 to 300 observation .

%latex.default(fa_summary_df, caption = paste("Summary of exploratory and confirmatory factor analysis for the",     in_title), size = "small", longtable = T, ctable = F, landscape = F,     rowlabel = "", where = "!htbp", file = filename, append = T)%
\setlongtables{\footnotesize
\begin{longtable}{lrrr}\caption{Summary of factor analysis for the adapted Portuguese IMMS} \tabularnewline
\hline\hline
\multicolumn{1}{l}{}&\multicolumn{1}{c}{MR1}&\multicolumn{1}{c}{MR2}&\multicolumn{1}{c}{MR3}\tabularnewline
\hline
\endfirsthead\caption[]{\em (continued)} \tabularnewline
\hline
\multicolumn{1}{l}{}&\multicolumn{1}{c}{MR1}&\multicolumn{1}{c}{MR2}&\multicolumn{1}{c}{MR3}\tabularnewline
\hline
\endhead
\hline
\multicolumn{4}{r}{\tiny CFI: 0.966; TLI: 0.989; df: 18.619; $chi^2$: 22.291; p-value: 0.25; RMSEA: 0.043 [0, 0.104];}
\endfoot
\label{tab:imms-factor-loading}
\emph{A: Attention} & & & \tabularnewline
Item12: \emph{A forma como a informação foi organizada no ambiente ...}&$ 0.857$&$-0.198$&$ 0.203$\tabularnewline
Item19: \emph{O feedback ou outros elementos fornecidos na atividade, ...}&$ 0.785$&$-0.004$&$ 0.243$\tabularnewline
Item04: \emph{O ambiente e tarefas da atividade foram atraentes}&$ 0.738$&$-0.304$&$ 0.204$\tabularnewline
Item20: \emph{A variedade de tarefas e coisas no ambiente, ajudou a ...}&$ 0.726$&$-0.133$&$ 0.270$\tabularnewline
Item16: \emph{As tarefas e sua organização na atividade transmitiram a ...}&$ 0.693$&$-0.241$&$ 0.334$\tabularnewline
Item01: \emph{Houve algo interessante no início desta atividade que chamou ...}&$ 0.653$&$-0.337$&$ 0.234$\tabularnewline \hline
\emph{R: Relevance} & & & \tabularnewline
Item15: \emph{A quantidade de tarefas repetitivas na atividade na atividade me ...}&$-0.087$&$ 0.614$&$-0.125$\tabularnewline
Item21: \emph{O ambiente e as tarefas da atividade foram chatos ou entediantes}&$-0.321$&$ 0.662$&$-0.144$\tabularnewline
Item10: \emph{O ambiente em que foi executada a atividade pareceu sem graça ...}&$-0.347$&$ 0.625$&$-0.076$\tabularnewline
Item08: \emph{A atividade foi muito abstrata que foi difícil manter minha atenção}&$ 0.011$&$ 0.484$&$-0.179$\tabularnewline \hline
\emph{S: Satisfaction} & & & \tabularnewline
Item13: \emph{A atividade teve coisas que estimularam minha curiosidade}&$ 0.307$&$-0.360$&$ 0.810$\tabularnewline
Item14: \emph{Eu realmente gostei de participar na atividade}&$ 0.434$&$-0.343$&$ 0.644$\tabularnewline
Item17: \emph{Aprendi algumas coisas que foram surpreendentes e/ou inesperadas}&$ 0.388$&$-0.104$&$ 0.568$\tabularnewline \hline
SS loadings&$ 3.995$&$ 2.019$&$ 1.849$\tabularnewline
Cumulative Var&$ 0.307$&$ 0.463$&$ 0.605$\tabularnewline
Proportion Explained&$ 0.508$&$ 0.257$&$ 0.235$\tabularnewline
\hline
\end{longtable}
}

\subsection{Reliability Tests of the Adapted Portuguese IMMS}

The overall and internal consistency of the adapted Portuguese IMMS had been evaluated by reliability tests in the global sample, and in the data collected over each empirical study (the second and third empirical studies). \autoref{tab:reliability-IMMS} shows the results of the reliability tests in which the Cronbach's alpha ($\alpha$) for the Level of Motivation have good overall consistency ($\alpha=0.909$) for the global sample and the data collected in the second and third empirical studies with values greater than 0.80. The Cronbach's $\alpha$ in the scales of \emph{A: Attention}, \emph{R: Relevance}, \emph{S: Satisfaction} indicate good consistency and high reliability for all the samples with values greater than $0.70$ and $0.80$. The only exception had been found in the scale of \emph{R: Relevance} for the data collected over the third empirical study in which the Cronbach's $\alpha$ with value $0.66$ indicates questionable reliability, but this value is close to $0.70$, thereby the reliability is considered acceptable.

%latex.default(rel_summary_df, caption = paste("Result of reliability analysis for the",     in_title), size = "small", longtable = T, ctable = F, landscape = F,     rowlabel = "", where = "!htbp", file = filename, append = T)%
\setlongtables{\small
\begin{longtable}{lrrr}\caption{Result of reliability analysis for the adapted Portuguese IMMS} \tabularnewline
\hline\hline
\multicolumn{1}{l}{Cronbach's alpha ($\alpha$)}&\multicolumn{1}{c}{Global}&\multicolumn{1}{c}{Second Study}&\multicolumn{1}{c}{Third Study}\tabularnewline
\hline
\endfirsthead\caption[]{\em (continued)} \tabularnewline
\hline
\multicolumn{1}{l}{Cronbach's alpha ($\alpha$)}&\multicolumn{1}{c}{Global}&\multicolumn{1}{c}{Second Study}&\multicolumn{1}{c}{Third Study}\tabularnewline
\hline
\endhead
\hline
\endfoot
\label{tab:reliability-IMMS}
\emph{Level of Motivation}&$0.909$&$0.930$&$0.874$\tabularnewline
\emph{A: Attention}&$0.918$&$0.930$&$0.900$\tabularnewline
\emph{R: Relevance}&$0.728$&$0.748$&$0.696$\tabularnewline
\emph{S: Satisfaction}&$0.851$&$0.836$&$0.876$\tabularnewline
\hline
\end{longtable}}

Separate reliability tests had also been conducted in the adapted Portuguese IMMS for the collected data in each empirical study and by dividing this data into: responses from students who participated in non-gamified CL sessions (\emph{non-gamified}), responses from students who participated in ontology-based CL sessions (\emph{ont-gamified}), and responses from students who participated in CL sessions that had been gamified without using ontologies (\emph{w/o-gamified}). \autoref{tab:reliability-IMMS-samples} shows the results of these reliability tests, where the Cronbach's $\alpha$ in the majority of scales and groups indicate good ($\alpha$ in $0.80$s) and excellent ($\alpha$ in $0.90$s) internal consistency. The only questionable internal consistency occurs in the scale of \emph{R: Relevance} for the data collected over the third study in the \aspas{\emph{w/o-gamified}} group with a Cronbach's $\alpha$ of $0.684$, but this value is close to the threshold of 0.7 which by this internal consistency is considered as acceptable.

%latex.default(rel_summary_df, caption = paste("Result of reliability analysis for the",     in_title), size = "small", longtable = T, ctable = F, landscape = F,     rowlabel = "", where = "!htbp", file = filename, append = T)%
\setlongtables{\small
\begin{longtable}{lrrrr}\caption{Results of reliability tests in the adapted Portuguese IMMS for each empirical study} \tabularnewline
\hline\hline
\multicolumn{1}{l}{Cronbach's alpha ($\alpha$)}&\multicolumn{1}{c}{Global}&\multicolumn{1}{c}{\emph{non-gamified}}&\multicolumn{1}{c}{\emph{ont-gamified}}&\multicolumn{1}{c}{\emph{w/o-gamified}}\tabularnewline
\hline
\endfirsthead\caption[]{\em (continued)} \tabularnewline
\hline
\multicolumn{1}{l}{Cronbach's alpha ($\alpha$)}&\multicolumn{1}{c}{Global}&\multicolumn{1}{c}{\emph{non-gamified}}&\multicolumn{1}{c}{\emph{ont-gamified}}&\multicolumn{1}{c}{\emph{w/o-gamified}}\tabularnewline
\hline
\endhead
\hline
\endfoot
\label{tab:reliability-IMMS-samples} 
\emph{Second study}: Level of Motivation&$0.930$&$0.932$&$0.926$& \tabularnewline
\emph{Second study}: Attention&$0.930$&$0.935$&$0.915$& \tabularnewline
\emph{Second study}: Relevance&$0.748$&$0.728$&$0.784$& \tabularnewline
\emph{Second study}: Satisfaction&$0.836$&$0.851$&$0.817$& \tabularnewline \hline
\emph{Third study}: Level of Motivation&$0.874$& &$0.886$&$0.866$\tabularnewline
\emph{Third study}: Attention&$0.900$& &$0.924$&$0.881$\tabularnewline
\emph{Third study}: Relevance&$0.696$& &$0.725$&$0.684$\tabularnewline
\emph{Third study}: Satisfaction&$0.876$& &$0.884$&$0.889$\tabularnewline
\hline
\end{longtable}}





 
\chapter[IRT-based Models for Measuring Motivation and Learning Outcomes]{Item Response Theory-based Models for Measuring Motivation and Learning Outcomes}
\label{appendix:irt-models}

For the empirical studies conducted in this PhD thesis dissertation, statistical instruments based on the Item Response Theory (IRT) had been used to estimate the participants' motivation and the learning outcomes. Instead to use the average scores of motivation surveys as measurement of motivation, Rating Scale Model (RSM) is used to estimate the intrinsic motivation and the level of motivation. The learning outcomes had been calculated as gains in skill/knowledge by stacking pre-test and post-test data with Generalized Partial Credit Model (GPCM) to estimate the changes in skill/knowledge when the data were gathered from multiple choice knowledge questionnaires and programming tasks. The first section (\autoref{sec:irt-motivation}) details the construction and validation procedure of IRT-based models employed in the construction of instruments for measuring the motivation and learning outcomes in the empirical studies. The second section (\autoref{sec:irt-learning-outcomes}) details the procedure for stacking the pre-test data and post-test data with IRT-based models for estimating changes in the latent trait estimates from the pre-test to post-test phase.

The rest of sections are organized as follows:
\begin{itemize}
\item
\autoref{sec:irt-motivation-pilot-study} presents the validation and results of the RSM-based instrument used to estimate the intrinsic motivation in the pilot empirical study;
\item
\autoref{sec:irt-motivation-first-study} presents the validation and results of the RSM-based instrument used to estimate the intrinsic motivation in the first empirical study;
\item
\autoref{sec:irt-motivation-second-study} presents the validation and results of the RSM-based instrument used to estimate the intrinsic motivation in the second empirical study;
\item
\autoref{sec:irt-intrinsic-motivation-third-study} presents the validation and results of the RSM-based instrument used to estimate the intrinsic motivation in the third empirical study;
\item
\autoref{sec:irt-level-motivation-third-study} presents the validation and results of the RSM-based instrument used to estimate the level of motivation in the third empirical study;
\item
\autoref{sec:irt-learning-outcomes-pilot-study} presents the validation and stacking procedure for estimating gains in skill/knowledge of the pilot empirical study;
\item
\autoref{sec:irt-learning-outcomes-first-study} presents the validation and stacking procedure for estimating gains in skill/knowledge of the first empirical study;
\item
\autoref{sec:irt-learning-outcomes-second-study} presents the validation and stacking procedure for estimating gains in skill/knowledge of the second empirical study; and
\item
\autoref{sec:irt-learning-outcomes-third-study} presents the validation and stacking procedure for estimating gains in skill/knowledge of the third empirical study;
\end{itemize}

%%%%%%%%%%%%%%%%%%%%%%%%%%%%%%%%%%%%%%%%%%%%%%%%%%%

\section{Construction and Validation of IRT-based Models}
\label{sec:irt-motivation}

Let be $i = \{1,2, ..., I\}$ the items in a set of responses; and let be $x = \{0,1, ..., X\}$ the categories of responses for the item $i$; then, the probability that a person $n$ scores $x$ on the item $i$ is described in a nonparametric IRT-based model by the item response model \cite{AdamsWu2007, AdamsWilsonWu1997} as:

$P(X_{n,i} = x | \theta_{n}) \propto exp(b_{i,x} \theta_{n} + a_{i,x} \xi)$.

where the symbol \aspas{$\propto$} means that the probabilities in the responses are normalized such that $\sum_{x=0}^{X} P(X_{n,i} = x | \theta_{n}) = 1$; the parameter $a_{i,x} \xi$ is the item intercept ($AXsi$) related to the location on latent trait; and the parameter $b_{i,x}$ is the slope related to the item discrimination.

As this item response model is a generalization of nonparametric models (such as Rasch model, Rating Scale Model - RSM, Partial Credit Model - PCM, General Partial Credit Model - GPCM, and Nominal Response Model - NRM), to be used as an instrument for measuring unidimensional latent traits such as the motivation and skill/knowledge of participants in the empirical studies, three fundamental assumptions related to unidimensional nonparametric models must be checked. These assumptions are the the unidimensionality of data structure, the local independence of items, and the monotonicity of the item characteristic curve. The unidimensionality determines whether items in the instrument measure only one latent trait $\theta$, the local independence verifies the statistical relationship between examinees’ responses for each pair of items in the instrument, and the monotonicity checks the relationship between the item responses and the latent trait $\theta$ measured by the instrument.

After to check these three fundamental assumptions, the values for the intercept and slope parameters are estimated by means of the Marginal Maximum Likelihood (MML) method \cite{BockAitkin1981}; then, the latent trait $\theta$ that represents the measurement of motivation or skill/knowledge for the participants in the empirical studies are computed by the Weighted Likelihood Estimator (WLE) \cite{Warm1989}. 

\subsection{Checking Assumptions}
\label{sec:checking-assumptions-irt-motivation}

\subsubsection*{Test of Unidimensionality}

Currently, there are a variety of statistic methods to assess the dimensionality of IRT-based models \cite{Hattie1985, NandakumarYuLiStout1998}, but not one of them is universal to determine the dimensionality. The most common statistic methods are based on factor analysis with eigenvalue-greater-than-one rule, ratio of first-to-second eigenvalues, parallel analysis, Root Means Square Error of Approximation (RMSEA) or chi-square tests. For the data gathered by means of motivation surveys, the unidimensional Confirmatory Factor Analysis (CFA) \cite{Brown2014} and the DETECT analysis \cite{StoutHabingDouglasKimRoussosZhang1996, Zhang2007} had been carried out to determine the dimensionality of IRT-based models.

Indices based on factor analysis, Chi-square (${\chi}^2$), Adjusted Goodness of Fit Index (AGFI), Tucker-Lewis Index (TLI), and Comparative Fit Index (CFI) are used in the unidimensional CFA as indices to evaluate whether the items in  is unidimensional \cite{Brown2014}. Lower values of the Chi-square (${\chi}^2$) indicates best fit. Values of AGFI, TLI and CFI are considered acceptable for the range of $0.90$ to $0.95$, and they indicate good fit when these values are higher than $0.95$. The DETECT analysis under a conditional covariance-based nonparametric multidimensionality assessment computes the indices DETECT, ASSI and RATIO \cite{Zhang2007}, where DETECT index greater than $1.00$ indicates strong multidimensionality, DETECT index between $0.40$ and $1.00$ indicates moderate multidimensionality, DETECT index between $0.20$ and $0.40$ indicates weak multidimensionality, and DETECT index lower than $0.20$ indicates essential unidimensionality. \emph{Essential unidimensionality} in the data structure is indicated when the ASSI $< 0.25$ and RATIO $< 0.36$, and \emph{essential deviation from unidimensionality} is indicated when the ASSI $> 0.25$ and RATIO $> 0.36$.

The test of unidimensionality had been carried out in R software version 3.4.3 \cite{RCoreTeam2017} in which the lavaan package version 0.5 \cite{Rosseel2012} and the sirt package version 2.6 \cite{Robitzsch2018} had been used to conduct the unidimensional CFA and the DETECT analysis, respectively. The R scripts for the test of unidimensionality are available at the URL: \url{https://geiser.github.io/phd-thesis-evaluation/}

\subsubsection*{Test of Local Independence}

For unidimensional IRT models:

\begin{citacao}
\aspas{Local independence means that when abilities influencing test performance are held constant, examinees' responses to any pair of items are statistically independent. In other words, after taking examinees' abilities into account, no relationship exists between examinees' responses to different items. Simply put, this means that the abilities specified in the model are the only factors influencing examinees' responses to test items} \cite{HambletonSwaminathanRogers1991}
\end{citacao}

Thus, the independence is tested by the $Q3$ statistic of item pairs $i$ and $j$ in which the correlation of items $i$ and $j$ is calculated as $Q3_{i,j} = Cor(e_{n,i}; e_{n,j})$, where $e_{n,i} = X_{n,i} - E(X_{n,i})$ represents the residual between the response of a person $n$ for the item $i$ and the expected response. According to the null test in the condition of independence, the effect size of model fit is defined by the average of absolute values of adjusted correlation Q3 ($MADaQ3$), and by the maximum adjusted correlation Q3 ($maxaQ3$). In this sense, under local independence the average of absolute of adjusted correlation Q3 is slightly smaller than zero ($MADaQ3 \approx 0$), and the null condition is not rejected ($p > 0.05$).

The TAM: Test analysis modules package version 2.10 \cite{RobitzschKieferWu2018} is employed to carried out the test of local independence in R software version 3.4.3 \cite{RCoreTeam2017}. The R scripts for the test of local independence are available at the URL: \url{https://geiser.github.io/phd-thesis-evaluation/}

\subsubsection*{Test of Monotonicity}

For evaluating the manifest monotonicity in the IRT-based models, the Mokken scale analysis \cite{Mokken1971, VanderArk2007} had been carried out with the data gathered through motivation surveys. In this analysis, the monotone homogeneity model and the double monotonicity model are used to check the assumptions of monotonicity. Employing these models, the \emph{item step response function} $P(X_{i} \geq x | \theta)$ calculates the ordering of the scores for each item $i$ reflecting the hypothesized ordering on the latent trait $\theta$. The violation of monotonicity in this function  is indicated at a significance level $\alpha = 0.05$ when the criteria $minvi$ is greater than $0.03$.

The test of monotonicity is carried out in R software version 3.4.3 \cite{RCoreTeam2017} by employing the mokken package version 2.8.10 \cite{VanderArk2012,VanderArk2007}. The R scripts for the test of monotonicity are available at the URL: \url{https://geiser.github.io/phd-thesis-evaluation/}

\subsection{Estimating Item Parameters}
\label{sec:estimating-parameters-irt-motivation}

Employing the Marginal Maximum Likelihood (MML) method \cite{BockAitkin1981}, the item intercepts \emph{AXsi} ($a_{i,k} \xi$) and the slopes related to the item discrimination ($b_{i,x}$) had been calculated by the TAM: Test analysis modules package version 2.10 \cite{RobitzschKieferWu2018} in the R software version 3.4.3 \cite{RCoreTeam2017}. The R scripts used for estimating the item parameters are available at the URL: \url{https://geiser.github.io/phd-thesis-evaluation/}

\subsection{Obtaining the Latent Trait Estimates}
\label{sec:obtaining-latent-trait-irt-motivation}

The latent trait estimates (intrinsic motivation, level of motivation and skill/knowledge) is calculated by the Weighted Likelihood Estimator (WLE) \cite{Warm1989} in which the latent trait distribution is assumed as a normal distribution with mean of $\mu = 0$ and units in \emph{logits}. These estimates had been calculated in the R software version 3.4.3 \cite{RCoreTeam2017} employing the TAM: Test analysis modules package version 2.10 \cite{RobitzschKieferWu2018}. The R scripts used to obtain the latent trait estimates are available at the URL: \url{https://geiser.github.io/phd-thesis-evaluation/}

%%%%%%%%%%%%%%%%%%%%%%%%%%%%%%%%%%%%%%%%%%%%%%%%%%%

\section{Stacking Procedure with IRT-based Models}
\label{sec:irt-learning-outcomes}

Traditionally, the measures of changes in latent traits estimates (e.g. motivation, mood, and skill/knowledge) are calculated as a difference of scores in the IRT-base gathering instruments. Such difference is calculated by subtracting the initial score obtained in a pre-test phase from the final score obtained in post-test phase, but this measurement causes some errors of measurements and misinterpretation \cite{Lord1956, Lord1958}. In addition, the variance, correlations, and reliability of score difference are dependent of population. To overcome these difficulties, different statistical methods such as residual change scores, and multi-wave methods have been proposed for measuring changes in latent traits estimates \cite{DimitrovRumrill2003, RogosaWillett1985}, but the use of IRT-based models is the most effective in solving the classical problems in the measurement of change in latent trait estimates \cite{GluckSpiel1997, QueirozPrimiCarvalhoEnumo2013}.

Measurement of change in the latent trait estimates using IRT-based models presents a challenge in which the measurement from Time 1 to Time 2 should also consider a change in the item parameters. To measure this change, it is necessary to define a reference frame encompassing both times in one unambiguous representation. This process of placing Time 1 data and Time 2 data together in an unique frame of reference is known as \emph{stacking procedure} \cite{Wright2003}. For the empirical studies conducted in this dissertation, the stacking procedure involves the treating of formative assessments as source of data in which the Time 1 is the pre-test phase and Time 2 is the post-test phase. As these data are gathered from programming tasks and multiple knowledge choice questionnaires, the General Partial Credit Model (GPCM) \cite{MastersWright1996} had been used as instrument to estimate the skill/knowledge. With this model, the measure of gains in skills and knowledge is carried out in three steps: (1) Data verification, (2) Item splitting, and (3) Calculating changes. This stacking procedure had been carried out in the R software version 3.4.3 \cite{RCoreTeam2017} using the TAM: Test analysis modules package version 2.10 \cite{RobitzschKieferWu2018}. The R scripts for the stacking procedure with GPCM are available at the URL: \url{https://geiser.github.io/phd-thesis-evaluation/}

\subsection{Step 1: Data Verification}

The data verification consists into carried out GPCM analyses for the data gathered in pre-test phase and post-test phase, independently. This verification aims to detect and eliminate gross errors in the data entry. As result of these analyses, items and observations that distort or degrade the measurement system had been removed from the stacked analysis. For the identification of these items and observations, \emph{Infit} and \emph{Outfit} statistics are used in which mean-square values greater than $2$ indicate the distortion and degradation. The stability of the reference frame is also obtained in this step by plotting the item parameters estimated by the GPCM analyses with the post-test data (Time 2 data) against those item parameters estimated with the pre-test data (Time 1). In this plot, a close fit to the identity line indicates stability in the reference frame. % Figure illustrate this data verification - adapted for the thesis author

Prior to the data verification, the responses gathered from multiple choice questionnaires and the programming tasks are scored according to the rules described below.

\subsubsection*{Scoring-rule for Multiple Choice Questionnaires}

Let $NBC$ be the number of correct responses which have been checked, $NM$ be the number of wrong responses; and $NMC$ be the number of wrong responses which have been checked; then, the scoring rule for a n-th question in a multiple choice questionnaire is given by:

$ score(n) = \begin{cases}
0 & \text{if } NBC = 0 \text{ or }  \\
(NBC)(NM+1) - NMC & \text{otherwise}
\end{cases}$

\subsubsection*{Construction of Guttman-based Scoring-rules for Programming Tasks}

Guttman-based scoring rules for programming tasks \cite{Guttman2017} are scoring rules based on the principle of Guttman scale in which a unidimensional scale is defined as an aggregation of different indicators. In this sense, a Guttman-based scoring rule consists in a function that defines the combination of indicators based on a set of thresholds. For example, giving the indicators of correctness ($Q$) and time ($T$); and the thresholds of $Q = 1$ when the programming task has been solved adequately, and $T_{n} = 1$ when the time to solve the programming task is less than n-th percentile; then, a Guttman-structure scoring rule can be defined by the cartesian product $Q{\times}T_{75}{\times}T_{50}{\times}T_{25}$ as follows ($x$ denotes either of 0 and 1):

\begin{center}\scriptsize
\begin{tabular}{ll}
$(0,x,x,x) = 0$ & when the solution is incorrect\\
 & and the solving time is irrelevant\\
$(1,0,x,x) = 1$ & when the solution is correct\\
& and the solving time is greater than 75-th percentile (3rd quartile)\\
$(1,1,0,x) = 2$ & when the solution is correct\\
& and the solving time is greater than 50-th percentile (median)\\
$(1,1,1,0) = 3$ & when the solution is correct\\
 & and the solving time is greater than 25-th percentile (1st quartile)\\
$(1,1,1,1) = 4$ & when the solution is correct\\
 & and the solving time is less than 25-th percentile (1st quartile)\\
\end{tabular}
\end{center}

Let $P_{i}$ be a programming task solved by the participants during the pre-test and post-test phases, it has been scored according to the following four Guttman-based scoring rules: 

\begin{center}\scriptsize
\begin{tabular}{lll}
$P_{i}S_{1}$:&\multicolumn{2}{l}{$Q$}   \\
& $(0) = 0$ &  when the solution is incorrect \\
& $(1) = 0$ &  when the solution is correct \\
& & \\
$P_{i}S_{2}$:&\multicolumn{2}{l}{$Q{\times}T_{50}$} \\
& $(0,x) = 0$ & when the solution is incorrect\\
&  & and the solving time is irrelevant\\
& $(1,0) = 1$ & when the solution is correct\\
& & and the solving time is greater than median\\
& $(1,1) = 2$ & when the solution is correct\\
& & and the solving time is less than median\\
& & \\
$P_{i}S_{3}$:&\multicolumn{2}{l}{$Q{\times}T_{67}{\times}T_{33}$} \\
& $(0,x,x) = 0$ & when the solution is incorrect\\
&  & and the solving time is irrelevant\\
& $(1,0,x) = 1$ & when the solution is correct\\
& & and the solving time is greater than 33-th percentile\\
& $(1,1,0) = 2$ & when the solution is correct\\
& & and the solving time is greater than 67-th percentile\\
& $(1,1,1) = 3$ & when the solution is correct\\
&  & and the solving time is less than 67-th percentile\\
& & \\
$P_{i}S_{4}$:&\multicolumn{2}{l}{$Q{\times}T_{75}{\times}T_{50}{\times}T_{25}$} \\
& $(0,x,x,x) = 0$ & when the solution is incorrect\\
&  & and the solving time is irrelevant\\
& $(1,0,x,x) = 1$ & when the solution is correct\\
& & and the solving time is greater than 75-th percentile (3rd quartile)\\
& $(1,1,0,x) = 2$ & when the solution is correct\\
& & and the solving time is greater than 50-th percentile (median)\\
& $(1,1,1,0) = 3$ & when the solution is correct\\
&  & and the solving time is greater than 25-th percentile (1st quartile)\\
& $(1,1,1,1) = 4$ & when the solution is correct\\
&  & and the solving time is less than 25-th percentile (1st quartile)\\
\end{tabular}
\end{center}

After scoring the programing tasks with the four Guttman-based scoring rules ($P_{i}S_{1}$, $P_{i}S_{2}$, $P_{i}S_{3}$ and $P_{i}S_{4}$) defined above, each possible combination of rules is tested one by one using the GPCM and a set of programming tasks related to the pre-test phase or post-test phase. With the results of these tests, the measurement instrument of skill/knowledge for the pre-test phase or post-test phase is built employing the combination of rules that best fits with the data gathered over the empirical studies. The chosen set of Guttman-based scoring rules is the one that has best indices in the tests of unidimentionality, local independence and monotonicity for the GPCM (detailed in \autoref{sec:checking-assumptions-irt-motivation}).

\subsection{Step 2: Item Splitting}

In this step, data gathered from the pre-test phase (Time 1) and post-test phase (Time 2) are stacked together vertically, so that each participant in the empirical study appears twice times and each item appears once time. With these stacked data, the item parameters are estimated employing the MML method in the GPCM. These item parameters are used to plot the stability of reference frame, where: (1) items that are away from the identity line are \aspas{\emph{splitting}} into two separate items by splitting their responses into two data sets with missing data at the other time point in which the item is defined; and (2) items that are close to the identity line defines the calibration items for calculating the changes in skill/knowledge.

\subsection{Step 3: Calculating Changes in Latent Trait Estimates}

For calculating the changes in skill/knowledge as changes in the latent trait estimates, the post-test phase (Time 2) is installed as the benchmark to measure the change from the pre-test phase (Time 1). Item parameters ($D2$) and skill/knowledge ($B2$) for the calibration of measurement system are obtained by a GPCM using data gathered from the post-test phase (Time 2). These item parameters ($D2$) are applied in the GPCM with the data gathered from the pre-test phase (Time 1) for estimating the skill/knowledge ($B1$) and the item parameters for the split items ($D1$).

With the skill/knowledge measured in the pre-test phase (Time 1, $B1$) against the skill/knowledge in the post-test phase (Time 2, $B2$), the changes in the skill/knowledge are calculated as $B2-B1$ that define an unambiguously frame of reference.

%%%%%%%%%%%%%%%%%%%%%%%%%%%%%%%%%%%%%%%%%%%%%%%%%%%

\section{RSM-based Instrument for Measuring the Intrinsic Motivation in the Pilot Empirical Study}
\label{sec:irt-motivation-pilot-study}

\subsection{Checking Assumptions}

\subsubsection*{Test of Unidimensionality}

\autoref{tab:test-unidimensionality-irt-motivation-pilot-study} shows the results for the test of unidimensionality in which the goodness of fit statistics indicate moderate multidimensionality ($0.40 < DETECT < 1.00$) to measure the intrinsic motivation with a DETECT index of $0.565$. Essential unidimensionality ($ASSI < 0.25$ and $RATIO < 0.36$) is indicated by the ASSI and RATIO indices with values of $0.020$ and $0.015$, respectively. The index of $AGFI = 0.945$ in the unidimensional CFA indicates an acceptable fit for measuring the \emph{Intrinsic Motivation}. The sub-scales of \emph{Interest/Enjoyment}, \emph{Perceived Choice}, \emph{Pressure/Tension} and \emph{Effort/Importance} have a good fit indicated by the AGFI index with values greater than $0.95$. A good fit with the unidimensional CFA is indicated by the TLI and CFI indices for all the sub-scales with exception of the \emph{Perceived Choice}. The ASSI index indicates essential unidimensionality in the sub-scales of \emph{Interest/Enjoyment} and \emph{Perceived Choice}, and it indicates an essential deviation from unidimensionality in the sub-scales of \emph{Pressure/Tension} and \emph{Effort/Importance}. Essential unidimensionality is indicated by the RATIO index in the sub-scales of \emph{Interest/Enjoyment} and \emph{Effort/Importance}, and essential deviation from unidimensionality is indicated in the sub-scale of \emph{Perceived Choice} and \emph{Pressure/Tension} by this index.

%latex.default(data_df, caption = paste("Goodness of fit statistics related to the test of unidimensionality",     "in RSM-based instruments", in_title), size = "small", longtable = T,     ctable = F, landscape = F, where = "!htbp", file = filename,     append = T)%
\setlongtables{\scriptsize
\begin{longtable}{lrrrrrrrr}\caption{Goodness of fit statistics related to the test of unidimensionality in the RSM-based instrument for measuring the intrinsic motivation in the pilot empirical study}
\tabularnewline
\hline\hline
\multicolumn{1}{l}{}&\multicolumn{1}{c}{df}&\multicolumn{1}{c}{chisq}&\multicolumn{1}{c}{AGFI}&\multicolumn{1}{c}{TLI}&\multicolumn{1}{c}{CFI}&\multicolumn{1}{c}{DETECT}&\multicolumn{1}{c}{ASSI}&\multicolumn{1}{c}{RATIO}\tabularnewline
\hline
\endfirsthead\caption[]{\em (continued)} \tabularnewline
\hline
\multicolumn{1}{l}{}&\multicolumn{1}{c}{df}&\multicolumn{1}{c}{chisq}&\multicolumn{1}{c}{AGFI}&\multicolumn{1}{c}{TLI}&\multicolumn{1}{c}{CFI}&\multicolumn{1}{c}{DETECT}&\multicolumn{1}{c}{ASSI}&\multicolumn{1}{c}{RATIO}\tabularnewline
\hline
\endhead
\hline
\multicolumn{9}{r}{\tiny df: degree of freedom; AGFI: Adjusted Goodness of Fit Index; CFI: Comparative Fit Index; TLI: Tucker-Lewis Index;}
\endfoot
\label{tab:test-unidimensionality-irt-motivation-pilot-study}
Intrinsic Motivation&$8.451$&$19.955$&$0.945$&$0.690$&$0.729$&$ 0.565$&$0.020$&$0.015$\tabularnewline
Interest/Enjoyment&$2.468$&$1.426$&$0.998$&$1.023$&$1.000$&$ 0.716$&$0.067$&$0.049$\tabularnewline
Perceived Choice&$3.064$&$9.713$&$0.978$&$0.711$&$0.788$&$22.998$&$0.200$&$0.714$\tabularnewline
Pressure/Tension&$1.982$&$0.534$&$0.998$&$1.117$&$1.000$&$17.068$&$0.333$&$0.873$\tabularnewline
Effort/Importance&$0.000$&$0.000$&$1.000$&$1.000$&$1.000$&$10.746$&$0.333$&$0.358$\tabularnewline
\hline
\end{longtable}}

\subsubsection*{Test of Local Independence}

Results from the test of local independence in the RSM-based instrument for measuring the intrinsic motivation in the pilot empirical study are summarized in \autoref{tab:test-local-independence-irt-motivation-pilot-study}. According to the p-values, the null condition of local independence is not rejected in any of the four sub-scales of RSM-based instrument. The Standardized Root Mean Squared Residual (SRMSR) indicates a good fit ($< 0.10$) for the sub-scales of \emph{Interest/Enjoyment} and \emph{Effort/Importance}, and acceptable fit ($0.10$s) for the \emph{Perceived Choice} and \emph{Pressure/Tension}.

%latex.default(data_df, caption = paste("Q3 statistics related to the test of local independence",     "in the RSM-based instrument", in_title), size = "small",     longtable = T, ctable = F, landscape = F, where = "!htbp",     file = filename, append = T)%
\setlongtables{\scriptsize
\begin{longtable}{lrrrrr}\caption{Item residual correlation statistics related to the test of local independence in the RSM-based instrument for measuring the intrinsic motivation in the pilot empirical study} \tabularnewline
\hline\hline
\multicolumn{1}{l}{}&\multicolumn{1}{c}{max.chisq}&\multicolumn{1}{c}{maxaQ3}&\multicolumn{1}{c}{MADaQ3}&\multicolumn{1}{c}{SRMSR}&\multicolumn{1}{c}{p.value}\tabularnewline
\hline
\endfirsthead\caption[]{\em (continued)} \tabularnewline
\hline
\multicolumn{1}{l}{}&\multicolumn{1}{c}{max.chisq}&\multicolumn{1}{c}{maxaQ3}&\multicolumn{1}{c}{MADaQ3}&\multicolumn{1}{c}{SRMSR}&\multicolumn{1}{c}{p.value}\tabularnewline
\hline
\endhead
\hline
\multicolumn{6}{r}{\tiny aQ3: adjusted correlation of item residuals; maxaQ3: maximum aQ3;}\tabularnewline
\multicolumn{6}{r}{\tiny MADaQ3: Median Absolute Deviation of aQ3;}
\endfoot
\label{tab:test-local-independence-irt-motivation-pilot-study}
Interest/Enjoyment&$498.445$&$0.353$&$0.157$&$0.083$&$1.000$\tabularnewline
Perceived Choice&$114.058$&$0.500$&$0.248$&$0.165$&$0.093$\tabularnewline
Pressure/Tension&$ 36.673$&$0.302$&$0.214$&$0.153$&$0.696$\tabularnewline
Effort/Importance&$ 38.718$&$0.066$&$0.044$&$0.037$&$1.000$\tabularnewline
\hline
\end{longtable}}
%\begin{flushright}{\tiny Q3: correlation of item residuals; maxaQ3: maximum adjusted correlation Q3; MADaQ3: Median Absolute Deviation in the adjusted Q3; }\end{flushright}

\subsubsection*{Test of Monotonicity}

\autoref{tab:test-monotonicity-irt-motivation-pilot-study} summarizes the test of monotonicity in the RSM-based instrument for measuring the intrinsic motivation in the pilot empirical study. These results indicates that there are no one violation of monotonicity in the items at the significance level $\alpha = 0.05$.

%latex.default(data_df, caption = paste("Summary of the violations of monotonicity",     "in the RSM-based instrument", in_title), size = "small",     longtable = T, ctable = F, landscape = F, where = "!htbp",     file = filename, append = T)%
\setlongtables{\scriptsize
\begin{longtable}{lrrrrrrrrrr}\caption{Test of monotonicity in the RSM-based instrument for measuring the intrinsic motivation in the pilot empirical study} \tabularnewline
\hline\hline
\multicolumn{1}{l}{}&\multicolumn{1}{c}{ItemH}&\multicolumn{1}{c}{ac}&\multicolumn{1}{c}{vi}&\multicolumn{1}{c}{vi/ac}&\multicolumn{1}{c}{maxvi}&\multicolumn{1}{c}{sum}&\multicolumn{1}{c}{sum/ac}&\multicolumn{1}{c}{zmax}&\multicolumn{1}{c}{zsig}&\multicolumn{1}{c}{crit}\tabularnewline
\hline
\endfirsthead\caption[]{\em (continued)} \tabularnewline
\hline
\multicolumn{1}{l}{}&\multicolumn{1}{c}{ItemH}&\multicolumn{1}{c}{ac}&\multicolumn{1}{c}{vi}&\multicolumn{1}{c}{vi/ac}&\multicolumn{1}{c}{maxvi}&\multicolumn{1}{c}{sum}&\multicolumn{1}{c}{sum/ac}&\multicolumn{1}{c}{zmax}&\multicolumn{1}{c}{zsig}&\multicolumn{1}{c}{crit}\tabularnewline
\hline
\endhead
\hline
\multicolumn{11}{r}{\tiny vi: number of violations; vi/ac: proportion of active pairs; maxvi: maximum violations;}\tabularnewline
\multicolumn{11}{r}{\tiny sum: sum of all violations; zmax: maximum z-value; zsig: number of significant z-values; crit: critical value}
\endfoot
\label{tab:test-monotonicity-irt-motivation-pilot-study}
Interest/Enjoyment:Item22IE&$0.85$&$0$&$0$&$$&$0.00$&$0.0$&$$&$0$&$0$&$  0$\tabularnewline
Interest/Enjoyment:Item09IE&$0.79$&$0$&$0$&$$&$0.00$&$0.0$&$$&$0$&$0$&$  0$\tabularnewline
Interest/Enjoyment:Item12IE&$0.81$&$0$&$0$&$$&$0.00$&$0.0$&$$&$0$&$0$&$  0$\tabularnewline
Interest/Enjoyment:Item24IE&$0.77$&$4$&$0$&$0.0$&$0.00$&$0.0$&$0.00$&$0$&$0$&$  0$\tabularnewline
Interest/Enjoyment:Item21IE&$0.72$&$0$&$0$&$$&$0.00$&$0.0$&$$&$0$&$0$&$  0$\tabularnewline
Interest/Enjoyment:Item01IE&$0.69$&$0$&$0$&$$&$0.00$&$0.0$&$$&$0$&$0$&$  0$\tabularnewline
Perceived Choice:Item17PC&$0.60$&$4$&$0$&$0.0$&$0.00$&$0.0$&$0.00$&$0$&$0$&$  0$\tabularnewline
Perceived Choice:Item15PC&$0.47$&$4$&$0$&$0.0$&$0.00$&$0.0$&$0.00$&$0$&$0$&$  0$\tabularnewline
Perceived Choice:Item06PC&$0.52$&$1$&$0$&$0.0$&$0.00$&$0.0$&$0.00$&$0$&$0$&$  0$\tabularnewline
Perceived Choice:Item02PC&$0.47$&$5$&$0$&$0.0$&$0.00$&$0.0$&$0.00$&$0$&$0$&$  0$\tabularnewline
Perceived Choice:Item08PC&$0.38$&$0$&$0$&$$&$0.00$&$0.0$&$$&$0$&$0$&$  0$\tabularnewline
Pressure/Tension:Item16PT&$0.53$&$0$&$0$&$$&$0.00$&$0.0$&$$&$0$&$0$&$  0$\tabularnewline
Pressure/Tension:Item14PT&$0.45$&$3$&$0$&$0.0$&$0.00$&$0.0$&$0.00$&$0$&$0$&$  0$\tabularnewline
Pressure/Tension:Item18PT&$0.56$&$4$&$0$&$0.0$&$0.00$&$0.0$&$0.00$&$0$&$0$&$  0$\tabularnewline
Pressure/Tension:Item11PT&$0.36$&$0$&$0$&$$&$0.00$&$0.0$&$$&$0$&$0$&$  0$\tabularnewline
Effort/Importance:Item13EI&$0.46$&$0$&$0$&$$&$0.00$&$0.0$&$$&$0$&$0$&$  0$\tabularnewline
Effort/Importance:Item03EI&$0.44$&$0$&$0$&$$&$0.00$&$0.0$&$$&$0$&$0$&$  0$\tabularnewline
Effort/Importance:Item07EI&$0.48$&$0$&$0$&$$&$0.00$&$0.0$&$$&$0$&$0$&$  0$\tabularnewline
\hline
\end{longtable}}
%\begin{flushright}{\tiny vi: numer of violations; vi/ac: proportion of active pairs; maxvi: maximum violations; sum: sum of all violations; zmax: maximum z-value; zsig: number of significant z-values; crit: Critical value }\end{flushright}

\subsection{Item Parameters}

\autoref{tab:item-parameters-interest-enjoyment-pilot-study} shows the estimated parameters for the RSM-based instrument used to measure the \emph{Interest/Enjoyment} in the pilot empirical study. These parameters had been calculated using the MML method \cite{BockAitkin1981}, so that the value in row \aspas{B.Cat$x$} and column \aspas{$i$} is the item slope $b_{i,x}$ of item $i$ in the category \aspas{$x$}, and the value in the row \aspas{AXsi.Cat$x$} and column \aspas{$i$} is the item intercept $a_{i,x}\xi$ of item $i$ in the category \aspas{$x$}. According to the Infit/Outfit statistics of items, no one mean-square value is greater than $2.0$ indicating that the measurement system of \emph{Interest/Enjoyment} is not distorted or degraded by the items.

%latex.default(estimated_params_df, caption = paste("Estimated parameters in the RSM-based instrument",     "for measuring the", lname), size = "small", longtable = T,     ctable = F, landscape = F, where = "!htbp", file = filename,     append = T)%
\setlongtables{\scriptsize
\begin{longtable}{lrrrrrr}\caption{Estimated parameters in the RSM-based instrument for measuring the interest/enjoyment in the pilot empirical study} \tabularnewline
\hline\hline
\multicolumn{1}{l}{}&\multicolumn{1}{c}{Item01IE}&\multicolumn{1}{c}{Item09IE}&\multicolumn{1}{c}{Item12IE}&\multicolumn{1}{c}{Item21IE}&\multicolumn{1}{c}{Item22IE}&\multicolumn{1}{c}{Item24IE}\tabularnewline
\hline
\endfirsthead\caption[]{\em (continued)} \tabularnewline
\hline
\multicolumn{1}{l}{}&\multicolumn{1}{c}{Item01IE}&\multicolumn{1}{c}{Item09IE}&\multicolumn{1}{c}{Item12IE}&\multicolumn{1}{c}{Item21IE}&\multicolumn{1}{c}{Item22IE}&\multicolumn{1}{c}{Item24IE}\tabularnewline
\hline
\endhead
\hline
\endfoot
\label{tab:item-parameters-interest-enjoyment-pilot-study}
xsi.item&$ 0.888$&$-0.023$&$ 0.368$&$-0.132$&$ 0.332$&$ 0.472$\tabularnewline
B.Cat0&$ 0.000$&$ 0.000$&$ 0.000$&$ 0.000$&$ 0.000$&$ 0.000$\tabularnewline
B.Cat1&$ 1.000$&$ 1.000$&$ 1.000$&$ 1.000$&$ 1.000$&$ 1.000$\tabularnewline
B.Cat2&$ 2.000$&$ 2.000$&$ 2.000$&$ 2.000$&$ 2.000$&$ 2.000$\tabularnewline
B.Cat3&$ 3.000$&$ 3.000$&$ 3.000$&$ 3.000$&$ 3.000$&$ 3.000$\tabularnewline
B.Cat4&$ 4.000$&$ 4.000$&$ 4.000$&$ 4.000$&$ 4.000$&$ 4.000$\tabularnewline
B.Cat5&$ 5.000$&$ 5.000$&$ 5.000$&$ 5.000$&$ 5.000$&$ 5.000$\tabularnewline
B.Cat6&$ 6.000$&$ 6.000$&$ 6.000$&$ 6.000$&$ 6.000$&$ 6.000$\tabularnewline
AXsi.Cat0&$ 0.000$&$ 0.000$&$ 0.000$&$ 0.000$&$ 0.000$&$ 0.000$\tabularnewline
AXsi.Cat1&$ 0.451$&$ 1.362$&$ 0.971$&$ 1.471$&$ 1.007$&$ 0.867$\tabularnewline
AXsi.Cat2&$ 0.349$&$ 2.172$&$ 1.390$&$ 2.390$&$ 1.460$&$ 1.180$\tabularnewline
AXsi.Cat3&$ 0.783$&$ 3.517$&$ 2.345$&$ 3.844$&$ 2.450$&$ 2.030$\tabularnewline
AXsi.Cat4&$-0.153$&$ 3.493$&$ 1.929$&$ 3.929$&$ 2.070$&$ 1.510$\tabularnewline
AXsi.Cat5&$-2.115$&$ 2.442$&$ 0.487$&$ 2.987$&$ 0.663$&$-0.037$\tabularnewline
AXsi.Cat6&$-5.328$&$ 0.140$&$-2.205$&$ 0.794$&$-1.995$&$-2.834$\tabularnewline
\hline
Outfit&$ 1.532$&$ 0.815$&$ 0.773$&$ 1.058$&$ 0.580$&$ 0.836$\tabularnewline
Infit&$ 1.386$&$ 0.882$&$ 0.801$&$ 1.284$&$ 0.635$&$ 0.957$\tabularnewline
\hline
\end{longtable}}

\autoref{tab:item-parameters-perceived-choice-pilot-study} shows the estimated parameters for the measurement instrument of \emph{Perceived Choice} in which the Infit/Outfit statistics of items indicate that no one item distorts or degrades the measurement system with mean-square greater than $2.0$.

%latex.default(estimated_params_df, caption = paste("Estimated parameters in the RSM-based instrument",     "for measuring the", lname), size = "small", longtable = T,     ctable = F, landscape = F, where = "!htbp", file = filename,     append = T)%
\setlongtables{\scriptsize
\begin{longtable}{lrrrrr}\caption{Estimated parameters in the RSM-based instrument for measuring the perceived choice in the pilot empirical study} \tabularnewline
\hline\hline
\multicolumn{1}{l}{}&\multicolumn{1}{c}{Item02PC}&\multicolumn{1}{c}{Item06PC}&\multicolumn{1}{c}{Item08PC}&\multicolumn{1}{c}{Item15PC}&\multicolumn{1}{c}{Item17PC}\tabularnewline
\hline
\endfirsthead\caption[]{\em (continued)} \tabularnewline
\hline
\multicolumn{1}{l}{}&\multicolumn{1}{c}{Item02PC}&\multicolumn{1}{c}{Item06PC}&\multicolumn{1}{c}{Item08PC}&\multicolumn{1}{c}{Item15PC}&\multicolumn{1}{c}{Item17PC}\tabularnewline
\hline
\endhead
\hline
\endfoot
\label{tab:item-parameters-perceived-choice-pilot-study}
xsi.item&$ 0.112$&$-0.237$&$-0.491$&$ 0.030$&$-0.469$\tabularnewline
B.Cat0&$ 0.000$&$ 0.000$&$ 0.000$&$ 0.000$&$ 0.000$\tabularnewline
B.Cat1&$ 1.000$&$ 1.000$&$ 1.000$&$ 1.000$&$ 1.000$\tabularnewline
B.Cat2&$ 2.000$&$ 2.000$&$ 2.000$&$ 2.000$&$ 2.000$\tabularnewline
B.Cat3&$ 3.000$&$ 3.000$&$ 3.000$&$ 3.000$&$ 3.000$\tabularnewline
B.Cat4&$ 4.000$&$ 4.000$&$ 4.000$&$ 4.000$&$ 4.000$\tabularnewline
B.Cat5&$ 5.000$&$ 5.000$&$ 5.000$&$ 5.000$&$ 5.000$\tabularnewline
B.Cat6&$ 6.000$&$ 6.000$&$ 6.000$&$ 6.000$&$ 6.000$\tabularnewline
AXsi.Cat0&$ 0.000$&$ 0.000$&$ 0.000$&$ 0.000$&$ 0.000$\tabularnewline
AXsi.Cat1&$ 1.185$&$ 1.534$&$ 1.788$&$ 1.267$&$ 1.766$\tabularnewline
AXsi.Cat2&$ 0.898$&$ 1.597$&$ 2.104$&$ 1.062$&$ 2.061$\tabularnewline
AXsi.Cat3&$ 2.082$&$ 3.130$&$ 3.891$&$ 2.328$&$ 3.826$\tabularnewline
AXsi.Cat4&$ 1.103$&$ 2.500$&$ 3.515$&$ 1.431$&$ 3.428$\tabularnewline
AXsi.Cat5&$ 0.368$&$ 2.115$&$ 3.383$&$ 0.779$&$ 3.275$\tabularnewline
AXsi.Cat6&$-0.674$&$ 1.422$&$ 2.944$&$-0.181$&$ 2.814$\tabularnewline
\hline
Outfit&$ 1.066$&$ 0.968$&$ 1.452$&$ 1.025$&$ 0.700$\tabularnewline
Infit&$ 1.007$&$ 0.994$&$ 1.375$&$ 1.001$&$ 0.704$\tabularnewline
\hline
\end{longtable}}


\autoref{tab:item-parameters-pressure-tension-pilot-study} shows the estimated parameters for the measurement instrument of \emph{Pressure/Tension} in the pilot empirical study in which the Infit/Outfit statistics of items indicate that no one item distorts or degrades the measurement system with mean-square greater than $2.0$.

%latex.default(estimated_params_df, caption = paste("Estimated parameters in the RSM-based instrument",     "for measuring the", lname), size = "small", longtable = T,     ctable = F, landscape = F, where = "!htbp", file = filename,     append = T)%
\setlongtables{\scriptsize
\begin{longtable}{lrrrr}\caption{Estimated parameters in the RSM-based instrument for measuring the pressure/tension in the pilot empirical study} \tabularnewline
\hline\hline
\multicolumn{1}{l}{}&\multicolumn{1}{c}{Item11PT}&\multicolumn{1}{c}{Item14PT}&\multicolumn{1}{c}{Item16PT}&\multicolumn{1}{c}{Item18PT}\tabularnewline
\hline
\endfirsthead\caption[]{\em (continued)} \tabularnewline
\hline
\multicolumn{1}{l}{}&\multicolumn{1}{c}{Item11PT}&\multicolumn{1}{c}{Item14PT}&\multicolumn{1}{c}{Item16PT}&\multicolumn{1}{c}{Item18PT}\tabularnewline
\hline
\endhead
\hline
\endfoot
\label{tab:item-parameters-pressure-tension-pilot-study}
xsi.item&$-0.036$&$ 0.288$&$-0.054$&$ 0.347$\tabularnewline
B.Cat0&$ 0.000$&$ 0.000$&$ 0.000$&$ 0.000$\tabularnewline
B.Cat1&$ 1.000$&$ 1.000$&$ 1.000$&$ 1.000$\tabularnewline
B.Cat2&$ 2.000$&$ 2.000$&$ 2.000$&$ 2.000$\tabularnewline
B.Cat3&$ 3.000$&$ 3.000$&$ 3.000$&$ 3.000$\tabularnewline
B.Cat4&$ 4.000$&$ 4.000$&$ 4.000$&$ 4.000$\tabularnewline
B.Cat5&$ 5.000$&$ 0.000$&$ 0.000$&$ 0.000$\tabularnewline
B.Cat6&$ 6.000$&$ 0.000$&$ 0.000$&$ 0.000$\tabularnewline
AXsi.Cat0&$ 0.000$&$ 0.000$&$ 0.000$&$ 0.000$\tabularnewline
AXsi.Cat1&$ 1.034$&$ 0.272$&$ 0.614$&$ 0.213$\tabularnewline
AXsi.Cat2&$ 1.249$&$-0.275$&$ 0.410$&$-0.392$\tabularnewline
AXsi.Cat3&$ 2.096$&$-0.190$&$ 0.837$&$-0.367$\tabularnewline
AXsi.Cat4&$ 1.895$&$-1.152$&$ 0.217$&$-1.387$\tabularnewline
AXsi.Cat5&$ 1.711$&$$&$$&$$\tabularnewline
AXsi.Cat6&$ 0.216$&$$&$$&$$\tabularnewline
\hline
Outfit&$ 1.355$&$ 1.000$&$ 0.861$&$ 0.850$\tabularnewline
Infit&$ 1.361$&$ 0.915$&$ 0.842$&$ 0.919$\tabularnewline
\hline
\end{longtable}}

\autoref{tab:item-parameters-effort-importance-pilot-study} shows the estimated parameters for the measurement instrument of \emph{Effort/Importance} in which the Infit/Outfit statistics of items indicate that no one item distorts or degrades the measurement system with mean-square greater than $2.0$.

%latex.default(estimated_params_df, caption = paste("Estimated parameters in the RSM-based instrument",     "for measuring the", lname), size = "small", longtable = T,     ctable = F, landscape = F, where = "!htbp", file = filename,     append = T)%
\setlongtables{\scriptsize
\begin{longtable}{lrrr}\caption{Estimated parameters in the RSM-based instrument for measuring the effort/importance in the pilot empirical study} \tabularnewline
\hline\hline
\multicolumn{1}{l}{}&\multicolumn{1}{c}{Item03EI}&\multicolumn{1}{c}{Item07EI}&\multicolumn{1}{c}{Item13EI}\tabularnewline
\hline
\endfirsthead\caption[]{\em (continued)} \tabularnewline
\hline
\multicolumn{1}{l}{}&\multicolumn{1}{c}{Item03EI}&\multicolumn{1}{c}{Item07EI}&\multicolumn{1}{c}{Item13EI}\tabularnewline
\hline
\endhead
\hline
\endfoot
\label{tab:item-parameters-effort-importance-pilot-study}
xsi.item&$-1.793$&$-1.571$&$-1.620$\tabularnewline
B.Cat0&$ 0.000$&$ 0.000$&$ 0.000$\tabularnewline
B.Cat1&$ 1.000$&$ 1.000$&$ 1.000$\tabularnewline
B.Cat2&$ 2.000$&$ 2.000$&$ 2.000$\tabularnewline
B.Cat3&$ 3.000$&$ 3.000$&$ 3.000$\tabularnewline
B.Cat4&$ 4.000$&$ 4.000$&$ 4.000$\tabularnewline
B.Cat5&$ 5.000$&$ 5.000$&$ 5.000$\tabularnewline
B.Cat6&$ 6.000$&$ 6.000$&$ 6.000$\tabularnewline
AXsi.Cat0&$ 0.000$&$ 0.000$&$ 0.000$\tabularnewline
AXsi.Cat1&$ 7.416$&$ 7.195$&$ 7.243$\tabularnewline
AXsi.Cat2&$ 9.886$&$ 9.444$&$ 9.541$\tabularnewline
AXsi.Cat3&$ 9.821$&$ 9.158$&$ 9.303$\tabularnewline
AXsi.Cat4&$10.943$&$10.059$&$10.252$\tabularnewline
AXsi.Cat5&$10.732$&$ 9.626$&$ 9.868$\tabularnewline
AXsi.Cat6&$10.756$&$ 9.429$&$ 9.719$\tabularnewline
\hline
Outfit&$ 1.012$&$ 1.063$&$ 0.987$\tabularnewline
Infit&$ 0.992$&$ 1.035$&$ 1.030$\tabularnewline
\hline
\end{longtable}}


\subsection{Intrinsic Motivation as Latent Trait Estimates}

\autoref{tab:intrinsic-motivation-estimates-pilot-study} shows the latent trait estimates by the RSM-based instrument for measuring the \emph{Intrinsic motivation} in the pilot empirical study.

%latex.default(data_df, caption = paste("Latent trait estimates and person model fit of the RSM-based instrument",     in_title), size = "scriptsize", longtable = T, ctable = F,     landscape = T, rowlabel = "", where = "!htbp", file = filename,     append = T)%
\setlongtables\begin{landscape}{\scriptsize
\begin{longtable}{l|rrrr|rrrr|rrrr|rrrr|rrrr}\caption{Latent trait estimates and person model fit of the RSM-based instrument for measuring the intrinsic motivation in the pilot empirical study} \tabularnewline
\hline\hline
\multicolumn{1}{l}{}&\multicolumn{4}{|c}{Intrinsic Motivation}&\multicolumn{4}{|c}{Interest/Enjoyment}&\multicolumn{4}{|c}{Perceived Choice}&\multicolumn{4}{|c}{Pressure/Tension}&\multicolumn{4}{|c}{Effort/Importance} \tabularnewline
\multicolumn{1}{l}{UserID}&\multicolumn{1}{|c}{theta}&\multicolumn{1}{c}{error}&\multicolumn{1}{c}{Outfit}&\multicolumn{1}{c}{Infit}&\multicolumn{1}{|c}{theta}&\multicolumn{1}{c}{error}&\multicolumn{1}{c}{Outfit}&\multicolumn{1}{c}{Infit}&\multicolumn{1}{|c}{theta}&\multicolumn{1}{c}{error}&\multicolumn{1}{c}{Outfit}&\multicolumn{1}{c}{Infit}&\multicolumn{1}{|c}{theta}&\multicolumn{1}{c}{error}&\multicolumn{1}{c}{Outfit}&\multicolumn{1}{c}{Infit}&\multicolumn{1}{|c}{theta}&\multicolumn{1}{c}{error}&\multicolumn{1}{c}{Outfit}&\multicolumn{1}{c}{Infit}\tabularnewline
\hline
\endfirsthead\caption[]{\em (continued)} \tabularnewline
\hline
\multicolumn{1}{l}{}&\multicolumn{4}{|c}{Intrinsic Motivation}&\multicolumn{4}{|c}{Interest/Enjoyment}&\multicolumn{4}{|c}{Perceived Choice}&\multicolumn{4}{|c}{Pressure/Tension}&\multicolumn{4}{|c}{Effort/Importance} \tabularnewline
\multicolumn{1}{l}{UserID}&\multicolumn{1}{|c}{theta}&\multicolumn{1}{c}{error}&\multicolumn{1}{c}{Outfit}&\multicolumn{1}{c}{Infit}&\multicolumn{1}{|c}{theta}&\multicolumn{1}{c}{error}&\multicolumn{1}{c}{Outfit}&\multicolumn{1}{c}{Infit}&\multicolumn{1}{|c}{theta}&\multicolumn{1}{c}{error}&\multicolumn{1}{c}{Outfit}&\multicolumn{1}{c}{Infit}&\multicolumn{1}{|c}{theta}&\multicolumn{1}{c}{error}&\multicolumn{1}{c}{Outfit}&\multicolumn{1}{c}{Infit}&\multicolumn{1}{|c}{theta}&\multicolumn{1}{c}{error}&\multicolumn{1}{c}{Outfit}&\multicolumn{1}{c}{Infit}\tabularnewline
\hline
\endhead
\hline
\endfoot
\label{tab:intrinsic-motivation-estimates-pilot-study}
10116&$-0.404$&$0.154$&$0.569$&$0.557$&$-0.114$&$0.340$&$0.991$&$1.063$&$-0.842$&$0.347$&$0.487$&$0.456$&$ 0.207$&$0.387$&$0.482$&$0.499$&$-0.595$&$0.444$&$0.020$&$0.021$\tabularnewline
10120&$ 0.969$&$0.240$&$0.595$&$0.566$&$ 1.906$&$0.483$&$0.791$&$0.888$&$ 1.372$&$0.492$&$0.530$&$0.588$&$-1.126$&$0.578$&$0.398$&$0.306$&$ 0.694$&$0.577$&$0.864$&$0.879$\tabularnewline
10121&$ 0.050$&$0.161$&$0.661$&$0.694$&$ 0.399$&$0.382$&$1.481$&$1.713$&$ 0.587$&$0.361$&$0.402$&$0.425$&$ 0.345$&$0.396$&$0.487$&$0.452$&$-0.119$&$0.426$&$0.017$&$0.017$\tabularnewline
10122&$-0.171$&$0.155$&$1.833$&$1.867$&$-2.052$&$0.572$&$0.878$&$0.690$&$ 1.170$&$0.443$&$0.441$&$0.480$&$-0.488$&$0.420$&$1.565$&$1.940$&$-0.119$&$0.426$&$1.391$&$1.366$\tabularnewline
10123&$-0.311$&$0.154$&$0.497$&$0.495$&$-0.832$&$0.330$&$0.240$&$0.260$&$ 0.236$&$0.340$&$0.375$&$0.387$&$ 0.207$&$0.387$&$0.821$&$0.793$&$-0.430$&$0.428$&$0.672$&$0.685$\tabularnewline
10126&$ 0.490$&$0.188$&$0.917$&$0.803$&$ 1.273$&$0.443$&$0.452$&$0.457$&$ 1.170$&$0.443$&$0.156$&$0.122$&$-0.488$&$0.420$&$0.854$&$0.996$&$-0.119$&$0.426$&$1.376$&$1.400$\tabularnewline
10127&$ 0.183$&$0.167$&$1.085$&$1.042$&$ 1.906$&$0.483$&$0.740$&$0.808$&$ 0.236$&$0.340$&$0.780$&$0.781$&$-0.195$&$0.390$&$0.330$&$0.359$&$-0.779$&$0.473$&$0.434$&$0.443$\tabularnewline
10128&$-0.521$&$0.155$&$2.364$&$2.405$&$-3.515$&$1.347$&$0.087$&$0.096$&$ 0.587$&$0.361$&$3.270$&$3.014$&$-0.336$&$0.401$&$2.149$&$2.637$&$-2.817$&$1.034$&$0.766$&$0.763$\tabularnewline
10129&$-0.794$&$0.164$&$0.817$&$0.850$&$-1.785$&$0.486$&$0.427$&$0.471$&$-0.622$&$0.335$&$0.418$&$0.391$&$ 3.173$&$1.487$&$0.122$&$0.133$&$ 0.043$&$0.438$&$1.521$&$1.516$\tabularnewline
10130&$-0.147$&$0.155$&$0.870$&$0.817$&$ 0.553$&$0.395$&$0.119$&$0.119$&$-0.305$&$0.330$&$0.676$&$0.674$&$ 1.413$&$0.633$&$0.328$&$0.321$&$ 0.694$&$0.577$&$0.282$&$0.288$\tabularnewline
10131&$ 0.239$&$0.170$&$1.986$&$2.120$&$ 2.400$&$0.530$&$2.083$&$2.163$&$-0.959$&$0.358$&$1.745$&$1.866$&$-0.860$&$0.496$&$0.247$&$0.199$&$ 0.429$&$0.504$&$0.534$&$0.572$\tabularnewline
10132&$-0.474$&$0.154$&$1.148$&$1.144$&$-3.515$&$1.347$&$0.087$&$0.096$&$-0.199$&$0.330$&$1.129$&$1.136$&$-0.195$&$0.390$&$0.252$&$0.298$&$ 0.221$&$0.462$&$0.454$&$0.421$\tabularnewline
10134&$-0.666$&$0.159$&$0.281$&$0.298$&$-1.285$&$0.375$&$0.139$&$0.130$&$-0.731$&$0.340$&$0.743$&$0.778$&$ 0.345$&$0.396$&$0.232$&$0.232$&$-0.595$&$0.444$&$0.020$&$0.021$\tabularnewline
10135&$-0.025$&$0.159$&$1.089$&$1.052$&$ 0.399$&$0.382$&$1.010$&$1.099$&$ 0.349$&$0.345$&$2.289$&$2.180$&$ 0.345$&$0.396$&$0.861$&$0.901$&$-0.274$&$0.422$&$0.757$&$0.757$\tabularnewline
10136&$-1.021$&$0.178$&$0.958$&$0.997$&$-3.515$&$1.347$&$0.087$&$0.096$&$-1.371$&$0.418$&$0.510$&$0.401$&$ 0.073$&$0.383$&$1.402$&$1.309$&$-1.869$&$0.792$&$0.419$&$0.393$\tabularnewline
10137&$ 0.969$&$0.240$&$0.854$&$0.798$&$ 1.474$&$0.455$&$0.324$&$0.317$&$ 1.372$&$0.492$&$1.290$&$1.524$&$-2.621$&$1.373$&$0.129$&$0.145$&$ 0.694$&$0.577$&$1.150$&$1.098$\tabularnewline
10138&$-0.001$&$0.159$&$0.790$&$0.756$&$ 0.553$&$0.395$&$0.711$&$0.716$&$-0.410$&$0.330$&$0.489$&$0.492$&$ 0.345$&$0.396$&$0.297$&$0.304$&$ 1.902$&$1.193$&$0.177$&$0.179$\tabularnewline
10139&$-0.099$&$0.157$&$1.163$&$1.175$&$-0.223$&$0.334$&$0.920$&$0.888$&$ 0.125$&$0.336$&$2.016$&$2.018$&$-0.488$&$0.420$&$1.866$&$1.854$&$-0.595$&$0.444$&$0.269$&$0.266$\tabularnewline
10140&$-0.617$&$0.157$&$0.514$&$0.490$&$-1.045$&$0.346$&$0.503$&$0.513$&$-0.199$&$0.330$&$0.115$&$0.115$&$ 0.845$&$0.472$&$1.194$&$1.258$&$-1.004$&$0.524$&$0.275$&$0.277$\tabularnewline
10141&$-0.218$&$0.154$&$0.484$&$0.470$&$ 0.002$&$0.349$&$0.430$&$0.449$&$-0.410$&$0.330$&$0.483$&$0.485$&$-0.060$&$0.384$&$0.718$&$0.632$&$-0.430$&$0.428$&$0.455$&$0.450$\tabularnewline
10142&$ 0.076$&$0.162$&$0.798$&$0.791$&$ 0.553$&$0.395$&$0.872$&$0.821$&$-0.305$&$0.330$&$0.364$&$0.365$&$-0.860$&$0.496$&$1.680$&$1.816$&$-0.119$&$0.426$&$0.500$&$0.494$\tabularnewline
10143&$ 0.102$&$0.163$&$0.125$&$0.127$&$ 0.125$&$0.358$&$0.118$&$0.119$&$ 0.349$&$0.345$&$0.140$&$0.137$&$-0.060$&$0.384$&$0.136$&$0.149$&$ 0.429$&$0.504$&$0.028$&$0.029$\tabularnewline
10145&$-0.357$&$0.154$&$0.864$&$0.851$&$-0.631$&$0.324$&$0.606$&$0.600$&$-0.842$&$0.347$&$0.785$&$0.731$&$ 0.493$&$0.411$&$0.398$&$0.413$&$ 1.902$&$1.193$&$0.177$&$0.179$\tabularnewline
10146&$ 0.327$&$0.176$&$0.384$&$0.356$&$ 0.895$&$0.420$&$0.102$&$0.108$&$ 0.349$&$0.345$&$0.135$&$0.136$&$-0.336$&$0.401$&$1.140$&$1.388$&$ 0.694$&$0.577$&$0.282$&$0.288$\tabularnewline
10148&$ 0.076$&$0.162$&$1.010$&$0.904$&$ 0.257$&$0.370$&$0.588$&$0.542$&$ 0.714$&$0.373$&$0.766$&$0.739$&$ 0.845$&$0.472$&$1.200$&$1.263$&$ 0.694$&$0.577$&$0.301$&$0.323$\tabularnewline
10149&$-0.716$&$0.161$&$1.830$&$1.804$&$-0.936$&$0.337$&$3.322$&$3.288$&$-1.083$&$0.371$&$2.091$&$1.735$&$ 3.173$&$1.487$&$0.122$&$0.133$&$ 0.221$&$0.462$&$2.008$&$2.064$\tabularnewline
10152&$ 0.562$&$0.194$&$2.241$&$1.987$&$ 5.221$&$1.494$&$0.080$&$0.089$&$ 0.714$&$0.373$&$0.625$&$0.606$&$-0.195$&$0.390$&$0.201$&$0.184$&$-0.595$&$0.444$&$2.314$&$2.363$\tabularnewline
10153&$-0.001$&$0.159$&$0.675$&$0.682$&$ 0.553$&$0.395$&$1.292$&$1.194$&$-0.199$&$0.330$&$0.880$&$0.877$&$-0.336$&$0.401$&$0.073$&$0.082$&$-0.274$&$0.422$&$0.853$&$0.854$\tabularnewline
10154&$-0.741$&$0.162$&$0.489$&$0.499$&$-1.160$&$0.358$&$1.153$&$1.092$&$-1.218$&$0.391$&$0.078$&$0.091$&$ 0.493$&$0.411$&$0.679$&$0.655$&$-0.430$&$0.428$&$0.455$&$0.450$\tabularnewline
10158&$ 1.160$&$0.269$&$1.112$&$1.220$&$ 3.026$&$0.623$&$0.865$&$0.850$&$ 0.587$&$0.361$&$0.827$&$0.784$&$-2.621$&$1.373$&$0.129$&$0.145$&$ 1.902$&$1.193$&$0.177$&$0.179$\tabularnewline
\hline
\end{longtable}}\end{landscape}

\newpage
%%%%%%%%%%%%%%%%%%%%%%%%%%%%%%%%%%%%%%%%%%%%%%%%%%%
\section{RSM-based Instrument for Measuring the Intrinsic Motivation in the First Empirical Study}
\label{sec:irt-motivation-first-study}

\subsection{Checking Assumptions}

\subsubsection*{Test of Unidimensionality}

\autoref{tab:test-unidimensionality-irt-motivation-first-study} shows the results for the test of unidimensionality in which the goodness of fit statistics indicate weak multidimensionality ($0.20 < DETECT < 0.40$) to measure the intrinsic motivation with a DETECT index of $0.227$. Essential unidimensionality ($ASSI < 0.25$ and $RATIO < 0.36$) in the data structure is indicated for the intrinsic motivation by the ASSI and RATIO indices with values of $0.221$ and $0.008$, respectively. The index of $AGFI = 0.980$ in the unidimensional CFA indicates an acceptable fit for measuring the \emph{Intrinsic Motivation}. The sub-scales of \emph{Interest/Enjoyment}, \emph{Perceived Choice}, \emph{Pressure/Tension} and \emph{Effort/Importance} have a good fit indicated by the AGFI index with values greater than $0.95$. A good fit with the unidimensional CFA is indicated by the TLI and CFI indices for all the sub-scales. The ASSI index indicates essential unidimensionality in the data structure for the sub-scales of \emph{Pressure/Tension} and \emph{Effort/Importance}, essential deviation from unidimensionality is indicated in the data structure of the scales: \emph{Interest/Enjoyment} and \emph{Perceived Choice}. The Ratio index in all the sub-scales indicate essential deviation from unidimensionality.

%latex.default(data_df, caption = paste("Goodness of fit statistics related to the test of unidimensionality",     "in RSM-based instruments", in_title), size = "small", longtable = T,     ctable = F, landscape = F, where = "!htbp", file = filename,     append = T)%
\setlongtables{\scriptsize
\begin{longtable}{lrrrrrrrr}\caption{Goodness of fit statistics related to the test of unidimensionality in the RSM-based instrument for measuring the intrinsic motivation in the first empirical study}
\tabularnewline
\hline\hline
\multicolumn{1}{l}{}&\multicolumn{1}{c}{df}&\multicolumn{1}{c}{chisq}&\multicolumn{1}{c}{AGFI}&\multicolumn{1}{c}{TLI}&\multicolumn{1}{c}{CFI}&\multicolumn{1}{c}{DETECT}&\multicolumn{1}{c}{ASSI}&\multicolumn{1}{c}{RATIO}\tabularnewline
\hline
\endfirsthead\caption[]{\em (continued)} \tabularnewline
\hline
\multicolumn{1}{l}{}&\multicolumn{1}{c}{df}&\multicolumn{1}{c}{chisq}&\multicolumn{1}{c}{AGFI}&\multicolumn{1}{c}{TLI}&\multicolumn{1}{c}{CFI}&\multicolumn{1}{c}{DETECT}&\multicolumn{1}{c}{ASSI}&\multicolumn{1}{c}{RATIO}\tabularnewline
\hline
\endhead
\hline
\multicolumn{9}{r}{\tiny df: degree of freedom; AGFI: Adjusted Goodness of Fit Index; CFI: Comparative Fit Index; TLI: Tucker-Lewis Index;}
\endfoot
\label{tab:test-unidimensionality-irt-motivation-first-study}
Intrinsic Motivation&$16.260$&$44.744$&$0.980$&$0.649$&$0.456$&$ 0.227$&$0.221$&$0.008$\tabularnewline
Interest/Enjoyment&$ 2.040$&$ 2.245$&$0.996$&$0.993$&$0.994$&$ 8.710$&$0.467$&$0.553$\tabularnewline
Perceived Choice&$ 2.848$&$ 3.640$&$0.997$&$0.983$&$0.975$&$12.694$&$0.400$&$0.605$\tabularnewline
Pressure/Tension&$ 1.676$&$ 2.886$&$0.979$&$0.924$&$0.925$&$ 5.922$&$0.333$&$0.480$\tabularnewline
Effort/Importance&$ 0.000$&$ 0.000$&$1.000$&$1.000$&$1.000$&$13.237$&$0.333$&$0.564$\tabularnewline
\hline
\end{longtable}}

\subsubsection*{Test of Local Independence}

Results from the test of local independence in the RSM-based instrument for measuring the intrinsic motivation in the first empirical study are summarized in \autoref{tab:test-local-independence-irt-motivation-first-study}. Although the null condition of local independence are rejected in the sub-scale of \emph{Interest/Enjoyment} and \emph{Pressure/Tension}, their Standardized Root Mean Squared Residual (SRMSR) indicates a good fit ($< 0.10$) for the sub-scale of \emph{Pressure/Tension}, and and acceptable fit ($0.10$s) for the \emph{Interest/Enjoyment}, \emph{Perceived Choice} and \emph{Effort/Importance}.

%latex.default(data_df, caption = paste("Q3 statistics related to the test of local independence",     "in the RSM-based instrument", in_title), size = "small",     longtable = T, ctable = F, landscape = F, where = "!htbp",     file = filename, append = T)%
\setlongtables{\scriptsize
\begin{longtable}{lrrrrr}\caption{Item residual correlation statistics related to the test of local independence in the RSM-based instrument for measuring the intrinsic motivation in the first empirical study} \tabularnewline
\hline\hline
\multicolumn{1}{l}{}&\multicolumn{1}{c}{max.chisq}&\multicolumn{1}{c}{maxaQ3}&\multicolumn{1}{c}{MADaQ3}&\multicolumn{1}{c}{SRMSR}&\multicolumn{1}{c}{p.value}\tabularnewline
\hline
\endfirsthead\caption[]{\em (continued)} \tabularnewline
\hline
\multicolumn{1}{l}{}&\multicolumn{1}{c}{max.chisq}&\multicolumn{1}{c}{maxaQ3}&\multicolumn{1}{c}{MADaQ3}&\multicolumn{1}{c}{SRMSR}&\multicolumn{1}{c}{p.value}\tabularnewline
\hline
\endhead
\hline
\multicolumn{6}{r}{\tiny aQ3: adjusted correlation of item residuals; maxaQ3: maximum aQ3;}\tabularnewline
\multicolumn{6}{r}{\tiny MADaQ3: Median Absolute Deviation of aQ3;}
\endfoot
\label{tab:test-local-independence-irt-motivation-first-study}
Intrinsic Motivation&$147.064$&$0.616$&$0.190$&$0.225$&$0.000$\tabularnewline
Interest/Enjoyment&$519.944$&$0.495$&$0.200$&$0.107$&$0.003$\tabularnewline
Perceived Choice&$ 67.919$&$0.355$&$0.149$&$0.106$&$0.073$\tabularnewline
Pressure/Tension&$ 42.092$&$0.397$&$0.159$&$0.099$&$0.017$\tabularnewline
Effort/Importance&$ 49.082$&$0.057$&$0.038$&$0.015$&$1.000$\tabularnewline
\hline
\end{longtable}}
%\begin{flushright}{\tiny Q3: correlation of item residuals; maxaQ3: maximum adjusted correlation Q3; MADaQ3: Median Absolute Deviation in the adjusted Q3; }\end{flushright}

\subsubsection*{Test of Monotonicity}

\autoref{tab:test-monotonicity-irt-motivation-first-study} summarizes the test of monotonicity in the RSM-based instrument for measuring the intrinsic motivation in the first empirical study. These results indicates that there are no one violation of monotonicity in the items at the significance level $\alpha = 0.05$.

%latex.default(data_df, caption = paste("Summary of the violations of monotonicity",     "in the RSM-based instrument", in_title), size = "small",     longtable = T, ctable = F, landscape = F, where = "!htbp",     file = filename, append = T)%
\setlongtables{\scriptsize
\begin{longtable}{lrrrrrrrrrr}\caption{Test of monotonicity in the RSM-based instrument for measuring the intrinsic motivation in the first empirical study} \tabularnewline
\hline\hline
\multicolumn{1}{l}{}&\multicolumn{1}{c}{ItemH}&\multicolumn{1}{c}{ac}&\multicolumn{1}{c}{vi}&\multicolumn{1}{c}{vi/ac}&\multicolumn{1}{c}{maxvi}&\multicolumn{1}{c}{sum}&\multicolumn{1}{c}{sum/ac}&\multicolumn{1}{c}{zmax}&\multicolumn{1}{c}{zsig}&\multicolumn{1}{c}{crit}\tabularnewline
\hline
\endfirsthead\caption[]{\em (continued)} \tabularnewline
\hline
\multicolumn{1}{l}{}&\multicolumn{1}{c}{ItemH}&\multicolumn{1}{c}{ac}&\multicolumn{1}{c}{vi}&\multicolumn{1}{c}{vi/ac}&\multicolumn{1}{c}{maxvi}&\multicolumn{1}{c}{sum}&\multicolumn{1}{c}{sum/ac}&\multicolumn{1}{c}{zmax}&\multicolumn{1}{c}{zsig}&\multicolumn{1}{c}{crit}\tabularnewline
\hline
\endhead
\hline
\multicolumn{11}{r}{\tiny vi: number of violations; vi/ac: proportion of active pairs; maxvi: maximum violations;}\tabularnewline
\multicolumn{11}{r}{\tiny sum: sum of all violations; zmax: maximum z-value; zsig: number of significant z-values; crit: critical value}
\endfoot
\label{tab:test-monotonicity-irt-motivation-first-study}
Intrinsic Motivation.Item22IE&$0.38$&$0$&$0$&$$&$0.00$&$0.00$&$$&$0$&$0$&$ 0$\tabularnewline
Intrinsic Motivation.Item09IE&$0.47$&$0$&$0$&$$&$0.00$&$0.00$&$$&$0$&$0$&$ 0$\tabularnewline
Intrinsic Motivation.Item12IE&$0.43$&$0$&$0$&$$&$0.00$&$0.00$&$$&$0$&$0$&$ 0$\tabularnewline
Intrinsic Motivation.Item24IE&$0.28$&$6$&$1$&$0.00$&$0.00$&$0.00$&$0.01$&$0$&$0$&$ 0$\tabularnewline
Intrinsic Motivation.Item21IE&$0.39$&$0$&$0$&$$&$0.00$&$0.00$&$$&$0$&$0$&$ 0$\tabularnewline
Intrinsic Motivation.Item01IE&$0.36$&$4$&$0$&$0.00$&$0.00$&$0.00$&$0.00$&$0$&$0$&$ 0$\tabularnewline
Intrinsic Motivation.Item17PC&$0.44$&$3$&$0$&$0.00$&$0.00$&$0.00$&$0.00$&$0$&$0$&$ 0$\tabularnewline
Intrinsic Motivation.Item15PC&$0.36$&$0$&$0$&$$&$0.00$&$0.00$&$$&$0$&$0$&$ 0$\tabularnewline
Intrinsic Motivation.Item06PC&$0.26$&$0$&$0$&$$&$0.00$&$0.00$&$$&$0$&$0$&$ 0$\tabularnewline
Intrinsic Motivation.Item02PC&$0.37$&$0$&$0$&$$&$0.00$&$0.00$&$$&$0$&$0$&$ 0$\tabularnewline
Intrinsic Motivation.Item08PC&$0.41$&$0$&$0$&$$&$0.00$&$0.00$&$$&$0$&$0$&$ 0$\tabularnewline
Intrinsic Motivation.Item16PT&$0.07$&$0$&$0$&$$&$0.00$&$0.00$&$$&$0$&$0$&$ 0$\tabularnewline
Intrinsic Motivation.Item14PT&$0.28$&$2$&$0$&$0.00$&$0.00$&$0.00$&$0.00$&$0$&$0$&$ 0$\tabularnewline
Intrinsic Motivation.Item18PT&$0.36$&$0$&$0$&$$&$0.00$&$0.00$&$$&$0$&$0$&$ 0$\tabularnewline
Intrinsic Motivation.Item13EI&$0.10$&$0$&$0$&$$&$0.00$&$0.00$&$$&$0$&$0$&$ 0$\tabularnewline
Intrinsic Motivation.Item03EI&$0.11$&$5$&$1$&$0.00$&$0.00$&$0.00$&$0.00$&$0$&$0$&$ 0$\tabularnewline
Intrinsic Motivation.Item07EI&$0.12$&$0$&$0$&$$&$0.00$&$0.00$&$$&$0$&$0$&$ 0$\tabularnewline
Interest/Enjoyment.Item22IE&$0.59$&$0$&$0$&$$&$0.00$&$0.00$&$$&$0$&$0$&$ 0$\tabularnewline
Interest/Enjoyment.Item09IE&$0.66$&$0$&$0$&$$&$0.00$&$0.00$&$$&$0$&$0$&$ 0$\tabularnewline
Interest/Enjoyment.Item12IE&$0.69$&$3$&$0$&$0.00$&$0.00$&$0.00$&$0.00$&$0$&$0$&$ 0$\tabularnewline
Interest/Enjoyment.Item24IE&$0.57$&$5$&$0$&$0.00$&$0.00$&$0.00$&$0.00$&$0$&$0$&$ 0$\tabularnewline
Interest/Enjoyment.Item21IE&$0.59$&$0$&$0$&$$&$0.00$&$0.00$&$$&$0$&$0$&$ 0$\tabularnewline
Interest/Enjoyment.Item01IE&$0.57$&$0$&$0$&$$&$0.00$&$0.00$&$$&$0$&$0$&$ 0$\tabularnewline
Perceived Choice.Item17PC&$0.69$&$0$&$0$&$$&$0.00$&$0.00$&$$&$0$&$0$&$ 0$\tabularnewline
Perceived Choice.Item15PC&$0.63$&$0$&$0$&$$&$0.00$&$0.00$&$$&$0$&$0$&$ 0$\tabularnewline
Perceived Choice.Item06PC&$0.53$&$0$&$0$&$$&$0.00$&$0.00$&$$&$0$&$0$&$ 0$\tabularnewline
Perceived Choice.Item02PC&$0.59$&$3$&$0$&$0.00$&$0.00$&$0.00$&$0.00$&$0$&$0$&$ 0$\tabularnewline
Perceived Choice.Item08PC&$0.62$&$0$&$0$&$$&$0.00$&$0.00$&$$&$0$&$0$&$ 0$\tabularnewline
Pressure/Tension.Item16PT&$0.63$&$0$&$0$&$$&$0.00$&$0.00$&$$&$0$&$0$&$ 0$\tabularnewline
Pressure/Tension.Item14PT&$0.61$&$0$&$0$&$$&$0.00$&$0.00$&$$&$0$&$0$&$ 0$\tabularnewline
Pressure/Tension.Item18PT&$0.54$&$0$&$0$&$$&$0.00$&$0.00$&$$&$0$&$0$&$ 0$\tabularnewline
Pressure/Tension.Item11PT&$0.56$&$0$&$0$&$$&$0.00$&$0.00$&$$&$0$&$0$&$ 0$\tabularnewline
Effort/Importance.Item13EI&$0.44$&$0$&$0$&$$&$0.00$&$0.00$&$$&$0$&$0$&$ 0$\tabularnewline
Effort/Importance.Item03EI&$0.48$&$0$&$0$&$$&$0.00$&$0.00$&$$&$0$&$0$&$ 0$\tabularnewline
Effort/Importance.Item07EI&$0.47$&$0$&$0$&$$&$0.00$&$0.00$&$$&$0$&$0$&$ 0$\tabularnewline
\hline
\end{longtable}}
%\begin{flushright}{\tiny vi: numer of violations; vi/ac: proportion of active pairs; maxvi: maximum violations; sum: sum of all violations; zmax: maximum z-value; zsig: number of significant z-values; crit: Critical value }\end{flushright}

\subsection{Item Parameters}

\autoref{tab:item-parameters-interest-enjoyment-first-study} shows the estimated parameters for the RSM-based instrument used to measure the \emph{Interest/Enjoyment} in the first empirical study. These parameters had been calculated using the MML method \cite{BockAitkin1981}, so that the value in row \aspas{B.Cat$x$} and column \aspas{$i$} is the item slope $b_{i,x}$ of item $i$ in the category \aspas{$x$}, and the value in the row \aspas{AXsi.Cat$x$} and column \aspas{$i$} is the item intercept $a_{i,x}\xi$ of item $i$ in the category \aspas{$x$}. According to the Infit/Outfit statistics of items, no one mean-square value is greater than $2.0$ indicating that the measurement system of \emph{Interest/Enjoyment} is not distorted or degraded by the items.

%latex.default(estimated_params_df, caption = paste("Estimated parameters in the RSM-based instrument",     "for measuring the", lname), size = "small", longtable = T,     ctable = F, landscape = F, where = "!htbp", file = filename,     append = T)%
\setlongtables{\scriptsize
\begin{longtable}{lrrrrrr}\caption{Estimated parameters in the RSM-based instrument for measuring the interest/enjoyment in the first empirical study} \tabularnewline
\hline\hline
\multicolumn{1}{l}{}&\multicolumn{1}{c}{Item01IE}&\multicolumn{1}{c}{Item09IE}&\multicolumn{1}{c}{Item12IE}&\multicolumn{1}{c}{Item21IE}&\multicolumn{1}{c}{Item22IE}&\multicolumn{1}{c}{Item24IE}\tabularnewline
\hline
\endfirsthead\caption[]{\em (continued)} \tabularnewline
\hline
\multicolumn{1}{l}{}&\multicolumn{1}{c}{Item01IE}&\multicolumn{1}{c}{Item09IE}&\multicolumn{1}{c}{Item12IE}&\multicolumn{1}{c}{Item21IE}&\multicolumn{1}{c}{Item22IE}&\multicolumn{1}{c}{Item24IE}\tabularnewline
\hline
\endhead
\hline
\endfoot
\label{tab:item-parameters-interest-enjoyment-first-study}
xsi.item&$-0.353$&$-0.989$&$-0.529$&$-0.661$&$-0.661$&$-0.307$\tabularnewline
B.Cat0&$ 0.000$&$ 0.000$&$ 0.000$&$ 0.000$&$ 0.000$&$ 0.000$\tabularnewline
B.Cat1&$ 1.000$&$ 1.000$&$ 1.000$&$ 1.000$&$ 1.000$&$ 1.000$\tabularnewline
B.Cat2&$ 2.000$&$ 2.000$&$ 2.000$&$ 2.000$&$ 2.000$&$ 2.000$\tabularnewline
B.Cat3&$ 3.000$&$ 3.000$&$ 3.000$&$ 3.000$&$ 3.000$&$ 3.000$\tabularnewline
B.Cat4&$ 4.000$&$ 4.000$&$ 4.000$&$ 4.000$&$ 4.000$&$ 4.000$\tabularnewline
B.Cat5&$ 5.000$&$ 5.000$&$ 5.000$&$ 5.000$&$ 5.000$&$ 5.000$\tabularnewline
B.Cat6&$ 6.000$&$ 6.000$&$ 6.000$&$ 6.000$&$ 6.000$&$ 6.000$\tabularnewline
AXsi.Cat0&$ 0.000$&$ 0.000$&$ 0.000$&$ 0.000$&$ 0.000$&$ 0.000$\tabularnewline
AXsi.Cat1&$ 1.419$&$ 2.055$&$ 1.595$&$ 1.727$&$ 1.727$&$ 1.373$\tabularnewline
AXsi.Cat2&$ 2.794$&$ 4.066$&$ 3.145$&$ 3.408$&$ 3.408$&$ 2.701$\tabularnewline
AXsi.Cat3&$ 4.736$&$ 6.644$&$ 5.262$&$ 5.657$&$ 5.657$&$ 4.595$\tabularnewline
AXsi.Cat4&$ 4.182$&$ 6.726$&$ 4.884$&$ 5.411$&$ 5.411$&$ 3.995$\tabularnewline
AXsi.Cat5&$ 3.581$&$ 6.762$&$ 4.458$&$ 5.117$&$ 5.117$&$ 3.348$\tabularnewline
AXsi.Cat6&$ 2.121$&$ 5.937$&$ 3.173$&$ 3.963$&$ 3.963$&$ 1.840$\tabularnewline
\hline
Outfit&$ 1.044$&$ 0.756$&$ 0.625$&$ 0.996$&$ 1.003$&$ 1.402$\tabularnewline
Infit&$ 1.039$&$ 0.775$&$ 0.640$&$ 1.106$&$ 1.101$&$ 1.427$\tabularnewline
\hline
\end{longtable}}

\autoref{tab:item-parameters-perceived-choice-first-study} shows the estimated parameters for the measurement instrument of \emph{Perceived Choice} in which the Infit/Outfit statistics of items indicate that no one item distorts or degrades the measurement system with mean-square greater than $2.0$.

\newpage
%latex.default(estimated_params_df, caption = paste("Estimated parameters in the RSM-based instrument",     "for measuring the", lname), size = "small", longtable = T,     ctable = F, landscape = F, where = "!htbp", file = filename,     append = T)%
\setlongtables{\scriptsize
\begin{longtable}{lrrrrr}\caption{Estimated parameters in the RSM-based instrument for measuring the perceived choice in the first empirical study} \tabularnewline
\hline\hline
\multicolumn{1}{l}{}&\multicolumn{1}{c}{Item02PC}&\multicolumn{1}{c}{Item06PC}&\multicolumn{1}{c}{Item08PC}&\multicolumn{1}{c}{Item15PC}&\multicolumn{1}{c}{Item17PC}\tabularnewline
\hline
\endfirsthead\caption[]{\em (continued)} \tabularnewline
\hline
\multicolumn{1}{l}{}&\multicolumn{1}{c}{Item02PC}&\multicolumn{1}{c}{Item06PC}&\multicolumn{1}{c}{Item08PC}&\multicolumn{1}{c}{Item15PC}&\multicolumn{1}{c}{Item17PC}\tabularnewline
\hline
\endhead
\hline
\endfoot
\label{tab:item-parameters-perceived-choice-first-study}
xsi.item&$-0.541$&$-0.688$&$-1.117$&$-0.292$&$-1.284$\tabularnewline
B.Cat0&$ 0.000$&$ 0.000$&$ 0.000$&$ 0.000$&$ 0.000$\tabularnewline
B.Cat1&$ 1.000$&$ 1.000$&$ 1.000$&$ 1.000$&$ 1.000$\tabularnewline
B.Cat2&$ 2.000$&$ 2.000$&$ 2.000$&$ 2.000$&$ 2.000$\tabularnewline
B.Cat3&$ 3.000$&$ 3.000$&$ 3.000$&$ 3.000$&$ 3.000$\tabularnewline
B.Cat4&$ 4.000$&$ 4.000$&$ 4.000$&$ 4.000$&$ 4.000$\tabularnewline
B.Cat5&$ 5.000$&$ 5.000$&$ 5.000$&$ 5.000$&$ 5.000$\tabularnewline
B.Cat6&$ 6.000$&$ 6.000$&$ 6.000$&$ 6.000$&$ 6.000$\tabularnewline
AXsi.Cat0&$ 0.000$&$ 0.000$&$ 0.000$&$ 0.000$&$ 0.000$\tabularnewline
AXsi.Cat1&$ 1.607$&$ 1.754$&$ 2.183$&$ 1.358$&$ 2.350$\tabularnewline
AXsi.Cat2&$ 2.434$&$ 2.729$&$ 3.587$&$ 1.937$&$ 3.920$\tabularnewline
AXsi.Cat3&$ 3.944$&$ 4.386$&$ 5.674$&$ 3.199$&$ 6.174$\tabularnewline
AXsi.Cat4&$ 3.608$&$ 4.197$&$ 5.914$&$ 2.614$&$ 6.581$\tabularnewline
AXsi.Cat5&$ 3.659$&$ 4.396$&$ 6.542$&$ 2.417$&$ 7.375$\tabularnewline
AXsi.Cat6&$ 3.245$&$ 4.129$&$ 6.704$&$ 1.754$&$ 7.704$\tabularnewline
\hline
Outfit&$ 1.046$&$ 1.396$&$ 0.895$&$ 1.013$&$ 0.619$\tabularnewline
Infit&$ 1.090$&$ 1.313$&$ 0.835$&$ 1.071$&$ 0.701$\tabularnewline
\hline
\end{longtable}}


\autoref{tab:item-parameters-pressure-tension-first-study} shows the estimated parameters for the measurement instrument of \emph{Pressure/Tension} in the first empirical study in which the Infit/Outfit statistics of items indicate that no one item distorts or degrades the measurement system with mean-square greater than $2.0$.

%latex.default(estimated_params_df, caption = paste("Estimated parameters in the RSM-based instrument",     "for measuring the", lname), size = "small", longtable = T,     ctable = F, landscape = F, where = "!htbp", file = filename,     append = T)%
\setlongtables{\scriptsize
\begin{longtable}{lrrrr}\caption{Estimated parameters in the RSM-based instrument for measuring the pressure/tension in the first empirical study} \tabularnewline
\hline\hline
\multicolumn{1}{l}{}&\multicolumn{1}{c}{Item11PT}&\multicolumn{1}{c}{Item14PT}&\multicolumn{1}{c}{Item16PT}&\multicolumn{1}{c}{Item18PT}\tabularnewline
\hline
\endfirsthead\caption[]{\em (continued)} \tabularnewline
\hline
\multicolumn{1}{l}{}&\multicolumn{1}{c}{Item11PT}&\multicolumn{1}{c}{Item14PT}&\multicolumn{1}{c}{Item16PT}&\multicolumn{1}{c}{Item18PT}\tabularnewline
\hline
\endhead
\hline
\endfoot
\label{tab:item-parameters-pressure-tension-first-study}
xsi.item&$ 1.344$&$ 2.034$&$ 1.310$&$ 1.765$\tabularnewline
B.Cat0&$ 0.000$&$ 0.000$&$ 0.000$&$ 0.000$\tabularnewline
B.Cat1&$ 1.000$&$ 1.000$&$ 1.000$&$ 1.000$\tabularnewline
B.Cat2&$ 2.000$&$ 2.000$&$ 2.000$&$ 2.000$\tabularnewline
B.Cat3&$ 3.000$&$ 3.000$&$ 3.000$&$ 3.000$\tabularnewline
B.Cat4&$ 4.000$&$ 4.000$&$ 4.000$&$ 4.000$\tabularnewline
B.Cat5&$ 5.000$&$ 0.000$&$ 0.000$&$ 0.000$\tabularnewline
B.Cat6&$ 6.000$&$ 0.000$&$ 0.000$&$ 0.000$\tabularnewline
AXsi.Cat0&$ 0.000$&$ 0.000$&$ 0.000$&$ 0.000$\tabularnewline
AXsi.Cat1&$-0.810$&$-1.566$&$-0.843$&$-1.298$\tabularnewline
AXsi.Cat2&$-1.985$&$-3.498$&$-2.051$&$-2.960$\tabularnewline
AXsi.Cat3&$-2.780$&$-5.050$&$-2.880$&$-4.243$\tabularnewline
AXsi.Cat4&$-5.108$&$-8.135$&$-5.241$&$-7.059$\tabularnewline
AXsi.Cat5&$-6.783$&$$&$$&$$\tabularnewline
AXsi.Cat6&$-8.065$&$$&$$&$$\tabularnewline
\hline
Outfit&$ 0.998$&$ 0.879$&$ 0.892$&$ 1.110$\tabularnewline
Infit&$ 1.276$&$ 0.827$&$ 0.922$&$ 1.111$\tabularnewline
\hline
\end{longtable}}

\autoref{tab:item-parameters-effort-importance-first-study} shows the estimated parameters for the measurement instrument of \emph{Effort/Importance} in which the Infit/Outfit statistics of items indicate that no one item distorts or degrades the measurement system with mean-square greater than $2.0$.

%latex.default(estimated_params_df, caption = paste("Estimated parameters in the RSM-based instrument",     "for measuring the", lname), size = "small", longtable = T,     ctable = F, landscape = F, where = "!htbp", file = filename,     append = T)%
\setlongtables{\scriptsize
\begin{longtable}{lrrr}\caption{Estimated parameters in the RSM-based instrument for measuring the effort/importance in the first empirical study} \tabularnewline
\hline\hline
\multicolumn{1}{l}{}&\multicolumn{1}{c}{Item03EI}&\multicolumn{1}{c}{Item07EI}&\multicolumn{1}{c}{Item13EI}\tabularnewline
\hline
\endfirsthead\caption[]{\em (continued)} \tabularnewline
\hline
\multicolumn{1}{l}{}&\multicolumn{1}{c}{Item03EI}&\multicolumn{1}{c}{Item07EI}&\multicolumn{1}{c}{Item13EI}\tabularnewline
\hline
\endhead
\hline
\endfoot
\label{tab:item-parameters-effort-importance-first-study}
xsi.item&$-1.543$&$-1.880$&$-2.684$\tabularnewline
B.Cat0&$ 0.000$&$ 0.000$&$ 0.000$\tabularnewline
B.Cat1&$ 1.000$&$ 1.000$&$ 1.000$\tabularnewline
B.Cat2&$ 2.000$&$ 2.000$&$ 2.000$\tabularnewline
B.Cat3&$ 3.000$&$ 3.000$&$ 3.000$\tabularnewline
B.Cat4&$ 4.000$&$ 4.000$&$ 4.000$\tabularnewline
B.Cat5&$ 5.000$&$ 5.000$&$ 5.000$\tabularnewline
B.Cat6&$ 6.000$&$ 6.000$&$ 6.000$\tabularnewline
AXsi.Cat0&$ 0.000$&$ 0.000$&$ 0.000$\tabularnewline
AXsi.Cat1&$ 6.967$&$ 7.305$&$ 8.108$\tabularnewline
AXsi.Cat2&$ 8.636$&$ 9.312$&$10.918$\tabularnewline
AXsi.Cat3&$ 9.827$&$10.840$&$13.250$\tabularnewline
AXsi.Cat4&$10.075$&$11.426$&$14.639$\tabularnewline
AXsi.Cat5&$10.097$&$11.785$&$15.802$\tabularnewline
AXsi.Cat6&$ 9.256$&$11.282$&$16.102$\tabularnewline
\hline
Outfit&$ 0.861$&$ 1.087$&$ 1.070$\tabularnewline
Infit&$ 0.821$&$ 1.174$&$ 1.162$\tabularnewline
\hline
\end{longtable}}


\subsection{Intrinsic Motivation as Latent Trait Estimates}

\autoref{tab:intrinsic-motivation-estimates-first-study} shows the latent trait estimates by the RSM-based instrument for measuring the \emph{Intrinsic motivation} in the first empirical study.

%latex.default(data_df, caption = paste("Latent trait estimates and person model fit of the RSM-based instrument",     in_title), size = "scriptsize", longtable = T, ctable = F,     landscape = T, rowlabel = "", where = "!htbp", file = filename,     append = T)%
\setlongtables\begin{landscape}{\scriptsize
\begin{longtable}{l|rrrr|rrrr|rrrr|rrrr|rrrr}\caption{Latent trait estimates and person model fit of the RSM-based instrument for measuring the intrinsic motivation in the first empirical study} \tabularnewline
\hline\hline
\multicolumn{1}{l}{}&\multicolumn{4}{|c}{Intrinsic Motivation}&\multicolumn{4}{|c}{Interest/Enjoyment}&\multicolumn{4}{|c}{Perceived Choice}&\multicolumn{4}{|c}{Pressure/Tension}&\multicolumn{4}{|c}{Effort/Importance} \tabularnewline
\multicolumn{1}{l}{UserID}&\multicolumn{1}{|c}{theta}&\multicolumn{1}{c}{error}&\multicolumn{1}{c}{Outfit}&\multicolumn{1}{c}{Infit}&\multicolumn{1}{|c}{theta}&\multicolumn{1}{c}{error}&\multicolumn{1}{c}{Outfit}&\multicolumn{1}{c}{Infit}&\multicolumn{1}{|c}{theta}&\multicolumn{1}{c}{error}&\multicolumn{1}{c}{Outfit}&\multicolumn{1}{c}{Infit}&\multicolumn{1}{|c}{theta}&\multicolumn{1}{c}{error}&\multicolumn{1}{c}{Outfit}&\multicolumn{1}{c}{Infit}&\multicolumn{1}{|c}{theta}&\multicolumn{1}{c}{error}&\multicolumn{1}{c}{Outfit}&\multicolumn{1}{c}{Infit}\tabularnewline
\hline
\endfirsthead\caption[]{\em (continued)} \tabularnewline
\hline
\multicolumn{1}{l}{}&\multicolumn{4}{|c}{Intrinsic Motivation}&\multicolumn{4}{|c}{Interest/Enjoyment}&\multicolumn{4}{|c}{Perceived Choice}&\multicolumn{4}{|c}{Pressure/Tension}&\multicolumn{4}{|c}{Effort/Importance} \tabularnewline
\multicolumn{1}{l}{UserID}&\multicolumn{1}{|c}{theta}&\multicolumn{1}{c}{error}&\multicolumn{1}{c}{Outfit}&\multicolumn{1}{c}{Infit}&\multicolumn{1}{|c}{theta}&\multicolumn{1}{c}{error}&\multicolumn{1}{c}{Outfit}&\multicolumn{1}{c}{Infit}&\multicolumn{1}{|c}{theta}&\multicolumn{1}{c}{error}&\multicolumn{1}{c}{Outfit}&\multicolumn{1}{c}{Infit}&\multicolumn{1}{|c}{theta}&\multicolumn{1}{c}{error}&\multicolumn{1}{c}{Outfit}&\multicolumn{1}{c}{Infit}&\multicolumn{1}{|c}{theta}&\multicolumn{1}{c}{error}&\multicolumn{1}{c}{Outfit}&\multicolumn{1}{c}{Infit}\tabularnewline
\hline
\endhead
\hline
\endfoot
\label{tab:intrinsic-motivation-estimates-first-study}
10169&$ 0.081$&$0.188$&$0.888$&$0.884$&$ 0.774$&$0.401$&$0.405$&$0.433$&$-0.345$&$0.341$&$1.313$&$1.394$&$ 0.822$&$0.443$&$0.896$&$1.063$&$ 0.214$&$0.594$&$0.216$&$0.194$\tabularnewline
10170&$ 0.435$&$0.216$&$0.331$&$0.341$&$ 0.339$&$0.388$&$0.339$&$0.339$&$ 0.813$&$0.497$&$0.318$&$0.249$&$-0.914$&$1.254$&$0.135$&$0.149$&$ 0.214$&$0.594$&$0.256$&$0.188$\tabularnewline
10171&$ 0.435$&$0.216$&$0.568$&$0.585$&$ 0.774$&$0.401$&$0.317$&$0.325$&$ 0.432$&$0.415$&$0.202$&$0.149$&$-0.007$&$0.714$&$0.299$&$0.321$&$-0.083$&$0.547$&$1.865$&$2.092$\tabularnewline
10172&$ 0.012$&$0.184$&$0.999$&$0.735$&$ 0.339$&$0.388$&$0.336$&$0.333$&$ 0.010$&$0.363$&$0.536$&$0.587$&$-0.914$&$1.254$&$0.135$&$0.149$&$-1.296$&$0.509$&$1.656$&$1.655$\tabularnewline
10174&$ 0.046$&$0.186$&$0.467$&$0.460$&$ 0.045$&$0.393$&$0.810$&$0.799$&$-0.565$&$0.334$&$0.044$&$0.045$&$-0.007$&$0.714$&$0.299$&$0.321$&$ 1.115$&$0.851$&$0.303$&$0.292$\tabularnewline
10175&$ 0.582$&$0.233$&$2.160$&$1.196$&$ 0.774$&$0.401$&$0.317$&$0.325$&$ 2.481$&$1.316$&$0.105$&$0.119$&$-0.914$&$1.254$&$0.135$&$0.149$&$-0.587$&$0.505$&$3.794$&$3.400$\tabularnewline
10176&$-0.179$&$0.176$&$0.651$&$0.702$&$-0.622$&$0.399$&$0.340$&$0.354$&$ 0.279$&$0.393$&$1.286$&$1.460$&$ 1.277$&$0.383$&$0.045$&$0.047$&$-0.823$&$0.499$&$0.796$&$0.836$\tabularnewline
10178&$ 0.152$&$0.192$&$0.554$&$0.392$&$ 0.194$&$0.389$&$0.058$&$0.058$&$ 0.279$&$0.393$&$0.537$&$0.513$&$ 0.990$&$0.414$&$0.669$&$0.586$&$ 0.214$&$0.594$&$0.256$&$0.188$\tabularnewline
10179&$-0.413$&$0.169$&$0.513$&$0.532$&$-0.944$&$0.376$&$0.155$&$0.162$&$-1.210$&$0.340$&$0.129$&$0.119$&$ 0.822$&$0.443$&$0.131$&$0.133$&$ 1.115$&$0.851$&$0.303$&$0.292$\tabularnewline
10181&$ 0.152$&$0.192$&$0.669$&$0.743$&$-0.447$&$0.403$&$0.984$&$0.992$&$ 0.432$&$0.415$&$0.764$&$0.654$&$-0.914$&$1.254$&$0.135$&$0.149$&$ 0.214$&$0.594$&$1.183$&$1.322$\tabularnewline
10183&$-0.209$&$0.175$&$0.441$&$0.436$&$-0.790$&$0.389$&$0.316$&$0.340$&$ 0.010$&$0.363$&$0.863$&$0.820$&$ 1.277$&$0.383$&$0.098$&$0.109$&$-0.083$&$0.547$&$0.104$&$0.101$\tabularnewline
10184&$ 0.116$&$0.190$&$0.662$&$0.648$&$-0.275$&$0.402$&$0.257$&$0.253$&$ 0.140$&$0.376$&$1.263$&$1.339$&$ 0.990$&$0.414$&$0.528$&$0.497$&$ 1.115$&$0.851$&$0.303$&$0.292$\tabularnewline
10185&$-0.385$&$0.170$&$0.159$&$0.141$&$-0.790$&$0.389$&$0.054$&$0.055$&$-0.565$&$0.334$&$0.295$&$0.282$&$ 0.990$&$0.414$&$0.383$&$0.393$&$-0.587$&$0.505$&$0.339$&$0.307$\tabularnewline
10186&$ 0.189$&$0.195$&$1.271$&$1.106$&$-0.111$&$0.398$&$1.043$&$1.019$&$ 0.605$&$0.448$&$1.257$&$1.383$&$ 1.407$&$0.376$&$0.389$&$0.301$&$ 2.256$&$1.456$&$0.164$&$0.191$\tabularnewline
10187&$ 0.826$&$0.269$&$2.906$&$1.891$&$ 3.747$&$1.435$&$0.083$&$0.086$&$ 0.432$&$0.415$&$1.176$&$1.163$&$ 1.658$&$0.376$&$1.921$&$1.937$&$ 1.115$&$0.851$&$0.303$&$0.292$\tabularnewline
10188&$-0.269$&$0.173$&$0.261$&$0.248$&$-0.944$&$0.376$&$0.205$&$0.205$&$-0.231$&$0.346$&$0.243$&$0.233$&$ 0.372$&$0.563$&$0.530$&$0.438$&$-0.823$&$0.499$&$0.101$&$0.100$\tabularnewline
10189&$-0.269$&$0.173$&$1.213$&$1.010$&$-0.790$&$0.389$&$1.412$&$1.439$&$-0.113$&$0.354$&$0.599$&$0.616$&$ 1.658$&$0.376$&$1.136$&$1.096$&$ 0.581$&$0.677$&$0.154$&$0.093$\tabularnewline
10190&$-0.117$&$0.178$&$0.697$&$0.774$&$-0.275$&$0.402$&$0.272$&$0.268$&$ 0.279$&$0.393$&$1.277$&$1.423$&$ 0.625$&$0.488$&$0.148$&$0.100$&$-1.296$&$0.509$&$0.472$&$0.498$\tabularnewline
10191&$ 0.081$&$0.188$&$1.074$&$1.161$&$ 0.774$&$0.401$&$0.821$&$0.878$&$-0.113$&$0.354$&$1.116$&$1.184$&$-0.007$&$0.714$&$0.316$&$0.348$&$-1.057$&$0.501$&$1.676$&$1.705$\tabularnewline
10192&$ 0.081$&$0.188$&$1.990$&$2.285$&$ 1.281$&$0.457$&$1.176$&$1.304$&$-0.565$&$0.334$&$3.047$&$3.084$&$-0.914$&$1.254$&$0.135$&$0.149$&$-1.057$&$0.501$&$1.495$&$1.458$\tabularnewline
10193&$-0.148$&$0.177$&$0.463$&$0.473$&$ 0.045$&$0.393$&$0.301$&$0.301$&$-0.565$&$0.334$&$0.648$&$0.671$&$-0.007$&$0.714$&$0.299$&$0.321$&$-0.823$&$0.499$&$0.356$&$0.360$\tabularnewline
10195&$ 0.116$&$0.190$&$0.477$&$0.417$&$ 0.482$&$0.389$&$0.097$&$0.098$&$-0.456$&$0.337$&$0.694$&$0.648$&$-0.914$&$1.254$&$0.135$&$0.149$&$ 0.214$&$0.594$&$0.256$&$0.188$\tabularnewline
10196&$-0.179$&$0.176$&$0.960$&$0.999$&$-0.790$&$0.389$&$0.054$&$0.055$&$-0.672$&$0.332$&$2.341$&$2.358$&$-0.914$&$1.254$&$0.135$&$0.149$&$ 0.581$&$0.677$&$0.678$&$0.730$\tabularnewline
10197&$ 0.306$&$0.204$&$1.249$&$1.259$&$ 0.045$&$0.393$&$0.562$&$0.568$&$ 1.495$&$0.741$&$0.470$&$0.516$&$-0.914$&$1.254$&$0.135$&$0.149$&$-0.587$&$0.505$&$2.915$&$3.082$\tabularnewline
10198&$ 3.433$&$1.352$&$0.030$&$0.034$&$ 3.747$&$1.435$&$0.083$&$0.086$&$ 2.481$&$1.316$&$0.105$&$0.119$&$-0.914$&$1.254$&$0.135$&$0.149$&$ 2.256$&$1.456$&$0.164$&$0.191$\tabularnewline
10199&$ 0.046$&$0.186$&$1.503$&$1.549$&$-0.790$&$0.389$&$2.383$&$2.418$&$ 2.481$&$1.316$&$0.105$&$0.119$&$ 1.277$&$0.383$&$1.793$&$1.898$&$-0.823$&$0.499$&$0.101$&$0.100$\tabularnewline
10200&$ 0.390$&$0.212$&$1.049$&$0.755$&$ 1.495$&$0.495$&$0.461$&$0.473$&$ 0.010$&$0.363$&$0.341$&$0.350$&$-0.914$&$1.254$&$0.135$&$0.149$&$-0.587$&$0.505$&$1.515$&$1.343$\tabularnewline
10201&$ 0.637$&$0.240$&$1.181$&$0.671$&$ 1.495$&$0.495$&$0.942$&$1.004$&$ 0.279$&$0.393$&$0.242$&$0.281$&$ 0.822$&$0.443$&$0.902$&$0.618$&$ 2.256$&$1.456$&$0.164$&$0.191$\tabularnewline
10202&$-0.607$&$0.166$&$1.613$&$1.457$&$-0.790$&$0.389$&$0.054$&$0.055$&$-1.210$&$0.340$&$0.914$&$0.856$&$ 4.700$&$1.542$&$0.119$&$0.147$&$ 2.256$&$1.456$&$0.164$&$0.191$\tabularnewline
10203&$ 0.081$&$0.188$&$2.427$&$2.880$&$-0.790$&$0.389$&$8.029$&$7.999$&$ 2.481$&$1.316$&$0.105$&$0.119$&$-0.007$&$0.714$&$0.316$&$0.348$&$-0.823$&$0.499$&$0.796$&$0.836$\tabularnewline
10204&$ 0.983$&$0.298$&$1.019$&$1.077$&$ 2.625$&$0.830$&$0.476$&$0.536$&$ 0.605$&$0.448$&$1.257$&$1.383$&$-0.914$&$1.254$&$0.135$&$0.149$&$ 0.214$&$0.594$&$0.498$&$0.462$\tabularnewline
10206&$ 0.081$&$0.188$&$0.802$&$0.674$&$-0.275$&$0.402$&$0.520$&$0.525$&$ 1.082$&$0.578$&$0.504$&$0.410$&$ 1.658$&$0.376$&$0.425$&$0.464$&$-0.083$&$0.547$&$0.104$&$0.101$\tabularnewline
10208&$ 0.582$&$0.233$&$1.369$&$1.071$&$ 1.495$&$0.495$&$0.379$&$0.346$&$ 0.140$&$0.376$&$0.917$&$0.940$&$ 0.625$&$0.488$&$1.162$&$1.350$&$ 2.256$&$1.456$&$0.164$&$0.191$\tabularnewline
10209&$-0.662$&$0.166$&$1.676$&$1.710$&$-1.436$&$0.337$&$0.586$&$0.597$&$-2.665$&$0.591$&$0.229$&$0.180$&$-0.914$&$1.254$&$0.135$&$0.149$&$ 2.256$&$1.456$&$0.164$&$0.191$\tabularnewline
10210&$-0.716$&$0.166$&$1.157$&$1.122$&$-1.210$&$0.352$&$1.032$&$0.992$&$-1.441$&$0.354$&$0.066$&$0.068$&$ 2.074$&$0.423$&$0.873$&$1.001$&$ 1.115$&$0.851$&$0.303$&$0.292$\tabularnewline
10211&$-0.413$&$0.169$&$1.845$&$1.928$&$ 1.281$&$0.457$&$0.565$&$0.578$&$-1.324$&$0.346$&$0.741$&$0.764$&$ 0.822$&$0.443$&$0.740$&$0.814$&$-2.515$&$0.681$&$0.463$&$0.364$\tabularnewline
10212&$ 0.435$&$0.216$&$1.137$&$0.867$&$ 1.495$&$0.495$&$0.770$&$0.812$&$ 0.432$&$0.415$&$0.713$&$0.827$&$ 1.140$&$0.395$&$0.373$&$0.451$&$ 0.214$&$0.594$&$0.216$&$0.194$\tabularnewline
10213&$-0.117$&$0.178$&$1.603$&$1.681$&$ 0.626$&$0.393$&$1.743$&$1.790$&$-0.456$&$0.337$&$1.967$&$2.002$&$-0.914$&$1.254$&$0.135$&$0.149$&$-1.820$&$0.555$&$0.539$&$0.451$\tabularnewline
10214&$-0.634$&$0.166$&$0.420$&$0.397$&$-1.540$&$0.332$&$0.482$&$0.499$&$-0.885$&$0.332$&$0.140$&$0.141$&$ 0.822$&$0.443$&$0.742$&$0.866$&$-1.296$&$0.509$&$0.617$&$0.639$\tabularnewline
10215&$ 0.531$&$0.227$&$0.983$&$0.952$&$ 0.194$&$0.389$&$0.780$&$0.781$&$ 2.481$&$1.316$&$0.105$&$0.119$&$-0.914$&$1.254$&$0.135$&$0.149$&$ 0.214$&$0.594$&$1.821$&$1.335$\tabularnewline
10216&$ 0.347$&$0.208$&$1.114$&$1.342$&$ 0.045$&$0.393$&$3.186$&$3.120$&$ 0.813$&$0.497$&$0.948$&$1.241$&$-0.007$&$0.714$&$0.866$&$0.660$&$ 0.581$&$0.677$&$0.154$&$0.093$\tabularnewline
10217&$ 0.152$&$0.192$&$0.965$&$1.090$&$-0.944$&$0.376$&$0.591$&$0.609$&$ 1.495$&$0.741$&$0.385$&$0.451$&$-0.914$&$1.254$&$0.135$&$0.149$&$-0.083$&$0.547$&$1.077$&$1.102$\tabularnewline
10218&$-0.298$&$0.172$&$0.812$&$0.794$&$-1.326$&$0.343$&$0.387$&$0.387$&$ 0.010$&$0.363$&$0.740$&$0.683$&$ 1.658$&$0.376$&$0.227$&$0.198$&$ 0.581$&$0.677$&$0.632$&$0.507$\tabularnewline
10219&$-0.385$&$0.170$&$0.454$&$0.427$&$-1.083$&$0.364$&$0.171$&$0.174$&$-0.456$&$0.337$&$0.676$&$0.666$&$-0.914$&$1.254$&$0.135$&$0.149$&$-1.296$&$0.509$&$0.275$&$0.263$\tabularnewline
10220&$ 0.531$&$0.227$&$1.140$&$1.367$&$ 0.626$&$0.393$&$1.926$&$1.943$&$ 0.813$&$0.497$&$1.707$&$1.863$&$-0.914$&$1.254$&$0.135$&$0.149$&$ 0.214$&$0.594$&$1.183$&$1.322$\tabularnewline
10221&$ 0.116$&$0.190$&$0.353$&$0.320$&$-0.275$&$0.402$&$0.110$&$0.109$&$ 0.010$&$0.363$&$0.579$&$0.523$&$-0.914$&$1.254$&$0.135$&$0.149$&$ 0.581$&$0.677$&$0.154$&$0.093$\tabularnewline
10222&$-1.430$&$0.190$&$0.779$&$0.811$&$-4.054$&$1.200$&$0.092$&$0.099$&$-2.382$&$0.505$&$0.215$&$0.148$&$ 1.787$&$0.385$&$0.258$&$0.195$&$-0.823$&$0.499$&$0.101$&$0.100$\tabularnewline
10223&$ 0.637$&$0.240$&$0.738$&$0.466$&$ 1.495$&$0.495$&$0.346$&$0.303$&$ 0.813$&$0.497$&$0.164$&$0.108$&$-0.007$&$0.714$&$0.316$&$0.348$&$-0.344$&$0.520$&$0.845$&$0.793$\tabularnewline
10224&$-0.179$&$0.176$&$1.965$&$2.040$&$ 1.757$&$0.553$&$0.376$&$0.354$&$-1.564$&$0.364$&$1.337$&$1.314$&$ 0.372$&$0.563$&$1.401$&$1.249$&$-0.823$&$0.499$&$0.101$&$0.100$\tabularnewline
10226&$-0.413$&$0.169$&$0.841$&$0.835$&$-0.944$&$0.376$&$0.610$&$0.685$&$-1.324$&$0.346$&$0.251$&$0.256$&$ 0.372$&$0.563$&$1.401$&$1.249$&$ 1.115$&$0.851$&$0.303$&$0.292$\tabularnewline
10227&$-0.825$&$0.167$&$0.743$&$0.729$&$-1.643$&$0.331$&$1.498$&$1.501$&$-0.672$&$0.332$&$0.324$&$0.321$&$ 2.074$&$0.423$&$0.205$&$0.235$&$-1.296$&$0.509$&$0.326$&$0.334$\tabularnewline
10228&$-0.239$&$0.174$&$0.522$&$0.406$&$-0.790$&$0.389$&$0.275$&$0.279$&$-0.345$&$0.341$&$0.448$&$0.464$&$ 0.990$&$0.414$&$1.138$&$0.983$&$ 0.214$&$0.594$&$0.256$&$0.188$\tabularnewline
10230&$-0.441$&$0.169$&$1.544$&$1.607$&$-1.744$&$0.331$&$1.999$&$2.063$&$ 0.010$&$0.363$&$2.000$&$1.786$&$ 1.533$&$0.373$&$1.534$&$1.752$&$-0.587$&$0.505$&$1.301$&$1.300$\tabularnewline
10231&$ 1.664$&$0.476$&$0.874$&$0.975$&$ 3.747$&$1.435$&$0.083$&$0.086$&$ 1.495$&$0.741$&$0.273$&$0.309$&$-0.914$&$1.254$&$0.135$&$0.149$&$ 0.581$&$0.677$&$0.678$&$0.730$\tabularnewline
10232&$ 0.189$&$0.195$&$0.477$&$0.528$&$ 0.626$&$0.393$&$1.330$&$1.273$&$-0.113$&$0.354$&$0.313$&$0.345$&$ 1.407$&$0.376$&$1.445$&$1.785$&$ 0.214$&$0.594$&$0.498$&$0.462$\tabularnewline
10233&$-0.689$&$0.166$&$0.712$&$0.715$&$-1.950$&$0.341$&$0.518$&$0.527$&$-0.992$&$0.333$&$0.935$&$0.941$&$-0.007$&$0.714$&$0.602$&$0.575$&$-0.587$&$0.505$&$0.042$&$0.043$\tabularnewline
10234&$-0.580$&$0.167$&$0.808$&$0.818$&$-1.326$&$0.343$&$1.571$&$1.543$&$-0.885$&$0.332$&$0.978$&$0.983$&$-0.007$&$0.714$&$0.866$&$0.660$&$-1.296$&$0.509$&$0.326$&$0.334$\tabularnewline
10237&$-0.021$&$0.182$&$0.792$&$0.859$&$ 0.045$&$0.393$&$0.988$&$0.961$&$ 0.010$&$0.363$&$0.999$&$1.060$&$-0.914$&$1.254$&$0.135$&$0.149$&$-1.057$&$0.501$&$1.059$&$1.029$\tabularnewline
10238&$-0.716$&$0.166$&$0.825$&$0.771$&$-0.944$&$0.376$&$0.330$&$0.302$&$-1.324$&$0.346$&$0.586$&$0.541$&$ 2.074$&$0.423$&$0.873$&$1.001$&$-0.083$&$0.547$&$0.104$&$0.101$\tabularnewline
10240&$-0.743$&$0.166$&$0.948$&$0.943$&$-1.210$&$0.352$&$0.707$&$0.686$&$-2.168$&$0.453$&$0.379$&$0.357$&$-0.914$&$1.254$&$0.135$&$0.149$&$-0.823$&$0.499$&$0.391$&$0.391$\tabularnewline
\hline
\end{longtable}}\end{landscape}


\newpage
%%%%%%%%%%%%%%%%%%%%%%%%%%%%%%%%%%%%%%%%%%%%%%%%%%%
\section{RSM-based Instrument for Measuring the Level of Motivation in the Second Empirical Study}
\label{sec:irt-motivation-second-study}

\subsection{Checking Assumptions}

\subsubsection*{Test of Unidimensionality}

\autoref{tab:test-unidimensionality-irt-motivation-second-study} shows the results for the test of unidimensionality in which the goodness of fit statistics indicate strong multidimensionality ($DETECT > 1.00$) with a DETECT index of $2.305$.
Essential unidimensionality ($ASSI < 0.25$ and $RATIO < 0.36$) in the data structure is indicated for the level of motivation by the ASSI and RATIO indices with values of $0.212$ and $0.152$, respectively.
The index of $AGFI = 0.990$ in the unidimensional CFA indicates an acceptable fit for measuring the \emph{Level of Motivation}.
The sub-scales of \emph{Attention}, \emph{Relevance}, and \emph{Satisfaction} have a good fit indicated by the AGFI index with values greater than $0.95$.
A good fit with the unidimensional CFA is indicated by the TLI and CFI indices for all the sub-scales.
The ASSI index indicates essential unidimensionality in the data structure for the sub-scale of \emph{Attention}, essential deviation from unidimensionality is indicated in the data structure of the scales: \emph{Relevance} and \emph{Satisfaction}.
The Ratio index in all the sub-scales indicate essential deviation from unidimensionality.

%latex.default(data_df, caption = paste("Goodness of fit statistics related to the test of unidimensionality",     "in RSM-based instruments", in_title), size = "small", longtable = T,     ctable = F, landscape = F, where = "!htbp", file = filename,     append = T)%
\setlongtables{\scriptsize
\begin{longtable}{lrrrrrrrr}\caption{Goodness of fit statistics related to the test of unidimensionality in the RSM-based instrument for measuring the level of motivation in the second empirical study}
\tabularnewline
\hline\hline
\multicolumn{1}{l}{}&\multicolumn{1}{c}{df}&\multicolumn{1}{c}{chisq}&\multicolumn{1}{c}{AGFI}&\multicolumn{1}{c}{TLI}&\multicolumn{1}{c}{CFI}&\multicolumn{1}{c}{DETECT}&\multicolumn{1}{c}{ASSI}&\multicolumn{1}{c}{RATIO}\tabularnewline
\hline
\endfirsthead\caption[]{\em (continued)} \tabularnewline
\hline
\multicolumn{1}{l}{}&\multicolumn{1}{c}{df}&\multicolumn{1}{c}{chisq}&\multicolumn{1}{c}{AGFI}&\multicolumn{1}{c}{TLI}&\multicolumn{1}{c}{CFI}&\multicolumn{1}{c}{DETECT}&\multicolumn{1}{c}{ASSI}&\multicolumn{1}{c}{RATIO}\tabularnewline
\hline
\endhead
\hline
\multicolumn{9}{r}{\tiny df: degree of freedom; AGFI: Adjusted Goodness of Fit Index; CFI: Comparative Fit Index; TLI: Tucker-Lewis Index;}
\endfoot
\label{tab:test-unidimensionality-irt-motivation-second-study}
Level of Motivation&$6.883$&$11.704$&$0.990$&$0.951$&$0.912$&$ 2.305$&$0.212$&$0.152$\tabularnewline
Attention&$4.632$&$ 4.865$&$0.998$&$0.998$&$0.993$&$ 5.641$&$0.067$&$0.506$\tabularnewline
Relevance&$1.427$&$ 1.133$&$0.997$&$1.018$&$1.000$&$19.151$&$0.667$&$0.669$\tabularnewline
Satisfaction&$0.000$&$ 0.000$&$1.000$&$1.000$&$1.000$&$10.099$&$0.333$&$0.495$\tabularnewline
\hline
\end{longtable}}

\subsubsection*{Test of Local Independence}

Results from the test of local independence in the RSM-based instrument for measuring the level of motivation in the second empirical study are summarized in \autoref{tab:test-local-independence-irt-motivation-second-study}.
The null condition of local independence is only rejected in the sub-scale of \emph{Relevance} but its Standardized Root Mean Squared Residual (SRMSR) indicates an acceptable fit ($0.10$s) with value of $0.120$.
The null condition of local independence are not rejected in the sub-scale of \emph{Attention} and \emph{Satisfaction}, and their SRMSRs indicates a good fit ($< 0.10$).

%latex.default(data_df, caption = paste("Q3 statistics related to the test of local independence",     "in the RSM-based instrument", in_title), size = "small",     longtable = T, ctable = F, landscape = F, where = "!htbp",     file = filename, append = T)%
\setlongtables{\scriptsize
\begin{longtable}{lrrrrr}\caption{Item residual correlation statistics related to the test of local independence in the RSM-based instrument for measuring the level of motivation in the second empirical study} \tabularnewline
\hline\hline
\multicolumn{1}{l}{}&\multicolumn{1}{c}{max.chisq}&\multicolumn{1}{c}{maxaQ3}&\multicolumn{1}{c}{MADaQ3}&\multicolumn{1}{c}{SRMSR}&\multicolumn{1}{c}{p.value}\tabularnewline
\hline
\endfirsthead\caption[]{\em (continued)} \tabularnewline
\hline
\multicolumn{1}{l}{}&\multicolumn{1}{c}{max.chisq}&\multicolumn{1}{c}{maxaQ3}&\multicolumn{1}{c}{MADaQ3}&\multicolumn{1}{c}{SRMSR}&\multicolumn{1}{c}{p.value}\tabularnewline
\hline
\endhead
\hline
\multicolumn{6}{r}{\tiny aQ3: adjusted correlation of item residuals; maxaQ3: maximum aQ3;}\tabularnewline
\multicolumn{6}{r}{\tiny MADaQ3: Median Absolute Deviation of aQ3;}
\endfoot
\label{tab:test-local-independence-irt-motivation-second-study}
Level of Motivation&$288.358$&$0.485$&$0.163$&$0.124$&$0.022$\tabularnewline
Attention&$286.339$&$0.237$&$0.134$&$0.061$&$1.000$\tabularnewline
Relevance&$ 68.696$&$0.366$&$0.151$&$0.120$&$0.040$\tabularnewline
Satisfaction&$ 54.715$&$0.237$&$0.158$&$0.074$&$0.237$\tabularnewline
\hline
\end{longtable}}
%\begin{flushright}{\tiny Q3: correlation of item residuals; maxaQ3: maximum adjusted correlation Q3; MADaQ3: Median Absolute Deviation in the adjusted Q3; }\end{flushright}

\subsubsection*{Test of Monotonicity}

\autoref{tab:test-monotonicity-irt-motivation-second-study} summarizes the test of monotonicity in the RSM-based instrument for measuring the level of motivation in the second empirical study. These results indicates that there are no one violation of monotonicity in the items at the significance level $\alpha = 0.05$.

%latex.default(data_df, caption = paste("Summary of the violations of monotonicity",     "in the RSM-based instrument", in_title), size = "small",     longtable = T, ctable = F, landscape = F, where = "!htbp",     file = filename, append = T)%
\setlongtables{\scriptsize
\begin{longtable}{lrrrrrrrrrr}\caption{Test of monotonicity in the RSM-based instrument for measuring the level of motivation in the second empirical study} \tabularnewline
\hline\hline
\multicolumn{1}{l}{}&\multicolumn{1}{c}{ItemH}&\multicolumn{1}{c}{ac}&\multicolumn{1}{c}{vi}&\multicolumn{1}{c}{vi/ac}&\multicolumn{1}{c}{maxvi}&\multicolumn{1}{c}{sum}&\multicolumn{1}{c}{sum/ac}&\multicolumn{1}{c}{zmax}&\multicolumn{1}{c}{zsig}&\multicolumn{1}{c}{crit}\tabularnewline
\hline
\endfirsthead\caption[]{\em (continued)} \tabularnewline
\hline
\multicolumn{1}{l}{}&\multicolumn{1}{c}{ItemH}&\multicolumn{1}{c}{ac}&\multicolumn{1}{c}{vi}&\multicolumn{1}{c}{vi/ac}&\multicolumn{1}{c}{maxvi}&\multicolumn{1}{c}{sum}&\multicolumn{1}{c}{sum/ac}&\multicolumn{1}{c}{zmax}&\multicolumn{1}{c}{zsig}&\multicolumn{1}{c}{crit}\tabularnewline
\hline
\endhead
\hline
\multicolumn{11}{r}{\tiny vi: number of violations; vi/ac: proportion of active pairs; maxvi: maximum violations;}\tabularnewline
\multicolumn{11}{r}{\tiny sum: sum of all violations; zmax: maximum z-value; zsig: number of significant z-values; crit: critical value}
\endfoot
\label{tab:test-monotonicity-irt-motivation-second-study}
Level of Motivation.Item12A&$0.64$&$2$&$0$&$  0$&$0$&$0$&$  0$&$0$&$0$&$0$\tabularnewline
Level of Motivation.Item19A&$0.52$&$0$&$0$&$$&$0$&$0$&$$&$0$&$0$&$0$\tabularnewline
Level of Motivation.Item04A&$0.64$&$0$&$0$&$$&$0$&$0$&$$&$0$&$0$&$0$\tabularnewline
Level of Motivation.Item20A&$0.59$&$0$&$0$&$$&$0$&$0$&$$&$0$&$0$&$0$\tabularnewline
Level of Motivation.Item16A&$0.62$&$0$&$0$&$$&$0$&$0$&$$&$0$&$0$&$0$\tabularnewline
Level of Motivation.Item01A&$0.63$&$0$&$0$&$$&$0$&$0$&$$&$0$&$0$&$0$\tabularnewline
Level of Motivation.Item15R&$0.42$&$0$&$0$&$$&$0$&$0$&$$&$0$&$0$&$0$\tabularnewline
Level of Motivation.Item21R&$0.59$&$0$&$0$&$$&$0$&$0$&$$&$0$&$0$&$0$\tabularnewline
Level of Motivation.Item10R&$0.53$&$5$&$0$&$  0$&$0$&$0$&$  0$&$0$&$0$&$0$\tabularnewline
Level of Motivation.Item13S&$0.58$&$4$&$0$&$  0$&$0$&$0$&$  0$&$0$&$0$&$0$\tabularnewline
Level of Motivation.Item14S&$0.60$&$4$&$0$&$  0$&$0$&$0$&$  0$&$0$&$0$&$0$\tabularnewline
Level of Motivation.Item17S&$0.51$&$0$&$0$&$$&$0$&$0$&$$&$0$&$0$&$0$\tabularnewline
Attention.Item12A&$0.77$&$2$&$0$&$  0$&$0$&$0$&$  0$&$0$&$0$&$0$\tabularnewline
Attention.Item19A&$0.66$&$0$&$0$&$$&$0$&$0$&$$&$0$&$0$&$0$\tabularnewline
Attention.Item04A&$0.70$&$0$&$0$&$$&$0$&$0$&$$&$0$&$0$&$0$\tabularnewline
Attention.Item20A&$0.72$&$0$&$0$&$$&$0$&$0$&$$&$0$&$0$&$0$\tabularnewline
Attention.Item16A&$0.70$&$0$&$0$&$$&$0$&$0$&$$&$0$&$0$&$0$\tabularnewline
Attention.Item01A&$0.71$&$0$&$0$&$$&$0$&$0$&$$&$0$&$0$&$0$\tabularnewline
Relevance.Item15R&$0.43$&$0$&$0$&$$&$0$&$0$&$$&$0$&$0$&$0$\tabularnewline
Relevance.Item21R&$0.53$&$0$&$0$&$$&$0$&$0$&$$&$0$&$0$&$0$\tabularnewline
Relevance.Item10R&$0.49$&$0$&$0$&$$&$0$&$0$&$$&$0$&$0$&$0$\tabularnewline
Relevance.Item08R&$0.35$&$0$&$0$&$$&$0$&$0$&$$&$0$&$0$&$0$\tabularnewline
Satisfaction.Item13S&$0.70$&$0$&$0$&$$&$0$&$0$&$$&$0$&$0$&$0$\tabularnewline
Satisfaction.Item14S&$0.66$&$0$&$0$&$$&$0$&$0$&$$&$0$&$0$&$0$\tabularnewline
Satisfaction.Item17S&$0.61$&$0$&$0$&$$&$0$&$0$&$$&$0$&$0$&$0$\tabularnewline
\hline
\end{longtable}}
%\begin{flushright}{\tiny vi: numer of violations; vi/ac: proportion of active pairs; maxvi: maximum violations; sum: sum of all violations; zmax: maximum z-value; zsig: number of significant z-values; crit: Critical value }\end{flushright}

\subsection{Item Parameters}

\autoref{tab:item-parameters-attention-second-study} shows the estimated parameters for the RSM-based instrument used to measure the \emph{Attention} in the second empirical study.
These parameters had been calculated using the MML method \cite{BockAitkin1981}, so that the value in row \aspas{B.Cat$x$} and column \aspas{$i$} is the item slope $b_{i,x}$ of item $i$ in the category \aspas{$x$}, and the value in the row \aspas{AXsi.Cat$x$} and column \aspas{$i$} is the item intercept $a_{i,x}\xi$ of item $i$ in the category \aspas{$x$}.
According to the Infit/Outfit statistics of items, no one mean-square value is greater than $2.0$ indicating that the measurement system of \emph{Attention} is not distorted or degraded by the items.

%latex.default(estimated_params_df, caption = paste("Estimated parameters in the RSM-based instrument",     "for measuring the", lname), size = "small", longtable = T,     ctable = F, landscape = F, where = "!htbp", file = filename,     append = T)%
\setlongtables{\scriptsize
\begin{longtable}{lrrrrrr}\caption{Estimated parameters in the RSM-based instrument for measuring the attention in the second empirical study} \tabularnewline
\hline\hline
\multicolumn{1}{l}{}&\multicolumn{1}{c}{Item01A}&\multicolumn{1}{c}{Item04A}&\multicolumn{1}{c}{Item12A}&\multicolumn{1}{c}{Item16A}&\multicolumn{1}{c}{Item19A}&\multicolumn{1}{c}{Item20A}\tabularnewline
\hline
\endfirsthead\caption[]{\em (continued)} \tabularnewline
\hline
\multicolumn{1}{l}{}&\multicolumn{1}{c}{Item01A}&\multicolumn{1}{c}{Item04A}&\multicolumn{1}{c}{Item12A}&\multicolumn{1}{c}{Item16A}&\multicolumn{1}{c}{Item19A}&\multicolumn{1}{c}{Item20A}\tabularnewline
\hline
\endhead
\hline
\endfoot
\label{tab:item-parameters-attention-second-study}
xsi.item&$-0.004$&$ 0.053$&$ 0.015$&$-0.290$&$ 0.034$&$ 0.090$\tabularnewline
B.Cat0&$ 0.000$&$ 0.000$&$ 0.000$&$ 0.000$&$ 0.000$&$ 0.000$\tabularnewline
B.Cat1&$ 1.000$&$ 1.000$&$ 1.000$&$ 1.000$&$ 1.000$&$ 1.000$\tabularnewline
B.Cat2&$ 2.000$&$ 2.000$&$ 2.000$&$ 2.000$&$ 2.000$&$ 2.000$\tabularnewline
B.Cat3&$ 3.000$&$ 3.000$&$ 3.000$&$ 3.000$&$ 3.000$&$ 3.000$\tabularnewline
B.Cat4&$ 4.000$&$ 4.000$&$ 4.000$&$ 4.000$&$ 4.000$&$ 4.000$\tabularnewline
B.Cat5&$ 5.000$&$ 5.000$&$ 5.000$&$ 5.000$&$ 5.000$&$ 5.000$\tabularnewline
B.Cat6&$ 6.000$&$ 6.000$&$ 6.000$&$ 6.000$&$ 6.000$&$ 6.000$\tabularnewline
AXsi.Cat0&$ 0.000$&$ 0.000$&$ 0.000$&$ 0.000$&$ 0.000$&$ 0.000$\tabularnewline
AXsi.Cat1&$ 1.688$&$ 1.632$&$ 1.670$&$ 1.974$&$ 1.651$&$ 1.594$\tabularnewline
AXsi.Cat2&$ 2.833$&$ 2.720$&$ 2.795$&$ 3.405$&$ 2.757$&$ 2.644$\tabularnewline
AXsi.Cat3&$ 4.141$&$ 3.971$&$ 4.084$&$ 4.999$&$ 4.028$&$ 3.858$\tabularnewline
AXsi.Cat4&$ 3.591$&$ 3.365$&$ 3.516$&$ 4.735$&$ 3.440$&$ 3.214$\tabularnewline
AXsi.Cat5&$ 1.928$&$ 1.645$&$ 1.833$&$ 3.357$&$ 1.739$&$ 1.456$\tabularnewline
AXsi.Cat6&$ 0.023$&$-0.317$&$-0.090$&$ 1.739$&$-0.203$&$-0.543$\tabularnewline
\hline
Outfit&$ 1.048$&$ 1.029$&$ 0.606$&$ 0.961$&$ 1.313$&$ 1.046$\tabularnewline
Infit&$ 1.119$&$ 1.075$&$ 0.630$&$ 0.995$&$ 1.425$&$ 0.863$\tabularnewline
\hline
\end{longtable}}

\autoref{tab:item-parameters-relevance-second-study} shows the estimated parameters for the measurement instrument of \emph{Relevance} in which the Infit/Outfit statistics of items indicate that no one item distorts or degrades the measurement system with mean-square greater than $2.0$.

%latex.default(estimated_params_df, caption = paste("Estimated parameters in the RSM-based instrument",     "for measuring the", lname), size = "small", longtable = T,     ctable = F, landscape = F, where = "!htbp", file = filename,     append = T)%
\setlongtables{\scriptsize
\begin{longtable}{lrrrrr}\caption{Estimated parameters in the RSM-based instrument for measuring the relevance in the second empirical study} \tabularnewline
\hline\hline
\multicolumn{1}{l}{}&\multicolumn{1}{c}{Item08R}&\multicolumn{1}{c}{Item10R}&\multicolumn{1}{c}{Item15R}&\multicolumn{1}{c}{Item21R}\tabularnewline
\hline
\endfirsthead\caption[]{\em (continued)} \tabularnewline
\hline
\multicolumn{1}{l}{}&\multicolumn{1}{c}{Item08R}&\multicolumn{1}{c}{Item10R}&\multicolumn{1}{c}{Item15R}&\multicolumn{1}{c}{Item21R}\tabularnewline
\hline
\endhead
\hline
\endfoot
\label{tab:item-parameters-relevance-second-study}
xsi.item&$-1.152$&$-0.777$&$-0.821$&$-0.766$\tabularnewline
B.Cat0&$ 0.000$&$ 0.000$&$ 0.000$&$ 0.000$\tabularnewline
B.Cat1&$ 1.000$&$ 1.000$&$ 1.000$&$ 1.000$\tabularnewline
B.Cat2&$ 2.000$&$ 2.000$&$ 2.000$&$ 2.000$\tabularnewline
B.Cat3&$ 3.000$&$ 3.000$&$ 3.000$&$ 3.000$\tabularnewline
B.Cat4&$ 4.000$&$ 4.000$&$ 4.000$&$ 4.000$\tabularnewline
B.Cat5&$ 5.000$&$ 5.000$&$ 5.000$&$ 5.000$\tabularnewline
B.Cat6&$ 6.000$&$ 6.000$&$ 6.000$&$ 6.000$\tabularnewline
AXsi.Cat0&$ 0.000$&$ 0.000$&$ 0.000$&$ 0.000$\tabularnewline
AXsi.Cat1&$ 4.435$&$ 4.060$&$ 4.104$&$ 4.049$\tabularnewline
AXsi.Cat2&$ 5.226$&$ 4.475$&$ 4.563$&$ 4.453$\tabularnewline
AXsi.Cat3&$ 6.531$&$ 5.405$&$ 5.537$&$ 5.372$\tabularnewline
AXsi.Cat4&$ 6.807$&$ 5.305$&$ 5.481$&$ 5.262$\tabularnewline
AXsi.Cat5&$ 6.965$&$ 5.089$&$ 5.309$&$ 5.034$\tabularnewline
AXsi.Cat6&$ 6.913$&$ 4.662$&$ 4.925$&$ 4.596$\tabularnewline
\hline
Outfit&$ 1.231$&$ 0.989$&$ 0.948$&$ 0.832$\tabularnewline
Infit&$ 1.209$&$ 1.020$&$ 0.902$&$ 0.849$\tabularnewline
\hline
\end{longtable}}


\autoref{tab:item-parameters-satisfaction-second-study} shows the estimated parameters for the measurement instrument of \emph{Satisfaction} in the second empirical study in which the Infit/Outfit statistics of items indicate that no one item distorts or degrades the measurement system with mean-square greater than $2.0$.

%latex.default(estimated_params_df, caption = paste("Estimated parameters in the RSM-based instrument",     "for measuring the", lname), size = "small", longtable = T,     ctable = F, landscape = F, where = "!htbp", file = filename,     append = T)%
\setlongtables{\scriptsize
\begin{longtable}{lrrr}\caption{Estimated parameters in the RSM-based instrument for measuring the satisfaction in the second empirical study} \tabularnewline
\hline\hline
\multicolumn{1}{l}{estimated}&\multicolumn{1}{c}{Item13S}&\multicolumn{1}{c}{Item14S}&\multicolumn{1}{c}{Item17S}\tabularnewline
\hline
\endfirsthead\caption[]{\em (continued)} \tabularnewline
\hline
\multicolumn{1}{l}{estimated}&\multicolumn{1}{c}{Item13S}&\multicolumn{1}{c}{Item14S}&\multicolumn{1}{c}{Item17S}\tabularnewline
\hline
\endhead
\hline
\endfoot
\label{tab:item-parameters-satisfaction-second-study}
xsi.item&$-0.415$&$-0.344$&$ 0.025$\tabularnewline
B.Cat0&$ 0.000$&$ 0.000$&$ 0.000$\tabularnewline
B.Cat1&$ 1.000$&$ 1.000$&$ 1.000$\tabularnewline
B.Cat2&$ 2.000$&$ 2.000$&$ 2.000$\tabularnewline
B.Cat3&$ 3.000$&$ 3.000$&$ 3.000$\tabularnewline
B.Cat4&$ 4.000$&$ 4.000$&$ 4.000$\tabularnewline
B.Cat5&$ 5.000$&$ 5.000$&$ 5.000$\tabularnewline
B.Cat6&$ 6.000$&$ 6.000$&$ 6.000$\tabularnewline
AXsi.Cat0&$ 0.000$&$ 0.000$&$ 0.000$\tabularnewline
AXsi.Cat1&$ 2.615$&$ 2.544$&$ 2.175$\tabularnewline
AXsi.Cat2&$ 3.805$&$ 3.662$&$ 2.924$\tabularnewline
AXsi.Cat3&$ 5.326$&$ 5.112$&$ 4.004$\tabularnewline
AXsi.Cat4&$ 5.125$&$ 4.839$&$ 3.363$\tabularnewline
AXsi.Cat5&$ 4.154$&$ 3.797$&$ 1.952$\tabularnewline
AXsi.Cat6&$ 2.492$&$ 2.064$&$-0.151$\tabularnewline
\hline
Outfit&$ 0.939$&$ 1.000$&$ 1.064$\tabularnewline
Infit&$ 0.957$&$ 1.039$&$ 1.025$\tabularnewline
\hline
\end{longtable}}

\subsection{Level of Motivation as Latent Trait Estimates}

\autoref{tab:level-motivation-estimates-second-study} shows the latent trait estimates by the RSM-based instrument for measuring the \emph{Level of Motivation} in the second empirical study.

%latex.default(data_df, caption = paste("Latent trait estimates and person model fit of the RSM-based instrument",     in_title), size = "scriptsize", longtable = T, ctable = F,     landscape = T, rowlabel = "", where = "!htbp", file = filename,     append = T)%
\setlongtables\begin{landscape}{\scriptsize
\begin{longtable}{l|rrrr|rrrr|rrrr|rrrr}\caption{Latent trait estimates and person model fit of the RSM-based instrument for measuring the level of motivation in the second empirical study} \tabularnewline
\hline\hline
\multicolumn{1}{l}{}&\multicolumn{4}{|c}{Level of Motivation}&\multicolumn{4}{|c}{Attention}&\multicolumn{4}{|c}{Relevance}&\multicolumn{4}{|c}{Satisfaction} \tabularnewline
\multicolumn{1}{l}{UserID}&\multicolumn{1}{|c}{theta}&\multicolumn{1}{c}{error}&\multicolumn{1}{c}{Outfit}&\multicolumn{1}{c}{Infit}&\multicolumn{1}{|c}{theta}&\multicolumn{1}{c}{error}&\multicolumn{1}{c}{Outfit}&\multicolumn{1}{c}{Infit}&\multicolumn{1}{|c}{theta}&\multicolumn{1}{c}{error}&\multicolumn{1}{c}{Outfit}&\multicolumn{1}{c}{Infit}&\multicolumn{1}{|c}{theta}&\multicolumn{1}{c}{error}&\multicolumn{1}{c}{Outfit}&\multicolumn{1}{c}{Infit}\tabularnewline
\hline
\endfirsthead\caption[]{\em (continued)} \tabularnewline
\hline
\multicolumn{1}{l}{}&\multicolumn{4}{|c}{Level of Motivation}&\multicolumn{4}{|c}{Attention}&\multicolumn{4}{|c}{Relevance}&\multicolumn{4}{|c}{Satisfaction} \tabularnewline
\multicolumn{1}{l}{UserID}&\multicolumn{1}{|c}{theta}&\multicolumn{1}{c}{error}&\multicolumn{1}{c}{Outfit}&\multicolumn{1}{c}{Infit}&\multicolumn{1}{|c}{theta}&\multicolumn{1}{c}{error}&\multicolumn{1}{c}{Outfit}&\multicolumn{1}{c}{Infit}&\multicolumn{1}{|c}{theta}&\multicolumn{1}{c}{error}&\multicolumn{1}{c}{Outfit}&\multicolumn{1}{c}{Infit}&\multicolumn{1}{|c}{theta}&\multicolumn{1}{c}{error}&\multicolumn{1}{c}{Outfit}&\multicolumn{1}{c}{Infit}\tabularnewline
\hline
\endhead
\hline
\endfoot
\label{tab:level-motivation-estimates-second-study}
10169&$-1.408$&$0.260$&$0.453$&$0.528$&$-1.957$&$0.419$&$0.224$&$0.269$&$-0.869$&$0.400$&$0.922$&$0.875$&$-1.982$&$0.611$&$0.042$&$0.044$\tabularnewline
10170&$ 0.226$&$0.248$&$0.641$&$0.640$&$ 0.568$&$0.435$&$0.852$&$0.854$&$ 0.006$&$0.365$&$0.808$&$0.789$&$-0.348$&$0.554$&$0.362$&$0.361$\tabularnewline
10171&$ 0.802$&$0.262$&$0.363$&$0.344$&$ 0.758$&$0.428$&$0.134$&$0.132$&$ 1.438$&$0.726$&$0.420$&$0.433$&$ 0.978$&$0.578$&$0.195$&$0.194$\tabularnewline
10172&$ 0.288$&$0.249$&$1.043$&$1.056$&$ 0.758$&$0.428$&$1.437$&$1.471$&$ 0.265$&$0.389$&$0.309$&$0.295$&$-0.645$&$0.536$&$0.893$&$0.905$\tabularnewline
10174&$-0.628$&$0.231$&$1.109$&$1.119$&$-0.952$&$0.357$&$0.786$&$0.796$&$-1.028$&$0.425$&$0.916$&$0.948$&$-0.348$&$0.554$&$2.933$&$3.002$\tabularnewline
10175&$ 1.319$&$0.294$&$1.086$&$1.139$&$ 1.764$&$0.425$&$1.141$&$1.166$&$ 2.358$&$1.265$&$0.133$&$0.136$&$ 0.978$&$0.578$&$1.191$&$1.186$\tabularnewline
10176&$-0.838$&$0.232$&$0.270$&$0.269$&$-1.070$&$0.353$&$0.022$&$0.020$&$-1.028$&$0.425$&$0.406$&$0.390$&$-0.910$&$0.524$&$0.408$&$0.412$\tabularnewline
10178&$ 0.413$&$0.252$&$0.227$&$0.226$&$ 0.758$&$0.428$&$0.164$&$0.163$&$-0.114$&$0.359$&$2.052$&$1.945$&$-0.014$&$0.570$&$0.145$&$0.142$\tabularnewline
10179&$ 0.103$&$0.246$&$0.289$&$0.284$&$-0.041$&$0.428$&$0.082$&$0.079$&$ 0.132$&$0.374$&$0.182$&$0.175$&$-0.014$&$0.570$&$1.042$&$1.040$\tabularnewline
10181&$-0.248$&$0.238$&$1.060$&$1.074$&$-0.041$&$0.428$&$2.256$&$2.269$&$-0.351$&$0.358$&$0.286$&$0.287$&$-0.645$&$0.536$&$0.691$&$0.675$\tabularnewline
10183&$-0.628$&$0.231$&$0.627$&$0.626$&$-0.952$&$0.357$&$0.855$&$0.868$&$-0.726$&$0.383$&$0.404$&$0.383$&$-0.348$&$0.554$&$0.041$&$0.043$\tabularnewline
10184&$ 0.870$&$0.265$&$0.437$&$0.449$&$ 1.436$&$0.413$&$0.571$&$0.575$&$ 0.579$&$0.441$&$0.314$&$0.280$&$ 0.978$&$0.578$&$0.561$&$0.565$\tabularnewline
10185&$-0.134$&$0.240$&$0.261$&$0.265$&$-0.041$&$0.428$&$0.531$&$0.538$&$-0.114$&$0.359$&$0.187$&$0.184$&$-0.348$&$0.554$&$0.041$&$0.043$\tabularnewline
10186&$ 0.043$&$0.244$&$2.338$&$2.347$&$-0.829$&$0.364$&$3.208$&$3.043$&$ 0.006$&$0.365$&$0.997$&$0.981$&$ 1.292$&$0.594$&$0.773$&$0.749$\tabularnewline
10187&$-0.191$&$0.239$&$1.134$&$1.102$&$ 0.367$&$0.439$&$0.732$&$0.732$&$-1.212$&$0.462$&$0.162$&$0.147$&$ 0.333$&$0.575$&$0.078$&$0.078$\tabularnewline
10188&$ 0.043$&$0.244$&$0.427$&$0.424$&$-0.041$&$0.428$&$0.548$&$0.551$&$ 0.132$&$0.374$&$0.722$&$0.696$&$ 0.333$&$0.575$&$0.324$&$0.324$\tabularnewline
10189&$-1.985$&$0.335$&$1.655$&$2.149$&$-3.114$&$0.777$&$0.617$&$0.577$&$-1.028$&$0.425$&$2.393$&$2.356$&$-4.308$&$1.561$&$0.157$&$0.161$\tabularnewline
10190&$-0.522$&$0.233$&$2.803$&$2.767$&$-1.665$&$0.377$&$1.204$&$1.170$&$-0.114$&$0.359$&$2.233$&$2.246$&$ 1.628$&$0.631$&$1.793$&$1.868$\tabularnewline
10191&$ 1.237$&$0.287$&$0.858$&$0.730$&$ 1.937$&$0.438$&$0.537$&$0.501$&$ 0.412$&$0.410$&$0.784$&$0.837$&$ 1.628$&$0.631$&$0.036$&$0.040$\tabularnewline
10192&$ 1.702$&$0.340$&$4.317$&$3.505$&$ 4.367$&$1.348$&$0.085$&$0.086$&$ 0.412$&$0.410$&$2.788$&$2.952$&$ 1.628$&$0.631$&$1.793$&$1.868$\tabularnewline
10193&$-0.628$&$0.231$&$0.731$&$0.720$&$-0.557$&$0.386$&$0.386$&$0.366$&$-0.233$&$0.357$&$1.303$&$1.295$&$-1.408$&$0.532$&$0.651$&$0.673$\tabularnewline
10196&$-0.469$&$0.233$&$0.919$&$0.912$&$-0.557$&$0.386$&$1.244$&$1.252$&$-0.869$&$0.400$&$0.343$&$0.324$&$-0.014$&$0.570$&$0.192$&$0.189$\tabularnewline
10197&$ 0.164$&$0.247$&$1.334$&$1.338$&$ 0.161$&$0.437$&$2.566$&$2.560$&$ 0.006$&$0.365$&$1.057$&$1.049$&$-0.014$&$0.570$&$1.042$&$1.040$\tabularnewline
10198&$ 2.516$&$0.511$&$1.764$&$2.085$&$ 4.367$&$1.348$&$0.085$&$0.086$&$ 0.779$&$0.489$&$2.488$&$2.022$&$ 1.628$&$0.631$&$1.793$&$1.868$\tabularnewline
10200&$ 0.226$&$0.248$&$0.470$&$0.460$&$ 0.367$&$0.439$&$0.240$&$0.239$&$-0.233$&$0.357$&$0.325$&$0.330$&$ 0.978$&$0.578$&$0.148$&$0.146$\tabularnewline
10201&$ 0.670$&$0.257$&$0.289$&$0.293$&$ 0.568$&$0.435$&$0.240$&$0.238$&$ 0.132$&$0.374$&$0.701$&$0.640$&$ 1.628$&$0.631$&$0.036$&$0.040$\tabularnewline
10202&$-0.248$&$0.238$&$0.185$&$0.187$&$-0.230$&$0.415$&$0.346$&$0.347$&$-0.726$&$0.383$&$0.711$&$0.779$&$-0.348$&$0.554$&$0.041$&$0.043$\tabularnewline
10203&$ 4.338$&$1.368$&$0.043$&$0.045$&$ 4.367$&$1.348$&$0.085$&$0.086$&$ 2.358$&$1.265$&$0.133$&$0.136$&$ 3.664$&$1.437$&$0.164$&$0.171$\tabularnewline
10204&$ 1.953$&$0.381$&$1.280$&$1.112$&$ 3.360$&$0.789$&$0.508$&$0.529$&$ 0.779$&$0.489$&$0.601$&$0.604$&$ 2.024$&$0.704$&$0.955$&$0.991$\tabularnewline
10206&$ 0.043$&$0.244$&$0.499$&$0.496$&$-0.402$&$0.400$&$0.435$&$0.447$&$-0.114$&$0.359$&$1.308$&$1.267$&$ 0.333$&$0.575$&$0.371$&$0.371$\tabularnewline
10208&$-0.469$&$0.233$&$0.553$&$0.543$&$-0.402$&$0.400$&$0.445$&$0.454$&$-0.595$&$0.370$&$0.464$&$0.455$&$-0.645$&$0.536$&$0.353$&$0.352$\tabularnewline
10209&$-1.545$&$0.273$&$1.163$&$1.163$&$-2.357$&$0.510$&$1.388$&$1.668$&$-0.595$&$0.370$&$1.646$&$1.700$&$-1.408$&$0.532$&$1.341$&$1.382$\tabularnewline
10210&$-0.522$&$0.233$&$1.262$&$1.233$&$-0.952$&$0.357$&$0.765$&$0.744$&$-0.114$&$0.359$&$1.136$&$1.159$&$-0.645$&$0.536$&$2.513$&$2.444$\tabularnewline
10211&$-1.700$&$0.292$&$0.547$&$0.564$&$-2.652$&$0.599$&$0.425$&$0.418$&$-0.351$&$0.358$&$0.929$&$0.932$&$-2.386$&$0.708$&$0.261$&$0.212$\tabularnewline
10212&$-0.522$&$0.233$&$0.419$&$0.417$&$-0.829$&$0.364$&$0.405$&$0.408$&$ 0.006$&$0.365$&$0.131$&$0.129$&$-1.159$&$0.522$&$0.535$&$0.542$\tabularnewline
10213&$ 0.350$&$0.250$&$0.675$&$0.682$&$ 0.367$&$0.439$&$1.132$&$1.129$&$ 0.006$&$0.365$&$0.125$&$0.124$&$ 1.292$&$0.594$&$0.391$&$0.389$\tabularnewline
10214&$ 0.164$&$0.247$&$0.440$&$0.433$&$ 0.367$&$0.439$&$0.252$&$0.251$&$ 0.412$&$0.410$&$0.131$&$0.141$&$-0.645$&$0.536$&$0.353$&$0.352$\tabularnewline
10215&$ 0.540$&$0.254$&$1.450$&$1.450$&$ 0.367$&$0.439$&$1.269$&$1.272$&$ 2.358$&$1.265$&$0.133$&$0.136$&$-0.014$&$0.570$&$2.171$&$2.202$\tabularnewline
10216&$-0.134$&$0.240$&$1.427$&$1.454$&$-0.402$&$0.400$&$2.949$&$3.013$&$ 0.006$&$0.365$&$0.819$&$0.790$&$ 0.333$&$0.575$&$0.788$&$0.787$\tabularnewline
10217&$ 0.540$&$0.254$&$0.328$&$0.332$&$ 0.758$&$0.428$&$0.535$&$0.530$&$ 0.412$&$0.410$&$0.131$&$0.141$&$ 0.663$&$0.574$&$0.451$&$0.451$\tabularnewline
10218&$-0.134$&$0.240$&$0.852$&$0.870$&$-0.952$&$0.357$&$0.926$&$0.908$&$ 0.132$&$0.374$&$0.202$&$0.198$&$ 0.663$&$0.574$&$0.038$&$0.038$\tabularnewline
10219&$-1.052$&$0.238$&$1.485$&$1.466$&$-1.301$&$0.353$&$1.751$&$1.756$&$-1.212$&$0.462$&$1.333$&$1.265$&$-0.910$&$0.524$&$0.911$&$0.914$\tabularnewline
10220&$-0.248$&$0.238$&$4.191$&$4.152$&$-2.357$&$0.510$&$2.272$&$2.082$&$ 2.358$&$1.265$&$0.133$&$0.136$&$ 1.628$&$0.631$&$1.793$&$1.868$\tabularnewline
10221&$-0.838$&$0.232$&$0.611$&$0.610$&$-0.698$&$0.373$&$0.440$&$0.452$&$-0.726$&$0.383$&$0.834$&$0.882$&$-1.408$&$0.532$&$0.346$&$0.347$\tabularnewline
10223&$-1.475$&$0.266$&$0.513$&$0.456$&$-1.803$&$0.395$&$0.694$&$0.670$&$-1.212$&$0.462$&$0.388$&$0.391$&$-1.673$&$0.559$&$0.452$&$0.427$\tabularnewline
10224&$ 0.870$&$0.265$&$0.829$&$0.794$&$ 1.274$&$0.413$&$0.875$&$0.876$&$ 0.579$&$0.441$&$1.065$&$1.082$&$ 1.292$&$0.594$&$0.391$&$0.389$\tabularnewline
10226&$ 0.939$&$0.268$&$0.617$&$0.610$&$ 0.938$&$0.421$&$0.802$&$0.811$&$ 0.779$&$0.489$&$1.047$&$0.919$&$ 1.292$&$0.594$&$0.339$&$0.343$\tabularnewline
10227&$ 1.159$&$0.281$&$0.246$&$0.270$&$ 1.764$&$0.425$&$0.483$&$0.490$&$ 0.579$&$0.441$&$0.038$&$0.038$&$ 1.292$&$0.594$&$0.089$&$0.090$\tabularnewline
10228&$-0.076$&$0.242$&$0.136$&$0.134$&$-0.041$&$0.428$&$0.178$&$0.177$&$-0.114$&$0.359$&$0.115$&$0.117$&$-0.348$&$0.554$&$0.041$&$0.043$\tabularnewline
10230&$ 1.159$&$0.281$&$1.190$&$1.158$&$ 2.124$&$0.459$&$0.627$&$0.640$&$ 0.579$&$0.441$&$0.625$&$0.616$&$ 0.978$&$0.578$&$2.064$&$2.074$\tabularnewline
10231&$-0.575$&$0.232$&$0.964$&$0.953$&$-1.417$&$0.357$&$0.522$&$0.521$&$ 0.006$&$0.365$&$0.994$&$0.976$&$ 0.333$&$0.575$&$0.692$&$0.692$\tabularnewline
10232&$ 1.010$&$0.271$&$0.297$&$0.276$&$ 1.109$&$0.416$&$0.067$&$0.068$&$ 1.040$&$0.568$&$0.452$&$0.442$&$ 1.292$&$0.594$&$0.339$&$0.343$\tabularnewline
10233&$-0.785$&$0.232$&$2.458$&$2.444$&$ 0.161$&$0.437$&$3.419$&$3.431$&$-1.028$&$0.425$&$0.231$&$0.229$&$-3.014$&$0.910$&$0.539$&$0.552$\tabularnewline
10234&$ 0.103$&$0.246$&$1.444$&$1.421$&$ 0.568$&$0.435$&$1.133$&$1.129$&$ 0.132$&$0.374$&$2.005$&$2.061$&$-0.348$&$0.554$&$0.883$&$0.908$\tabularnewline
10237&$-0.076$&$0.242$&$0.527$&$0.517$&$-0.402$&$0.400$&$0.337$&$0.325$&$ 0.579$&$0.441$&$0.281$&$0.250$&$-0.645$&$0.536$&$0.691$&$0.675$\tabularnewline
10238&$ 0.103$&$0.246$&$0.222$&$0.219$&$ 0.367$&$0.439$&$0.240$&$0.239$&$-0.351$&$0.358$&$0.176$&$0.177$&$ 0.333$&$0.575$&$0.078$&$0.078$\tabularnewline
10240&$-1.475$&$0.266$&$0.147$&$0.166$&$-1.803$&$0.395$&$0.019$&$0.023$&$-0.869$&$0.400$&$0.024$&$0.024$&$-2.386$&$0.708$&$0.462$&$0.493$\tabularnewline
10242&$-2.395$&$0.417$&$0.964$&$1.247$&$-4.171$&$1.373$&$0.085$&$0.087$&$-1.028$&$0.425$&$0.578$&$0.545$&$-4.308$&$1.561$&$0.157$&$0.161$\tabularnewline
\hline
\end{longtable}}\end{landscape}


\newpage
%%%%%%%%%%%%%%%%%%%%%%%%%%%%%%%%%%%%%%%%%%%%%%%%%%%
\section{RSM-based Instrument for Measuring the Intrinsic Motivation in the Third Empirical Study}
\label{sec:irt-intrinsic-motivation-third-study}

\subsection{Checking Assumptions}

\subsubsection*{Test of Unidimensionality}

\autoref{tab:test-unidimensionality-irt-motivation-third-study} shows the results for the test of unidimensionality in which the goodness of fit statistics indicate strong multidimensionality ($DETECT > 1.00$) to measure the intrinsic motivation with a DETECT index of $6.096$. 
Essential unidimensionality ($ASSI < 0.25$ and $RATIO < 0.36$) in the data structure is indicated for the intrinsic motivation by the ASSI and RATIO indices with values of $0.242$ and $0.128$, respectively.
The index of $AGFI = 0.899$ in the unidimensional CFA indicates an acceptable fit for measuring the \emph{Intrinsic Motivation}.
The sub-scales of \emph{Interest/Enjoyment}, \emph{Perceived Choice}, \emph{Pressure/Tension} and \emph{Effort/Importance} have a good fit indicated by the AGFI index with values greater than $0.95$.
A good fit with the unidimensional CFA is indicated by the TLI and CFI indices for all the sub-scales.
The ASSI index indicates essential unidimensionality in the data structure for the sub-scales of \emph{Interest/Enjoyment} and \emph{Pressure/Tension}, essential deviation from unidimensionality is indicated in the data structure of the scales: \emph{Perceived Choice} and \emph{Effort/Importance}.
The Ratio index in all the sub-scales indicate essential deviation from unidimensionality.

%latex.default(data_df, caption = paste("Goodness of fit statistics related to the test of unidimensionality",     "in RSM-based instruments", in_title), size = "small", longtable = T,     ctable = F, landscape = F, where = "!htbp", file = filename,     append = T)%
\setlongtables{\scriptsize
\begin{longtable}{lrrrrrrrr}\caption{Goodness of fit statistics related to the test of unidimensionality in the RSM-based instrument for measuring the intrinsic motivation in the third empirical study}
\tabularnewline
\hline\hline
\multicolumn{1}{l}{}&\multicolumn{1}{c}{df}&\multicolumn{1}{c}{chisq}&\multicolumn{1}{c}{AGFI}&\multicolumn{1}{c}{TLI}&\multicolumn{1}{c}{CFI}&\multicolumn{1}{c}{DETECT}&\multicolumn{1}{c}{ASSI}&\multicolumn{1}{c}{RATIO}\tabularnewline
\hline
\endfirsthead\caption[]{\em (continued)} \tabularnewline
\hline
\multicolumn{1}{l}{}&\multicolumn{1}{c}{df}&\multicolumn{1}{c}{chisq}&\multicolumn{1}{c}{AGFI}&\multicolumn{1}{c}{TLI}&\multicolumn{1}{c}{CFI}&\multicolumn{1}{c}{DETECT}&\multicolumn{1}{c}{ASSI}&\multicolumn{1}{c}{RATIO}\tabularnewline
\hline
\endhead
\hline
\multicolumn{9}{r}{\tiny df: degree of freedom; AGFI: Adjusted Goodness of Fit Index; CFI: Comparative Fit Index; TLI: Tucker-Lewis Index;}
\endfoot
\label{tab:test-unidimensionality-irt-motivation-third-study}
Intrinsic Motivation&$9.598$&$54.026$&$0.899$&$0.156$&$0.226$&$ 6.096$&$0.242$&$0.128$\tabularnewline
Interest/Enjoyment&$5.419$&$ 8.602$&$0.996$&$0.963$&$0.887$&$ 6.028$&$0.200$&$0.507$\tabularnewline
Perceived Choice&$4.040$&$ 3.576$&$0.998$&$1.006$&$1.000$&$12.178$&$0.600$&$0.803$\tabularnewline
Pressure/Tension&$1.765$&$ 1.113$&$0.999$&$1.014$&$1.000$&$17.469$&$0.000$&$0.644$\tabularnewline
Effort/Importance&$0.000$&$ 0.000$&$1.000$&$1.000$&$1.000$&$17.820$&$0.333$&$0.776$\tabularnewline
\hline
\end{longtable}}

\subsubsection*{Test of Local Independence}

Results from the test of local independence in the RSM-based instrument for measuring the intrinsic motivation in the third empirical study are summarized in \autoref{tab:test-local-independence-irt-motivation-third-study}.
The Standardized Root Mean Squared Residual (SRMSR) indicates a good fit ($< 0.10$) for the sub-scales of \emph{Interest/Enjoyment}, \emph{Perceived Choice} and \emph{Effort/Importance}. An acceptable fit ($0.10$s) has found for the sub-scale of \emph{Pressure/Tension} with value of $0.189$. The null condition of local independence is not rejected in the sub-scales of \emph{Interest/Enjoyment}, \emph{Perceived Choice} and \emph{Effort/Importance}. 

%latex.default(data_df, caption = paste("Q3 statistics related to the test of local independence",     "in the RSM-based instrument", in_title), size = "small",     longtable = T, ctable = F, landscape = F, where = "!htbp",     file = filename, append = T)%
\setlongtables{\scriptsize
\begin{longtable}{lrrrrr}\caption{Item residual correlation statistics related to the test of local independence in the RSM-based instrument for measuring the intrinsic motivation in the third empirical study} \tabularnewline
\hline\hline
\multicolumn{1}{l}{}&\multicolumn{1}{c}{max.chisq}&\multicolumn{1}{c}{maxaQ3}&\multicolumn{1}{c}{MADaQ3}&\multicolumn{1}{c}{SRMSR}&\multicolumn{1}{c}{p.value}\tabularnewline
\hline
\endfirsthead\caption[]{\em (continued)} \tabularnewline
\hline
\multicolumn{1}{l}{}&\multicolumn{1}{c}{max.chisq}&\multicolumn{1}{c}{maxaQ3}&\multicolumn{1}{c}{MADaQ3}&\multicolumn{1}{c}{SRMSR}&\multicolumn{1}{c}{p.value}\tabularnewline
\hline
\endhead
\hline
\multicolumn{6}{r}{\tiny aQ3: adjusted correlation of item residuals; maxaQ3: maximum aQ3;}\tabularnewline
\multicolumn{6}{r}{\tiny MADaQ3: Median Absolute Deviation of aQ3;}
\endfoot
\label{tab:test-local-independence-irt-motivation-third-study}
Intrinsic Motivation&$ 169.797$&$0.738$&$0.276$&$0.274$&$0.000$\tabularnewline
Interest/Enjoyment&$1034.187$&$0.329$&$0.147$&$0.096$&$0.339$\tabularnewline
Perceived Choice&$ 180.200$&$0.316$&$0.113$&$0.054$&$0.284$\tabularnewline
Pressure/Tension&$  64.855$&$0.500$&$0.338$&$0.189$&$0.003$\tabularnewline
Effort/Importance&$  52.489$&$0.211$&$0.141$&$0.081$&$0.430$\tabularnewline
\hline
\end{longtable}}
%\begin{flushright}{\tiny Q3: correlation of item residuals; maxaQ3: maximum adjusted correlation Q3; MADaQ3: Median Absolute Deviation in the adjusted Q3; }\end{flushright}

\subsubsection*{Test of Monotonicity}

\autoref{tab:test-monotonicity-irt-motivation-third-study} summarizes the test of monotonicity in the RSM-based instrument for measuring the intrinsic motivation in the third empirical study. These results indicates that there are no one violation of monotonicity in the items at the significance level $\alpha = 0.05$.

%latex.default(data_df, caption = paste("Summary of the violations of monotonicity",     "in the RSM-based instrument", in_title), size = "small",     longtable = T, ctable = F, landscape = F, where = "!htbp",     file = filename, append = T)%
\setlongtables{\scriptsize
\begin{longtable}{lrrrrrrrrrr}\caption{Test of monotonicity in the RSM-based instrument for measuring the intrinsic motivation in the third empirical study} \tabularnewline
\hline\hline
\multicolumn{1}{l}{}&\multicolumn{1}{c}{ItemH}&\multicolumn{1}{c}{ac}&\multicolumn{1}{c}{vi}&\multicolumn{1}{c}{vi/ac}&\multicolumn{1}{c}{maxvi}&\multicolumn{1}{c}{sum}&\multicolumn{1}{c}{sum/ac}&\multicolumn{1}{c}{zmax}&\multicolumn{1}{c}{zsig}&\multicolumn{1}{c}{crit}\tabularnewline
\hline
\endfirsthead\caption[]{\em (continued)} \tabularnewline
\hline
\multicolumn{1}{l}{}&\multicolumn{1}{c}{ItemH}&\multicolumn{1}{c}{ac}&\multicolumn{1}{c}{vi}&\multicolumn{1}{c}{vi/ac}&\multicolumn{1}{c}{maxvi}&\multicolumn{1}{c}{sum}&\multicolumn{1}{c}{sum/ac}&\multicolumn{1}{c}{zmax}&\multicolumn{1}{c}{zsig}&\multicolumn{1}{c}{crit}\tabularnewline
\hline
\endhead
\hline
\multicolumn{11}{r}{\tiny vi: number of violations; vi/ac: proportion of active pairs; maxvi: maximum violations;}\tabularnewline
\multicolumn{11}{r}{\tiny sum: sum of all violations; zmax: maximum z-value; zsig: number of significant z-values; crit: critical value}
\endfoot
\label{tab:test-monotonicity-irt-motivation-third-study}
Intrinsic Motivation.Item22IE&$0.33$&$0$&$0$&$$&$0.00$&$0.00$&$$&$0.00$&$0$&$ 0$\tabularnewline
Intrinsic Motivation.Item09IE&$0.33$&$0$&$0$&$$&$0.00$&$0.00$&$$&$0.00$&$0$&$ 0$\tabularnewline
Intrinsic Motivation.Item12IE&$0.28$&$5$&$1$&$0.2$&$0.08$&$0.08$&$0.02$&$0.53$&$0$&$51$\tabularnewline
Intrinsic Motivation.Item24IE&$0.18$&$0$&$0$&$$&$0.00$&$0.00$&$$&$0.00$&$0$&$ 0$\tabularnewline
Intrinsic Motivation.Item21IE&$0.25$&$6$&$0$&$0.0$&$0.00$&$0.00$&$0.00$&$0.00$&$0$&$ 0$\tabularnewline
Intrinsic Motivation.Item01IE&$0.29$&$4$&$0$&$0.0$&$0.00$&$0.00$&$0.00$&$0.00$&$0$&$ 0$\tabularnewline
Intrinsic Motivation.Item17PC&$0.33$&$0$&$0$&$$&$0.00$&$0.00$&$$&$0.00$&$0$&$ 0$\tabularnewline
Intrinsic Motivation.Item15PC&$0.32$&$6$&$0$&$0.0$&$0.00$&$0.00$&$0.00$&$0.00$&$0$&$ 0$\tabularnewline
Intrinsic Motivation.Item06PC&$0.38$&$0$&$0$&$$&$0.00$&$0.00$&$$&$0.00$&$0$&$ 0$\tabularnewline
Intrinsic Motivation.Item02PC&$0.33$&$4$&$0$&$0.0$&$0.00$&$0.00$&$0.00$&$0.00$&$0$&$ 0$\tabularnewline
Intrinsic Motivation.Item08PC&$0.39$&$0$&$0$&$$&$0.00$&$0.00$&$$&$0.00$&$0$&$ 0$\tabularnewline
Intrinsic Motivation.Item16PT&$0.16$&$4$&$0$&$0.0$&$0.00$&$0.00$&$0.00$&$0.00$&$0$&$ 0$\tabularnewline
Intrinsic Motivation.Item14PT&$0.27$&$4$&$0$&$0.0$&$0.00$&$0.00$&$0.00$&$0.00$&$0$&$ 0$\tabularnewline
Intrinsic Motivation.Item18PT&$0.28$&$4$&$0$&$0.0$&$0.00$&$0.00$&$0.00$&$0.00$&$0$&$ 0$\tabularnewline
Intrinsic Motivation.Item11PT&$0.11$&$0$&$0$&$$&$0.00$&$0.00$&$$&$0.00$&$0$&$ 0$\tabularnewline
Intrinsic Motivation.Item13EI&$0.19$&$4$&$0$&$0.0$&$0.00$&$0.00$&$0.00$&$0.00$&$0$&$ 0$\tabularnewline
Intrinsic Motivation.Item03EI&$0.06$&$5$&$0$&$0.0$&$0.00$&$0.00$&$0.00$&$0.00$&$0$&$ 0$\tabularnewline
Intrinsic Motivation.Item07EI&$0.14$&$4$&$0$&$0.0$&$0.00$&$0.00$&$0.00$&$0.00$&$0$&$ 0$\tabularnewline
Interest/Enjoyment.Item22IE&$0.76$&$0$&$0$&$$&$0.00$&$0.00$&$$&$0.00$&$0$&$ 0$\tabularnewline
Interest/Enjoyment.Item09IE&$0.73$&$0$&$0$&$$&$0.00$&$0.00$&$$&$0.00$&$0$&$ 0$\tabularnewline
Interest/Enjoyment.Item12IE&$0.66$&$0$&$0$&$$&$0.00$&$0.00$&$$&$0.00$&$0$&$ 0$\tabularnewline
Interest/Enjoyment.Item24IE&$0.61$&$0$&$0$&$$&$0.00$&$0.00$&$$&$0.00$&$0$&$ 0$\tabularnewline
Interest/Enjoyment.Item21IE&$0.69$&$0$&$0$&$$&$0.00$&$0.00$&$$&$0.00$&$0$&$ 0$\tabularnewline
Interest/Enjoyment.Item01IE&$0.60$&$0$&$0$&$$&$0.00$&$0.00$&$$&$0.00$&$0$&$ 0$\tabularnewline
Perceived Choice.Item17PC&$0.69$&$4$&$0$&$0.0$&$0.00$&$0.00$&$0.00$&$0.00$&$0$&$ 0$\tabularnewline
Perceived Choice.Item15PC&$0.63$&$0$&$0$&$$&$0.00$&$0.00$&$$&$0.00$&$0$&$ 0$\tabularnewline
Perceived Choice.Item06PC&$0.71$&$0$&$0$&$$&$0.00$&$0.00$&$$&$0.00$&$0$&$ 0$\tabularnewline
Perceived Choice.Item02PC&$0.69$&$0$&$0$&$$&$0.00$&$0.00$&$$&$0.00$&$0$&$ 0$\tabularnewline
Perceived Choice.Item08PC&$0.69$&$4$&$0$&$0.0$&$0.00$&$0.00$&$0.00$&$0.00$&$0$&$ 0$\tabularnewline
Pressure/Tension.Item16PT&$0.64$&$0$&$0$&$$&$0.00$&$0.00$&$$&$0.00$&$0$&$ 0$\tabularnewline
Pressure/Tension.Item14PT&$0.65$&$4$&$0$&$0.0$&$0.00$&$0.00$&$0.00$&$0.00$&$0$&$ 0$\tabularnewline
Pressure/Tension.Item18PT&$0.65$&$4$&$0$&$0.0$&$0.00$&$0.00$&$0.00$&$0.00$&$0$&$ 0$\tabularnewline
Pressure/Tension.Item11PT&$0.44$&$0$&$0$&$$&$0.00$&$0.00$&$$&$0.00$&$0$&$ 0$\tabularnewline
Effort/Importance.Item13EI&$0.63$&$0$&$0$&$$&$0.00$&$0.00$&$$&$0.00$&$0$&$ 0$\tabularnewline
Effort/Importance.Item03EI&$0.53$&$5$&$0$&$0.0$&$0.00$&$0.00$&$0.00$&$0.00$&$0$&$ 0$\tabularnewline
Effort/Importance.Item07EI&$0.60$&$0$&$0$&$$&$0.00$&$0.00$&$$&$0.00$&$0$&$ 0$\tabularnewline
\hline
\end{longtable}}
%\begin{flushright}{\tiny vi: numer of violations; vi/ac: proportion of active pairs; maxvi: maximum violations; sum: sum of all violations; zmax: maximum z-value; zsig: number of significant z-values; crit: Critical value }\end{flushright}

\subsection{Item Parameters}

\autoref{tab:item-parameters-interest-enjoyment-third-study} shows the estimated parameters for the RSM-based instrument used to measure the \emph{Interest/Enjoyment} in the third empirical study. These parameters had been calculated using the MML method \cite{BockAitkin1981}, so that the value in row \aspas{B.Cat$x$} and column \aspas{$i$} is the item slope $b_{i,x}$ of item $i$ in the category \aspas{$x$}, and the value in the row \aspas{AXsi.Cat$x$} and column \aspas{$i$} is the item intercept $a_{i,x}\xi$ of item $i$ in the category \aspas{$x$}. According to the Infit/Outfit statistics of items, no one mean-square value is greater than $2.0$ indicating that the measurement system of \emph{Interest/Enjoyment} is not distorted or degraded by the items.

%latex.default(estimated_params_df, caption = paste("Estimated parameters in the RSM-based instrument",     "for measuring the", lname), size = "small", longtable = T,     ctable = F, landscape = F, where = "!htbp", file = filename,     append = T)%
\setlongtables{\scriptsize
\begin{longtable}{lrrrrrr}\caption{Estimated parameters in the RSM-based instrument for measuring the interest/enjoyment in the third empirical study} \tabularnewline
\hline\hline
\multicolumn{1}{l}{}&\multicolumn{1}{c}{Item01IE}&\multicolumn{1}{c}{Item09IE}&\multicolumn{1}{c}{Item12IE}&\multicolumn{1}{c}{Item21IE}&\multicolumn{1}{c}{Item22IE}&\multicolumn{1}{c}{Item24IE}\tabularnewline
\hline
\endfirsthead\caption[]{\em (continued)} \tabularnewline
\hline
\multicolumn{1}{l}{}&\multicolumn{1}{c}{Item01IE}&\multicolumn{1}{c}{Item09IE}&\multicolumn{1}{c}{Item12IE}&\multicolumn{1}{c}{Item21IE}&\multicolumn{1}{c}{Item22IE}&\multicolumn{1}{c}{Item24IE}\tabularnewline
\hline
\endhead
\hline
\endfoot
\label{tab:item-parameters-interest-enjoyment-third-study}
xsi.item&$ 0.912$&$ 0.615$&$ 0.570$&$ 0.080$&$ 0.615$&$ 0.797$\tabularnewline
B.Cat0&$ 0.000$&$ 0.000$&$ 0.000$&$ 0.000$&$ 0.000$&$ 0.000$\tabularnewline
B.Cat1&$ 1.000$&$ 1.000$&$ 1.000$&$ 1.000$&$ 1.000$&$ 1.000$\tabularnewline
B.Cat2&$ 2.000$&$ 2.000$&$ 2.000$&$ 2.000$&$ 2.000$&$ 2.000$\tabularnewline
B.Cat3&$ 3.000$&$ 3.000$&$ 3.000$&$ 3.000$&$ 3.000$&$ 3.000$\tabularnewline
B.Cat4&$ 4.000$&$ 4.000$&$ 4.000$&$ 4.000$&$ 4.000$&$ 4.000$\tabularnewline
B.Cat5&$ 5.000$&$ 5.000$&$ 5.000$&$ 5.000$&$ 5.000$&$ 5.000$\tabularnewline
B.Cat6&$ 6.000$&$ 6.000$&$ 6.000$&$ 6.000$&$ 6.000$&$ 6.000$\tabularnewline
AXsi.Cat0&$ 0.000$&$ 0.000$&$ 0.000$&$ 0.000$&$ 0.000$&$ 0.000$\tabularnewline
AXsi.Cat1&$ 0.877$&$ 1.174$&$ 1.219$&$ 1.709$&$ 1.174$&$ 0.992$\tabularnewline
AXsi.Cat2&$ 1.646$&$ 2.241$&$ 2.331$&$ 3.311$&$ 2.241$&$ 1.876$\tabularnewline
AXsi.Cat3&$ 1.780$&$ 2.672$&$ 2.808$&$ 4.278$&$ 2.672$&$ 2.125$\tabularnewline
AXsi.Cat4&$-0.028$&$ 1.161$&$ 1.342$&$ 3.302$&$ 1.161$&$ 0.431$\tabularnewline
AXsi.Cat5&$-2.330$&$-0.844$&$-0.617$&$ 1.833$&$-0.844$&$-1.755$\tabularnewline
AXsi.Cat6&$-5.473$&$-3.690$&$-3.418$&$-0.478$&$-3.690$&$-4.784$\tabularnewline
\hline
Outfit&$ 1.503$&$ 0.664$&$ 1.009$&$ 0.937$&$ 0.521$&$ 1.309$\tabularnewline
Infit&$ 1.537$&$ 0.694$&$ 0.996$&$ 0.926$&$ 0.516$&$ 1.425$\tabularnewline
\hline
\end{longtable}}

\autoref{tab:item-parameters-perceived-choice-third-study} shows the estimated parameters for the measurement instrument of \emph{Perceived Choice} in which the Infit/Outfit statistics of items indicate that no one item distorts or degrades the measurement system with mean-square greater than $2.0$.

%latex.default(estimated_params_df, caption = paste("Estimated parameters in the RSM-based instrument",     "for measuring the", lname), size = "small", longtable = T,     ctable = F, landscape = F, where = "!htbp", file = filename,     append = T)%
\setlongtables{\scriptsize
\begin{longtable}{lrrrrr}\caption{Estimated parameters in the RSM-based instrument for measuring the perceived choice in the third empirical study} \tabularnewline
\hline\hline
\multicolumn{1}{l}{}&\multicolumn{1}{c}{Item02PC}&\multicolumn{1}{c}{Item06PC}&\multicolumn{1}{c}{Item08PC}&\multicolumn{1}{c}{Item15PC}&\multicolumn{1}{c}{Item17PC}\tabularnewline
\hline
\endfirsthead\caption[]{\em (continued)} \tabularnewline
\hline
\multicolumn{1}{l}{}&\multicolumn{1}{c}{Item02PC}&\multicolumn{1}{c}{Item06PC}&\multicolumn{1}{c}{Item08PC}&\multicolumn{1}{c}{Item15PC}&\multicolumn{1}{c}{Item17PC}\tabularnewline
\hline
\endhead
\hline
\endfoot
\label{tab:item-parameters-perceived-choice-third-study}
xsi.item&$-0.210$&$-0.119$&$-0.393$&$ 0.192$&$-0.356$\tabularnewline
B.Cat0&$ 0.000$&$ 0.000$&$ 0.000$&$ 0.000$&$ 0.000$\tabularnewline
B.Cat1&$ 1.000$&$ 1.000$&$ 1.000$&$ 1.000$&$ 1.000$\tabularnewline
B.Cat2&$ 2.000$&$ 2.000$&$ 2.000$&$ 2.000$&$ 2.000$\tabularnewline
B.Cat3&$ 3.000$&$ 3.000$&$ 3.000$&$ 3.000$&$ 3.000$\tabularnewline
B.Cat4&$ 4.000$&$ 4.000$&$ 4.000$&$ 4.000$&$ 4.000$\tabularnewline
B.Cat5&$ 5.000$&$ 5.000$&$ 5.000$&$ 5.000$&$ 5.000$\tabularnewline
B.Cat6&$ 6.000$&$ 6.000$&$ 6.000$&$ 6.000$&$ 6.000$\tabularnewline
AXsi.Cat0&$ 0.000$&$ 0.000$&$ 0.000$&$ 0.000$&$ 0.000$\tabularnewline
AXsi.Cat1&$ 2.122$&$ 2.031$&$ 2.305$&$ 1.720$&$ 2.268$\tabularnewline
AXsi.Cat2&$ 3.359$&$ 3.177$&$ 3.725$&$ 2.555$&$ 3.652$\tabularnewline
AXsi.Cat3&$ 4.097$&$ 3.824$&$ 4.645$&$ 2.890$&$ 4.535$\tabularnewline
AXsi.Cat4&$ 3.435$&$ 3.071$&$ 4.166$&$ 1.826$&$ 4.020$\tabularnewline
AXsi.Cat5&$ 2.634$&$ 2.178$&$ 3.548$&$ 0.622$&$ 3.365$\tabularnewline
AXsi.Cat6&$ 1.261$&$ 0.714$&$ 2.357$&$-1.153$&$ 2.138$\tabularnewline
\hline
Outfit&$ 1.005$&$ 0.955$&$ 0.873$&$ 1.267$&$ 0.937$\tabularnewline
Infit&$ 1.028$&$ 0.917$&$ 0.911$&$ 1.240$&$ 0.991$\tabularnewline
\hline
\end{longtable}}


\autoref{tab:item-parameters-pressure-tension-third-study} shows the estimated parameters for the measurement instrument of \emph{Pressure/Tension} in the third empirical study in which the Infit/Outfit statistics of items indicate that no one item distorts or degrades the measurement system with mean-square greater than $2.0$.

%latex.default(estimated_params_df, caption = paste("Estimated parameters in the RSM-based instrument",     "for measuring the", lname), size = "small", longtable = T,     ctable = F, landscape = F, where = "!htbp", file = filename,     append = T)%
\setlongtables{\scriptsize
\begin{longtable}{lrrrr}\caption{Estimated parameters in the RSM-based instrument for measuring the pressure/tension in the third empirical study} \tabularnewline
\hline\hline
\multicolumn{1}{l}{}&\multicolumn{1}{c}{Item11PT}&\multicolumn{1}{c}{Item14PT}&\multicolumn{1}{c}{Item16PT}&\multicolumn{1}{c}{Item18PT}\tabularnewline
\hline
\endfirsthead\caption[]{\em (continued)} \tabularnewline
\hline
\multicolumn{1}{l}{}&\multicolumn{1}{c}{Item11PT}&\multicolumn{1}{c}{Item14PT}&\multicolumn{1}{c}{Item16PT}&\multicolumn{1}{c}{Item18PT}\tabularnewline
\hline
\endhead
\hline
\endfoot
\label{tab:item-parameters-pressure-tension-third-study}
xsi.item&$ 0.123$&$-0.009$&$ 0.113$&$-0.081$\tabularnewline
B.Cat0&$ 0.000$&$ 0.000$&$ 0.000$&$ 0.000$\tabularnewline
B.Cat1&$ 1.000$&$ 1.000$&$ 1.000$&$ 1.000$\tabularnewline
B.Cat2&$ 2.000$&$ 2.000$&$ 2.000$&$ 2.000$\tabularnewline
B.Cat3&$ 3.000$&$ 3.000$&$ 3.000$&$ 3.000$\tabularnewline
B.Cat4&$ 4.000$&$ 4.000$&$ 4.000$&$ 4.000$\tabularnewline
B.Cat5&$ 5.000$&$ 0.000$&$ 0.000$&$ 0.000$\tabularnewline
B.Cat6&$ 6.000$&$ 0.000$&$ 0.000$&$ 0.000$\tabularnewline
AXsi.Cat0&$ 0.000$&$ 0.000$&$ 0.000$&$ 0.000$\tabularnewline
AXsi.Cat1&$ 0.222$&$-0.022$&$-0.145$&$ 0.049$\tabularnewline
AXsi.Cat2&$ 0.418$&$-0.072$&$-0.316$&$ 0.071$\tabularnewline
AXsi.Cat3&$ 1.149$&$ 0.415$&$ 0.048$&$ 0.628$\tabularnewline
AXsi.Cat4&$ 1.017$&$ 0.038$&$-0.451$&$ 0.322$\tabularnewline
AXsi.Cat5&$-1.232$&$$&$$&$$\tabularnewline
AXsi.Cat6&$-0.736$&$$&$$&$$\tabularnewline
\hline
Outfit&$ 1.530$&$ 0.747$&$ 0.778$&$ 0.744$\tabularnewline
Infit&$ 1.583$&$ 0.819$&$ 0.910$&$ 0.866$\tabularnewline
\hline
\end{longtable}}

\autoref{tab:item-parameters-effort-importance-third-study} shows the estimated parameters for the measurement instrument of \emph{Effort/Importance} in which the Infit/Outfit statistics of items indicate that no one item distorts or degrades the measurement system with mean-square greater than $2.0$.

%latex.default(estimated_params_df, caption = paste("Estimated parameters in the RSM-based instrument",     "for measuring the", lname), size = "small", longtable = T,     ctable = F, landscape = F, where = "!htbp", file = filename,     append = T)%
\setlongtables{\scriptsize
\begin{longtable}{lrrr}\caption{Estimated parameters in the RSM-based instrument for measuring the effort/importance in the third empirical study} \tabularnewline
\hline\hline
\multicolumn{1}{l}{}&\multicolumn{1}{c}{Item03EI}&\multicolumn{1}{c}{Item07EI}&\multicolumn{1}{c}{Item13EI}\tabularnewline
\hline
\endfirsthead\caption[]{\em (continued)} \tabularnewline
\hline
\multicolumn{1}{l}{}&\multicolumn{1}{c}{Item03EI}&\multicolumn{1}{c}{Item07EI}&\multicolumn{1}{c}{Item13EI}\tabularnewline
\hline
\endhead
\hline
\endfoot
\label{tab:item-parameters-effort-importance-third-study}
xsi.item&$-1.278$&$-2.012$&$-2.309$\tabularnewline
B.Cat0&$ 0.000$&$ 0.000$&$ 0.000$\tabularnewline
B.Cat1&$ 1.000$&$ 1.000$&$ 1.000$\tabularnewline
B.Cat2&$ 2.000$&$ 2.000$&$ 2.000$\tabularnewline
B.Cat3&$ 3.000$&$ 3.000$&$ 3.000$\tabularnewline
B.Cat4&$ 4.000$&$ 4.000$&$ 4.000$\tabularnewline
B.Cat5&$ 5.000$&$ 5.000$&$ 5.000$\tabularnewline
B.Cat6&$ 6.000$&$ 6.000$&$ 6.000$\tabularnewline
AXsi.Cat0&$ 0.000$&$ 0.000$&$ 0.000$\tabularnewline
AXsi.Cat1&$ 8.240$&$ 8.975$&$ 9.272$\tabularnewline
AXsi.Cat2&$ 9.465$&$10.933$&$11.527$\tabularnewline
AXsi.Cat3&$10.365$&$12.567$&$13.458$\tabularnewline
AXsi.Cat4&$ 9.821$&$12.758$&$13.946$\tabularnewline
AXsi.Cat5&$ 8.680$&$12.351$&$13.837$\tabularnewline
AXsi.Cat6&$ 7.667$&$12.072$&$13.855$\tabularnewline
\hline
Outfit&$ 1.148$&$ 0.923$&$ 0.818$\tabularnewline
Infit&$ 1.160$&$ 0.973$&$ 0.903$\tabularnewline
\hline
\end{longtable}}


\subsection{Intrinsic Motivation as Latent Trait Estimates}

\autoref{tab:intrinsic-motivation-estimates-third-study} shows the latent trait estimates by the RSM-based instrument for measuring the \emph{Intrinsic motivation} in the third empirical study.

%latex.default(data_df, caption = paste("Latent trait estimates and person model fit of the RSM-based instrument",     in_title), size = "scriptsize", longtable = T, ctable = F,     landscape = T, rowlabel = "", where = "!htbp", file = filename,     append = T)%
\setlongtables\begin{landscape}{\scriptsize
\begin{longtable}{l|rrrr|rrrr|rrrr|rrrr|rrrr}\caption{Latent trait estimates and person model fit of the RSM-based instrument for measuring the intrinsic motivation in the third empirical study} \tabularnewline
\hline\hline
\multicolumn{1}{l}{}&\multicolumn{4}{|c}{Intrinsic Motivation}&\multicolumn{4}{|c}{Interest/Enjoyment}&\multicolumn{4}{|c}{Perceived Choice}&\multicolumn{4}{|c}{Pressure/Tension}&\multicolumn{4}{|c}{Effort/Importance} \tabularnewline
\multicolumn{1}{l}{UserID}&\multicolumn{1}{|c}{theta}&\multicolumn{1}{c}{error}&\multicolumn{1}{c}{Outfit}&\multicolumn{1}{c}{Infit}&\multicolumn{1}{|c}{theta}&\multicolumn{1}{c}{error}&\multicolumn{1}{c}{Outfit}&\multicolumn{1}{c}{Infit}&\multicolumn{1}{|c}{theta}&\multicolumn{1}{c}{error}&\multicolumn{1}{c}{Outfit}&\multicolumn{1}{c}{Infit}&\multicolumn{1}{|c}{theta}&\multicolumn{1}{c}{error}&\multicolumn{1}{c}{Outfit}&\multicolumn{1}{c}{Infit}&\multicolumn{1}{|c}{theta}&\multicolumn{1}{c}{error}&\multicolumn{1}{c}{Outfit}&\multicolumn{1}{c}{Infit}\tabularnewline
\hline
\endfirsthead\caption[]{\em (continued)} \tabularnewline
\hline
\multicolumn{1}{l}{}&\multicolumn{4}{|c}{Intrinsic Motivation}&\multicolumn{4}{|c}{Interest/Enjoyment}&\multicolumn{4}{|c}{Perceived Choice}&\multicolumn{4}{|c}{Pressure/Tension}&\multicolumn{4}{|c}{Effort/Importance} \tabularnewline
\multicolumn{1}{l}{UserID}&\multicolumn{1}{|c}{theta}&\multicolumn{1}{c}{error}&\multicolumn{1}{c}{Outfit}&\multicolumn{1}{c}{Infit}&\multicolumn{1}{|c}{theta}&\multicolumn{1}{c}{error}&\multicolumn{1}{c}{Outfit}&\multicolumn{1}{c}{Infit}&\multicolumn{1}{|c}{theta}&\multicolumn{1}{c}{error}&\multicolumn{1}{c}{Outfit}&\multicolumn{1}{c}{Infit}&\multicolumn{1}{|c}{theta}&\multicolumn{1}{c}{error}&\multicolumn{1}{c}{Outfit}&\multicolumn{1}{c}{Infit}&\multicolumn{1}{|c}{theta}&\multicolumn{1}{c}{error}&\multicolumn{1}{c}{Outfit}&\multicolumn{1}{c}{Infit}\tabularnewline
\hline
\endhead
\hline
\endfoot
\label{tab:intrinsic-motivation-estimates-third-study}
10169&$ 0.455$&$0.163$&$2.935$&$2.794$&$ 5.357$&$1.439$&$0.082$&$0.087$&$ 0.808$&$0.387$&$0.502$&$0.507$&$ 2.258$&$1.137$&$0.144$&$0.139$&$-1.073$&$0.572$&$1.960$&$1.819$\tabularnewline
10170&$ 0.481$&$0.164$&$0.870$&$0.875$&$ 0.721$&$0.443$&$0.521$&$0.550$&$ 1.258$&$0.420$&$0.508$&$0.499$&$-0.345$&$0.363$&$2.636$&$2.908$&$-0.222$&$0.517$&$0.021$&$0.023$\tabularnewline
10171&$ 1.203$&$0.219$&$0.256$&$0.237$&$ 2.448$&$0.432$&$0.483$&$0.493$&$ 1.630$&$0.478$&$0.173$&$0.149$&$-0.752$&$0.439$&$0.195$&$0.193$&$ 2.330$&$1.198$&$0.174$&$0.222$\tabularnewline
10172&$-0.028$&$0.161$&$1.282$&$1.267$&$ 1.795$&$0.408$&$0.438$&$0.421$&$-1.085$&$0.424$&$0.263$&$0.263$&$ 0.498$&$0.411$&$0.878$&$0.972$&$-1.411$&$0.601$&$0.874$&$0.793$\tabularnewline
10174&$ 0.124$&$0.160$&$0.770$&$0.753$&$ 1.116$&$0.430$&$0.972$&$0.972$&$ 0.087$&$0.394$&$0.204$&$0.204$&$ 0.498$&$0.411$&$0.878$&$0.972$&$-0.481$&$0.528$&$1.531$&$1.646$\tabularnewline
10175&$ 0.857$&$0.183$&$0.700$&$0.702$&$ 0.512$&$0.443$&$0.056$&$0.057$&$ 2.181$&$0.619$&$0.786$&$0.940$&$-1.956$&$1.082$&$0.145$&$0.149$&$ 0.802$&$0.573$&$0.820$&$0.881$\tabularnewline
10176&$-0.157$&$0.163$&$0.439$&$0.440$&$ 0.305$&$0.437$&$0.108$&$0.111$&$-0.572$&$0.411$&$0.598$&$0.594$&$ 0.674$&$0.445$&$0.545$&$0.580$&$-1.073$&$0.572$&$0.522$&$0.488$\tabularnewline
10179&$ 0.250$&$0.160$&$1.246$&$1.264$&$ 0.109$&$0.427$&$1.898$&$2.006$&$-1.261$&$0.432$&$0.735$&$0.716$&$-1.244$&$0.630$&$0.355$&$0.367$&$ 1.124$&$0.635$&$0.419$&$0.328$\tabularnewline
10181&$-0.264$&$0.166$&$0.870$&$0.907$&$-1.250$&$0.401$&$0.482$&$0.485$&$ 0.087$&$0.394$&$1.561$&$1.586$&$ 0.098$&$0.359$&$0.958$&$0.930$&$-1.776$&$0.641$&$0.547$&$0.523$\tabularnewline
10183&$-0.210$&$0.164$&$0.492$&$0.501$&$-0.685$&$0.385$&$0.208$&$0.209$&$ 0.239$&$0.389$&$0.914$&$0.932$&$ 0.887$&$0.493$&$0.215$&$0.187$&$-1.073$&$0.572$&$0.919$&$0.864$\tabularnewline
10184&$-0.002$&$0.160$&$0.269$&$0.265$&$ 0.305$&$0.437$&$0.212$&$0.212$&$-0.071$&$0.399$&$0.426$&$0.432$&$ 0.218$&$0.370$&$0.074$&$0.071$&$-1.073$&$0.572$&$0.023$&$0.023$\tabularnewline
10185&$ 0.074$&$0.160$&$0.992$&$0.981$&$ 1.469$&$0.415$&$1.025$&$1.037$&$-0.572$&$0.411$&$1.788$&$1.777$&$ 0.498$&$0.411$&$0.292$&$0.281$&$-0.481$&$0.528$&$0.296$&$0.316$\tabularnewline
10186&$ 0.200$&$0.160$&$1.326$&$1.329$&$ 0.305$&$0.437$&$3.062$&$3.146$&$ 0.385$&$0.386$&$0.850$&$0.869$&$ 1.152$&$0.567$&$0.537$&$0.601$&$ 2.330$&$1.198$&$0.174$&$0.222$\tabularnewline
10188&$ 0.535$&$0.165$&$0.967$&$1.006$&$-0.543$&$0.389$&$0.755$&$0.787$&$ 1.630$&$0.478$&$0.878$&$0.864$&$-1.956$&$1.082$&$0.145$&$0.149$&$ 0.024$&$0.513$&$0.072$&$0.072$\tabularnewline
10189&$-0.028$&$0.161$&$0.808$&$0.797$&$-1.405$&$0.416$&$0.537$&$0.560$&$-0.572$&$0.411$&$0.191$&$0.192$&$-0.597$&$0.401$&$0.637$&$0.671$&$ 0.523$&$0.538$&$0.454$&$0.415$\tabularnewline
10190&$-0.291$&$0.166$&$0.595$&$0.593$&$-0.396$&$0.395$&$0.398$&$0.393$&$-0.912$&$0.418$&$1.482$&$1.470$&$ 0.674$&$0.445$&$1.164$&$1.371$&$-0.762$&$0.547$&$0.586$&$0.586$\tabularnewline
10191&$ 0.023$&$0.160$&$0.952$&$0.964$&$-1.104$&$0.391$&$0.400$&$0.381$&$-0.742$&$0.415$&$0.211$&$0.212$&$-1.956$&$1.082$&$0.145$&$0.149$&$-0.222$&$0.517$&$0.323$&$0.344$\tabularnewline
10192&$ 0.225$&$0.160$&$1.019$&$1.005$&$ 0.512$&$0.443$&$3.013$&$2.969$&$ 0.528$&$0.384$&$0.461$&$0.462$&$ 0.498$&$0.411$&$0.014$&$0.015$&$ 0.523$&$0.538$&$3.664$&$3.845$\tabularnewline
10193&$ 0.200$&$0.160$&$2.406$&$2.445$&$-2.287$&$0.601$&$0.524$&$0.561$&$ 3.716$&$1.382$&$0.102$&$0.107$&$-0.233$&$0.355$&$2.258$&$2.442$&$ 0.024$&$0.513$&$2.149$&$2.043$\tabularnewline
10197&$ 0.175$&$0.160$&$0.583$&$0.584$&$-0.072$&$0.416$&$1.626$&$1.626$&$ 0.668$&$0.384$&$0.926$&$0.923$&$ 0.098$&$0.359$&$0.473$&$0.459$&$-0.222$&$0.517$&$0.021$&$0.023$\tabularnewline
10198&$ 0.074$&$0.160$&$1.088$&$1.079$&$ 0.512$&$0.443$&$2.179$&$2.148$&$-0.402$&$0.408$&$3.557$&$3.492$&$ 0.498$&$0.411$&$0.014$&$0.015$&$ 0.523$&$0.538$&$0.525$&$0.577$\tabularnewline
10199&$-0.079$&$0.161$&$0.411$&$0.400$&$-0.543$&$0.389$&$0.639$&$0.666$&$-0.235$&$0.404$&$1.112$&$1.103$&$-0.014$&$0.353$&$0.212$&$0.207$&$-0.762$&$0.547$&$0.205$&$0.198$\tabularnewline
10200&$ 0.326$&$0.161$&$0.902$&$0.869$&$ 1.116$&$0.430$&$0.521$&$0.497$&$ 0.528$&$0.384$&$1.087$&$1.095$&$ 0.887$&$0.493$&$0.390$&$0.351$&$ 1.542$&$0.765$&$0.184$&$0.130$\tabularnewline
10201&$ 0.403$&$0.162$&$0.561$&$0.568$&$ 1.795$&$0.408$&$1.147$&$1.163$&$ 0.087$&$0.394$&$0.265$&$0.263$&$-0.124$&$0.351$&$0.208$&$0.204$&$ 0.024$&$0.513$&$0.258$&$0.245$\tabularnewline
10202&$ 0.200$&$0.160$&$0.369$&$0.382$&$-0.543$&$0.389$&$0.077$&$0.071$&$ 0.808$&$0.387$&$0.845$&$0.861$&$ 0.098$&$0.359$&$0.153$&$0.164$&$ 0.523$&$0.538$&$0.021$&$0.020$\tabularnewline
10203&$ 0.673$&$0.171$&$1.183$&$1.072$&$ 0.721$&$0.443$&$0.865$&$0.902$&$ 1.869$&$0.531$&$1.313$&$1.584$&$-0.014$&$0.353$&$1.566$&$1.531$&$ 2.330$&$1.198$&$0.174$&$0.222$\tabularnewline
10204&$-1.007$&$0.205$&$1.102$&$1.272$&$-3.681$&$1.306$&$0.087$&$0.095$&$-2.752$&$0.671$&$0.419$&$0.418$&$ 2.258$&$1.137$&$0.144$&$0.139$&$ 0.024$&$0.513$&$1.477$&$1.543$\tabularnewline
10206&$ 0.403$&$0.162$&$0.267$&$0.279$&$ 0.109$&$0.427$&$0.786$&$0.800$&$ 1.100$&$0.404$&$0.115$&$0.116$&$-0.124$&$0.351$&$0.081$&$0.081$&$ 0.523$&$0.538$&$0.021$&$0.020$\tabularnewline
10208&$ 0.403$&$0.162$&$0.126$&$0.130$&$ 0.305$&$0.437$&$0.108$&$0.111$&$ 0.528$&$0.384$&$0.445$&$0.449$&$-0.465$&$0.378$&$0.035$&$0.037$&$ 0.523$&$0.538$&$0.021$&$0.020$\tabularnewline
10209&$-0.237$&$0.165$&$1.597$&$1.644$&$-1.104$&$0.391$&$0.743$&$0.727$&$-2.752$&$0.671$&$0.419$&$0.418$&$-1.244$&$0.630$&$0.397$&$0.401$&$-0.222$&$0.517$&$0.435$&$0.409$\tabularnewline
10210&$-0.780$&$0.188$&$1.937$&$2.140$&$-3.681$&$1.306$&$0.087$&$0.095$&$-2.752$&$0.671$&$0.419$&$0.418$&$ 2.258$&$1.137$&$0.144$&$0.139$&$ 2.330$&$1.198$&$0.174$&$0.222$\tabularnewline
10213&$-0.079$&$0.161$&$2.095$&$2.075$&$ 0.305$&$0.437$&$3.401$&$3.535$&$-2.752$&$0.671$&$0.419$&$0.418$&$ 0.218$&$0.370$&$1.597$&$1.525$&$ 2.330$&$1.198$&$0.174$&$0.222$\tabularnewline
10214&$-0.375$&$0.169$&$1.851$&$1.920$&$-2.722$&$0.762$&$0.275$&$0.329$&$-1.261$&$0.432$&$2.481$&$2.574$&$ 0.349$&$0.386$&$1.469$&$1.526$&$ 2.330$&$1.198$&$0.174$&$0.222$\tabularnewline
10215&$ 0.377$&$0.161$&$1.305$&$1.319$&$ 0.512$&$0.443$&$1.219$&$1.216$&$ 0.951$&$0.394$&$1.433$&$1.442$&$-0.345$&$0.363$&$2.636$&$2.908$&$-0.481$&$0.528$&$2.122$&$2.112$\tabularnewline
10216&$ 0.824$&$0.180$&$0.669$&$0.717$&$ 1.297$&$0.422$&$2.059$&$1.974$&$ 1.258$&$0.420$&$0.508$&$0.499$&$-1.956$&$1.082$&$0.145$&$0.149$&$ 0.523$&$0.538$&$0.455$&$0.438$\tabularnewline
10217&$ 0.403$&$0.162$&$0.749$&$0.729$&$ 0.924$&$0.438$&$0.483$&$0.470$&$ 0.087$&$0.394$&$0.677$&$0.699$&$-1.956$&$1.082$&$0.145$&$0.149$&$-0.762$&$0.547$&$0.128$&$0.127$\tabularnewline
10218&$-0.028$&$0.161$&$0.361$&$0.358$&$ 0.512$&$0.443$&$0.056$&$0.057$&$-0.402$&$0.408$&$0.301$&$0.299$&$ 0.349$&$0.386$&$0.621$&$0.685$&$-0.762$&$0.547$&$0.205$&$0.198$\tabularnewline
10219&$-0.237$&$0.165$&$2.223$&$2.282$&$-1.574$&$0.438$&$1.936$&$1.702$&$-0.742$&$0.415$&$3.805$&$3.809$&$-1.956$&$1.082$&$0.145$&$0.149$&$-2.719$&$0.854$&$0.301$&$0.304$\tabularnewline
10220&$ 0.562$&$0.166$&$1.539$&$1.641$&$-0.963$&$0.386$&$1.536$&$1.549$&$ 1.869$&$0.531$&$1.941$&$2.034$&$-1.956$&$1.082$&$0.145$&$0.149$&$ 0.802$&$0.573$&$0.820$&$0.881$\tabularnewline
10221&$-0.131$&$0.162$&$0.539$&$0.554$&$-0.824$&$0.383$&$0.100$&$0.100$&$-0.912$&$0.418$&$0.933$&$0.923$&$-0.014$&$0.353$&$0.960$&$0.970$&$ 0.268$&$0.520$&$0.127$&$0.131$\tabularnewline
10223&$ 0.275$&$0.160$&$0.574$&$0.546$&$ 1.634$&$0.410$&$0.165$&$0.162$&$ 0.087$&$0.394$&$0.204$&$0.204$&$ 0.098$&$0.359$&$0.463$&$0.438$&$-0.481$&$0.528$&$0.829$&$0.774$\tabularnewline
10224&$-0.105$&$0.162$&$0.456$&$0.456$&$ 0.512$&$0.443$&$0.056$&$0.057$&$-0.402$&$0.408$&$0.176$&$0.177$&$ 0.887$&$0.493$&$0.363$&$0.325$&$-0.762$&$0.547$&$1.357$&$1.257$\tabularnewline
10226&$-1.237$&$0.228$&$0.625$&$0.561$&$-1.994$&$0.519$&$1.113$&$1.469$&$-2.752$&$0.671$&$0.419$&$0.418$&$ 2.258$&$1.137$&$0.144$&$0.139$&$-2.719$&$0.854$&$0.301$&$0.304$\tabularnewline
10227&$ 0.175$&$0.160$&$0.643$&$0.614$&$ 0.924$&$0.438$&$1.125$&$1.129$&$ 0.385$&$0.386$&$0.160$&$0.163$&$ 0.218$&$0.370$&$0.636$&$0.640$&$-0.762$&$0.547$&$0.949$&$1.009$\tabularnewline
10228&$ 0.049$&$0.160$&$1.940$&$1.938$&$-2.287$&$0.601$&$0.351$&$0.333$&$ 1.869$&$0.531$&$0.406$&$0.381$&$ 0.098$&$0.359$&$1.548$&$1.637$&$ 0.024$&$0.513$&$2.322$&$2.252$\tabularnewline
10230&$ 0.149$&$0.160$&$0.660$&$0.668$&$ 1.116$&$0.430$&$0.165$&$0.163$&$-0.071$&$0.399$&$0.641$&$0.645$&$ 0.349$&$0.386$&$1.326$&$1.429$&$-0.222$&$0.517$&$0.640$&$0.621$\tabularnewline
10231&$ 0.703$&$0.173$&$1.179$&$1.084$&$ 1.116$&$0.430$&$0.934$&$0.913$&$ 1.869$&$0.531$&$0.438$&$0.479$&$-1.956$&$1.082$&$0.145$&$0.149$&$-1.073$&$0.572$&$0.700$&$0.652$\tabularnewline
10232&$ 0.481$&$0.164$&$0.289$&$0.281$&$ 1.116$&$0.430$&$0.227$&$0.222$&$ 0.385$&$0.386$&$0.302$&$0.298$&$-0.233$&$0.355$&$0.496$&$0.538$&$ 1.124$&$0.635$&$0.408$&$0.398$\tabularnewline
10234&$-0.613$&$0.179$&$1.334$&$1.408$&$-1.766$&$0.470$&$0.225$&$0.193$&$-2.093$&$0.524$&$0.271$&$0.262$&$-0.465$&$0.378$&$0.843$&$0.894$&$-2.719$&$0.854$&$0.301$&$0.304$\tabularnewline
10237&$ 0.023$&$0.160$&$0.698$&$0.689$&$ 0.512$&$0.443$&$1.727$&$1.715$&$ 0.087$&$0.394$&$0.793$&$0.764$&$ 0.887$&$0.493$&$0.390$&$0.351$&$-0.222$&$0.517$&$0.323$&$0.344$\tabularnewline
10238&$-0.105$&$0.162$&$0.666$&$0.673$&$ 0.305$&$0.437$&$0.108$&$0.111$&$-1.085$&$0.424$&$0.563$&$0.535$&$ 1.152$&$0.567$&$0.480$&$0.520$&$ 0.802$&$0.573$&$0.197$&$0.151$\tabularnewline
\hline
\end{longtable}}\end{landscape}

\newpage
%%%%%%%%%%%%%%%%%%%%%%%%%%%%%%%%%%%%%%%%%%%%%%%%%%%
\section{RSM-based Instrument for Measuring the Level of Motivation in the Third Empirical Study}
\label{sec:irt-level-motivation-third-study}


\subsection{Checking Assumptions}

\subsubsection*{Test of Unidimensionality}

\autoref{tab:test-unidimensionality-irt-motivation-third-study} shows the results for the test of unidimensionality in which the goodness of fit statistics indicate a strong multidimensionality ($DETECT > 1.00$) for the \emph{Level of Motivation} with a DETECT index of $6.634$.
Essential unidimensionality ($ASSI < 0.25$ and $RATIO < 0.36$) in the data structure is indicated for the level of motivation by the ASSI and RATIO indices with values of $0.242$ and $0.279$, respectively.
The index of $AGFI = 0.980$ in the unidimensional CFA indicates an acceptable fit for measuring the \emph{Level of Motivation}.
The sub-scales of \emph{Attention}, \emph{Relevance}, and \emph{Satisfaction} have a good fit indicated by the AGFI index with values greater than $0.95$.
A good fit with the unidimensional CFA is indicated by the TLI and CFI indices for all the sub-scales.
The ASSI and RATIO indices indicate essential deviation from unidimensionality in all the sub-scales.

%latex.default(data_df, caption = paste("Goodness of fit statistics related to the test of unidimensionality",     "in RSM-based instruments", in_title), size = "small", longtable = T,     ctable = F, landscape = F, where = "!htbp", file = filename,     append = T)%
\setlongtables{\scriptsize
\begin{longtable}{lrrrrrrrr}\caption{Goodness of fit statistics related to the test of unidimensionality in the RSM-based instrument for measuring the level of motivation in the third empirical study}
\tabularnewline
\hline\hline
\multicolumn{1}{l}{}&\multicolumn{1}{c}{df}&\multicolumn{1}{c}{chisq}&\multicolumn{1}{c}{AGFI}&\multicolumn{1}{c}{TLI}&\multicolumn{1}{c}{CFI}&\multicolumn{1}{c}{DETECT}&\multicolumn{1}{c}{ASSI}&\multicolumn{1}{c}{RATIO}\tabularnewline
\hline
\endfirsthead\caption[]{\em (continued)} \tabularnewline
\hline
\multicolumn{1}{l}{}&\multicolumn{1}{c}{df}&\multicolumn{1}{c}{chisq}&\multicolumn{1}{c}{AGFI}&\multicolumn{1}{c}{TLI}&\multicolumn{1}{c}{CFI}&\multicolumn{1}{c}{DETECT}&\multicolumn{1}{c}{ASSI}&\multicolumn{1}{c}{RATIO}\tabularnewline
\hline
\endhead
\hline
\multicolumn{9}{r}{\tiny df: degree of freedom; AGFI: Adjusted Goodness of Fit Index; CFI: Comparative Fit Index; TLI: Tucker-Lewis Index;}
\endfoot
\label{tab:test-unidimensionality-irt-motivation-third-study}
Level of Motivation&$9.196$&$20.051$&$0.980$&$0.844$&$0.749$&$ 6.634$&$0.242$&$0.279$\tabularnewline
Attention&$4.337$&$ 3.955$&$0.997$&$1.006$&$1.000$&$ 7.166$&$0.333$&$0.621$\tabularnewline
Relevance&$1.762$&$ 1.090$&$0.997$&$1.063$&$1.000$&$14.903$&$0.333$&$0.529$\tabularnewline
Satisfaction&$0.000$&$ 0.000$&$1.000$&$1.000$&$1.000$&$18.022$&$0.333$&$0.939$\tabularnewline
\hline
\end{longtable}}

\subsubsection*{Test of Local Independence}

Results from the test of local independence in the RSM-based instrument for measuring the level of motivation in the third empirical study are summarized in \autoref{tab:test-local-independence-irt-motivation-third-study}.
The null condition of local independence is not rejected in all the sub-scales.
The Standardized Root Mean Squared Residual (SRMSR) indicates good fits ($< 0.10$) for all the sub-scales.

%latex.default(data_df, caption = paste("Q3 statistics related to the test of local independence",     "in the RSM-based instrument", in_title), size = "small",     longtable = T, ctable = F, landscape = F, where = "!htbp",     file = filename, append = T)%
\setlongtables{\scriptsize
\begin{longtable}{lrrrrr}\caption{Item residual correlation statistics related to the test of local independence in the RSM-based instrument for measuring the level of motivation in the third empirical study} \tabularnewline
\hline\hline
\multicolumn{1}{l}{}&\multicolumn{1}{c}{max.chisq}&\multicolumn{1}{c}{maxaQ3}&\multicolumn{1}{c}{MADaQ3}&\multicolumn{1}{c}{SRMSR}&\multicolumn{1}{c}{p.value}\tabularnewline
\hline
\endfirsthead\caption[]{\em (continued)} \tabularnewline
\hline
\multicolumn{1}{l}{}&\multicolumn{1}{c}{max.chisq}&\multicolumn{1}{c}{maxaQ3}&\multicolumn{1}{c}{MADaQ3}&\multicolumn{1}{c}{SRMSR}&\multicolumn{1}{c}{p.value}\tabularnewline
\hline
\endhead
\hline
\multicolumn{6}{r}{\tiny aQ3: adjusted correlation of item residuals; maxaQ3: maximum aQ3;}\tabularnewline
\multicolumn{6}{r}{\tiny MADaQ3: Median Absolute Deviation of aQ3;}
\endfoot
\label{tab:test-local-independence-irt-motivation-third-study}
Level of Motivation&$138.794$&$0.601$&$0.201$&$0.213$&$0.002$\tabularnewline
Attention&$282.594$&$0.335$&$0.139$&$0.086$&$0.284$\tabularnewline
Relevance&$110.345$&$0.176$&$0.072$&$0.066$&$1.000$\tabularnewline
Satisfaction&$ 62.109$&$0.282$&$0.188$&$0.073$&$0.144$\tabularnewline
\hline
\end{longtable}}
%\begin{flushright}{\tiny Q3: correlation of item residuals; maxaQ3: maximum adjusted correlation Q3; MADaQ3: Median Absolute Deviation in the adjusted Q3; }\end{flushright}

\subsubsection*{Test of Monotonicity}

\autoref{tab:test-monotonicity-irt-motivation-third-study} summarizes the test of monotonicity in the RSM-based instrument for measuring the level of motivation in the third empirical study. These results indicates that there are no one violation of monotonicity in the items at the significance level $\alpha = 0.05$.

%latex.default(data_df, caption = paste("Summary of the violations of monotonicity",     "in the RSM-based instrument", in_title), size = "small",     longtable = T, ctable = F, landscape = F, where = "!htbp",     file = filename, append = T)%
\setlongtables{\scriptsize
\begin{longtable}{lrrrrrrrrrr}\caption{Test of monotonicity in the RSM-based instrument for measuring the level of motivation in the third empirical study} \tabularnewline
\hline\hline
\multicolumn{1}{l}{}&\multicolumn{1}{c}{ItemH}&\multicolumn{1}{c}{ac}&\multicolumn{1}{c}{vi}&\multicolumn{1}{c}{vi/ac}&\multicolumn{1}{c}{maxvi}&\multicolumn{1}{c}{sum}&\multicolumn{1}{c}{sum/ac}&\multicolumn{1}{c}{zmax}&\multicolumn{1}{c}{zsig}&\multicolumn{1}{c}{crit}\tabularnewline
\hline
\endfirsthead\caption[]{\em (continued)} \tabularnewline
\hline
\multicolumn{1}{l}{}&\multicolumn{1}{c}{ItemH}&\multicolumn{1}{c}{ac}&\multicolumn{1}{c}{vi}&\multicolumn{1}{c}{vi/ac}&\multicolumn{1}{c}{maxvi}&\multicolumn{1}{c}{sum}&\multicolumn{1}{c}{sum/ac}&\multicolumn{1}{c}{zmax}&\multicolumn{1}{c}{zsig}&\multicolumn{1}{c}{crit}\tabularnewline
\hline
\endhead
\hline
\multicolumn{11}{r}{\tiny vi: number of violations; vi/ac: proportion of active pairs; maxvi: maximum violations;}\tabularnewline
\multicolumn{11}{r}{\tiny sum: sum of all violations; zmax: maximum z-value; zsig: number of significant z-values; crit: critical value}
\endfoot
\label{tab:test-monotonicity-irt-motivation-third-study}
Level of Motivation.Item12A&$0.53$&$0$&$0$&$$&$0$&$0$&$$&$0$&$0$&$0$\tabularnewline
Level of Motivation.Item19A&$0.47$&$0$&$0$&$$&$0$&$0$&$$&$0$&$0$&$0$\tabularnewline
Level of Motivation.Item04A&$0.47$&$0$&$0$&$$&$0$&$0$&$$&$0$&$0$&$0$\tabularnewline
Level of Motivation.Item20A&$0.45$&$0$&$0$&$$&$0$&$0$&$$&$0$&$0$&$0$\tabularnewline
Level of Motivation.Item16A&$0.51$&$5$&$0$&$  0$&$0$&$0$&$  0$&$0$&$0$&$0$\tabularnewline
Level of Motivation.Item01A&$0.44$&$0$&$0$&$$&$0$&$0$&$$&$0$&$0$&$0$\tabularnewline
Level of Motivation.Item21R&$0.24$&$0$&$0$&$$&$0$&$0$&$$&$0$&$0$&$0$\tabularnewline
Level of Motivation.Item10R&$0.29$&$0$&$0$&$$&$0$&$0$&$$&$0$&$0$&$0$\tabularnewline
Level of Motivation.Item08R&$0.12$&$0$&$0$&$$&$0$&$0$&$$&$0$&$0$&$0$\tabularnewline
Level of Motivation.Item13S&$0.49$&$0$&$0$&$$&$0$&$0$&$$&$0$&$0$&$0$\tabularnewline
Level of Motivation.Item14S&$0.52$&$3$&$0$&$  0$&$0$&$0$&$  0$&$0$&$0$&$0$\tabularnewline
Level of Motivation.Item17S&$0.42$&$0$&$0$&$$&$0$&$0$&$$&$0$&$0$&$0$\tabularnewline
Attention.Item12A&$0.68$&$5$&$0$&$  0$&$0$&$0$&$  0$&$0$&$0$&$0$\tabularnewline
Attention.Item19A&$0.66$&$0$&$0$&$$&$0$&$0$&$$&$0$&$0$&$0$\tabularnewline
Attention.Item04A&$0.66$&$0$&$0$&$$&$0$&$0$&$$&$0$&$0$&$0$\tabularnewline
Attention.Item20A&$0.59$&$0$&$0$&$$&$0$&$0$&$$&$0$&$0$&$0$\tabularnewline
Attention.Item16A&$0.66$&$4$&$0$&$  0$&$0$&$0$&$  0$&$0$&$0$&$0$\tabularnewline
Attention.Item01A&$0.54$&$0$&$0$&$$&$0$&$0$&$$&$0$&$0$&$0$\tabularnewline
Relevance.Item15R&$0.35$&$0$&$0$&$$&$0$&$0$&$$&$0$&$0$&$0$\tabularnewline
Relevance.Item21R&$0.41$&$0$&$0$&$$&$0$&$0$&$$&$0$&$0$&$0$\tabularnewline
Relevance.Item10R&$0.42$&$0$&$0$&$$&$0$&$0$&$$&$0$&$0$&$0$\tabularnewline
Relevance.Item08R&$0.35$&$0$&$0$&$$&$0$&$0$&$$&$0$&$0$&$0$\tabularnewline
Satisfaction.Item13S&$0.77$&$3$&$0$&$  0$&$0$&$0$&$  0$&$0$&$0$&$0$\tabularnewline
Satisfaction.Item14S&$0.75$&$0$&$0$&$$&$0$&$0$&$$&$0$&$0$&$0$\tabularnewline
Satisfaction.Item17S&$0.68$&$4$&$0$&$  0$&$0$&$0$&$  0$&$0$&$0$&$0$\tabularnewline
\hline
\end{longtable}}


\subsection{Item Parameters}

\autoref{tab:item-parameters-attention-third-study} shows the estimated parameters for the RSM-based instrument used to measure the \emph{Attention} in the third empirical study.
These parameters had been calculated using the MML method \cite{BockAitkin1981}, so that the value in row \aspas{B.Cat$x$} and column \aspas{$i$} is the item slope $b_{i,x}$ of item $i$ in the category \aspas{$x$}, and the value in the row \aspas{AXsi.Cat$x$} and column \aspas{$i$} is the item intercept $a_{i,x}\xi$ of item $i$ in the category \aspas{$x$}.
According to the Infit/Outfit statistics of items, no one mean-square value is greater than $2.0$ indicating that the measurement system of \emph{Attention} is not distorted or degraded by the items.

%latex.default(estimated_params_df, caption = paste("Estimated parameters in the RSM-based instrument",     "for measuring the", lname), size = "small", longtable = T,     ctable = F, landscape = F, where = "!htbp", file = filename,     append = T)%
\setlongtables{\scriptsize
\begin{longtable}{lrrrrrr}\caption{Estimated parameters in the RSM-based instrument for measuring the attention in the third empirical study} \tabularnewline
\hline\hline
\multicolumn{1}{l}{}&\multicolumn{1}{c}{Item01A}&\multicolumn{1}{c}{Item04A}&\multicolumn{1}{c}{Item12A}&\multicolumn{1}{c}{Item16A}&\multicolumn{1}{c}{Item19A}&\multicolumn{1}{c}{Item20A}\tabularnewline
\hline
\endfirsthead\caption[]{\em (continued)} \tabularnewline
\hline
\multicolumn{1}{l}{}&\multicolumn{1}{c}{Item01A}&\multicolumn{1}{c}{Item04A}&\multicolumn{1}{c}{Item12A}&\multicolumn{1}{c}{Item16A}&\multicolumn{1}{c}{Item19A}&\multicolumn{1}{c}{Item20A}\tabularnewline
\hline
\endhead
\hline
\endfoot
\label{tab:item-parameters-attention-third-study}
xsi.item&$ 0.321$&$ 0.340$&$ 0.303$&$-0.029$&$ 0.229$&$ 0.376$\tabularnewline
B.Cat0&$ 0.000$&$ 0.000$&$ 0.000$&$ 0.000$&$ 0.000$&$ 0.000$\tabularnewline
B.Cat1&$ 1.000$&$ 1.000$&$ 1.000$&$ 1.000$&$ 1.000$&$ 1.000$\tabularnewline
B.Cat2&$ 2.000$&$ 2.000$&$ 2.000$&$ 2.000$&$ 2.000$&$ 2.000$\tabularnewline
B.Cat3&$ 3.000$&$ 3.000$&$ 3.000$&$ 3.000$&$ 3.000$&$ 3.000$\tabularnewline
B.Cat4&$ 4.000$&$ 4.000$&$ 4.000$&$ 4.000$&$ 4.000$&$ 4.000$\tabularnewline
B.Cat5&$ 5.000$&$ 5.000$&$ 5.000$&$ 5.000$&$ 5.000$&$ 5.000$\tabularnewline
B.Cat6&$ 6.000$&$ 6.000$&$ 6.000$&$ 6.000$&$ 6.000$&$ 6.000$\tabularnewline
AXsi.Cat0&$ 0.000$&$ 0.000$&$ 0.000$&$ 0.000$&$ 0.000$&$ 0.000$\tabularnewline
AXsi.Cat1&$ 1.292$&$ 1.274$&$ 1.311$&$ 1.643$&$ 1.384$&$ 1.237$\tabularnewline
AXsi.Cat2&$ 1.608$&$ 1.571$&$ 1.645$&$ 2.309$&$ 1.792$&$ 1.497$\tabularnewline
AXsi.Cat3&$ 2.891$&$ 2.836$&$ 2.946$&$ 3.943$&$ 3.167$&$ 2.725$\tabularnewline
AXsi.Cat4&$ 1.786$&$ 1.712$&$ 1.859$&$ 3.188$&$ 2.154$&$ 1.565$\tabularnewline
AXsi.Cat5&$-0.044$&$-0.136$&$ 0.048$&$ 1.709$&$ 0.416$&$-0.320$\tabularnewline
AXsi.Cat6&$-1.928$&$-2.038$&$-1.817$&$ 0.176$&$-1.375$&$-2.259$\tabularnewline
\hline
Outfit&$ 1.231$&$ 0.856$&$ 0.781$&$ 0.795$&$ 1.244$&$ 1.013$\tabularnewline
Infit&$ 1.207$&$ 0.838$&$ 0.824$&$ 0.806$&$ 1.294$&$ 1.105$\tabularnewline
\hline
\end{longtable}}

\autoref{tab:item-parameters-relevance-third-study} shows the estimated parameters for the measurement instrument of \emph{Relevance} in which the Infit/Outfit statistics of items indicate that no one item distorts or degrades the measurement system with mean-square greater than $2.0$.

%latex.default(estimated_params_df, caption = paste("Estimated parameters in the RSM-based instrument",     "for measuring the", lname), size = "small", longtable = T,     ctable = F, landscape = F, where = "!htbp", file = filename,     append = T)%
\setlongtables{\scriptsize
\begin{longtable}{lrrrrr}\caption{Estimated parameters in the RSM-based instrument for measuring the relevance in the third empirical study} \tabularnewline
\hline\hline
\multicolumn{1}{l}{}&\multicolumn{1}{c}{Item08R}&\multicolumn{1}{c}{Item10R}&\multicolumn{1}{c}{Item15R}&\multicolumn{1}{c}{Item21R}\tabularnewline
\hline
\endfirsthead\caption[]{\em (continued)} \tabularnewline
\hline
\multicolumn{1}{l}{}&\multicolumn{1}{c}{Item08R}&\multicolumn{1}{c}{Item10R}&\multicolumn{1}{c}{Item15R}&\multicolumn{1}{c}{Item21R}\tabularnewline
\hline
\endhead
\hline
\endfoot
\label{tab:item-parameters-relevance-third-study}
xsi.item&$-0.953$&$-0.773$&$-0.539$&$-0.677$\tabularnewline
B.Cat0&$ 0.000$&$ 0.000$&$ 0.000$&$ 0.000$\tabularnewline
B.Cat1&$ 1.000$&$ 1.000$&$ 1.000$&$ 1.000$\tabularnewline
B.Cat2&$ 2.000$&$ 2.000$&$ 2.000$&$ 2.000$\tabularnewline
B.Cat3&$ 3.000$&$ 3.000$&$ 3.000$&$ 3.000$\tabularnewline
B.Cat4&$ 4.000$&$ 4.000$&$ 4.000$&$ 4.000$\tabularnewline
B.Cat5&$ 5.000$&$ 5.000$&$ 5.000$&$ 5.000$\tabularnewline
B.Cat6&$ 6.000$&$ 6.000$&$ 6.000$&$ 6.000$\tabularnewline
AXsi.Cat0&$ 0.000$&$ 0.000$&$ 0.000$&$ 0.000$\tabularnewline
AXsi.Cat1&$ 3.925$&$ 3.745$&$ 3.511$&$ 3.649$\tabularnewline
AXsi.Cat2&$ 4.994$&$ 4.634$&$ 4.166$&$ 4.441$\tabularnewline
AXsi.Cat3&$ 6.466$&$ 5.925$&$ 5.223$&$ 5.636$\tabularnewline
AXsi.Cat4&$ 6.291$&$ 5.570$&$ 4.634$&$ 5.185$\tabularnewline
AXsi.Cat5&$ 6.096$&$ 5.195$&$ 4.024$&$ 4.713$\tabularnewline
AXsi.Cat6&$ 5.718$&$ 4.637$&$ 3.232$&$ 4.059$\tabularnewline
\hline
Outfit&$ 1.124$&$ 0.896$&$ 1.014$&$ 0.945$\tabularnewline
Infit&$ 1.142$&$ 0.898$&$ 0.986$&$ 0.960$\tabularnewline
\hline
\end{longtable}}


\autoref{tab:item-parameters-satisfaction-third-study} shows the estimated parameters for the measurement instrument of \emph{Satisfaction} in the third empirical study in which the Infit/Outfit statistics of items indicate that no one item distorts or degrades the measurement system with mean-square greater than $2.0$.

%latex.default(estimated_params_df, caption = paste("Estimated parameters in the RSM-based instrument",     "for measuring the", lname), size = "small", longtable = T,     ctable = F, landscape = F, where = "!htbp", file = filename,     append = T)%
\setlongtables{\scriptsize
\begin{longtable}{lrrr}\caption{Estimated parameters in the RSM-based instrument for measuring the satisfaction in the third empirical study} \tabularnewline
\hline\hline
\multicolumn{1}{l}{estimated}&\multicolumn{1}{c}{Item13S}&\multicolumn{1}{c}{Item14S}&\multicolumn{1}{c}{Item17S}\tabularnewline
\hline
\endfirsthead\caption[]{\em (continued)} \tabularnewline
\hline
\multicolumn{1}{l}{estimated}&\multicolumn{1}{c}{Item13S}&\multicolumn{1}{c}{Item14S}&\multicolumn{1}{c}{Item17S}\tabularnewline
\hline
\endhead
\hline
\endfoot
\label{tab:item-parameters-satisfaction-third-study}
xsi.item&$-0.267$&$-0.047$&$-0.378$\tabularnewline
B.Cat0&$ 0.000$&$ 0.000$&$ 0.000$\tabularnewline
B.Cat1&$ 1.000$&$ 1.000$&$ 1.000$\tabularnewline
B.Cat2&$ 2.000$&$ 2.000$&$ 2.000$\tabularnewline
B.Cat3&$ 3.000$&$ 3.000$&$ 3.000$\tabularnewline
B.Cat4&$ 4.000$&$ 4.000$&$ 4.000$\tabularnewline
B.Cat5&$ 5.000$&$ 5.000$&$ 5.000$\tabularnewline
B.Cat6&$ 6.000$&$ 6.000$&$ 6.000$\tabularnewline
AXsi.Cat0&$ 0.000$&$ 0.000$&$ 0.000$\tabularnewline
AXsi.Cat1&$ 2.315$&$ 2.095$&$ 2.426$\tabularnewline
AXsi.Cat2&$ 4.032$&$ 3.592$&$ 4.253$\tabularnewline
AXsi.Cat3&$ 5.337$&$ 4.677$&$ 5.669$\tabularnewline
AXsi.Cat4&$ 4.692$&$ 3.813$&$ 5.135$\tabularnewline
AXsi.Cat5&$ 3.596$&$ 2.496$&$ 4.149$\tabularnewline
AXsi.Cat6&$ 1.602$&$ 0.282$&$ 2.266$\tabularnewline
\hline
Outfit&$ 0.944$&$ 0.921$&$ 1.176$\tabularnewline
Infit&$ 0.938$&$ 0.889$&$ 1.163$\tabularnewline
\hline
\end{longtable}}

\subsection{Level of Motivation as Latent Trait Estimates}

\autoref{tab:level-motivation-estimates-third-study} shows the latent trait estimates by the RSM-based instrument for measuring the \emph{Level of Motivation} in the third empirical study.

%latex.default(data_df, caption = paste("Latent trait estimates and person model fit of the RSM-based instrument",     in_title), size = "scriptsize", longtable = T, ctable = F,     landscape = T, rowlabel = "", where = "!htbp", file = filename,     append = T)%
\setlongtables\begin{landscape}{\scriptsize
\begin{longtable}{l|rrrr|rrrr|rrrr|rrrr}\caption{Latent trait estimates and person model fit of the RSM-based instrument for measuring the level of motivation in the third empirical study} \tabularnewline
\hline\hline
\multicolumn{1}{l}{}&\multicolumn{4}{|c}{Level of Motivation}&\multicolumn{4}{|c}{Attention}&\multicolumn{4}{|c}{Relevance}&\multicolumn{4}{|c}{Satisfaction} \tabularnewline
\multicolumn{1}{l}{UserID}&\multicolumn{1}{|c}{theta}&\multicolumn{1}{c}{error}&\multicolumn{1}{c}{Outfit}&\multicolumn{1}{c}{Infit}&\multicolumn{1}{|c}{theta}&\multicolumn{1}{c}{error}&\multicolumn{1}{c}{Outfit}&\multicolumn{1}{c}{Infit}&\multicolumn{1}{|c}{theta}&\multicolumn{1}{c}{error}&\multicolumn{1}{c}{Outfit}&\multicolumn{1}{c}{Infit}&\multicolumn{1}{|c}{theta}&\multicolumn{1}{c}{error}&\multicolumn{1}{c}{Outfit}&\multicolumn{1}{c}{Infit}\tabularnewline
\hline
\endfirsthead\caption[]{\em (continued)} \tabularnewline
\hline
\multicolumn{1}{l}{}&\multicolumn{4}{|c}{Level of Motivation}&\multicolumn{4}{|c}{Attention}&\multicolumn{4}{|c}{Relevance}&\multicolumn{4}{|c}{Satisfaction} \tabularnewline
\multicolumn{1}{l}{UserID}&\multicolumn{1}{|c}{theta}&\multicolumn{1}{c}{error}&\multicolumn{1}{c}{Outfit}&\multicolumn{1}{c}{Infit}&\multicolumn{1}{|c}{theta}&\multicolumn{1}{c}{error}&\multicolumn{1}{c}{Outfit}&\multicolumn{1}{c}{Infit}&\multicolumn{1}{|c}{theta}&\multicolumn{1}{c}{error}&\multicolumn{1}{c}{Outfit}&\multicolumn{1}{c}{Infit}&\multicolumn{1}{|c}{theta}&\multicolumn{1}{c}{error}&\multicolumn{1}{c}{Outfit}&\multicolumn{1}{c}{Infit}\tabularnewline
\hline
\endhead
\hline
\endfoot
\label{tab:level-motivation-estimates-third-study}
10169&$ 2.348$&$0.490$&$2.109$&$1.238$&$ 4.278$&$1.300$&$0.087$&$0.088$&$ 0.527$&$0.414$&$0.808$&$0.827$&$ 3.858$&$1.474$&$0.162$&$0.165$\tabularnewline
10170&$ 0.329$&$0.228$&$0.723$&$0.690$&$ 0.844$&$0.411$&$0.646$&$0.647$&$ 0.374$&$0.405$&$1.525$&$1.464$&$ 0.133$&$0.611$&$0.166$&$0.168$\tabularnewline
10171&$ 0.820$&$0.243$&$0.388$&$0.375$&$ 1.167$&$0.395$&$0.342$&$0.338$&$ 1.074$&$0.501$&$0.152$&$0.122$&$ 1.115$&$0.573$&$1.097$&$1.090$\tabularnewline
10172&$ 0.226$&$0.226$&$1.159$&$1.121$&$ 1.167$&$0.395$&$0.285$&$0.284$&$-1.153$&$0.469$&$0.028$&$0.029$&$ 0.817$&$0.574$&$0.534$&$0.532$\tabularnewline
10174&$ 0.381$&$0.230$&$1.352$&$1.334$&$ 1.314$&$0.389$&$1.038$&$1.034$&$-0.761$&$0.437$&$0.810$&$0.808$&$ 0.817$&$0.574$&$0.744$&$0.752$\tabularnewline
10175&$ 1.000$&$0.253$&$0.322$&$0.305$&$ 1.314$&$0.389$&$0.321$&$0.322$&$ 1.074$&$0.501$&$0.500$&$0.498$&$ 1.416$&$0.588$&$0.354$&$0.350$\tabularnewline
10176&$ 0.025$&$0.222$&$0.287$&$0.288$&$ 0.470$&$0.421$&$0.296$&$0.296$&$-0.404$&$0.416$&$0.033$&$0.034$&$-0.292$&$0.620$&$0.016$&$0.016$\tabularnewline
10178&$-0.168$&$0.219$&$0.329$&$0.327$&$ 0.095$&$0.402$&$0.019$&$0.019$&$-0.761$&$0.437$&$0.429$&$0.409$&$-1.049$&$0.571$&$0.940$&$0.943$\tabularnewline
10179&$-0.073$&$0.220$&$0.504$&$0.507$&$-0.344$&$0.361$&$0.909$&$0.889$&$ 0.374$&$0.405$&$0.439$&$0.449$&$-0.292$&$0.620$&$0.717$&$0.712$\tabularnewline
10181&$-0.263$&$0.218$&$1.243$&$1.244$&$ 0.276$&$0.415$&$0.362$&$0.363$&$-0.078$&$0.403$&$0.581$&$0.583$&$-2.532$&$0.669$&$1.000$&$1.031$\tabularnewline
10183&$-0.876$&$0.228$&$0.138$&$0.148$&$-1.247$&$0.368$&$0.036$&$0.039$&$-1.153$&$0.469$&$0.268$&$0.261$&$-1.616$&$0.549$&$0.626$&$0.629$\tabularnewline
10184&$-0.024$&$0.221$&$0.501$&$0.502$&$-0.068$&$0.387$&$0.788$&$0.786$&$ 0.075$&$0.400$&$1.426$&$1.410$&$ 0.499$&$0.588$&$0.888$&$0.897$\tabularnewline
10185&$-0.168$&$0.219$&$1.030$&$1.007$&$-0.068$&$0.387$&$1.322$&$1.251$&$-0.761$&$0.437$&$1.270$&$1.272$&$ 0.499$&$0.588$&$0.274$&$0.273$\tabularnewline
10186&$ 0.277$&$0.227$&$2.241$&$2.257$&$-0.344$&$0.361$&$2.834$&$2.891$&$ 0.527$&$0.414$&$1.947$&$1.952$&$ 2.142$&$0.706$&$1.615$&$1.556$\tabularnewline
10188&$-0.073$&$0.220$&$0.845$&$0.818$&$-0.465$&$0.351$&$0.395$&$0.396$&$ 1.336$&$0.575$&$0.444$&$0.451$&$-1.343$&$0.553$&$0.021$&$0.021$\tabularnewline
10189&$-0.927$&$0.231$&$1.131$&$1.219$&$-2.220$&$0.603$&$0.452$&$0.438$&$-0.579$&$0.425$&$3.434$&$3.448$&$-1.616$&$0.549$&$0.101$&$0.100$\tabularnewline
10190&$ 0.025$&$0.222$&$0.989$&$0.992$&$ 1.011$&$0.402$&$0.801$&$0.787$&$-0.952$&$0.450$&$0.068$&$0.071$&$-0.704$&$0.599$&$0.282$&$0.285$\tabularnewline
10191&$-0.216$&$0.218$&$0.673$&$0.648$&$-0.465$&$0.351$&$0.534$&$0.549$&$-0.238$&$0.409$&$1.454$&$1.492$&$ 0.133$&$0.611$&$0.166$&$0.168$\tabularnewline
10192&$ 0.879$&$0.246$&$1.605$&$1.489$&$ 1.314$&$0.389$&$1.664$&$1.657$&$-0.078$&$0.403$&$1.499$&$1.489$&$ 2.705$&$0.876$&$0.441$&$0.438$\tabularnewline
10193&$-0.876$&$0.228$&$2.397$&$2.606$&$-2.710$&$0.795$&$0.624$&$0.593$&$ 1.727$&$0.726$&$0.478$&$0.480$&$-3.028$&$0.827$&$0.650$&$0.632$\tabularnewline
10197&$ 0.277$&$0.227$&$1.831$&$1.831$&$ 1.314$&$0.389$&$1.805$&$1.801$&$ 0.225$&$0.400$&$1.741$&$1.747$&$-1.049$&$0.571$&$1.104$&$1.087$\tabularnewline
10198&$ 1.270$&$0.276$&$0.907$&$0.934$&$ 1.881$&$0.399$&$1.273$&$1.284$&$ 0.374$&$0.405$&$0.301$&$0.306$&$ 2.705$&$0.876$&$0.673$&$0.662$\tabularnewline
10199&$-0.024$&$0.221$&$0.694$&$0.706$&$-0.344$&$0.361$&$1.124$&$1.177$&$-0.078$&$0.403$&$0.297$&$0.302$&$ 0.499$&$0.588$&$0.888$&$0.897$\tabularnewline
10200&$ 0.762$&$0.241$&$0.348$&$0.343$&$ 1.167$&$0.395$&$0.295$&$0.296$&$ 0.374$&$0.405$&$0.052$&$0.050$&$ 1.745$&$0.628$&$0.774$&$0.741$\tabularnewline
10201&$ 0.434$&$0.231$&$1.043$&$0.999$&$ 1.011$&$0.402$&$1.198$&$1.200$&$-0.761$&$0.437$&$0.892$&$0.883$&$ 1.416$&$0.588$&$0.126$&$0.128$\tabularnewline
10202&$-0.121$&$0.220$&$0.530$&$0.531$&$-0.579$&$0.345$&$0.545$&$0.552$&$-0.078$&$0.403$&$0.558$&$0.558$&$ 0.817$&$0.574$&$0.019$&$0.019$\tabularnewline
10203&$ 0.487$&$0.232$&$0.873$&$0.864$&$ 0.662$&$0.419$&$0.626$&$0.630$&$-0.404$&$0.416$&$0.174$&$0.172$&$ 2.142$&$0.706$&$1.419$&$1.418$\tabularnewline
10204&$-1.149$&$0.249$&$1.634$&$1.851$&$-3.843$&$1.425$&$0.084$&$0.085$&$-0.238$&$0.409$&$2.292$&$2.289$&$-1.887$&$0.562$&$1.179$&$1.179$\tabularnewline
10206&$ 0.434$&$0.231$&$0.250$&$0.253$&$ 0.662$&$0.419$&$0.572$&$0.566$&$ 0.075$&$0.400$&$0.058$&$0.057$&$ 1.115$&$0.573$&$0.141$&$0.140$\tabularnewline
10208&$-0.541$&$0.218$&$0.598$&$0.605$&$-1.012$&$0.347$&$0.144$&$0.141$&$-0.761$&$0.437$&$0.862$&$0.877$&$ 0.133$&$0.611$&$1.194$&$1.209$\tabularnewline
10209&$-0.728$&$0.222$&$0.750$&$0.775$&$-1.126$&$0.356$&$0.141$&$0.136$&$-0.238$&$0.409$&$1.818$&$1.867$&$-1.887$&$0.562$&$0.131$&$0.134$\tabularnewline
10210&$-2.299$&$0.486$&$0.310$&$0.352$&$-3.843$&$1.425$&$0.084$&$0.085$&$-1.625$&$0.536$&$1.169$&$1.082$&$-4.077$&$1.394$&$0.168$&$0.171$\tabularnewline
10213&$ 1.198$&$0.269$&$1.590$&$1.767$&$ 1.596$&$0.387$&$2.625$&$2.640$&$ 1.727$&$0.726$&$0.418$&$0.432$&$ 1.745$&$0.628$&$1.948$&$2.012$\tabularnewline
10214&$-0.681$&$0.220$&$0.506$&$0.500$&$-1.247$&$0.368$&$0.722$&$0.752$&$-0.952$&$0.450$&$0.663$&$0.675$&$-0.704$&$0.599$&$0.282$&$0.285$\tabularnewline
10215&$-0.168$&$0.219$&$2.231$&$2.186$&$-0.689$&$0.341$&$1.659$&$1.682$&$ 1.074$&$0.501$&$1.300$&$1.425$&$-1.887$&$0.562$&$1.044$&$1.062$\tabularnewline
10216&$ 0.434$&$0.231$&$0.834$&$0.844$&$-0.068$&$0.387$&$1.367$&$1.373$&$ 0.867$&$0.458$&$0.324$&$0.329$&$ 1.115$&$0.573$&$0.141$&$0.140$\tabularnewline
10217&$ 0.025$&$0.222$&$0.121$&$0.123$&$-0.068$&$0.387$&$0.319$&$0.308$&$ 0.225$&$0.400$&$0.036$&$0.036$&$-0.292$&$0.620$&$0.016$&$0.016$\tabularnewline
10218&$-0.121$&$0.220$&$0.456$&$0.461$&$ 0.095$&$0.402$&$0.774$&$0.760$&$-0.078$&$0.403$&$0.326$&$0.326$&$-1.049$&$0.571$&$0.217$&$0.213$\tabularnewline
10219&$-1.211$&$0.255$&$0.704$&$0.677$&$-1.378$&$0.386$&$0.608$&$0.584$&$-0.761$&$0.437$&$0.300$&$0.302$&$-4.077$&$1.394$&$0.168$&$0.171$\tabularnewline
10220&$-0.073$&$0.220$&$3.289$&$3.273$&$-1.919$&$0.507$&$2.147$&$2.032$&$ 1.074$&$0.501$&$1.300$&$1.425$&$ 1.745$&$0.628$&$1.948$&$2.012$\tabularnewline
10221&$-0.587$&$0.218$&$0.671$&$0.671$&$-0.796$&$0.340$&$0.442$&$0.442$&$-1.625$&$0.536$&$0.836$&$0.861$&$-0.292$&$0.620$&$0.564$&$0.565$\tabularnewline
10223&$ 0.075$&$0.223$&$0.247$&$0.239$&$-0.068$&$0.387$&$0.114$&$0.118$&$ 0.225$&$0.400$&$0.499$&$0.494$&$ 0.133$&$0.611$&$0.392$&$0.388$\tabularnewline
10224&$ 0.277$&$0.227$&$0.211$&$0.208$&$ 0.470$&$0.421$&$0.180$&$0.179$&$-0.078$&$0.403$&$0.070$&$0.070$&$ 0.817$&$0.574$&$0.459$&$0.457$\tabularnewline
10226&$-0.448$&$0.217$&$0.734$&$0.742$&$-0.344$&$0.361$&$1.402$&$1.306$&$-0.238$&$0.409$&$1.145$&$1.133$&$-1.616$&$0.549$&$0.101$&$0.100$\tabularnewline
10227&$ 0.939$&$0.250$&$0.382$&$0.354$&$ 1.596$&$0.387$&$0.179$&$0.181$&$ 0.225$&$0.400$&$0.418$&$0.420$&$ 1.745$&$0.628$&$0.023$&$0.024$\tabularnewline
10228&$-0.448$&$0.217$&$1.367$&$1.339$&$-1.247$&$0.368$&$1.180$&$1.209$&$ 0.075$&$0.400$&$1.960$&$1.968$&$-0.292$&$0.620$&$0.016$&$0.016$\tabularnewline
10230&$ 0.226$&$0.226$&$0.999$&$0.997$&$ 0.844$&$0.411$&$1.057$&$1.058$&$ 0.075$&$0.400$&$0.174$&$0.174$&$-0.292$&$0.620$&$3.518$&$3.532$\tabularnewline
10231&$ 0.226$&$0.226$&$0.479$&$0.468$&$ 0.095$&$0.402$&$0.019$&$0.019$&$-0.078$&$0.403$&$0.297$&$0.302$&$ 1.115$&$0.573$&$1.604$&$1.598$\tabularnewline
10232&$ 0.820$&$0.243$&$0.334$&$0.338$&$ 1.011$&$0.402$&$0.632$&$0.630$&$ 0.689$&$0.431$&$0.604$&$0.609$&$ 1.416$&$0.588$&$0.126$&$0.128$\tabularnewline
10234&$-0.402$&$0.217$&$2.340$&$2.335$&$-0.579$&$0.345$&$3.201$&$3.234$&$ 0.689$&$0.431$&$0.819$&$0.841$&$-2.532$&$0.669$&$0.353$&$0.353$\tabularnewline
10237&$-0.024$&$0.221$&$0.495$&$0.488$&$-0.213$&$0.373$&$0.549$&$0.529$&$-0.078$&$0.403$&$0.297$&$0.302$&$ 0.133$&$0.611$&$1.562$&$1.586$\tabularnewline
10238&$ 0.025$&$0.222$&$0.547$&$0.546$&$ 0.470$&$0.421$&$0.260$&$0.260$&$-0.078$&$0.403$&$0.207$&$0.210$&$-1.049$&$0.571$&$1.239$&$1.223$\tabularnewline
\hline
\end{longtable}}\end{landscape}

\newpage
%%%%%%%%%%%%%%%%%%%%%%%%%%%%%%%%%%%%%%%%%%%%%%%%%%%
\section{Stacking Procedure for Estimating Gains in Skill/Knowledge of the Pilot Empirical Study}
\label{sec:irt-learning-outcomes-pilot-study}


\subsection{Checking Assumptions}

\subsubsection*{Test of Unidimensionality}

\autoref{tab:test-unidimensionality-irt-gains-skill-knowledge-pilot-study} shows the results for the test of unidimensionality in which the goodness of fit statistics indicate essential unidimensionality ($DETECT > 1.00$) for the \emph{Pre-test} with a DETECT index of $0.009$. 
Strong multidimensionality for the \emph{Post-test} is indicated by the DETECT index with a value of $3.187$.
Essential unidimensionality in the data structure is indicated by the ASSI index in the \emph{Pre-test}.
The RATIO indices for the \emph{Pre-test} and \emph{Post-test} indicate essential deviation from unidimensionality.
The index of $AGFI = 0.990$ in the unidimensional CFA indicates an acceptable fit for measuring the skill/knowledge obtained in the \emph{Post-test}. The unidimensional CFA indicated by the TLI and CFI indices indicate unidimensionality.

%latex.default(data_df, caption = paste0("Goodness of fit statistics related to the test of unidimensionality",     "in the ", irt.short.name, "-based instrument ", in_title),     size = "small", longtable = T, ctable = F, landscape = F,     where = "!htbp", file = filename, append = T)%
\setlongtables{\scriptsize
\begin{longtable}{lrrrrrrrr}\caption{Goodness of fit statistics related to the test of unidimensionality in the GPCM-based instrument for measuring the gains in skill/knowledge of the pilot empirical study} \tabularnewline
\hline\hline
\multicolumn{1}{l}{}&\multicolumn{1}{c}{df}&\multicolumn{1}{c}{chisq}&\multicolumn{1}{c}{AGFI}&\multicolumn{1}{c}{TLI}&\multicolumn{1}{c}{CFI}&\multicolumn{1}{c}{DETECT}&\multicolumn{1}{c}{ASSI}&\multicolumn{1}{c}{RATIO}\tabularnewline
\hline
\endfirsthead\caption[]{\em (continued)} \tabularnewline
\hline
\multicolumn{1}{l}{}&\multicolumn{1}{c}{df}&\multicolumn{1}{c}{chisq}&\multicolumn{1}{c}{AGFI}&\multicolumn{1}{c}{TLI}&\multicolumn{1}{c}{CFI}&\multicolumn{1}{c}{DETECT}&\multicolumn{1}{c}{ASSI}&\multicolumn{1}{c}{RATIO}\tabularnewline
\hline
\endhead
\hline
\multicolumn{9}{r}{\tiny df: degree of freedom; AGFI: Adjusted Goodness of Fit Index; CFI: Comparative Fit Index; TLI: Tucker-Lewis Index;}
\endfoot
\label{tab:test-unidimensionality-irt-gains-skill-knowledge-pilot-study}
Pre-test&$2$&$2.591$&$0.548$&$ 0.912$&$0.971$&$0.009$&$0.167$&$0.998$\tabularnewline
Post-test&$2$&$0.387$&$0.990$&$-1.852$&$1.000$&$3.187$&$0.333$&$0.582$\tabularnewline
\hline
\end{longtable}}


\subsubsection*{Test of Local Independence}

Results from the test of local independence in the GPCM-based instrument for measuring gains in skill/knowledge of the pilot empirical study are summarized in \autoref{tab:test-local-independence-irt-gain-skill-knowledge-pilot-study}.
The null condition of local independence is not rejected in the \emph{Pre-test} and \emph{Post-test}.
The Standardized Root Mean Squared Residual (SRMSR) indicates a good fit ($< 0.10$) in the \emph{Pre-test} with value of $0.089$, and the SRMSR index in the \emph{Post-test} indicates acceptable good fit with value of $0.107$.

%latex.default(data_df, caption = paste0("Item residual correlation statistics",     "related to the test of local independence", "in the ", irt.short.name,     "-based instrument ", in_title), size = "small", longtable = T,     ctable = F, landscape = F, where = "!htbp", file = filename,     append = T)%
\setlongtables{\scriptsize
\begin{longtable}{lrrrrr}\caption{Item residual correlation statistics related to the test of local independence in the GPCM-based instrument for measuring gains in skill/knowledge of the pilot empirical study} \tabularnewline
\hline\hline
\multicolumn{1}{l}{}&\multicolumn{1}{c}{max.chisq}&\multicolumn{1}{c}{maxaQ3}&\multicolumn{1}{c}{MADaQ3}&\multicolumn{1}{c}{SRMSR}&\multicolumn{1}{c}{p.value}\tabularnewline
\hline
\endfirsthead\caption[]{\em (continued)} \tabularnewline
\hline
\multicolumn{1}{l}{}&\multicolumn{1}{c}{max.chisq}&\multicolumn{1}{c}{maxaQ3}&\multicolumn{1}{c}{MADaQ3}&\multicolumn{1}{c}{SRMSR}&\multicolumn{1}{c}{p.value}\tabularnewline
\hline
\endhead
\hline
\multicolumn{6}{r}{\tiny aQ3: adjusted correlation of item residuals; maxaQ3: maximum aQ3;}\tabularnewline
\multicolumn{6}{r}{\tiny MADaQ3: Median Absolute Deviation of aQ3;}
\endfoot
\label{tab:test-local-independence-irt-gain-skill-knowledge-pilot-study}
Pre-test&$5.639$&$0.225$&$0.086$&$0.089$&$1.000$\tabularnewline
Post-test&$9.115$&$0.288$&$0.116$&$0.107$&$0.767$\tabularnewline
\hline
\end{longtable}}

\subsection{Item Parameters}

\autoref{tab:item-parameters-pre-pilot-study} shows the estimated parameters for the GPCM-based instrument used to measure the pre-test skill/knowledge of the pilot empirical study.
These parameters had been calculated using the MML method \cite{BockAitkin1981}, so that the value in row \aspas{B.Cat$x$} and column \aspas{$i$} is the item slope $b_{i,x}$ of item $i$ in the category \aspas{$x$}, and the value in the row \aspas{AXsi.Cat$x$} and column \aspas{$i$} is the item intercept $a_{i,x}\xi$ of item $i$ in the category \aspas{$x$}.
According to the Infit/Outfit statistics of items, no one mean-square value is greater than $2.0$ indicating that the measurement system is not distorted or degraded by the items.

%latex.default(estimated_params_df, caption = paste0("Estimated parameters in the ",     irt.short.name, "-based instrument", " for measuring the ",     lname), size = "scriptsize", longtable = T, ctable = F, landscape = F,     where = "!htbp", file = filename, append = T)%
\setlongtables{\scriptsize
\begin{longtable}{lrrrr}\caption{Estimated parameters in the GPCM-based instrument for measuring the pre-test skill/knowledge of the pilot empirical study} \tabularnewline
\hline\hline
\multicolumn{1}{l}{}&\multicolumn{1}{c}{P1s0}&\multicolumn{1}{c}{P2s0}&\multicolumn{1}{c}{P3s2}&\multicolumn{1}{c}{P4s0}\tabularnewline
\hline
\endfirsthead\caption[]{\em (continued)} \tabularnewline
\hline
\multicolumn{1}{l}{}&\multicolumn{1}{c}{P1s0}&\multicolumn{1}{c}{P2s0}&\multicolumn{1}{c}{P3s2}&\multicolumn{1}{c}{P4s0}\tabularnewline
\hline
\endhead
\hline
\endfoot
\label{tab:item-parameters-pre-pilot-study}
xsi.item&$-0.643$&$ 0.433$&$  4.444$&$ 4.227$\tabularnewline
B.Cat0&$ 0.000$&$ 0.000$&$  0.000$&$ 0.000$\tabularnewline
B.Cat1&$ 1.000$&$ 1.000$&$  1.000$&$ 1.000$\tabularnewline
B.Cat2&$ 0.000$&$ 0.000$&$  2.000$&$ 0.000$\tabularnewline
B.Cat3&$ 0.000$&$ 0.000$&$  3.000$&$ 0.000$\tabularnewline
AXsi.Cat0&$ 0.000$&$ 0.000$&$  0.000$&$ 0.000$\tabularnewline
AXsi.Cat1&$ 0.643$&$-0.433$&$ -4.177$&$-4.227$\tabularnewline
AXsi.Cat2&$$&$$&$ -7.924$&$$\tabularnewline
AXsi.Cat3&$$&$$&$-13.332$&$$\tabularnewline
\hline
max.Outfit&$ 0.649$&$ 0.537$&$  0.994$&$ 0.386$\tabularnewline
max.Infit&$ 0.844$&$ 0.755$&$  1.500$&$ 0.961$\tabularnewline
\hline
\end{longtable}}

\autoref{tab:item-parameters-pos-pilot-study} shows the estimated parameters for the GPCM-based instrument used to measure the post-test skill/knowledge of the pilot empirical study.
These parameters had been calculated using the MML method \cite{BockAitkin1981}, so that the value in row \aspas{B.Cat$x$} and column \aspas{$i$} is the item slope $b_{i,x}$ of item $i$ in the category \aspas{$x$}, and the value in the row \aspas{AXsi.Cat$x$} and column \aspas{$i$} is the item intercept $a_{i,x}\xi$ of item $i$ in the category \aspas{$x$}.
According to the Infit/Outfit statistics of items, no one mean-square value is greater than $2.0$ indicating that the measurement system is not distorted or degraded by the items.

%latex.default(estimated_params_df, caption = paste0("Estimated parameters in the ",     irt.short.name, "-based instrument", " for measuring the ",     lname), size = "scriptsize", longtable = T, ctable = F, landscape = F,     where = "!htbp", file = filename, append = T)%
\setlongtables{\scriptsize
\begin{longtable}{lrrrr}\caption{Estimated parameters in the GPCM-based instrument for measuring the post-test skill/knowledge in the pilot empirical study} \tabularnewline
\hline\hline
\multicolumn{1}{l}{}&\multicolumn{1}{c}{PAs2}&\multicolumn{1}{c}{PBs3}&\multicolumn{1}{c}{PCs0}&\multicolumn{1}{c}{PDs0}\tabularnewline
\hline
\endfirsthead\caption[]{\em (continued)} \tabularnewline
\hline
\multicolumn{1}{l}{}&\multicolumn{1}{c}{PAs2}&\multicolumn{1}{c}{PBs3}&\multicolumn{1}{c}{PCs0}&\multicolumn{1}{c}{PDs0}\tabularnewline
\hline
\endhead
\hline
\endfoot
\label{tab:item-parameters-pos-pilot-study}
xsi.item&$-0.506$&$-0.040$&$-0.518$&$ 1.901$\tabularnewline
B.Cat0&$ 0.000$&$ 0.000$&$ 0.000$&$ 0.000$\tabularnewline
B.Cat1&$ 1.000$&$ 1.000$&$ 1.000$&$ 1.000$\tabularnewline
B.Cat2&$ 2.000$&$ 2.000$&$ 0.000$&$ 0.000$\tabularnewline
B.Cat3&$ 3.000$&$ 3.000$&$ 0.000$&$ 0.000$\tabularnewline
B.Cat4&$ 0.000$&$ 4.000$&$ 0.000$&$ 0.000$\tabularnewline
AXsi.Cat0&$ 0.000$&$ 0.000$&$ 0.000$&$ 0.000$\tabularnewline
AXsi.Cat1&$ 2.044$&$ 1.341$&$ 0.518$&$-1.901$\tabularnewline
AXsi.Cat2&$ 1.845$&$ 1.192$&$$&$$\tabularnewline
AXsi.Cat3&$ 1.518$&$ 0.708$&$$&$$\tabularnewline
AXsi.Cat4&$$&$ 0.161$&$$&$$\tabularnewline
\hline
max.Outfit&$ 1.198$&$ 1.915$&$ 1.060$&$ 0.927$\tabularnewline
max.Infit&$ 1.149$&$ 1.078$&$ 1.071$&$ 0.950$\tabularnewline
\hline
\end{longtable}}


\subsection{Gains in Skill/Knowledge as Latent Trait Estimates}

\autoref{tab:gain-skill-knowledge-estimates-pilot-study} shows the latent trait estimates by the GPCM-based instrument for measuring the gains in skill/knowledge of the pilot empirical study.

%latex.default(data_df, caption = paste0("Latent trait estimates and person model fit of the ",     irt.short.name, "-based instrument ", in_title), size = "scriptsize",     longtable = T, ctable = F, landscape = T, rowlabel = "",     where = "!htbp", file = filename, append = T)%
\setlongtables{\scriptsize
\begin{longtable}{l|rrrr|rrrr}\caption{Latent trait estimates and person model fit of the GPCM-based instrument for measuring gains in skill/knowledge of the pilot empirical study} \tabularnewline
\hline\hline
\multicolumn{1}{l}{}&\multicolumn{4}{|c}{Pre-test}&\multicolumn{4}{|c}{Post-test}\tabularnewline
\multicolumn{1}{l}{UserID}&\multicolumn{1}{|c}{theta}&\multicolumn{1}{c}{error}&\multicolumn{1}{c}{Outfit}&\multicolumn{1}{c}{Infit}&\multicolumn{1}{|c}{theta}&\multicolumn{1}{c}{error}&\multicolumn{1}{c}{Outfit}&\multicolumn{1}{c}{Infit}\tabularnewline
\hline
\endfirsthead\caption[]{\em (continued)} \tabularnewline
\hline
\multicolumn{1}{l}{}&\multicolumn{4}{|c}{Pre-test}&\multicolumn{4}{|c}{Post-test}\tabularnewline
\multicolumn{1}{l}{UserID}&\multicolumn{1}{|c}{theta}&\multicolumn{1}{c}{error}&\multicolumn{1}{c}{Outfit}&\multicolumn{1}{c}{Infit}&\multicolumn{1}{|c}{theta}&\multicolumn{1}{c}{error}&\multicolumn{1}{c}{Outfit}&\multicolumn{1}{c}{Infit}\tabularnewline
\hline
\endhead
\hline
\endfoot
\label{tab:gain-skill-knowledge-estimates-pilot-study}
10116&$ 3.666$&$0.940$&$0.160$&$0.129$&$ 0.259$&$0.639$&$0.814$&$1.117$\tabularnewline
10119&$-1.876$&$1.955$&$0.099$&$0.226$&$-3.280$&$1.766$&$0.216$&$0.232$\tabularnewline
10120&$ 4.987$&$1.035$&$0.121$&$0.112$&$-0.514$&$0.715$&$0.343$&$0.243$\tabularnewline
10121&$ 3.046$&$1.280$&$0.167$&$0.355$&$-1.684$&$1.085$&$0.422$&$0.422$\tabularnewline
10122&$ 3.046$&$1.280$&$0.167$&$0.355$&$ 0.310$&$0.659$&$1.253$&$0.704$\tabularnewline
10126&$ 2.620$&$1.298$&$0.148$&$0.199$&$-1.090$&$0.831$&$0.163$&$0.108$\tabularnewline
10127&$ 3.046$&$1.280$&$0.167$&$0.355$&$-1.684$&$1.085$&$0.422$&$0.422$\tabularnewline
10128&$ 2.620$&$1.298$&$0.148$&$0.199$&$ 0.606$&$0.660$&$0.184$&$0.109$\tabularnewline
10129&$-1.876$&$1.955$&$0.099$&$0.226$&$-0.094$&$0.656$&$0.351$&$0.368$\tabularnewline
10130&$-0.103$&$1.443$&$0.394$&$0.567$&$-0.477$&$0.723$&$0.431$&$0.262$\tabularnewline
10131&$-2.145$&$2.095$&$0.223$&$0.223$&$-0.052$&$0.668$&$0.581$&$0.226$\tabularnewline
10132&$-1.876$&$1.955$&$0.099$&$0.226$&$ 0.606$&$0.660$&$1.016$&$1.429$\tabularnewline
10133&$ 2.620$&$1.298$&$0.148$&$0.199$&$-1.921$&$1.050$&$0.377$&$0.405$\tabularnewline
10134&$-1.876$&$1.955$&$0.099$&$0.226$&$-0.514$&$0.715$&$0.343$&$0.243$\tabularnewline
10135&$ 2.620$&$1.298$&$0.148$&$0.199$&$ 0.259$&$0.639$&$1.477$&$1.865$\tabularnewline
10136&$-1.876$&$1.955$&$0.099$&$0.226$&$ 0.606$&$0.660$&$0.482$&$0.531$\tabularnewline
10137&$-1.874$&$1.962$&$0.131$&$0.228$&$-0.094$&$0.656$&$0.357$&$0.290$\tabularnewline
10138&$-1.876$&$1.955$&$0.099$&$0.226$&$ 1.010$&$0.733$&$0.309$&$0.243$\tabularnewline
10139&$-2.145$&$2.095$&$0.223$&$0.223$&$ 0.674$&$0.697$&$1.548$&$1.065$\tabularnewline
10140&$-0.103$&$1.443$&$0.394$&$0.567$&$ 0.259$&$0.639$&$1.703$&$0.703$\tabularnewline
10141&$-1.876$&$1.955$&$0.099$&$0.226$&$ 3.106$&$1.673$&$0.117$&$0.181$\tabularnewline
10143&$ 3.046$&$1.280$&$0.167$&$0.355$&$-3.002$&$2.068$&$0.379$&$0.379$\tabularnewline
10144&$ 4.271$&$0.903$&$0.300$&$0.360$&$ 0.606$&$0.660$&$1.102$&$0.605$\tabularnewline
10145&$ 4.987$&$1.035$&$0.121$&$0.112$&$ 0.606$&$0.660$&$0.184$&$0.109$\tabularnewline
10146&$-0.104$&$1.424$&$0.299$&$0.552$&$ 1.628$&$0.929$&$0.832$&$0.865$\tabularnewline
10148&$-1.876$&$1.955$&$0.099$&$0.226$&$ 1.010$&$0.733$&$1.138$&$1.138$\tabularnewline
10149&$-1.876$&$1.955$&$0.099$&$0.226$&$ 0.606$&$0.660$&$0.184$&$0.109$\tabularnewline
10152&$ 4.987$&$1.035$&$0.858$&$1.461$&$-0.514$&$0.715$&$0.446$&$0.384$\tabularnewline
10153&$-0.103$&$1.443$&$0.394$&$0.567$&$ 1.628$&$0.929$&$0.385$&$0.411$\tabularnewline
10154&$-2.145$&$2.095$&$0.223$&$0.223$&$-3.002$&$2.068$&$0.379$&$0.379$\tabularnewline
\hline
\end{longtable}}


%%%%%%%%%%%%%%%%%%%%%%%%%%%%%%%%%%%%%%%%%%%%%%%%%%%
\section{Stacking Procedure for Estimating Gains in Skill/Knowledge of the First Empirical Study}
\label{sec:irt-learning-outcomes-first-study}


\subsection{Checking Assumptions}

\subsubsection*{Test of Unidimensionality}

\autoref{tab:test-unidimensionality-irt-gains-skill-knowledge-first-study} shows the results for the test of unidimensionality in which the goodness of fit statistics indicate strong multidimensionality ($DETECT > 1.00$) for the \emph{Pre-test} and \emph{Post-test} with values of $33.268$ and $27.559$. 
Essential unidimensionality in the data structure is indicated by the ASSI index in the \emph{Pre-test} (with value of $0.056$).
The index AGFI in the \emph{Pre-test} and \emph{Post-test} indicate got fit with values greater than $0.95$
A good fit with the unidimensional CFA is indicated by the TLI and CFI indices for the pre-test and post-test.

%latex.default(data_df, caption = paste0("Goodness of fit statistics related to the test of unidimensionality",     "in the ", irt.short.name, "-based instrument ", in_title),     size = "small", longtable = T, ctable = F, landscape = F,     where = "!htbp", file = filename, append = T)%
\setlongtables{\scriptsize
\begin{longtable}{lrrrrrrrr}\caption{Goodness of fit statistics related to the test of unidimensionality in the GPCM-based instrument for measuring the gains in skill/knowledge of the first empirical study} \tabularnewline
\hline\hline
\multicolumn{1}{l}{}&\multicolumn{1}{c}{df}&\multicolumn{1}{c}{chisq}&\multicolumn{1}{c}{AGFI}&\multicolumn{1}{c}{TLI}&\multicolumn{1}{c}{CFI}&\multicolumn{1}{c}{DETECT}&\multicolumn{1}{c}{ASSI}&\multicolumn{1}{c}{RATIO}\tabularnewline
\hline
\endfirsthead\caption[]{\em (continued)} \tabularnewline
\hline
\multicolumn{1}{l}{}&\multicolumn{1}{c}{df}&\multicolumn{1}{c}{chisq}&\multicolumn{1}{c}{AGFI}&\multicolumn{1}{c}{TLI}&\multicolumn{1}{c}{CFI}&\multicolumn{1}{c}{DETECT}&\multicolumn{1}{c}{ASSI}&\multicolumn{1}{c}{RATIO}\tabularnewline
\hline
\endhead
\hline
\multicolumn{9}{r}{\tiny df: degree of freedom; AGFI: Adjusted Goodness of Fit Index; CFI: Comparative Fit Index; TLI: Tucker-Lewis Index;}
\endfoot
\label{tab:test-unidimensionality-irt-gains-skill-knowledge-first-study}
Pre-test&$27$&$27.576$&$0.969$&$0.981$&$0.986$&$33.264$&$0.056$&$0.508$\tabularnewline
Post-test&$14$&$16.087$&$0.997$&$0.875$&$0.917$&$27.559$&$0.333$&$0.528$\tabularnewline
\hline
\end{longtable}}


\subsubsection*{Test of Local Independence}

Results from the test of local independence in the GPCM-based instrument for measuring gains in skill/knowledge of the first empirical study are summarized in \autoref{tab:test-local-independence-irt-gain-skill-knowledge-first-study}.
The null condition of local independence is not rejected in the \emph{Pre-test} and \emph{Post-test}.
The Standardized Root Mean Squared Residual (SRMSR) indicates an acceptable fit ($0.10$s) in the \emph{Post-test}.

%latex.default(data_df, caption = paste0("Item residual correlation statistics",     "related to the test of local independence", "in the ", irt.short.name,     "-based instrument ", in_title), size = "small", longtable = T,     ctable = F, landscape = F, where = "!htbp", file = filename,     append = T)%
\setlongtables{\scriptsize
\begin{longtable}{lrrrrr}\caption{Item residual correlation statistics related to the test of local independence in the GPCM-based instrument for measuring gains in skill/knowledge of the first empirical study} \tabularnewline
\hline\hline
\multicolumn{1}{l}{}&\multicolumn{1}{c}{max.chisq}&\multicolumn{1}{c}{maxaQ3}&\multicolumn{1}{c}{MADaQ3}&\multicolumn{1}{c}{SRMSR}&\multicolumn{1}{c}{p.value}\tabularnewline
\hline
\endfirsthead\caption[]{\em (continued)} \tabularnewline
\hline
\multicolumn{1}{l}{}&\multicolumn{1}{c}{max.chisq}&\multicolumn{1}{c}{maxaQ3}&\multicolumn{1}{c}{MADaQ3}&\multicolumn{1}{c}{SRMSR}&\multicolumn{1}{c}{p.value}\tabularnewline
\hline
\endhead
\hline
\multicolumn{6}{r}{\tiny aQ3: adjusted correlation of item residuals; maxaQ3: maximum aQ3;}\tabularnewline
\multicolumn{6}{r}{\tiny MADaQ3: Median Absolute Deviation of aQ3;}
\endfoot
\label{tab:test-local-independence-irt-gain-skill-knowledge-first-study}
Pre-test&$51.415$&$0.370$&$0.126$&$0.251$&$0.176$\tabularnewline
Post-test&$90.705$&$0.326$&$0.125$&$0.166$&$0.345$\tabularnewline
\hline
\end{longtable}}

\subsection{Item Parameters}

\autoref{tab:item-parameters-pre-first-study} shows the estimated parameters for the GPCM-based instrument used to measure the pre-test skill/knowledge of the first empirical study.
These parameters had been calculated using the MML method \cite{BockAitkin1981}, so that the value in row \aspas{B.Cat$x$} and column \aspas{$i$} is the item slope $b_{i,x}$ of item $i$ in the category \aspas{$x$}, and the value in the row \aspas{AXsi.Cat$x$} and column \aspas{$i$} is the item intercept $a_{i,x}\xi$ of item $i$ in the category \aspas{$x$}.
According to the Infit/Outfit statistics of items, no one mean-square value is greater than $2.0$ indicating that the measurement system is not distorted or degraded by the items.

%latex.default(estimated_params_df, caption = paste0("Estimated parameters in the ",     irt.short.name, "-based instrument", " for measuring the ",     lname), size = "scriptsize", longtable = T, ctable = F, landscape = F,     where = "!htbp", file = filename, append = T)%
\setlongtables{\scriptsize
\begin{longtable}{lrrrrrrrrr}\caption{Estimated parameters in the GPCM-based instrument for measuring the pre-test skill/knowledge of the first empirical study} \tabularnewline
\hline\hline
\multicolumn{1}{l}{}&\multicolumn{1}{c}{An3}&\multicolumn{1}{c}{Ap1}&\multicolumn{1}{c}{Ap2}&\multicolumn{1}{c}{Ap3}&\multicolumn{1}{c}{Ev1}&\multicolumn{1}{c}{Ev2}&\multicolumn{1}{c}{P1s2}&\multicolumn{1}{c}{Un1}&\multicolumn{1}{c}{Un2}\tabularnewline
\hline
\endfirsthead\caption[]{\em (continued)} \tabularnewline
\hline
\multicolumn{1}{l}{}&\multicolumn{1}{c}{An3}&\multicolumn{1}{c}{Ap1}&\multicolumn{1}{c}{Ap2}&\multicolumn{1}{c}{Ap3}&\multicolumn{1}{c}{Ev1}&\multicolumn{1}{c}{Ev2}&\multicolumn{1}{c}{P1s2}&\multicolumn{1}{c}{Un1}&\multicolumn{1}{c}{Un2}\tabularnewline
\hline
\endhead
\hline
\endfoot
\label{tab:item-parameters-pre-first-study}
xsi.item&$-0.224$&$-0.662$&$-0.573$&$-0.196$&$-0.177$&$-0.054$&$-12.260$&$ -6.086$&$-0.093$\tabularnewline
B.Cat0&$ 0.000$&$ 0.000$&$ 0.000$&$ 0.000$&$ 0.000$&$ 0.000$&$  0.000$&$  0.000$&$ 0.000$\tabularnewline
B.Cat1&$ 1.000$&$ 1.000$&$ 1.000$&$ 1.000$&$ 1.000$&$ 1.000$&$  1.000$&$  1.000$&$ 1.000$\tabularnewline
B.Cat2&$ 2.000$&$ 2.000$&$ 0.000$&$ 2.000$&$ 2.000$&$ 2.000$&$  2.000$&$  2.000$&$ 2.000$\tabularnewline
B.Cat3&$ 3.000$&$ 3.000$&$ 0.000$&$ 3.000$&$ 3.000$&$ 3.000$&$  3.000$&$  3.000$&$ 3.000$\tabularnewline
B.Cat4&$ 4.000$&$ 4.000$&$ 0.000$&$ 4.000$&$ 4.000$&$ 4.000$&$  0.000$&$  4.000$&$ 4.000$\tabularnewline
B.Cat5&$ 5.000$&$ 5.000$&$ 0.000$&$ 5.000$&$ 5.000$&$ 5.000$&$  0.000$&$  5.000$&$ 5.000$\tabularnewline
B.Cat6&$ 6.000$&$ 6.000$&$ 0.000$&$ 6.000$&$ 6.000$&$ 6.000$&$  0.000$&$  6.000$&$ 6.000$\tabularnewline
B.Cat7&$ 7.000$&$ 0.000$&$ 0.000$&$ 7.000$&$ 7.000$&$ 7.000$&$  0.000$&$  7.000$&$ 0.000$\tabularnewline
B.Cat8&$ 8.000$&$ 0.000$&$ 0.000$&$ 8.000$&$ 8.000$&$ 8.000$&$  0.000$&$  8.000$&$ 0.000$\tabularnewline
B.Cat9&$ 9.000$&$ 0.000$&$ 0.000$&$ 0.000$&$ 9.000$&$ 9.000$&$  0.000$&$  9.000$&$ 0.000$\tabularnewline
B.Cat10&$10.000$&$ 0.000$&$ 0.000$&$ 0.000$&$10.000$&$10.000$&$  0.000$&$ 10.000$&$ 0.000$\tabularnewline
B.Cat11&$11.000$&$ 0.000$&$ 0.000$&$ 0.000$&$11.000$&$ 0.000$&$  0.000$&$ 11.000$&$ 0.000$\tabularnewline
B.Cat12&$12.000$&$ 0.000$&$ 0.000$&$ 0.000$&$12.000$&$ 0.000$&$  0.000$&$ 12.000$&$ 0.000$\tabularnewline
B.Cat13&$13.000$&$ 0.000$&$ 0.000$&$ 0.000$&$ 0.000$&$ 0.000$&$  0.000$&$ 13.000$&$ 0.000$\tabularnewline
B.Cat14&$14.000$&$ 0.000$&$ 0.000$&$ 0.000$&$ 0.000$&$ 0.000$&$  0.000$&$ 14.000$&$ 0.000$\tabularnewline
B.Cat15&$15.000$&$ 0.000$&$ 0.000$&$ 0.000$&$ 0.000$&$ 0.000$&$  0.000$&$ 15.000$&$ 0.000$\tabularnewline
B.Cat16&$16.000$&$ 0.000$&$ 0.000$&$ 0.000$&$ 0.000$&$ 0.000$&$  0.000$&$ 16.000$&$ 0.000$\tabularnewline
B.Cat17&$ 0.000$&$ 0.000$&$ 0.000$&$ 0.000$&$ 0.000$&$ 0.000$&$  0.000$&$ 17.000$&$ 0.000$\tabularnewline
B.Cat18&$ 0.000$&$ 0.000$&$ 0.000$&$ 0.000$&$ 0.000$&$ 0.000$&$  0.000$&$ 18.000$&$ 0.000$\tabularnewline
AXsi.Cat0&$ 0.000$&$ 0.000$&$ 0.000$&$ 0.000$&$ 0.000$&$ 0.000$&$  0.000$&$  0.000$&$ 0.000$\tabularnewline
AXsi.Cat1&$-7.050$&$-5.831$&$ 0.573$&$-7.073$&$-5.761$&$-7.447$&$ 36.897$&$  7.836$&$-2.079$\tabularnewline
AXsi.Cat2&$-8.618$&$ 1.387$&$$&$-6.930$&$-5.680$&$-8.616$&$ 36.897$&$ 17.905$&$-1.386$\tabularnewline
AXsi.Cat3&$-8.489$&$ 1.096$&$$&$ 0.587$&$-0.523$&$-7.518$&$ 36.779$&$ 31.574$&$ 1.417$\tabularnewline
AXsi.Cat4&$-7.076$&$-4.816$&$$&$ 1.526$&$-5.619$&$-0.337$&$$&$ 45.698$&$-1.386$\tabularnewline
AXsi.Cat5&$ 0.000$&$-4.747$&$$&$-6.317$&$-7.055$&$-0.075$&$$&$ 67.220$&$-2.079$\tabularnewline
AXsi.Cat6&$ 0.001$&$ 3.970$&$$&$-7.534$&$-7.059$&$-7.256$&$$&$108.163$&$ 0.560$\tabularnewline
AXsi.Cat7&$ 1.387$&$$&$$&$-6.345$&$-5.544$&$-8.887$&$$&$100.994$&$$\tabularnewline
AXsi.Cat8&$ 1.097$&$$&$$&$ 1.568$&$ 0.782$&$-8.856$&$$&$ 99.267$&$$\tabularnewline
AXsi.Cat9&$-5.125$&$$&$$&$$&$-4.940$&$-7.238$&$$&$ 99.338$&$$\tabularnewline
AXsi.Cat10&$-6.509$&$$&$$&$$&$-6.290$&$ 0.539$&$$&$101.019$&$$\tabularnewline
AXsi.Cat11&$-6.499$&$$&$$&$$&$-4.913$&$$&$$&$109.081$&$$\tabularnewline
AXsi.Cat12&$-5.251$&$$&$$&$$&$ 2.124$&$$&$$&$110.243$&$$\tabularnewline
AXsi.Cat13&$-0.007$&$$&$$&$$&$$&$$&$$&$103.017$&$$\tabularnewline
AXsi.Cat14&$ 1.099$&$$&$$&$$&$$&$$&$$&$101.713$&$$\tabularnewline
AXsi.Cat15&$ 2.197$&$$&$$&$$&$$&$$&$$&$102.736$&$$\tabularnewline
AXsi.Cat16&$ 3.584$&$$&$$&$$&$$&$$&$$&$107.459$&$$\tabularnewline
AXsi.Cat17&$$&$$&$$&$$&$$&$$&$$&$110.838$&$$\tabularnewline
AXsi.Cat18&$$&$$&$$&$$&$$&$$&$$&$109.550$&$$\tabularnewline
\hline
max.Outfit&$ 1.000$&$ 1.000$&$ 1.000$&$ 1.000$&$ 1.000$&$ 1.000$&$  1.000$&$  1.001$&$ 1.000$\tabularnewline
max.Infit&$ 1.000$&$ 1.000$&$ 1.000$&$ 1.000$&$ 1.000$&$ 1.000$&$  1.000$&$  1.001$&$ 1.000$\tabularnewline
\hline
\end{longtable}}

\autoref{tab:item-parameters-pos-first-study} shows the estimated parameters for the GPCM-based instrument used to measure the post-test skill/knowledge of the first empirical study.
These parameters had been calculated using the MML method \cite{BockAitkin1981}, so that the value in row \aspas{B.Cat$x$} and column \aspas{$i$} is the item slope $b_{i,x}$ of item $i$ in the category \aspas{$x$}, and the value in the row \aspas{AXsi.Cat$x$} and column \aspas{$i$} is the item intercept $a_{i,x}\xi$ of item $i$ in the category \aspas{$x$}.
According to the Infit/Outfit statistics of items, no one mean-square value is greater than $2.0$ indicating that the measurement system is not distorted or degraded by the items.

%latex.default(estimated_params_df, caption = paste0("Estimated parameters in the ",     irt.short.name, "-based instrument", " for measuring the ",     lname), size = "scriptsize", longtable = T, ctable = F, landscape = F,     where = "!htbp", file = filename, append = T)%
\setlongtables{\scriptsize
\begin{longtable}{lrrrrrrr}\caption{Estimated parameters in the GPCM-based instrument for measuring the post-test skill/knowledge in the first empirical study} \tabularnewline
\hline\hline
\multicolumn{1}{l}{}&\multicolumn{1}{c}{AnC}&\multicolumn{1}{c}{ApB}&\multicolumn{1}{c}{ApC}&\multicolumn{1}{c}{EvA}&\multicolumn{1}{c}{PAs3}&\multicolumn{1}{c}{ReB}&\multicolumn{1}{c}{UnB}\tabularnewline
\hline
\endfirsthead\caption[]{\em (continued)} \tabularnewline
\hline
\multicolumn{1}{l}{}&\multicolumn{1}{c}{AnC}&\multicolumn{1}{c}{ApB}&\multicolumn{1}{c}{ApC}&\multicolumn{1}{c}{EvA}&\multicolumn{1}{c}{PAs3}&\multicolumn{1}{c}{ReB}&\multicolumn{1}{c}{UnB}\tabularnewline
\hline
\endhead
\hline
\endfoot
\label{tab:item-parameters-pos-first-study}
xsi.item&$-0.264$&$-40.922$&$-0.412$&$-0.659$&$-26.881$&$ -7.149$&$-0.415$\tabularnewline
B.Cat0&$ 0.000$&$  0.000$&$ 0.000$&$ 0.000$&$  0.000$&$  0.000$&$ 0.000$\tabularnewline
B.Cat1&$ 1.000$&$  1.000$&$ 1.000$&$ 1.000$&$  1.000$&$  1.000$&$ 1.000$\tabularnewline
B.Cat2&$ 2.000$&$  2.000$&$ 2.000$&$ 2.000$&$  2.000$&$  2.000$&$ 2.000$\tabularnewline
B.Cat3&$ 3.000$&$  3.000$&$ 3.000$&$ 3.000$&$  3.000$&$  3.000$&$ 3.000$\tabularnewline
B.Cat4&$ 4.000$&$  4.000$&$ 4.000$&$ 4.000$&$  4.000$&$  4.000$&$ 4.000$\tabularnewline
B.Cat5&$ 5.000$&$  5.000$&$ 5.000$&$ 5.000$&$  0.000$&$  5.000$&$ 0.000$\tabularnewline
B.Cat6&$ 6.000$&$  6.000$&$ 6.000$&$ 6.000$&$  0.000$&$  6.000$&$ 0.000$\tabularnewline
B.Cat7&$ 7.000$&$  7.000$&$ 7.000$&$ 0.000$&$  0.000$&$  7.000$&$ 0.000$\tabularnewline
B.Cat8&$ 8.000$&$  8.000$&$ 8.000$&$ 0.000$&$  0.000$&$  8.000$&$ 0.000$\tabularnewline
B.Cat9&$ 9.000$&$  9.000$&$ 0.000$&$ 0.000$&$  0.000$&$  9.000$&$ 0.000$\tabularnewline
B.Cat10&$10.000$&$ 10.000$&$ 0.000$&$ 0.000$&$  0.000$&$ 10.000$&$ 0.000$\tabularnewline
B.Cat11&$11.000$&$ 11.000$&$ 0.000$&$ 0.000$&$  0.000$&$ 11.000$&$ 0.000$\tabularnewline
B.Cat12&$12.000$&$ 12.000$&$ 0.000$&$ 0.000$&$  0.000$&$ 12.000$&$ 0.000$\tabularnewline
B.Cat13&$13.000$&$  0.000$&$ 0.000$&$ 0.000$&$  0.000$&$ 13.000$&$ 0.000$\tabularnewline
B.Cat14&$14.000$&$  0.000$&$ 0.000$&$ 0.000$&$  0.000$&$ 14.000$&$ 0.000$\tabularnewline
B.Cat15&$ 0.000$&$  0.000$&$ 0.000$&$ 0.000$&$  0.000$&$ 15.000$&$ 0.000$\tabularnewline
B.Cat16&$ 0.000$&$  0.000$&$ 0.000$&$ 0.000$&$  0.000$&$ 16.000$&$ 0.000$\tabularnewline
B.Cat17&$ 0.000$&$  0.000$&$ 0.000$&$ 0.000$&$  0.000$&$ 17.000$&$ 0.000$\tabularnewline
B.Cat18&$ 0.000$&$  0.000$&$ 0.000$&$ 0.000$&$  0.000$&$ 18.000$&$ 0.000$\tabularnewline
B.Cat19&$ 0.000$&$  0.000$&$ 0.000$&$ 0.000$&$  0.000$&$ 19.000$&$ 0.000$\tabularnewline
B.Cat20&$ 0.000$&$  0.000$&$ 0.000$&$ 0.000$&$  0.000$&$ 20.000$&$ 0.000$\tabularnewline
B.Cat21&$ 0.000$&$  0.000$&$ 0.000$&$ 0.000$&$  0.000$&$ 21.000$&$ 0.000$\tabularnewline
B.Cat22&$ 0.000$&$  0.000$&$ 0.000$&$ 0.000$&$  0.000$&$ 22.000$&$ 0.000$\tabularnewline
B.Cat23&$ 0.000$&$  0.000$&$ 0.000$&$ 0.000$&$  0.000$&$ 23.000$&$ 0.000$\tabularnewline
B.Cat24&$ 0.000$&$  0.000$&$ 0.000$&$ 0.000$&$  0.000$&$ 24.000$&$ 0.000$\tabularnewline
B.Cat25&$ 0.000$&$  0.000$&$ 0.000$&$ 0.000$&$  0.000$&$ 25.000$&$ 0.000$\tabularnewline
B.Cat26&$ 0.000$&$  0.000$&$ 0.000$&$ 0.000$&$  0.000$&$ 26.000$&$ 0.000$\tabularnewline
B.Cat27&$ 0.000$&$  0.000$&$ 0.000$&$ 0.000$&$  0.000$&$ 27.000$&$ 0.000$\tabularnewline
B.Cat28&$ 0.000$&$  0.000$&$ 0.000$&$ 0.000$&$  0.000$&$ 28.000$&$ 0.000$\tabularnewline
B.Cat29&$ 0.000$&$  0.000$&$ 0.000$&$ 0.000$&$  0.000$&$ 29.000$&$ 0.000$\tabularnewline
B.Cat30&$ 0.000$&$  0.000$&$ 0.000$&$ 0.000$&$  0.000$&$ 30.000$&$ 0.000$\tabularnewline
B.Cat31&$ 0.000$&$  0.000$&$ 0.000$&$ 0.000$&$  0.000$&$ 31.000$&$ 0.000$\tabularnewline
B.Cat32&$ 0.000$&$  0.000$&$ 0.000$&$ 0.000$&$  0.000$&$ 32.000$&$ 0.000$\tabularnewline
B.Cat33&$ 0.000$&$  0.000$&$ 0.000$&$ 0.000$&$  0.000$&$ 33.000$&$ 0.000$\tabularnewline
B.Cat34&$ 0.000$&$  0.000$&$ 0.000$&$ 0.000$&$  0.000$&$ 34.000$&$ 0.000$\tabularnewline
B.Cat35&$ 0.000$&$  0.000$&$ 0.000$&$ 0.000$&$  0.000$&$ 35.000$&$ 0.000$\tabularnewline
B.Cat36&$ 0.000$&$  0.000$&$ 0.000$&$ 0.000$&$  0.000$&$ 36.000$&$ 0.000$\tabularnewline
B.Cat37&$ 0.000$&$  0.000$&$ 0.000$&$ 0.000$&$  0.000$&$ 37.000$&$ 0.000$\tabularnewline
B.Cat38&$ 0.000$&$  0.000$&$ 0.000$&$ 0.000$&$  0.000$&$ 38.000$&$ 0.000$\tabularnewline
B.Cat39&$ 0.000$&$  0.000$&$ 0.000$&$ 0.000$&$  0.000$&$ 39.000$&$ 0.000$\tabularnewline
B.Cat40&$ 0.000$&$  0.000$&$ 0.000$&$ 0.000$&$  0.000$&$ 40.000$&$ 0.000$\tabularnewline
AXsi.Cat0&$ 0.000$&$  0.000$&$ 0.000$&$ 0.000$&$  0.000$&$  0.000$&$ 0.000$\tabularnewline
AXsi.Cat1&$-6.026$&$ 63.868$&$-6.583$&$-5.856$&$107.842$&$  9.246$&$-6.877$\tabularnewline
AXsi.Cat2&$-7.906$&$127.595$&$-6.455$&$ 1.386$&$107.523$&$ 18.531$&$-7.055$\tabularnewline
AXsi.Cat3&$-8.366$&$191.328$&$ 2.080$&$-5.439$&$107.390$&$ 27.833$&$-0.135$\tabularnewline
AXsi.Cat4&$-7.959$&$258.331$&$ 3.044$&$-6.371$&$107.523$&$ 37.145$&$ 1.658$\tabularnewline
AXsi.Cat5&$-6.265$&$329.662$&$-4.344$&$-5.216$&$$&$ 46.459$&$$\tabularnewline
AXsi.Cat6&$ 1.104$&$402.199$&$-5.207$&$ 3.951$&$$&$ 55.736$&$$\tabularnewline
AXsi.Cat7&$-0.006$&$487.428$&$-4.150$&$$&$$&$ 64.998$&$$\tabularnewline
AXsi.Cat8&$-4.810$&$488.815$&$ 3.295$&$$&$$&$ 74.299$&$$\tabularnewline
AXsi.Cat9&$-6.196$&$488.527$&$$&$$&$$&$ 83.592$&$$\tabularnewline
AXsi.Cat10&$-6.451$&$487.428$&$$&$$&$$&$ 93.532$&$$\tabularnewline
AXsi.Cat11&$-6.376$&$489.731$&$$&$$&$$&$103.808$&$$\tabularnewline
AXsi.Cat12&$-4.726$&$491.066$&$$&$$&$$&$114.022$&$$\tabularnewline
AXsi.Cat13&$ 2.491$&$$&$$&$$&$$&$124.364$&$$\tabularnewline
AXsi.Cat14&$ 3.696$&$$&$$&$$&$$&$136.479$&$$\tabularnewline
AXsi.Cat15&$$&$$&$$&$$&$$&$146.873$&$$\tabularnewline
AXsi.Cat16&$$&$$&$$&$$&$$&$159.515$&$$\tabularnewline
AXsi.Cat17&$$&$$&$$&$$&$$&$171.435$&$$\tabularnewline
AXsi.Cat18&$$&$$&$$&$$&$$&$182.746$&$$\tabularnewline
AXsi.Cat19&$$&$$&$$&$$&$$&$196.735$&$$\tabularnewline
AXsi.Cat20&$$&$$&$$&$$&$$&$217.437$&$$\tabularnewline
AXsi.Cat21&$$&$$&$$&$$&$$&$241.704$&$$\tabularnewline
AXsi.Cat22&$$&$$&$$&$$&$$&$282.662$&$$\tabularnewline
AXsi.Cat23&$$&$$&$$&$$&$$&$282.649$&$$\tabularnewline
AXsi.Cat24&$$&$$&$$&$$&$$&$281.939$&$$\tabularnewline
AXsi.Cat25&$$&$$&$$&$$&$$&$277.050$&$$\tabularnewline
AXsi.Cat26&$$&$$&$$&$$&$$&$275.453$&$$\tabularnewline
AXsi.Cat27&$$&$$&$$&$$&$$&$274.899$&$$\tabularnewline
AXsi.Cat28&$$&$$&$$&$$&$$&$274.735$&$$\tabularnewline
AXsi.Cat29&$$&$$&$$&$$&$$&$275.542$&$$\tabularnewline
AXsi.Cat30&$$&$$&$$&$$&$$&$277.779$&$$\tabularnewline
AXsi.Cat31&$$&$$&$$&$$&$$&$283.146$&$$\tabularnewline
AXsi.Cat32&$$&$$&$$&$$&$$&$284.024$&$$\tabularnewline
AXsi.Cat33&$$&$$&$$&$$&$$&$278.717$&$$\tabularnewline
AXsi.Cat34&$$&$$&$$&$$&$$&$276.796$&$$\tabularnewline
AXsi.Cat35&$$&$$&$$&$$&$$&$276.101$&$$\tabularnewline
AXsi.Cat36&$$&$$&$$&$$&$$&$275.699$&$$\tabularnewline
AXsi.Cat37&$$&$$&$$&$$&$$&$276.319$&$$\tabularnewline
AXsi.Cat38&$$&$$&$$&$$&$$&$278.569$&$$\tabularnewline
AXsi.Cat39&$$&$$&$$&$$&$$&$282.075$&$$\tabularnewline
AXsi.Cat40&$$&$$&$$&$$&$$&$285.947$&$$\tabularnewline
\hline
max.Outfit&$ 1.007$&$  1.000$&$ 1.000$&$ 1.000$&$  1.000$&$  0.000$&$ 1.000$\tabularnewline
max.Infit&$ 1.007$&$  1.000$&$ 1.000$&$ 1.000$&$  1.000$&$  0.000$&$ 1.000$\tabularnewline
\hline
\end{longtable}}


\subsection{Gains in Skill/Knowledge as Latent Trait Estimates}

\autoref{tab:gain-skill-knowledge-estimates-first-study} shows the latent trait estimates by the GPCM-based instrument for measuring the gains in skill/knowledge of the first empirical study.

%latex.default(data_df, caption = paste0("Latent trait estimates and person model fit of the ",     irt.short.name, "-based instrument ", in_title), size = "scriptsize",     longtable = T, ctable = F, landscape = T, rowlabel = "",     where = "!htbp", file = filename, append = T)%
\setlongtables{\scriptsize
\begin{longtable}{l|rrrr|rrrr}\caption{Latent trait estimates and person model fit of the GPCM-based instrument for measuring gains in skill/knowledge of the first empirical study} \tabularnewline
\hline\hline
\multicolumn{1}{l}{}&\multicolumn{4}{|c}{Pre-test}&\multicolumn{4}{|c}{Post-test}\tabularnewline
\multicolumn{1}{l}{UserID}&\multicolumn{1}{|c}{theta}&\multicolumn{1}{c}{error}&\multicolumn{1}{c}{Outfit}&\multicolumn{1}{c}{Infit}&\multicolumn{1}{|c}{theta}&\multicolumn{1}{c}{error}&\multicolumn{1}{c}{Outfit}&\multicolumn{1}{c}{Infit}\tabularnewline
\hline
\endfirsthead\caption[]{\em (continued)} \tabularnewline
\hline
\multicolumn{1}{l}{}&\multicolumn{4}{|c}{Pre-test}&\multicolumn{4}{|c}{Post-test}\tabularnewline
\multicolumn{1}{l}{UserID}&\multicolumn{1}{|c}{theta}&\multicolumn{1}{c}{error}&\multicolumn{1}{c}{Outfit}&\multicolumn{1}{c}{Infit}&\multicolumn{1}{|c}{theta}&\multicolumn{1}{c}{error}&\multicolumn{1}{c}{Outfit}&\multicolumn{1}{c}{Infit}\tabularnewline
\hline
\endhead
\hline
\endfoot
\label{tab:gain-skill-knowledge-estimates-first-study}
10169&$-0.039$&$0.116$&$1.010$&$1.069$&$-0.181$&$0.103$&$0.285$&$0.243$\tabularnewline
10170&$-0.063$&$0.112$&$1.271$&$0.855$&$-0.121$&$0.108$&$0.182$&$0.121$\tabularnewline
10171&$-0.170$&$0.100$&$0.913$&$0.843$&$ 0.235$&$0.300$&$0.172$&$0.222$\tabularnewline
10174&$-0.063$&$0.112$&$0.403$&$0.547$&$-0.207$&$0.105$&$0.445$&$0.651$\tabularnewline
10175&$ 0.335$&$0.240$&$0.219$&$0.242$&$ 0.016$&$0.160$&$0.267$&$0.256$\tabularnewline
10176&$-0.084$&$0.108$&$0.596$&$0.238$&$-0.065$&$0.121$&$1.125$&$0.571$\tabularnewline
10178&$-0.397$&$0.126$&$2.367$&$1.774$&$-0.105$&$0.110$&$0.511$&$0.189$\tabularnewline
10179&$ 0.034$&$0.133$&$0.658$&$0.957$&$ 0.322$&$0.387$&$0.114$&$0.171$\tabularnewline
10181&$ 0.034$&$0.133$&$0.270$&$0.271$&$-0.074$&$0.119$&$1.015$&$0.462$\tabularnewline
10183&$-0.170$&$0.100$&$1.014$&$1.003$&$-0.076$&$0.117$&$0.625$&$0.533$\tabularnewline
10184&$ 0.115$&$0.154$&$0.637$&$0.510$&$ 0.048$&$0.179$&$1.462$&$0.704$\tabularnewline
10185&$-0.012$&$0.122$&$0.859$&$1.235$&$ 0.134$&$0.245$&$0.210$&$0.243$\tabularnewline
10186&$ 0.002$&$0.125$&$0.857$&$0.868$&$ 0.322$&$0.387$&$0.114$&$0.171$\tabularnewline
10187&$-0.115$&$0.104$&$0.706$&$0.760$&$-0.105$&$0.110$&$0.511$&$0.189$\tabularnewline
10188&$ 0.277$&$0.211$&$1.107$&$1.299$&$ 0.844$&$0.727$&$0.114$&$0.249$\tabularnewline
10189&$ 0.234$&$0.195$&$0.429$&$0.302$&$-0.007$&$0.147$&$1.102$&$0.540$\tabularnewline
10190&$-0.215$&$0.100$&$0.730$&$0.636$&$-0.007$&$0.147$&$1.090$&$0.557$\tabularnewline
10191&$ 0.281$&$0.216$&$0.455$&$0.578$&$ 0.322$&$0.387$&$0.114$&$0.171$\tabularnewline
10192&$-0.105$&$0.105$&$0.492$&$0.428$&$-0.025$&$0.138$&$0.423$&$0.331$\tabularnewline
10193&$ 0.035$&$0.132$&$0.884$&$0.986$&$ 0.016$&$0.160$&$1.381$&$0.617$\tabularnewline
10195&$-0.085$&$0.108$&$0.437$&$0.629$&$-0.123$&$0.107$&$0.645$&$1.053$\tabularnewline
10196&$-0.095$&$0.107$&$0.355$&$0.278$&$ 0.104$&$0.216$&$0.428$&$0.337$\tabularnewline
10197&$ 0.052$&$0.138$&$0.942$&$0.530$&$-0.025$&$0.138$&$0.423$&$0.331$\tabularnewline
10198&$-0.205$&$0.100$&$0.456$&$0.257$&$-0.040$&$0.131$&$0.391$&$0.354$\tabularnewline
10199&$-0.026$&$0.119$&$1.178$&$0.578$&$-0.025$&$0.138$&$1.337$&$0.582$\tabularnewline
10200&$ 0.115$&$0.154$&$0.657$&$0.440$&$-0.076$&$0.117$&$0.545$&$0.294$\tabularnewline
10201&$ 0.140$&$0.162$&$0.883$&$0.774$&$-0.040$&$0.131$&$0.391$&$0.354$\tabularnewline
10202&$-0.124$&$0.103$&$0.637$&$0.872$&$-0.074$&$0.119$&$1.017$&$0.560$\tabularnewline
10203&$-0.040$&$0.134$&$0.785$&$0.444$&$-0.635$&$0.211$&$0.168$&$0.153$\tabularnewline
10204&$ 0.335$&$0.240$&$1.075$&$0.693$&$-0.040$&$0.131$&$0.403$&$0.335$\tabularnewline
10206&$ 0.140$&$0.162$&$0.559$&$0.742$&$-0.053$&$0.126$&$1.354$&$0.629$\tabularnewline
10208&$-0.116$&$0.104$&$0.873$&$0.564$&$-0.025$&$0.138$&$0.423$&$0.331$\tabularnewline
10209&$-0.026$&$0.119$&$0.702$&$0.738$&$ 0.322$&$0.387$&$0.114$&$0.171$\tabularnewline
10210&$-0.084$&$0.108$&$1.296$&$0.899$&$-0.138$&$0.105$&$0.425$&$0.108$\tabularnewline
10212&$ 0.404$&$0.273$&$0.298$&$0.362$&$-0.172$&$0.103$&$0.384$&$0.513$\tabularnewline
10213&$ 0.200$&$0.184$&$0.584$&$0.883$&$ 0.235$&$0.300$&$0.172$&$0.222$\tabularnewline
10214&$ 0.140$&$0.164$&$0.629$&$0.756$&$-0.007$&$0.147$&$1.130$&$0.548$\tabularnewline
10215&$-0.038$&$0.116$&$0.264$&$0.311$&$ 0.048$&$0.179$&$0.440$&$0.313$\tabularnewline
10216&$-0.026$&$0.119$&$0.490$&$0.655$&$ 0.844$&$0.727$&$0.114$&$0.249$\tabularnewline
10217&$-0.025$&$0.119$&$0.792$&$0.816$&$ 0.235$&$0.300$&$0.172$&$0.222$\tabularnewline
10218&$ 0.019$&$0.128$&$0.765$&$0.414$&$-0.086$&$0.114$&$0.686$&$0.293$\tabularnewline
10219&$-0.039$&$0.116$&$1.280$&$1.535$&$-0.021$&$0.142$&$0.408$&$0.382$\tabularnewline
10220&$ 0.034$&$0.133$&$0.403$&$0.217$&$-0.002$&$0.152$&$1.267$&$0.573$\tabularnewline
10221&$-0.116$&$0.104$&$1.133$&$0.643$&$-0.189$&$0.104$&$1.693$&$0.575$\tabularnewline
10222&$ 0.052$&$0.138$&$0.752$&$0.484$&$-0.180$&$0.103$&$0.416$&$0.600$\tabularnewline
10223&$ 0.002$&$0.125$&$0.751$&$0.795$&$ 0.134$&$0.245$&$0.210$&$0.245$\tabularnewline
10224&$-0.196$&$0.100$&$0.810$&$0.797$&$-0.156$&$0.103$&$0.720$&$0.958$\tabularnewline
10226&$ 0.199$&$0.182$&$0.321$&$0.511$&$ 0.322$&$0.387$&$0.114$&$0.171$\tabularnewline
10227&$-0.161$&$0.100$&$0.381$&$0.324$&$-0.121$&$0.108$&$2.351$&$0.526$\tabularnewline
10230&$ 0.200$&$0.184$&$0.988$&$0.772$&$-0.121$&$0.108$&$0.117$&$0.089$\tabularnewline
10231&$-0.096$&$0.107$&$1.160$&$0.923$&$ 0.048$&$0.179$&$0.490$&$0.325$\tabularnewline
10232&$ 0.052$&$0.138$&$1.198$&$1.570$&$-0.065$&$0.121$&$1.010$&$0.813$\tabularnewline
10233&$-0.085$&$0.108$&$0.804$&$0.612$&$-0.037$&$0.134$&$0.359$&$0.394$\tabularnewline
10234&$ 0.140$&$0.164$&$0.681$&$0.703$&$-0.138$&$0.105$&$1.278$&$1.260$\tabularnewline
10237&$ 0.154$&$0.177$&$0.366$&$0.513$&$-0.074$&$0.119$&$0.917$&$1.057$\tabularnewline
10238&$ 0.052$&$0.138$&$0.279$&$0.290$&$-0.037$&$0.134$&$0.215$&$0.296$\tabularnewline
10240&$-0.310$&$0.109$&$2.264$&$2.377$&$-0.040$&$0.131$&$1.168$&$0.595$\tabularnewline
\hline
\end{longtable}}




%%%%%%%%%%%%%%%%%%%%%%%%%%%%%%%%%%%%%%%%%%%%%%%%%%%
\section{Stacking Procedure for Estimating Gains in Skill/Knowledge of the Second Empirical Study}
\label{sec:irt-learning-outcomes-second-study}


\subsection{Checking Assumptions}

\subsubsection*{Test of Unidimensionality}

\autoref{tab:test-unidimensionality-irt-gains-skill-knowledge-second-study} shows the results for the test of unidimensionality in which the goodness of fit statistics indicate strong multidimensionality ($DETECT > 1.00$) for the \emph{Pre-test} and \emph{Post-test} with values of $126.482$ and $79.021$. 
Essential unidimensionality in the data structure is indicated by the ASSI index in the \emph{Pre-test} (with value of $0.2$).
The index AGFI in the \emph{Pre-test} and \emph{Post-test} indicates good fit with values greater than $0.95$.
A good fit with the unidimensional CFA is indicated by the TLI and CFI indices for the pre-test and post-test.

%latex.default(data_df, caption = paste0("Goodness of fit statistics related to the test of unidimensionality",     "in the ", irt.short.name, "-based instrument ", in_title),     size = "small", longtable = T, ctable = F, landscape = F,     where = "!htbp", file = filename, append = T)%
\setlongtables{\scriptsize
\begin{longtable}{lrrrrrrrr}\caption{Goodness of fit statistics related to the test of unidimensionality in the GPCM-based instrument for measuring the gains in skill/knowledge of the second empirical study} \tabularnewline
\hline\hline
\multicolumn{1}{l}{}&\multicolumn{1}{c}{df}&\multicolumn{1}{c}{chisq}&\multicolumn{1}{c}{AGFI}&\multicolumn{1}{c}{TLI}&\multicolumn{1}{c}{CFI}&\multicolumn{1}{c}{DETECT}&\multicolumn{1}{c}{ASSI}&\multicolumn{1}{c}{RATIO}\tabularnewline
\hline
\endfirsthead\caption[]{\em (continued)} \tabularnewline
\hline
\multicolumn{1}{l}{}&\multicolumn{1}{c}{df}&\multicolumn{1}{c}{chisq}&\multicolumn{1}{c}{AGFI}&\multicolumn{1}{c}{TLI}&\multicolumn{1}{c}{CFI}&\multicolumn{1}{c}{DETECT}&\multicolumn{1}{c}{ASSI}&\multicolumn{1}{c}{RATIO}\tabularnewline
\hline
\endhead
\hline
\multicolumn{9}{r}{\tiny df: degree of freedom; AGFI: Adjusted Goodness of Fit Index; CFI: Comparative Fit Index; TLI: Tucker-Lewis Index;}
\endfoot
\label{tab:test-unidimensionality-irt-gains-skill-knowledge-second-study}
Pre-test&$9$&$8.908$&$0.970$&$1.003$&$1.000$&$126.482$&$0.2$&$0.712$\tabularnewline
Post-test&$5$&$7.052$&$0.931$&$0.813$&$0.907$&$ 79.021$&$0.4$&$0.515$\tabularnewline
\hline
\end{longtable}}

\subsubsection*{Test of Local Independence}

Results from the test of local independence in the GPCM-based instrument for measuring gains in skill/knowledge of the second empirical study are summarized in \autoref{tab:test-local-independence-irt-gain-skill-knowledge-second-study}.
The null condition of local independence is not rejected in the \emph{Pre-test} and \emph{Post-test}.

%latex.default(data_df, caption = paste0("Item residual correlation statistics",     "related to the test of local independence", "in the ", irt.short.name,     "-based instrument ", in_title), size = "small", longtable = T,     ctable = F, landscape = F, where = "!htbp", file = filename,     append = T)%
\setlongtables{\scriptsize
\begin{longtable}{lrrrrr}\caption{Item residual correlation statistics related to the test of local independence in the GPCM-based instrument for measuring gains in skill/knowledge of the second empirical study} \tabularnewline
\hline\hline
\multicolumn{1}{l}{}&\multicolumn{1}{c}{max.chisq}&\multicolumn{1}{c}{maxaQ3}&\multicolumn{1}{c}{MADaQ3}&\multicolumn{1}{c}{SRMSR}&\multicolumn{1}{c}{p.value}\tabularnewline
\hline
\endfirsthead\caption[]{\em (continued)} \tabularnewline
\hline
\multicolumn{1}{l}{}&\multicolumn{1}{c}{max.chisq}&\multicolumn{1}{c}{maxaQ3}&\multicolumn{1}{c}{MADaQ3}&\multicolumn{1}{c}{SRMSR}&\multicolumn{1}{c}{p.value}\tabularnewline
\hline
\endhead
\hline
\multicolumn{6}{r}{\tiny aQ3: adjusted correlation of item residuals; maxaQ3: maximum aQ3;}\tabularnewline
\multicolumn{6}{r}{\tiny MADaQ3: Median Absolute Deviation of aQ3;}
\endfoot
\label{tab:test-local-independence-irt-gain-skill-knowledge-second-study}
Pre-test&$336.290$&$0.399$&$0.132$&$0.349$&$0.051$\tabularnewline
Post-test&$ 77.018$&$0.303$&$0.136$&$0.309$&$0.397$\tabularnewline
\hline
\end{longtable}}

\subsection{Item Parameters}

\autoref{tab:item-parameters-pre-second-study} shows the estimated parameters for the GPCM-based instrument used to measure the pre-test skill/knowledge of the second empirical study.
These parameters had been calculated using the MML method \cite{BockAitkin1981}, so that the value in row \aspas{B.Cat$x$} and column \aspas{$i$} is the item slope $b_{i,x}$ of item $i$ in the category \aspas{$x$}, and the value in the row \aspas{AXsi.Cat$x$} and column \aspas{$i$} is the item intercept $a_{i,x}\xi$ of item $i$ in the category \aspas{$x$}.
According to the Infit/Outfit statistics of items, no one mean-square value is greater than $2.0$ indicating that the measurement system is not distorted or degraded by the items.

%latex.default(estimated_params_df, caption = paste0("Estimated parameters in the ",     irt.short.name, "-based instrument", " for measuring the ",     lname), size = "scriptsize", longtable = T, ctable = F, landscape = F,     where = "!htbp", file = filename, append = T)%
\setlongtables{\scriptsize
\begin{longtable}{lrrrrrrrrr}\caption{Estimated parameters in the GPCM-based instrument for measuring the pre-test skill/knowledge of the second empirical study} \tabularnewline
\hline\hline
\multicolumn{1}{l}{}&\multicolumn{1}{c}{An3a}&\multicolumn{1}{c}{An3b}&\multicolumn{1}{c}{Ap2a}&\multicolumn{1}{c}{Ap2b}&\multicolumn{1}{c}{P2s1}&\multicolumn{1}{c}{Un3}\tabularnewline
\hline
\endfirsthead\caption[]{\em (continued)} \tabularnewline
\hline
\multicolumn{1}{l}{}&\multicolumn{1}{c}{An3a}&\multicolumn{1}{c}{An3b}&\multicolumn{1}{c}{Ap2a}&\multicolumn{1}{c}{Ap2b}&\multicolumn{1}{c}{P2s1}&\multicolumn{1}{c}{Un3}\tabularnewline
\hline
\endhead
\hline
\endfoot
\label{tab:item-parameters-pre-second-study}
xsi.item&$-0.079$&$-0.010$&$-1.966$&$-0.013$&$-22.124$&$-0.058$\tabularnewline
B.Cat0&$ 0.000$&$ 0.000$&$ 0.000$&$ 0.000$&$  0.000$&$ 0.000$\tabularnewline
B.Cat1&$ 1.000$&$ 1.000$&$ 1.000$&$ 1.000$&$  1.000$&$ 1.000$\tabularnewline
B.Cat2&$ 2.000$&$ 2.000$&$ 0.000$&$ 2.000$&$  2.000$&$ 2.000$\tabularnewline
B.Cat3&$ 3.000$&$ 3.000$&$ 0.000$&$ 3.000$&$  0.000$&$ 3.000$\tabularnewline
B.Cat4&$ 4.000$&$ 4.000$&$ 0.000$&$ 4.000$&$  0.000$&$ 4.000$\tabularnewline
B.Cat5&$ 5.000$&$ 5.000$&$ 0.000$&$ 5.000$&$  0.000$&$ 5.000$\tabularnewline
B.Cat6&$ 6.000$&$ 6.000$&$ 0.000$&$ 6.000$&$  0.000$&$ 6.000$\tabularnewline
B.Cat7&$ 7.000$&$ 7.000$&$ 0.000$&$ 7.000$&$  0.000$&$ 7.000$\tabularnewline
B.Cat8&$ 8.000$&$ 8.000$&$ 0.000$&$ 8.000$&$  0.000$&$ 8.000$\tabularnewline
B.Cat9&$ 9.000$&$ 9.000$&$ 0.000$&$ 9.000$&$  0.000$&$ 9.000$\tabularnewline
B.Cat10&$10.000$&$10.000$&$ 0.000$&$10.000$&$  0.000$&$10.000$\tabularnewline
B.Cat11&$11.000$&$11.000$&$ 0.000$&$11.000$&$  0.000$&$11.000$\tabularnewline
B.Cat12&$12.000$&$12.000$&$ 0.000$&$12.000$&$  0.000$&$12.000$\tabularnewline
B.Cat13&$13.000$&$ 0.000$&$ 0.000$&$13.000$&$  0.000$&$13.000$\tabularnewline
B.Cat14&$14.000$&$ 0.000$&$ 0.000$&$14.000$&$  0.000$&$14.000$\tabularnewline
B.Cat15&$15.000$&$ 0.000$&$ 0.000$&$ 0.000$&$  0.000$&$15.000$\tabularnewline
B.Cat16&$16.000$&$ 0.000$&$ 0.000$&$ 0.000$&$  0.000$&$16.000$\tabularnewline
B.Cat17&$ 0.000$&$ 0.000$&$ 0.000$&$ 0.000$&$  0.000$&$17.000$\tabularnewline
B.Cat18&$ 0.000$&$ 0.000$&$ 0.000$&$ 0.000$&$  0.000$&$18.000$\tabularnewline
B.Cat19&$ 0.000$&$ 0.000$&$ 0.000$&$ 0.000$&$  0.000$&$19.000$\tabularnewline
B.Cat20&$ 0.000$&$ 0.000$&$ 0.000$&$ 0.000$&$  0.000$&$20.000$\tabularnewline
B.Cat21&$ 0.000$&$ 0.000$&$ 0.000$&$ 0.000$&$  0.000$&$21.000$\tabularnewline
B.Cat22&$ 0.000$&$ 0.000$&$ 0.000$&$ 0.000$&$  0.000$&$22.000$\tabularnewline
B.Cat23&$ 0.000$&$ 0.000$&$ 0.000$&$ 0.000$&$  0.000$&$23.000$\tabularnewline
B.Cat24&$ 0.000$&$ 0.000$&$ 0.000$&$ 0.000$&$  0.000$&$24.000$\tabularnewline
B.Cat25&$ 0.000$&$ 0.000$&$ 0.000$&$ 0.000$&$  0.000$&$25.000$\tabularnewline
B.Cat26&$ 0.000$&$ 0.000$&$ 0.000$&$ 0.000$&$  0.000$&$26.000$\tabularnewline
B.Cat27&$ 0.000$&$ 0.000$&$ 0.000$&$ 0.000$&$  0.000$&$27.000$\tabularnewline
B.Cat28&$ 0.000$&$ 0.000$&$ 0.000$&$ 0.000$&$  0.000$&$28.000$\tabularnewline
B.Cat29&$ 0.000$&$ 0.000$&$ 0.000$&$ 0.000$&$  0.000$&$29.000$\tabularnewline
B.Cat30&$ 0.000$&$ 0.000$&$ 0.000$&$ 0.000$&$  0.000$&$30.000$\tabularnewline
B.Cat31&$ 0.000$&$ 0.000$&$ 0.000$&$ 0.000$&$  0.000$&$31.000$\tabularnewline
B.Cat32&$ 0.000$&$ 0.000$&$ 0.000$&$ 0.000$&$  0.000$&$32.000$\tabularnewline
B.Cat33&$ 0.000$&$ 0.000$&$ 0.000$&$ 0.000$&$  0.000$&$33.000$\tabularnewline
B.Cat34&$ 0.000$&$ 0.000$&$ 0.000$&$ 0.000$&$  0.000$&$34.000$\tabularnewline
B.Cat35&$ 0.000$&$ 0.000$&$ 0.000$&$ 0.000$&$  0.000$&$35.000$\tabularnewline
B.Cat36&$ 0.000$&$ 0.000$&$ 0.000$&$ 0.000$&$  0.000$&$36.000$\tabularnewline
B.Cat37&$ 0.000$&$ 0.000$&$ 0.000$&$ 0.000$&$  0.000$&$37.000$\tabularnewline
B.Cat38&$ 0.000$&$ 0.000$&$ 0.000$&$ 0.000$&$  0.000$&$38.000$\tabularnewline
B.Cat39&$ 0.000$&$ 0.000$&$ 0.000$&$ 0.000$&$  0.000$&$39.000$\tabularnewline
B.Cat40&$ 0.000$&$ 0.000$&$ 0.000$&$ 0.000$&$  0.000$&$40.000$\tabularnewline
AXsi.Cat0&$ 0.000$&$ 0.000$&$ 0.000$&$ 0.000$&$  0.000$&$ 0.000$\tabularnewline
AXsi.Cat1&$-5.616$&$-7.424$&$ 1.966$&$-6.053$&$ 44.471$&$-5.448$\tabularnewline
AXsi.Cat2&$-5.666$&$-8.141$&$$&$-7.940$&$ 44.248$&$-7.656$\tabularnewline
AXsi.Cat3&$-0.287$&$-7.619$&$$&$-8.724$&$$&$-8.036$\tabularnewline
AXsi.Cat4&$ 0.813$&$-2.717$&$$&$-8.077$&$$&$-7.771$\tabularnewline
AXsi.Cat5&$-4.826$&$ 0.127$&$$&$-6.069$&$$&$-5.950$\tabularnewline
AXsi.Cat6&$-5.910$&$-0.762$&$$&$ 0.123$&$$&$-1.000$\tabularnewline
AXsi.Cat7&$-4.711$&$-7.448$&$$&$-1.682$&$$&$-1.025$\tabularnewline
AXsi.Cat8&$ 1.180$&$-9.296$&$$&$-6.004$&$$&$-4.663$\tabularnewline
AXsi.Cat9&$-4.363$&$-9.859$&$$&$-7.451$&$$&$-0.310$\tabularnewline
AXsi.Cat10&$-4.494$&$-9.344$&$$&$-7.985$&$$&$-0.321$\tabularnewline
AXsi.Cat11&$-0.709$&$-7.386$&$$&$-7.488$&$$&$-4.996$\tabularnewline
AXsi.Cat12&$ 1.019$&$ 0.126$&$$&$-6.370$&$$&$-6.747$\tabularnewline
AXsi.Cat13&$-4.583$&$$&$$&$-2.806$&$$&$-7.436$\tabularnewline
AXsi.Cat14&$-4.845$&$$&$$&$ 0.186$&$$&$-7.448$\tabularnewline
AXsi.Cat15&$-1.408$&$$&$$&$$&$$&$-6.499$\tabularnewline
AXsi.Cat16&$ 1.265$&$$&$$&$$&$$&$-4.881$\tabularnewline
AXsi.Cat17&$$&$$&$$&$$&$$&$-0.865$\tabularnewline
AXsi.Cat18&$$&$$&$$&$$&$$&$ 0.981$\tabularnewline
AXsi.Cat19&$$&$$&$$&$$&$$&$-0.262$\tabularnewline
AXsi.Cat20&$$&$$&$$&$$&$$&$-4.141$\tabularnewline
AXsi.Cat21&$$&$$&$$&$$&$$&$-5.534$\tabularnewline
AXsi.Cat22&$$&$$&$$&$$&$$&$-6.731$\tabularnewline
AXsi.Cat23&$$&$$&$$&$$&$$&$-7.145$\tabularnewline
AXsi.Cat24&$$&$$&$$&$$&$$&$-7.113$\tabularnewline
AXsi.Cat25&$$&$$&$$&$$&$$&$-6.642$\tabularnewline
AXsi.Cat26&$$&$$&$$&$$&$$&$-5.970$\tabularnewline
AXsi.Cat27&$$&$$&$$&$$&$$&$-4.529$\tabularnewline
AXsi.Cat28&$$&$$&$$&$$&$$&$-1.038$\tabularnewline
AXsi.Cat29&$$&$$&$$&$$&$$&$ 0.694$\tabularnewline
AXsi.Cat30&$$&$$&$$&$$&$$&$ 0.459$\tabularnewline
AXsi.Cat31&$$&$$&$$&$$&$$&$-3.326$\tabularnewline
AXsi.Cat32&$$&$$&$$&$$&$$&$-5.734$\tabularnewline
AXsi.Cat33&$$&$$&$$&$$&$$&$-6.869$\tabularnewline
AXsi.Cat34&$$&$$&$$&$$&$$&$-7.189$\tabularnewline
AXsi.Cat35&$$&$$&$$&$$&$$&$-7.135$\tabularnewline
AXsi.Cat36&$$&$$&$$&$$&$$&$-6.650$\tabularnewline
AXsi.Cat37&$$&$$&$$&$$&$$&$-6.214$\tabularnewline
AXsi.Cat38&$$&$$&$$&$$&$$&$-4.356$\tabularnewline
AXsi.Cat39&$$&$$&$$&$$&$$&$-0.379$\tabularnewline
AXsi.Cat40&$$&$$&$$&$$&$$&$ 2.319$\tabularnewline
\hline
max.Outfit&$ 1.008$&$ 1.000$&$ 1.000$&$ 1.005$&$  1.000$&$ 0.613$\tabularnewline
max.Infit&$ 1.008$&$ 1.000$&$ 1.000$&$ 1.005$&$  1.000$&$ 0.613$\tabularnewline
\hline
\end{longtable}}

\autoref{tab:item-parameters-pos-second-study} shows the estimated parameters for the GPCM-based instrument used to measure the post-test skill/knowledge of the second empirical study.
These parameters had been calculated using the MML method \cite{BockAitkin1981}, so that the value in row \aspas{B.Cat$x$} and column \aspas{$i$} is the item slope $b_{i,x}$ of item $i$ in the category \aspas{$x$}, and the value in the row \aspas{AXsi.Cat$x$} and column \aspas{$i$} is the item intercept $a_{i,x}\xi$ of item $i$ in the category \aspas{$x$}.
According to the Infit/Outfit statistics of items, no one mean-square value is greater than $2.0$ indicating that the measurement system is not distorted or degraded by the items.

%latex.default(estimated_params_df, caption = paste0("Estimated parameters in the ",     irt.short.name, "-based instrument", " for measuring the ",     lname), size = "scriptsize", longtable = T, ctable = F, landscape = F,     where = "!htbp", file = filename, append = T)%
\setlongtables{\scriptsize
\begin{longtable}{lrrrrrrr}\caption{Estimated parameters in the GPCM-based instrument for measuring the post-test skill/knowledge in the second empirical study} \tabularnewline
\hline\hline
\multicolumn{1}{l}{}&\multicolumn{1}{c}{AnC1}&\multicolumn{1}{c}{AnC2}&\multicolumn{1}{c}{ApB1}&\multicolumn{1}{c}{ApB2}&\multicolumn{1}{c}{PCs3}\tabularnewline
\hline
\endfirsthead\caption[]{\em (continued)} \tabularnewline
\hline
\multicolumn{1}{l}{}&\multicolumn{1}{c}{AnC1}&\multicolumn{1}{c}{AnC2}&\multicolumn{1}{c}{ApB1}&\multicolumn{1}{c}{ApB2}&\multicolumn{1}{c}{PCs3}\tabularnewline
\hline
\endhead
\hline
\endfoot
\label{tab:item-parameters-pos-second-study}
xsi.item&$-0.119$&$-0.133$&$-2.175$&$-0.106$&$-15.226$\tabularnewline
B.Cat0&$ 0.000$&$ 0.000$&$ 0.000$&$ 0.000$&$  0.000$\tabularnewline
B.Cat1&$ 1.000$&$ 1.000$&$ 1.000$&$ 1.000$&$  1.000$\tabularnewline
B.Cat2&$ 2.000$&$ 2.000$&$ 0.000$&$ 2.000$&$  2.000$\tabularnewline
B.Cat3&$ 3.000$&$ 3.000$&$ 0.000$&$ 3.000$&$  3.000$\tabularnewline
B.Cat4&$ 4.000$&$ 4.000$&$ 0.000$&$ 4.000$&$  4.000$\tabularnewline
B.Cat5&$ 5.000$&$ 5.000$&$ 0.000$&$ 5.000$&$  0.000$\tabularnewline
B.Cat6&$ 6.000$&$ 6.000$&$ 0.000$&$ 6.000$&$  0.000$\tabularnewline
B.Cat7&$ 7.000$&$ 7.000$&$ 0.000$&$ 7.000$&$  0.000$\tabularnewline
B.Cat8&$ 8.000$&$ 8.000$&$ 0.000$&$ 8.000$&$  0.000$\tabularnewline
B.Cat9&$ 9.000$&$ 9.000$&$ 0.000$&$ 9.000$&$  0.000$\tabularnewline
B.Cat10&$10.000$&$10.000$&$ 0.000$&$10.000$&$  0.000$\tabularnewline
B.Cat11&$11.000$&$11.000$&$ 0.000$&$11.000$&$  0.000$\tabularnewline
B.Cat12&$12.000$&$12.000$&$ 0.000$&$12.000$&$  0.000$\tabularnewline
B.Cat13&$13.000$&$13.000$&$ 0.000$&$13.000$&$  0.000$\tabularnewline
B.Cat14&$14.000$&$14.000$&$ 0.000$&$14.000$&$  0.000$\tabularnewline
B.Cat15&$15.000$&$15.000$&$ 0.000$&$ 0.000$&$  0.000$\tabularnewline
B.Cat16&$16.000$&$16.000$&$ 0.000$&$ 0.000$&$  0.000$\tabularnewline
B.Cat17&$17.000$&$17.000$&$ 0.000$&$ 0.000$&$  0.000$\tabularnewline
B.Cat18&$18.000$&$18.000$&$ 0.000$&$ 0.000$&$  0.000$\tabularnewline
AXsi.Cat0&$ 0.000$&$ 0.000$&$ 0.000$&$ 0.000$&$  0.000$\tabularnewline
AXsi.Cat1&$-5.378$&$-6.301$&$ 2.175$&$-5.704$&$ 61.918$\tabularnewline
AXsi.Cat2&$-0.701$&$-7.422$&$$&$-5.752$&$ 61.311$\tabularnewline
AXsi.Cat3&$-5.803$&$-6.199$&$$&$-1.807$&$ 61.466$\tabularnewline
AXsi.Cat4&$-5.661$&$ 1.386$&$$&$-1.808$&$ 60.906$\tabularnewline
AXsi.Cat5&$ 0.696$&$ 1.792$&$$&$-4.963$&$$\tabularnewline
AXsi.Cat6&$ 0.918$&$ 2.194$&$$&$ 0.613$&$$\tabularnewline
AXsi.Cat7&$-4.952$&$-3.722$&$$&$-0.407$&$$\tabularnewline
AXsi.Cat8&$-6.102$&$-5.093$&$$&$-5.151$&$$\tabularnewline
AXsi.Cat9&$-5.226$&$-5.264$&$$&$-7.286$&$$\tabularnewline
AXsi.Cat10&$-0.701$&$-3.726$&$$&$-8.101$&$$\tabularnewline
AXsi.Cat11&$ 0.700$&$ 2.074$&$$&$-8.113$&$$\tabularnewline
AXsi.Cat12&$ 1.951$&$ 2.299$&$$&$-7.278$&$$\tabularnewline
AXsi.Cat13&$-4.933$&$-3.907$&$$&$-5.135$&$$\tabularnewline
AXsi.Cat14&$-6.289$&$-5.710$&$$&$ 1.478$&$$\tabularnewline
AXsi.Cat15&$-5.946$&$-6.251$&$$&$$&$$\tabularnewline
AXsi.Cat16&$-4.969$&$-5.876$&$$&$$&$$\tabularnewline
AXsi.Cat17&$-0.702$&$-3.951$&$$&$$&$$\tabularnewline
AXsi.Cat18&$ 2.149$&$ 2.392$&$$&$$&$$\tabularnewline
\hline
max.Outfit&$ 1.007$&$ 1.000$&$ 1.000$&$ 1.009$&$  1.000$\tabularnewline
max.Infit&$ 1.007$&$ 1.000$&$ 1.000$&$ 1.009$&$  1.000$\tabularnewline
\hline
\end{longtable}}


\subsection{Gains in Skill/Knowledge as Latent Trait Estimates}

\autoref{tab:gain-skill-knowledge-estimates-second-study} shows the latent trait estimates by the GPCM-based instrument for measuring the gains in skill/knowledge of the second empirical study.

%latex.default(data_df, caption = paste0("Latent trait estimates and person model fit of the ",     irt.short.name, "-based instrument ", in_title), size = "scriptsize",     longtable = T, ctable = F, landscape = T, rowlabel = "",     where = "!htbp", file = filename, append = T)%
\setlongtables{\scriptsize
\begin{longtable}{l|rrrr|rrrr}\caption{Latent trait estimates and person model fit of the GPCM-based instrument for measuring gains in skill/knowledge of the second empirical study} \tabularnewline
\hline\hline
\multicolumn{1}{l}{}&\multicolumn{4}{|c}{Pre-test}&\multicolumn{4}{|c}{Post-test}\tabularnewline
\multicolumn{1}{l}{UserID}&\multicolumn{1}{|c}{theta}&\multicolumn{1}{c}{error}&\multicolumn{1}{c}{Outfit}&\multicolumn{1}{c}{Infit}&\multicolumn{1}{|c}{theta}&\multicolumn{1}{c}{error}&\multicolumn{1}{c}{Outfit}&\multicolumn{1}{c}{Infit}\tabularnewline
\hline
\endfirsthead\caption[]{\em (continued)} \tabularnewline
\hline
\multicolumn{1}{l}{}&\multicolumn{4}{|c}{Pre-test}&\multicolumn{4}{|c}{Post-test}\tabularnewline
\multicolumn{1}{l}{UserID}&\multicolumn{1}{|c}{theta}&\multicolumn{1}{c}{error}&\multicolumn{1}{c}{Outfit}&\multicolumn{1}{c}{Infit}&\multicolumn{1}{|c}{theta}&\multicolumn{1}{c}{error}&\multicolumn{1}{c}{Outfit}&\multicolumn{1}{c}{Infit}\tabularnewline
\hline
\endhead
\hline
\endfoot
\label{tab:gain-skill-knowledge-estimates-second-study}
10169&$ 0.009$&$0.067$&$1.001$&$0.908$&$-0.236$&$0.135$&$2.199$&$0.736$\tabularnewline
10170&$-0.040$&$0.060$&$0.628$&$0.356$&$-0.153$&$0.116$&$2.807$&$0.850$\tabularnewline
10172&$ 0.053$&$0.082$&$0.575$&$0.424$&$ 0.411$&$0.301$&$0.360$&$0.250$\tabularnewline
10174&$-0.119$&$0.077$&$2.530$&$0.879$&$-0.141$&$0.114$&$0.413$&$0.388$\tabularnewline
10175&$ 0.141$&$0.118$&$0.499$&$0.489$&$ 0.286$&$0.207$&$0.163$&$0.207$\tabularnewline
10176&$ 0.012$&$0.068$&$0.300$&$0.406$&$-0.028$&$0.109$&$1.031$&$1.270$\tabularnewline
10178&$ 0.005$&$0.066$&$0.600$&$0.700$&$-0.098$&$0.110$&$0.594$&$0.757$\tabularnewline
10179&$-0.007$&$0.063$&$0.529$&$0.778$&$ 0.047$&$0.117$&$0.177$&$0.185$\tabularnewline
10181&$-0.020$&$0.061$&$0.834$&$0.347$&$ 0.024$&$0.113$&$0.259$&$0.299$\tabularnewline
10183&$-0.204$&$0.112$&$0.616$&$0.784$&$-0.119$&$0.112$&$0.243$&$0.240$\tabularnewline
10184&$ 0.053$&$0.082$&$0.492$&$0.393$&$ 0.069$&$0.120$&$1.157$&$1.131$\tabularnewline
10185&$ 0.148$&$0.121$&$0.418$&$0.498$&$ 0.300$&$0.222$&$0.142$&$0.190$\tabularnewline
10186&$-0.013$&$0.065$&$0.344$&$0.201$&$-0.165$&$0.119$&$0.530$&$0.728$\tabularnewline
10187&$ 0.009$&$0.067$&$0.673$&$0.681$&$-0.038$&$0.109$&$0.117$&$0.114$\tabularnewline
10188&$ 0.307$&$0.205$&$0.301$&$0.196$&$-0.018$&$0.110$&$0.401$&$0.480$\tabularnewline
10189&$-0.093$&$0.068$&$1.485$&$0.165$&$-0.153$&$0.116$&$0.054$&$0.046$\tabularnewline
10190&$-0.010$&$0.063$&$0.702$&$0.333$&$ 0.047$&$0.117$&$0.216$&$0.237$\tabularnewline
10192&$ 0.005$&$0.066$&$1.455$&$0.382$&$-0.099$&$0.109$&$0.475$&$0.747$\tabularnewline
10196&$-0.063$&$0.062$&$0.347$&$0.097$&$ 0.069$&$0.120$&$0.237$&$0.183$\tabularnewline
10197&$ 0.012$&$0.068$&$0.589$&$0.568$&$-0.029$&$0.108$&$0.302$&$0.123$\tabularnewline
10198&$-0.350$&$0.182$&$0.218$&$0.123$&$-0.029$&$0.108$&$0.631$&$0.698$\tabularnewline
10200&$-0.082$&$0.065$&$0.388$&$0.222$&$-0.218$&$0.129$&$1.676$&$0.180$\tabularnewline
10201&$ 0.025$&$0.072$&$0.762$&$0.655$&$ 0.057$&$0.118$&$1.146$&$1.711$\tabularnewline
10202&$-0.078$&$0.065$&$2.116$&$1.299$&$-0.049$&$0.108$&$0.585$&$0.636$\tabularnewline
10203&$-0.195$&$0.108$&$0.366$&$0.301$&$-0.584$&$0.257$&$1.201$&$0.305$\tabularnewline
10204&$ 0.126$&$0.111$&$0.595$&$0.632$&$ 0.109$&$0.129$&$0.493$&$0.356$\tabularnewline
10206&$ 0.016$&$0.069$&$0.549$&$0.225$&$-0.049$&$0.108$&$0.606$&$0.970$\tabularnewline
10208&$ 0.216$&$0.151$&$0.335$&$0.257$&$ 0.139$&$0.138$&$0.205$&$0.265$\tabularnewline
10210&$-0.063$&$0.062$&$0.249$&$0.063$&$-0.019$&$0.109$&$0.634$&$0.810$\tabularnewline
10212&$ 0.016$&$0.069$&$0.369$&$0.380$&$-0.018$&$0.110$&$0.545$&$0.708$\tabularnewline
10213&$ 0.105$&$0.103$&$0.436$&$0.467$&$ 0.059$&$0.119$&$0.664$&$0.882$\tabularnewline
10214&$ 0.035$&$0.075$&$0.273$&$0.269$&$-0.019$&$0.109$&$0.531$&$0.485$\tabularnewline
10215&$-0.033$&$0.060$&$0.488$&$0.271$&$ 0.139$&$0.138$&$0.373$&$0.594$\tabularnewline
10217&$ 0.358$&$0.242$&$0.239$&$0.155$&$ 0.069$&$0.120$&$0.638$&$0.904$\tabularnewline
10218&$-0.011$&$0.063$&$0.706$&$0.392$&$-0.153$&$0.117$&$0.657$&$0.765$\tabularnewline
10219&$-0.062$&$0.062$&$0.150$&$0.061$&$ 0.059$&$0.119$&$0.172$&$0.177$\tabularnewline
10220&$-0.119$&$0.077$&$0.353$&$0.190$&$ 0.144$&$0.141$&$0.236$&$0.271$\tabularnewline
10221&$-0.119$&$0.077$&$0.832$&$0.905$&$-0.253$&$0.140$&$1.873$&$0.237$\tabularnewline
10224&$ 0.035$&$0.075$&$0.354$&$0.282$&$-0.009$&$0.109$&$1.109$&$0.894$\tabularnewline
10226&$-0.051$&$0.061$&$0.787$&$0.328$&$ 0.144$&$0.141$&$0.236$&$0.271$\tabularnewline
10227&$-0.059$&$0.061$&$1.654$&$0.122$&$ 0.023$&$0.112$&$0.768$&$0.870$\tabularnewline
10228&$-0.010$&$0.063$&$1.958$&$0.321$&$-0.130$&$0.113$&$1.394$&$1.609$\tabularnewline
10230&$ 0.032$&$0.074$&$0.185$&$0.292$&$ 0.139$&$0.138$&$0.530$&$0.746$\tabularnewline
10231&$ 0.216$&$0.151$&$0.335$&$0.257$&$ 0.250$&$0.186$&$0.489$&$0.322$\tabularnewline
10232&$-0.119$&$0.077$&$0.245$&$0.518$&$-0.049$&$0.108$&$0.730$&$0.877$\tabularnewline
10237&$ 0.005$&$0.066$&$1.148$&$2.103$&$ 0.300$&$0.222$&$0.142$&$0.190$\tabularnewline
10238&$-0.021$&$0.061$&$1.840$&$0.381$&$ 0.036$&$0.115$&$0.674$&$0.874$\tabularnewline
10240&$-0.033$&$0.060$&$0.968$&$1.356$&$-0.088$&$0.109$&$0.326$&$0.220$\tabularnewline
\hline
\end{longtable}}



%%%%%%%%%%%%%%%%%%%%%%%%%%%%%%%%%%%%%%%%%%%%%%%%%%%
\section{Stacking Procedure for Estimating Gains in Skill/Knowledge of the Third Empirical Study}
\label{sec:irt-learning-outcomes-third-study}


\subsection{Checking Assumptions}

\subsubsection*{Test of Unidimensionality}

\autoref{tab:test-unidimensionality-irt-gains-skill-knowledge-third-study} shows the results for the test of unidimensionality in which the goodness of fit statistics indicate strong multidimensionality ($DETECT > 1.00$) for the \emph{Pre-test} and \emph{Post-test}. 
Essential unidimensionality in the data structure is indicated by the ASSI index in the \emph{Pre-test} with value of $0.200$.
The index AGFI in the \emph{Pre-test} and \emph{Post-test} indicates good fit with values greater than $0.95$.
A good fit with the unidimensional CFA is indicated by the TLI and CFI indices for the pre-test and post-test.

%latex.default(data_df, caption = paste0("Goodness of fit statistics related to the test of unidimensionality",     "in the ", irt.short.name, "-based instrument ", in_title),     size = "small", longtable = T, ctable = F, landscape = F,     where = "!htbp", file = filename, append = T)%
\setlongtables{\scriptsize
\begin{longtable}{lrrrrrrrr}\caption{Goodness of fit statistics related to the test of unidimensionality in the GPCM-based instrument for measuring the gains in skill/knowledge of the third empirical study} \tabularnewline
\hline\hline
\multicolumn{1}{l}{}&\multicolumn{1}{c}{df}&\multicolumn{1}{c}{chisq}&\multicolumn{1}{c}{AGFI}&\multicolumn{1}{c}{TLI}&\multicolumn{1}{c}{CFI}&\multicolumn{1}{c}{DETECT}&\multicolumn{1}{c}{ASSI}&\multicolumn{1}{c}{RATIO}\tabularnewline
\hline
\endfirsthead\caption[]{\em (continued)} \tabularnewline
\hline
\multicolumn{1}{l}{}&\multicolumn{1}{c}{df}&\multicolumn{1}{c}{chisq}&\multicolumn{1}{c}{AGFI}&\multicolumn{1}{c}{TLI}&\multicolumn{1}{c}{CFI}&\multicolumn{1}{c}{DETECT}&\multicolumn{1}{c}{ASSI}&\multicolumn{1}{c}{RATIO}\tabularnewline
\hline
\endhead
\hline
\multicolumn{9}{r}{\tiny df: degree of freedom; AGFI: Adjusted Goodness of Fit Index; CFI: Comparative Fit Index; TLI: Tucker-Lewis Index;}
\endfoot
\label{tab:test-unidimensionality-irt-gains-skill-knowledge-third-study}
Pre-test&$ 9$&$8.664$&$0.964$&$1.036$&$1$&$194.158$&$0.200$&$0.457$\tabularnewline
Post-test&$14$&$9.815$&$0.962$&$2.826$&$1$&$258.820$&$0.429$&$0.645$\tabularnewline
\hline
\end{longtable}}

\subsubsection*{Test of Local Independence}

Results from the test of local independence in the GPCM-based instrument for measuring gains in skill/knowledge of the third empirical study are summarized in \autoref{tab:test-local-independence-irt-gain-skill-knowledge-third-study}.
The null condition of local independence is not rejected in the \emph{Pre-test} and \emph{Post-test}.

%latex.default(data_df, caption = paste0("Item residual correlation statistics",     "related to the test of local independence", "in the ", irt.short.name,     "-based instrument ", in_title), size = "small", longtable = T,     ctable = F, landscape = F, where = "!htbp", file = filename,     append = T)%
\setlongtables{\scriptsize
\begin{longtable}{lrrrrr}\caption{Item residual correlation statistics related to the test of local independence in the GPCM-based instrument for measuring gains in skill/knowledge of the third empirical study} \tabularnewline
\hline\hline
\multicolumn{1}{l}{}&\multicolumn{1}{c}{max.chisq}&\multicolumn{1}{c}{maxaQ3}&\multicolumn{1}{c}{MADaQ3}&\multicolumn{1}{c}{SRMSR}&\multicolumn{1}{c}{p.value}\tabularnewline
\hline
\endfirsthead\caption[]{\em (continued)} \tabularnewline
\hline
\multicolumn{1}{l}{}&\multicolumn{1}{c}{max.chisq}&\multicolumn{1}{c}{maxaQ3}&\multicolumn{1}{c}{MADaQ3}&\multicolumn{1}{c}{SRMSR}&\multicolumn{1}{c}{p.value}\tabularnewline
\hline
\endhead
\hline
\multicolumn{6}{r}{\tiny aQ3: adjusted correlation of item residuals; maxaQ3: maximum aQ3;}\tabularnewline
\multicolumn{6}{r}{\tiny MADaQ3: Median Absolute Deviation of aQ3;}
\endfoot
\label{tab:test-local-independence-irt-gain-skill-knowledge-third-study}
Pre-test&$ 285.450$&$0.361$&$0.142$&$0.188$&$0.112$\tabularnewline
Post-test&$6332.263$&$0.342$&$0.141$&$0.179$&$0.402$\tabularnewline
\hline
\end{longtable}}

\subsection{Item Parameters}

\autoref{tab:item-parameters-pre-third-study} shows the estimated parameters for the GPCM-based instrument used to measure the pre-test skill/knowledge of the third empirical study.
These parameters had been calculated using the MML method \cite{BockAitkin1981}, so that the value in row \aspas{B.Cat$x$} and column \aspas{$i$} is the item slope $b_{i,x}$ of item $i$ in the category \aspas{$x$}, and the value in the row \aspas{AXsi.Cat$x$} and column \aspas{$i$} is the item intercept $a_{i,x}\xi$ of item $i$ in the category \aspas{$x$}.
According to the Infit/Outfit statistics of items, no one mean-square value is greater than $2.0$ indicating that the measurement system is not distorted or degraded by the items.

%latex.default(estimated_params_df, caption = paste0("Estimated parameters in the ",     irt.short.name, "-based instrument", " for measuring the ",     lname), size = "scriptsize", longtable = T, ctable = F, landscape = F,     where = "!htbp", file = filename, append = T)%
\setlongtables{\scriptsize
\begin{longtable}{lrrrrrr}\caption{Estimated parameters in the GPCM-based instrument for measuring the pre-test skill/knowledge of the third empirical study} \tabularnewline
\hline\hline
\multicolumn{1}{l}{}&\multicolumn{1}{c}{An3a}&\multicolumn{1}{c}{An3b}&\multicolumn{1}{c}{Ap1}&\multicolumn{1}{c}{Ap3}&\multicolumn{1}{c}{Ev2}&\multicolumn{1}{c}{Un2}\tabularnewline
\hline
\endfirsthead\caption[]{\em (continued)} \tabularnewline
\hline
\multicolumn{1}{l}{}&\multicolumn{1}{c}{An3a}&\multicolumn{1}{c}{An3b}&\multicolumn{1}{c}{Ap1}&\multicolumn{1}{c}{Ap3}&\multicolumn{1}{c}{Ev2}&\multicolumn{1}{c}{Un2}\tabularnewline
\hline
\endhead
\hline
\endfoot
\label{tab:item-parameters-pre-third-study}
xsi.item&$ 0.030$&$ 0.065$&$-0.013$&$ 0.019$&$ 0.073$&$  0.006$\tabularnewline
B.Cat0&$ 0.000$&$ 0.000$&$ 0.000$&$ 0.000$&$ 0.000$&$  0.000$\tabularnewline
B.Cat1&$ 1.000$&$ 1.000$&$ 1.000$&$ 1.000$&$ 1.000$&$  1.000$\tabularnewline
B.Cat2&$ 2.000$&$ 2.000$&$ 2.000$&$ 2.000$&$ 2.000$&$  2.000$\tabularnewline
B.Cat3&$ 3.000$&$ 3.000$&$ 3.000$&$ 3.000$&$ 3.000$&$  3.000$\tabularnewline
B.Cat4&$ 4.000$&$ 4.000$&$ 4.000$&$ 4.000$&$ 4.000$&$  4.000$\tabularnewline
B.Cat5&$ 5.000$&$ 5.000$&$ 5.000$&$ 5.000$&$ 5.000$&$  5.000$\tabularnewline
B.Cat6&$ 6.000$&$ 6.000$&$ 6.000$&$ 6.000$&$ 6.000$&$  6.000$\tabularnewline
B.Cat7&$ 7.000$&$ 7.000$&$ 7.000$&$ 7.000$&$ 7.000$&$  7.000$\tabularnewline
B.Cat8&$ 8.000$&$ 8.000$&$ 8.000$&$ 8.000$&$ 8.000$&$  8.000$\tabularnewline
B.Cat9&$ 9.000$&$ 9.000$&$ 9.000$&$ 9.000$&$ 9.000$&$  9.000$\tabularnewline
B.Cat10&$10.000$&$10.000$&$10.000$&$10.000$&$10.000$&$ 10.000$\tabularnewline
B.Cat11&$11.000$&$11.000$&$11.000$&$11.000$&$11.000$&$ 11.000$\tabularnewline
B.Cat12&$12.000$&$12.000$&$12.000$&$12.000$&$12.000$&$ 12.000$\tabularnewline
B.Cat13&$13.000$&$13.000$&$13.000$&$13.000$&$13.000$&$ 13.000$\tabularnewline
B.Cat14&$14.000$&$14.000$&$14.000$&$14.000$&$14.000$&$ 14.000$\tabularnewline
B.Cat15&$ 0.000$&$ 0.000$&$15.000$&$15.000$&$15.000$&$ 15.000$\tabularnewline
B.Cat16&$ 0.000$&$ 0.000$&$16.000$&$16.000$&$16.000$&$ 16.000$\tabularnewline
B.Cat17&$ 0.000$&$ 0.000$&$17.000$&$17.000$&$17.000$&$ 17.000$\tabularnewline
B.Cat18&$ 0.000$&$ 0.000$&$18.000$&$18.000$&$18.000$&$ 18.000$\tabularnewline
B.Cat19&$ 0.000$&$ 0.000$&$19.000$&$19.000$&$ 0.000$&$ 19.000$\tabularnewline
B.Cat20&$ 0.000$&$ 0.000$&$20.000$&$20.000$&$ 0.000$&$ 20.000$\tabularnewline
B.Cat21&$ 0.000$&$ 0.000$&$21.000$&$21.000$&$ 0.000$&$ 21.000$\tabularnewline
B.Cat22&$ 0.000$&$ 0.000$&$22.000$&$22.000$&$ 0.000$&$ 22.000$\tabularnewline
B.Cat23&$ 0.000$&$ 0.000$&$ 0.000$&$ 0.000$&$ 0.000$&$ 23.000$\tabularnewline
B.Cat24&$ 0.000$&$ 0.000$&$ 0.000$&$ 0.000$&$ 0.000$&$ 24.000$\tabularnewline
B.Cat25&$ 0.000$&$ 0.000$&$ 0.000$&$ 0.000$&$ 0.000$&$ 25.000$\tabularnewline
B.Cat26&$ 0.000$&$ 0.000$&$ 0.000$&$ 0.000$&$ 0.000$&$ 26.000$\tabularnewline
B.Cat27&$ 0.000$&$ 0.000$&$ 0.000$&$ 0.000$&$ 0.000$&$ 27.000$\tabularnewline
B.Cat28&$ 0.000$&$ 0.000$&$ 0.000$&$ 0.000$&$ 0.000$&$ 28.000$\tabularnewline
B.Cat29&$ 0.000$&$ 0.000$&$ 0.000$&$ 0.000$&$ 0.000$&$ 29.000$\tabularnewline
B.Cat30&$ 0.000$&$ 0.000$&$ 0.000$&$ 0.000$&$ 0.000$&$ 30.000$\tabularnewline
B.Cat31&$ 0.000$&$ 0.000$&$ 0.000$&$ 0.000$&$ 0.000$&$ 31.000$\tabularnewline
B.Cat32&$ 0.000$&$ 0.000$&$ 0.000$&$ 0.000$&$ 0.000$&$ 32.000$\tabularnewline
B.Cat33&$ 0.000$&$ 0.000$&$ 0.000$&$ 0.000$&$ 0.000$&$ 33.000$\tabularnewline
B.Cat34&$ 0.000$&$ 0.000$&$ 0.000$&$ 0.000$&$ 0.000$&$ 34.000$\tabularnewline
B.Cat35&$ 0.000$&$ 0.000$&$ 0.000$&$ 0.000$&$ 0.000$&$ 35.000$\tabularnewline
B.Cat36&$ 0.000$&$ 0.000$&$ 0.000$&$ 0.000$&$ 0.000$&$ 36.000$\tabularnewline
B.Cat37&$ 0.000$&$ 0.000$&$ 0.000$&$ 0.000$&$ 0.000$&$ 37.000$\tabularnewline
B.Cat38&$ 0.000$&$ 0.000$&$ 0.000$&$ 0.000$&$ 0.000$&$ 38.000$\tabularnewline
B.Cat39&$ 0.000$&$ 0.000$&$ 0.000$&$ 0.000$&$ 0.000$&$ 39.000$\tabularnewline
B.Cat40&$ 0.000$&$ 0.000$&$ 0.000$&$ 0.000$&$ 0.000$&$ 40.000$\tabularnewline
B.Cat41&$ 0.000$&$ 0.000$&$ 0.000$&$ 0.000$&$ 0.000$&$ 41.000$\tabularnewline
B.Cat42&$ 0.000$&$ 0.000$&$ 0.000$&$ 0.000$&$ 0.000$&$ 42.000$\tabularnewline
AXsi.Cat0&$ 0.000$&$ 0.000$&$ 0.000$&$ 0.000$&$ 0.000$&$  0.000$\tabularnewline
AXsi.Cat1&$-6.608$&$-6.424$&$-6.249$&$-5.083$&$-6.162$&$ -6.234$\tabularnewline
AXsi.Cat2&$-7.773$&$-8.559$&$-8.205$&$-7.157$&$-7.963$&$ -7.998$\tabularnewline
AXsi.Cat3&$-8.053$&$-9.409$&$-8.939$&$-8.177$&$-8.692$&$ -8.967$\tabularnewline
AXsi.Cat4&$-6.692$&$-8.749$&$-9.472$&$-8.725$&$-9.030$&$ -9.913$\tabularnewline
AXsi.Cat5&$-3.028$&$-6.715$&$-9.537$&$-8.919$&$-8.693$&$-10.148$\tabularnewline
AXsi.Cat6&$-1.045$&$-1.541$&$-9.387$&$-8.781$&$-6.938$&$ -9.931$\tabularnewline
AXsi.Cat7&$-0.593$&$-2.375$&$-8.827$&$-8.305$&$-3.495$&$ -9.579$\tabularnewline
AXsi.Cat8&$-3.026$&$-3.577$&$-7.911$&$-7.450$&$-0.672$&$ -9.065$\tabularnewline
AXsi.Cat9&$-6.673$&$-7.170$&$-6.166$&$-5.841$&$-3.510$&$ -7.907$\tabularnewline
AXsi.Cat10&$-7.543$&$-7.541$&$-1.781$&$-0.760$&$-7.131$&$ -5.895$\tabularnewline
AXsi.Cat11&$-6.507$&$-6.311$&$-6.144$&$-1.604$&$-8.700$&$ -0.628$\tabularnewline
AXsi.Cat12&$-2.311$&$-3.585$&$-7.598$&$-3.588$&$-9.521$&$ -0.688$\tabularnewline
AXsi.Cat13&$-1.889$&$-2.883$&$-8.300$&$-5.826$&$-9.563$&$ -0.369$\tabularnewline
AXsi.Cat14&$-0.423$&$-0.905$&$-8.407$&$-7.696$&$-8.974$&$ -0.056$\tabularnewline
AXsi.Cat15&$$&$$&$-8.270$&$-8.421$&$-8.844$&$ -5.141$\tabularnewline
AXsi.Cat16&$$&$$&$-7.623$&$-8.996$&$-8.289$&$ -6.966$\tabularnewline
AXsi.Cat17&$$&$$&$-6.085$&$-9.055$&$-6.950$&$ -7.980$\tabularnewline
AXsi.Cat18&$$&$$&$-3.306$&$-8.914$&$-1.306$&$ -8.589$\tabularnewline
AXsi.Cat19&$$&$$&$-6.316$&$-8.382$&$$&$ -8.632$\tabularnewline
AXsi.Cat20&$$&$$&$-7.231$&$-7.435$&$$&$ -8.486$\tabularnewline
AXsi.Cat21&$$&$$&$-5.925$&$-5.716$&$$&$ -7.945$\tabularnewline
AXsi.Cat22&$$&$$&$ 0.284$&$-0.417$&$$&$ -6.822$\tabularnewline
AXsi.Cat23&$$&$$&$$&$$&$$&$ -5.445$\tabularnewline
AXsi.Cat24&$$&$$&$$&$$&$$&$ -2.093$\tabularnewline
AXsi.Cat25&$$&$$&$$&$$&$$&$ -1.456$\tabularnewline
AXsi.Cat26&$$&$$&$$&$$&$$&$ -0.586$\tabularnewline
AXsi.Cat27&$$&$$&$$&$$&$$&$  0.964$\tabularnewline
AXsi.Cat28&$$&$$&$$&$$&$$&$ -0.138$\tabularnewline
AXsi.Cat29&$$&$$&$$&$$&$$&$ -2.046$\tabularnewline
AXsi.Cat30&$$&$$&$$&$$&$$&$ -6.085$\tabularnewline
AXsi.Cat31&$$&$$&$$&$$&$$&$ -7.783$\tabularnewline
AXsi.Cat32&$$&$$&$$&$$&$$&$ -8.824$\tabularnewline
AXsi.Cat33&$$&$$&$$&$$&$$&$ -9.423$\tabularnewline
AXsi.Cat34&$$&$$&$$&$$&$$&$ -9.713$\tabularnewline
AXsi.Cat35&$$&$$&$$&$$&$$&$ -9.829$\tabularnewline
AXsi.Cat36&$$&$$&$$&$$&$$&$ -9.526$\tabularnewline
AXsi.Cat37&$$&$$&$$&$$&$$&$ -8.816$\tabularnewline
AXsi.Cat38&$$&$$&$$&$$&$$&$ -7.746$\tabularnewline
AXsi.Cat39&$$&$$&$$&$$&$$&$ -5.916$\tabularnewline
AXsi.Cat40&$$&$$&$$&$$&$$&$ -1.936$\tabularnewline
AXsi.Cat41&$$&$$&$$&$$&$$&$ -0.577$\tabularnewline
AXsi.Cat42&$$&$$&$$&$$&$$&$ -0.258$\tabularnewline
max.Outfit&$ 1.002$&$ 1.069$&$ 1.028$&$ 1.167$&$ 0.985$&$  0.580$\tabularnewline
max.Infit&$ 1.002$&$ 1.069$&$ 1.028$&$ 1.167$&$ 0.985$&$  0.580$\tabularnewline
\hline
\end{longtable}}

\autoref{tab:item-parameters-pos-third-study} shows the estimated parameters for the GPCM-based instrument used to measure the post-test skill/knowledge of the third empirical study.
These parameters had been calculated using the MML method \cite{BockAitkin1981}, so that the value in row \aspas{B.Cat$x$} and column \aspas{$i$} is the item slope $b_{i,x}$ of item $i$ in the category \aspas{$x$}, and the value in the row \aspas{AXsi.Cat$x$} and column \aspas{$i$} is the item intercept $a_{i,x}\xi$ of item $i$ in the category \aspas{$x$}.
According to the Infit/Outfit statistics of items, no one mean-square value is greater than $2.0$ indicating that the measurement system is not distorted or degraded by the items.

%latex.default(estimated_params_df, caption = paste0("Estimated parameters in the ",     irt.short.name, "-based instrument", " for measuring the ",     lname), size = "scriptsize", longtable = T, ctable = F, landscape = F,     where = "!htbp", file = filename, append = T)%
\setlongtables{\scriptsize
\begin{longtable}{lrrrrrrr}\caption{Estimated parameters in the GPCM-based instrument for measuring the post-test skill/knowledge in the third empirical study} \tabularnewline
\hline\hline
\multicolumn{1}{l}{}&\multicolumn{1}{c}{AnC1}&\multicolumn{1}{c}{AnC2}&\multicolumn{1}{c}{ApA}&\multicolumn{1}{c}{ApC}&\multicolumn{1}{c}{EvB}&\multicolumn{1}{c}{PGs3}&\multicolumn{1}{c}{ReB}\tabularnewline
\hline
\endfirsthead\caption[]{\em (continued)} \tabularnewline
\hline
\multicolumn{1}{l}{}&\multicolumn{1}{c}{AnC1}&\multicolumn{1}{c}{AnC2}&\multicolumn{1}{c}{ApA}&\multicolumn{1}{c}{ApC}&\multicolumn{1}{c}{EvB}&\multicolumn{1}{c}{PGs3}&\multicolumn{1}{c}{ReB}\tabularnewline
\hline
\endhead
\hline
\endfoot
\label{tab:item-parameters-pos-third-study}
xsi.item&$ 0.031$&$ 0.139$&$ 0.037$&$  0.095$&$  0.118$&$-16.012$&$-0.052$\tabularnewline
B.Cat0&$ 0.000$&$ 0.000$&$ 0.000$&$  0.000$&$  0.000$&$  0.000$&$ 0.000$\tabularnewline
B.Cat1&$ 1.000$&$ 1.000$&$ 1.000$&$  1.000$&$  1.000$&$  1.000$&$ 1.000$\tabularnewline
B.Cat2&$ 2.000$&$ 2.000$&$ 2.000$&$  2.000$&$  2.000$&$  2.000$&$ 2.000$\tabularnewline
B.Cat3&$ 3.000$&$ 3.000$&$ 3.000$&$  3.000$&$  3.000$&$  3.000$&$ 3.000$\tabularnewline
B.Cat4&$ 4.000$&$ 4.000$&$ 4.000$&$  4.000$&$  4.000$&$  4.000$&$ 4.000$\tabularnewline
B.Cat5&$ 5.000$&$ 5.000$&$ 5.000$&$  5.000$&$  5.000$&$  0.000$&$ 5.000$\tabularnewline
B.Cat6&$ 6.000$&$ 6.000$&$ 6.000$&$  6.000$&$  6.000$&$  0.000$&$ 6.000$\tabularnewline
B.Cat7&$ 7.000$&$ 7.000$&$ 7.000$&$  7.000$&$  7.000$&$  0.000$&$ 7.000$\tabularnewline
B.Cat8&$ 8.000$&$ 8.000$&$ 8.000$&$  8.000$&$  8.000$&$  0.000$&$ 8.000$\tabularnewline
B.Cat9&$ 9.000$&$ 9.000$&$ 9.000$&$  9.000$&$  9.000$&$  0.000$&$ 9.000$\tabularnewline
B.Cat10&$10.000$&$10.000$&$10.000$&$ 10.000$&$ 10.000$&$  0.000$&$10.000$\tabularnewline
B.Cat11&$11.000$&$11.000$&$11.000$&$ 11.000$&$ 11.000$&$  0.000$&$11.000$\tabularnewline
B.Cat12&$12.000$&$12.000$&$12.000$&$ 12.000$&$ 12.000$&$  0.000$&$12.000$\tabularnewline
B.Cat13&$13.000$&$13.000$&$13.000$&$ 13.000$&$ 13.000$&$  0.000$&$13.000$\tabularnewline
B.Cat14&$14.000$&$14.000$&$14.000$&$ 14.000$&$ 14.000$&$  0.000$&$14.000$\tabularnewline
B.Cat15&$ 0.000$&$ 0.000$&$ 0.000$&$ 15.000$&$ 15.000$&$  0.000$&$15.000$\tabularnewline
B.Cat16&$ 0.000$&$ 0.000$&$ 0.000$&$ 16.000$&$ 16.000$&$  0.000$&$16.000$\tabularnewline
B.Cat17&$ 0.000$&$ 0.000$&$ 0.000$&$ 17.000$&$ 17.000$&$  0.000$&$17.000$\tabularnewline
B.Cat18&$ 0.000$&$ 0.000$&$ 0.000$&$ 18.000$&$ 18.000$&$  0.000$&$18.000$\tabularnewline
B.Cat19&$ 0.000$&$ 0.000$&$ 0.000$&$  0.000$&$  0.000$&$  0.000$&$19.000$\tabularnewline
B.Cat20&$ 0.000$&$ 0.000$&$ 0.000$&$  0.000$&$  0.000$&$  0.000$&$20.000$\tabularnewline
B.Cat21&$ 0.000$&$ 0.000$&$ 0.000$&$  0.000$&$  0.000$&$  0.000$&$21.000$\tabularnewline
B.Cat22&$ 0.000$&$ 0.000$&$ 0.000$&$  0.000$&$  0.000$&$  0.000$&$22.000$\tabularnewline
B.Cat23&$ 0.000$&$ 0.000$&$ 0.000$&$  0.000$&$  0.000$&$  0.000$&$23.000$\tabularnewline
B.Cat24&$ 0.000$&$ 0.000$&$ 0.000$&$  0.000$&$  0.000$&$  0.000$&$24.000$\tabularnewline
B.Cat25&$ 0.000$&$ 0.000$&$ 0.000$&$  0.000$&$  0.000$&$  0.000$&$25.000$\tabularnewline
B.Cat26&$ 0.000$&$ 0.000$&$ 0.000$&$  0.000$&$  0.000$&$  0.000$&$26.000$\tabularnewline
B.Cat27&$ 0.000$&$ 0.000$&$ 0.000$&$  0.000$&$  0.000$&$  0.000$&$27.000$\tabularnewline
B.Cat28&$ 0.000$&$ 0.000$&$ 0.000$&$  0.000$&$  0.000$&$  0.000$&$28.000$\tabularnewline
B.Cat29&$ 0.000$&$ 0.000$&$ 0.000$&$  0.000$&$  0.000$&$  0.000$&$29.000$\tabularnewline
B.Cat30&$ 0.000$&$ 0.000$&$ 0.000$&$  0.000$&$  0.000$&$  0.000$&$30.000$\tabularnewline
B.Cat31&$ 0.000$&$ 0.000$&$ 0.000$&$  0.000$&$  0.000$&$  0.000$&$31.000$\tabularnewline
B.Cat32&$ 0.000$&$ 0.000$&$ 0.000$&$  0.000$&$  0.000$&$  0.000$&$32.000$\tabularnewline
B.Cat33&$ 0.000$&$ 0.000$&$ 0.000$&$  0.000$&$  0.000$&$  0.000$&$33.000$\tabularnewline
B.Cat34&$ 0.000$&$ 0.000$&$ 0.000$&$  0.000$&$  0.000$&$  0.000$&$34.000$\tabularnewline
B.Cat35&$ 0.000$&$ 0.000$&$ 0.000$&$  0.000$&$  0.000$&$  0.000$&$35.000$\tabularnewline
B.Cat36&$ 0.000$&$ 0.000$&$ 0.000$&$  0.000$&$  0.000$&$  0.000$&$36.000$\tabularnewline
B.Cat37&$ 0.000$&$ 0.000$&$ 0.000$&$  0.000$&$  0.000$&$  0.000$&$37.000$\tabularnewline
B.Cat38&$ 0.000$&$ 0.000$&$ 0.000$&$  0.000$&$  0.000$&$  0.000$&$38.000$\tabularnewline
B.Cat39&$ 0.000$&$ 0.000$&$ 0.000$&$  0.000$&$  0.000$&$  0.000$&$39.000$\tabularnewline
B.Cat40&$ 0.000$&$ 0.000$&$ 0.000$&$  0.000$&$  0.000$&$  0.000$&$40.000$\tabularnewline
B.Cat41&$ 0.000$&$ 0.000$&$ 0.000$&$  0.000$&$  0.000$&$  0.000$&$41.000$\tabularnewline
B.Cat42&$ 0.000$&$ 0.000$&$ 0.000$&$  0.000$&$  0.000$&$  0.000$&$42.000$\tabularnewline
B.Cat43&$ 0.000$&$ 0.000$&$ 0.000$&$  0.000$&$  0.000$&$  0.000$&$43.000$\tabularnewline
B.Cat44&$ 0.000$&$ 0.000$&$ 0.000$&$  0.000$&$  0.000$&$  0.000$&$44.000$\tabularnewline
B.Cat45&$ 0.000$&$ 0.000$&$ 0.000$&$  0.000$&$  0.000$&$  0.000$&$45.000$\tabularnewline
B.Cat46&$ 0.000$&$ 0.000$&$ 0.000$&$  0.000$&$  0.000$&$  0.000$&$46.000$\tabularnewline
B.Cat47&$ 0.000$&$ 0.000$&$ 0.000$&$  0.000$&$  0.000$&$  0.000$&$47.000$\tabularnewline
B.Cat48&$ 0.000$&$ 0.000$&$ 0.000$&$  0.000$&$  0.000$&$  0.000$&$48.000$\tabularnewline
B.Cat49&$ 0.000$&$ 0.000$&$ 0.000$&$  0.000$&$  0.000$&$  0.000$&$49.000$\tabularnewline
B.Cat50&$ 0.000$&$ 0.000$&$ 0.000$&$  0.000$&$  0.000$&$  0.000$&$50.000$\tabularnewline
B.Cat51&$ 0.000$&$ 0.000$&$ 0.000$&$  0.000$&$  0.000$&$  0.000$&$51.000$\tabularnewline
B.Cat52&$ 0.000$&$ 0.000$&$ 0.000$&$  0.000$&$  0.000$&$  0.000$&$52.000$\tabularnewline
B.Cat53&$ 0.000$&$ 0.000$&$ 0.000$&$  0.000$&$  0.000$&$  0.000$&$53.000$\tabularnewline
B.Cat54&$ 0.000$&$ 0.000$&$ 0.000$&$  0.000$&$  0.000$&$  0.000$&$54.000$\tabularnewline
B.Cat55&$ 0.000$&$ 0.000$&$ 0.000$&$  0.000$&$  0.000$&$  0.000$&$55.000$\tabularnewline
B.Cat56&$ 0.000$&$ 0.000$&$ 0.000$&$  0.000$&$  0.000$&$  0.000$&$56.000$\tabularnewline
AXsi.Cat0&$ 0.000$&$ 0.000$&$ 0.000$&$  0.000$&$  0.000$&$  0.000$&$ 0.000$\tabularnewline
AXsi.Cat1&$-6.016$&$-7.630$&$-6.477$&$ -6.260$&$ -6.520$&$ 64.180$&$-4.373$\tabularnewline
AXsi.Cat2&$-7.797$&$-8.765$&$-7.520$&$ -8.031$&$ -8.764$&$ 63.710$&$ 0.942$\tabularnewline
AXsi.Cat3&$-7.643$&$-7.791$&$-6.599$&$ -9.035$&$ -8.893$&$ 63.710$&$-4.505$\tabularnewline
AXsi.Cat4&$-6.231$&$-3.055$&$-3.183$&$ -8.561$&$ -8.163$&$ 64.046$&$-6.208$\tabularnewline
AXsi.Cat5&$-2.360$&$-1.946$&$-5.662$&$ -6.585$&$ -6.520$&$$&$-7.275$\tabularnewline
AXsi.Cat6&$-1.411$&$-0.336$&$-0.720$&$ -2.792$&$ -3.627$&$$&$-7.471$\tabularnewline
AXsi.Cat7&$-1.231$&$-2.351$&$-5.722$&$ -3.472$&$ -3.532$&$$&$-7.537$\tabularnewline
AXsi.Cat8&$-3.054$&$-3.051$&$-3.191$&$ -1.023$&$ -1.395$&$$&$-7.057$\tabularnewline
AXsi.Cat9&$-6.343$&$-8.386$&$-6.700$&$ -3.552$&$ -2.948$&$$&$-6.357$\tabularnewline
AXsi.Cat10&$-6.866$&$-9.280$&$-7.405$&$ -7.056$&$ -3.666$&$$&$-4.146$\tabularnewline
AXsi.Cat11&$-6.217$&$-8.264$&$-6.479$&$ -8.584$&$ -7.352$&$$&$ 0.181$\tabularnewline
AXsi.Cat12&$-3.074$&$-1.661$&$-3.165$&$ -9.695$&$ -9.256$&$$&$-0.127$\tabularnewline
AXsi.Cat13&$-2.980$&$-8.117$&$-5.683$&$-10.164$&$-10.319$&$$&$ 0.008$\tabularnewline
AXsi.Cat14&$-0.435$&$-1.946$&$-0.511$&$-10.467$&$-10.722$&$$&$-4.277$\tabularnewline
AXsi.Cat15&$$&$$&$$&$-10.173$&$-10.519$&$$&$-6.159$\tabularnewline
AXsi.Cat16&$$&$$&$$&$ -9.278$&$ -9.662$&$$&$-6.911$\tabularnewline
AXsi.Cat17&$$&$$&$$&$ -7.619$&$ -7.857$&$$&$-7.021$\tabularnewline
AXsi.Cat18&$$&$$&$$&$ -1.702$&$ -2.132$&$$&$-6.176$\tabularnewline
AXsi.Cat19&$$&$$&$$&$$&$$&$$&$-4.458$\tabularnewline
AXsi.Cat20&$$&$$&$$&$$&$$&$$&$ 1.042$\tabularnewline
AXsi.Cat21&$$&$$&$$&$$&$$&$$&$ 0.490$\tabularnewline
AXsi.Cat22&$$&$$&$$&$$&$$&$$&$-3.616$\tabularnewline
AXsi.Cat23&$$&$$&$$&$$&$$&$$&$-5.390$\tabularnewline
AXsi.Cat24&$$&$$&$$&$$&$$&$$&$-6.193$\tabularnewline
AXsi.Cat25&$$&$$&$$&$$&$$&$$&$-6.436$\tabularnewline
AXsi.Cat26&$$&$$&$$&$$&$$&$$&$-6.284$\tabularnewline
AXsi.Cat27&$$&$$&$$&$$&$$&$$&$-5.316$\tabularnewline
AXsi.Cat28&$$&$$&$$&$$&$$&$$&$-3.471$\tabularnewline
AXsi.Cat29&$$&$$&$$&$$&$$&$$&$ 0.672$\tabularnewline
AXsi.Cat30&$$&$$&$$&$$&$$&$$&$-3.567$\tabularnewline
AXsi.Cat31&$$&$$&$$&$$&$$&$$&$-5.196$\tabularnewline
AXsi.Cat32&$$&$$&$$&$$&$$&$$&$-5.974$\tabularnewline
AXsi.Cat33&$$&$$&$$&$$&$$&$$&$-6.083$\tabularnewline
AXsi.Cat34&$$&$$&$$&$$&$$&$$&$-6.261$\tabularnewline
AXsi.Cat35&$$&$$&$$&$$&$$&$$&$-5.425$\tabularnewline
AXsi.Cat36&$$&$$&$$&$$&$$&$$&$-3.647$\tabularnewline
AXsi.Cat37&$$&$$&$$&$$&$$&$$&$-0.202$\tabularnewline
AXsi.Cat38&$$&$$&$$&$$&$$&$$&$ 1.827$\tabularnewline
AXsi.Cat39&$$&$$&$$&$$&$$&$$&$ 0.003$\tabularnewline
AXsi.Cat40&$$&$$&$$&$$&$$&$$&$-3.899$\tabularnewline
AXsi.Cat41&$$&$$&$$&$$&$$&$$&$-5.746$\tabularnewline
AXsi.Cat42&$$&$$&$$&$$&$$&$$&$-6.678$\tabularnewline
AXsi.Cat43&$$&$$&$$&$$&$$&$$&$-6.958$\tabularnewline
AXsi.Cat44&$$&$$&$$&$$&$$&$$&$-6.454$\tabularnewline
AXsi.Cat45&$$&$$&$$&$$&$$&$$&$-5.064$\tabularnewline
AXsi.Cat46&$$&$$&$$&$$&$$&$$&$-3.569$\tabularnewline
AXsi.Cat47&$$&$$&$$&$$&$$&$$&$ 1.232$\tabularnewline
AXsi.Cat48&$$&$$&$$&$$&$$&$$&$ 0.582$\tabularnewline
AXsi.Cat49&$$&$$&$$&$$&$$&$$&$-3.531$\tabularnewline
AXsi.Cat50&$$&$$&$$&$$&$$&$$&$-5.357$\tabularnewline
AXsi.Cat51&$$&$$&$$&$$&$$&$$&$-6.125$\tabularnewline
AXsi.Cat52&$$&$$&$$&$$&$$&$$&$-6.413$\tabularnewline
AXsi.Cat53&$$&$$&$$&$$&$$&$$&$-6.322$\tabularnewline
AXsi.Cat54&$$&$$&$$&$$&$$&$$&$-5.461$\tabularnewline
AXsi.Cat55&$$&$$&$$&$$&$$&$$&$-3.401$\tabularnewline
AXsi.Cat56&$$&$$&$$&$$&$$&$$&$ 2.922$\tabularnewline
\hline
max.Outfit&$ 1.008$&$ 1.000$&$ 1.004$&$  1.205$&$  1.166$&$  1.000$&$ 0.674$\tabularnewline
max.Infit&$ 1.008$&$ 1.000$&$ 1.004$&$  1.205$&$  1.166$&$  1.000$&$ 0.674$\tabularnewline
\hline
\end{longtable}}


\subsection{Gains in Skill/Knowledge as Latent Trait Estimates}

\autoref{tab:gain-skill-knowledge-estimates-third-study} shows the latent trait estimates by the GPCM-based instrument for measuring the gains in skill/knowledge of the third empirical study.

%latex.default(data_df, caption = paste0("Latent trait estimates and person model fit of the ",     irt.short.name, "-based instrument ", in_title), size = "scriptsize",     longtable = T, ctable = F, landscape = T, rowlabel = "",     where = "!htbp", file = filename, append = T)%
\setlongtables{\scriptsize
\begin{longtable}{l|rrrr|rrrr}\caption{Latent trait estimates and person model fit of the GPCM-based instrument for measuring gains in skill/knowledge of the third empirical study} \tabularnewline
\hline\hline
\multicolumn{1}{l}{}&\multicolumn{4}{|c}{Pre-test}&\multicolumn{4}{|c}{Post-test}\tabularnewline
\multicolumn{1}{l}{UserID}&\multicolumn{1}{|c}{theta}&\multicolumn{1}{c}{error}&\multicolumn{1}{c}{Outfit}&\multicolumn{1}{c}{Infit}&\multicolumn{1}{|c}{theta}&\multicolumn{1}{c}{error}&\multicolumn{1}{c}{Outfit}&\multicolumn{1}{c}{Infit}\tabularnewline
\hline
\endfirsthead\caption[]{\em (continued)} \tabularnewline
\hline
\multicolumn{1}{l}{}&\multicolumn{4}{|c}{Pre-test}&\multicolumn{4}{|c}{Post-test}\tabularnewline
\multicolumn{1}{l}{UserID}&\multicolumn{1}{|c}{theta}&\multicolumn{1}{c}{error}&\multicolumn{1}{c}{Outfit}&\multicolumn{1}{c}{Infit}&\multicolumn{1}{|c}{theta}&\multicolumn{1}{c}{error}&\multicolumn{1}{c}{Outfit}&\multicolumn{1}{c}{Infit}\tabularnewline
\hline
\endhead
\hline
\endfoot
\label{tab:gain-skill-knowledge-estimates-third-study}
10169&$-0.083$&$0.058$&$   0.414$&$  0.320$&$ 0.074$&$0.062$&$  0.458$&$  0.359$\tabularnewline
10170&$ 0.017$&$0.048$&$   1.122$&$  0.975$&$-0.020$&$0.043$&$  0.436$&$  0.215$\tabularnewline
10172&$-0.026$&$0.049$&$   1.002$&$  1.593$&$-0.080$&$0.057$&$  0.695$&$  0.573$\tabularnewline
10174&$ 0.110$&$0.680$&$   1.375$&$  1.965$&$-0.026$&$0.043$&$  1.006$&$  0.375$\tabularnewline
10175&$ 0.034$&$0.049$&$   1.066$&$  0.894$&$ 0.210$&$0.085$&$  0.834$&$  0.913$\tabularnewline
10176&$ 0.078$&$0.054$&$   0.718$&$  0.627$&$-0.016$&$0.053$&$  0.790$&$  1.114$\tabularnewline
10178&$-0.023$&$0.055$&$   1.477$&$  1.851$&$ 0.011$&$0.048$&$  0.492$&$  0.313$\tabularnewline
10179&$ 0.068$&$0.049$&$   0.224$&$  0.249$&$-0.011$&$0.043$&$  0.395$&$  0.399$\tabularnewline
10181&$-0.059$&$0.053$&$   0.479$&$  0.647$&$ 0.015$&$0.049$&$  0.850$&$  0.642$\tabularnewline
10183&$-0.026$&$0.049$&$   1.056$&$  1.013$&$-0.020$&$0.043$&$  0.180$&$  0.111$\tabularnewline
10184&$ 0.055$&$0.051$&$   0.618$&$  0.482$&$-0.020$&$0.043$&$  0.867$&$  1.210$\tabularnewline
10185&$ 0.021$&$0.048$&$   1.222$&$  1.580$&$-0.010$&$0.043$&$  0.772$&$  0.970$\tabularnewline
10186&$ 0.013$&$0.048$&$   0.217$&$  0.185$&$ 0.007$&$0.047$&$  0.672$&$  0.260$\tabularnewline
10187&$-0.023$&$0.055$&$   1.477$&$  1.851$&$-0.003$&$0.045$&$  0.449$&$  0.728$\tabularnewline
10188&$ 0.007$&$0.048$&$   1.188$&$  0.961$&$ 0.161$&$0.073$&$  1.386$&$  1.198$\tabularnewline
10189&$-0.110$&$0.066$&$   0.177$&$  0.150$&$-0.041$&$0.044$&$  2.318$&$  0.626$\tabularnewline
10190&$-0.052$&$0.050$&$   0.310$&$  0.215$&$-0.088$&$0.061$&$  0.475$&$  0.428$\tabularnewline
10191&$ 0.086$&$0.056$&$   1.119$&$  0.949$&$-0.021$&$0.043$&$  0.732$&$  0.247$\tabularnewline
10192&$ 0.017$&$0.048$&$   0.884$&$  0.697$&$ 0.127$&$0.069$&$  1.557$&$  1.476$\tabularnewline
10193&$ 0.011$&$0.048$&$   0.375$&$  0.319$&$-0.029$&$0.043$&$  1.621$&$  0.877$\tabularnewline
10197&$ 0.021$&$0.048$&$   0.678$&$  0.582$&$-0.269$&$0.081$&$ 24.053$&$ 20.830$\tabularnewline
10198&$-0.057$&$0.052$&$   0.505$&$  0.698$&$ 0.032$&$0.053$&$  0.339$&$  0.234$\tabularnewline
10200&$-0.052$&$0.050$&$   0.453$&$  0.630$&$-0.012$&$0.045$&$  0.534$&$  0.426$\tabularnewline
10201&$ 0.034$&$0.049$&$   1.010$&$  0.796$&$-0.015$&$0.043$&$  0.890$&$  0.286$\tabularnewline
10202&$-0.077$&$0.056$&$   0.616$&$  0.885$&$ 0.113$&$0.067$&$  0.631$&$  0.796$\tabularnewline
10203&$ 0.549$&$0.104$&$1025.437$&$224.121$&$-0.082$&$0.058$&$  0.317$&$  0.126$\tabularnewline
10204&$-0.025$&$0.065$&$   0.312$&$  0.374$&$ 0.020$&$0.050$&$  1.124$&$  0.787$\tabularnewline
10206&$ 0.015$&$0.048$&$   1.215$&$  0.961$&$ 0.003$&$0.046$&$  0.572$&$  0.199$\tabularnewline
10209&$ 0.045$&$0.050$&$   0.663$&$  0.387$&$ 0.181$&$0.077$&$  0.563$&$  0.383$\tabularnewline
10210&$ 0.017$&$0.048$&$   0.943$&$  1.178$&$ 0.023$&$0.051$&$  1.055$&$  0.601$\tabularnewline
10213&$-0.003$&$0.048$&$   1.373$&$  1.118$&$ 0.003$&$0.046$&$  0.597$&$  0.732$\tabularnewline
10214&$-0.110$&$0.066$&$   0.686$&$  0.674$&$ 0.002$&$0.046$&$  1.450$&$  1.251$\tabularnewline
10215&$-0.101$&$0.262$&$   2.793$&$  3.233$&$ 0.536$&$0.138$&$926.235$&$194.704$\tabularnewline
10216&$ 0.110$&$0.680$&$   0.722$&$  0.846$&$ 0.067$&$0.061$&$  1.006$&$  0.783$\tabularnewline
10217&$ 0.041$&$0.050$&$   1.181$&$  0.970$&$ 0.128$&$0.069$&$  0.449$&$  0.316$\tabularnewline
10218&$ 0.009$&$0.048$&$   0.940$&$  0.939$&$-0.015$&$0.043$&$  0.911$&$  1.083$\tabularnewline
10219&$ 0.023$&$0.050$&$   0.624$&$  0.398$&$-0.012$&$0.045$&$  0.863$&$  0.319$\tabularnewline
10220&$-0.052$&$0.050$&$   0.831$&$  0.846$&$-0.012$&$0.045$&$  1.429$&$  0.402$\tabularnewline
10221&$-0.092$&$0.060$&$   0.516$&$  0.321$&$-0.205$&$0.070$&$  5.678$&$ 11.765$\tabularnewline
10223&$ 0.007$&$0.048$&$   1.188$&$  0.961$&$-0.011$&$0.043$&$  1.210$&$  0.812$\tabularnewline
10226&$-0.049$&$0.051$&$   0.406$&$  0.324$&$ 0.004$&$0.046$&$  0.493$&$  0.643$\tabularnewline
10227&$ 0.013$&$0.048$&$   0.706$&$  0.642$&$-0.264$&$0.075$&$  9.201$&$  6.981$\tabularnewline
10228&$-0.149$&$0.082$&$   0.787$&$  0.551$&$-0.264$&$0.075$&$  8.157$&$ 12.257$\tabularnewline
10230&$ 0.060$&$0.052$&$   0.525$&$  0.478$&$-0.079$&$0.056$&$  1.086$&$  0.678$\tabularnewline
10231&$ 0.477$&$0.083$&$ 144.754$&$ 74.848$&$ 0.036$&$0.054$&$  1.624$&$  1.293$\tabularnewline
10232&$ 0.050$&$0.051$&$   0.573$&$  0.424$&$-0.024$&$0.043$&$  0.171$&$  0.061$\tabularnewline
10237&$ 0.005$&$0.048$&$   0.854$&$  0.673$&$ 0.021$&$0.051$&$  0.567$&$  0.507$\tabularnewline
10238&$-0.003$&$0.447$&$   1.172$&$  1.195$&$-0.015$&$0.043$&$  0.662$&$  0.259$\tabularnewline
10240&$-0.003$&$0.447$&$   0.677$&$  0.614$&$-0.041$&$0.046$&$  0.737$&$  0.285$\tabularnewline
\hline
\end{longtable}}



\end{apendicesenv}
% ---

% ----------------------------------------------------------
% Anexos
% ----------------------------------------------------------

% ---
% Inicia os anexos
% ---
\begin{anexosenv}
 \chapter[Web-based Questionnaire for the Adapted Portuguese IMI]{Web-based Questionnaire for the Adapted Portuguese Version of the Intrinsic Motivation Inventory (Questionnaire of Motivation Used in the Pilot Empirical Study)}
\label{annex:IMI-pilot-study}
\includegraphics[width=1\textwidth]{images/annex/IMI-pilot-study.pdf}

\chapter[Paper-based Questionnaire for the Adapted Portuguese IMI]{Paper-based Questionnaire for the Adapted Portuguese Version of the Intrinsic Motivation Inventory (Questionnaire of Motivation Used in the First Empirical Study)}
\label{annex:IMI-first-study}
%\includegraphics[width=1\textwidth]{images/annex/IMI-pilot-study.pdf} % replace it

\chapter[Paper-based Questionnaire for the Adapted Portuguese IMMS]{Paper-based Questionnaire for the Adapted Portuguese Version of the Instructional Materials Motivation Survey (Questionnaire of Motivation Used in the Second Empirical Study)}
\label{annex:IMMS-second-study}

\chapter[Web-based Questionnaire for the Adapted Portuguese Version of IMI and IMMS]{Web-based Questionnaire for the Adapted Portuguese Version of the Intrinsic Motivation Inventory and the Adapted Portuguese Instructional Materials Motivation Survey (Questionnaire of Motivation Used in the Third Empirical Study)}
\label{annex:IMI-IMMS-third-study}
\includegraphics[width=1\textwidth]{images/annex/IMI-IMMS-third-study-01.pdf}
\newpage
\includegraphics[width=1\textwidth]{images/annex/IMI-IMMS-third-study-02.pdf}

\end{anexosenv}
% ---

\end{document}